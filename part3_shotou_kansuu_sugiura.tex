\part*{第3章:初等函数}
\addcontentsline{toc}{part}{\texorpdfstring{第3章:初等函数}{第3章:初等函数}}

\section*{p166--167:1}\label{p166--167:1}
\addcontentsline{toc}{section}{\texorpdfstring{p166--167:1}{p166--167:1}}

ラプラシアン(ラプラス作用素)$\Laplacian$は$\sum_{i=1}^n \frac{\partial^2}{\partial x_i^2}$で定義されていたことを思い出そう(p135).

\begin{tleftbar}
    \begin{proof}
        以下定理1.1と同様に$z = x + iy$とし,$f(z) = u + iv$と書く.このとき定義より,
        \[
            \Laplacian u = u_{x, x} + u_{y, y},
            \quad \Laplacian v = v_{x, x} + v_{y,y}
        \]
        となる%
        \footnote{%
            $f_{x_i, x_j} = \frac{\partial^2 f}{\partial x_i \partial x_j} = \frac{\partial}{\partial x_i} \left(\frac{\partial f}{\partial x_j} \right)$であったことに注意せよ.%
        }%
        .コーシー・リーマンの関係式(定理1.1)より,
        \[
            u_{x, x} = v_{y, x}, \quad u_{y, y} = -v_{x, y}
        \]
        が成り立つ.ここで仮定より $u, v$ は $C^2$ 級であるから,定理II.3.3により,$v_{y, x} = v_{x, y}$ である.よって
        \[
            u_{x, x} + u_{y, y} = v_{y, x} - v_{x, y} = 0
        \]
        が成り立つ.同様に
        \[
            v_{x, x} + v_{y, y} = -u_{y, x} + u_{x, y} = 0
        \]
        も成立する.
    \end{proof}
\end{tleftbar}

問題文において言及されているように,$u, v$の$C^2$級との仮定は実際には不要である.定理IX.3.1系2を参照せよ.


\section*{p166--167:2}\label{p166--167:2}
\addcontentsline{toc}{section}{\texorpdfstring{p166--167:2}{p166--167:2}}

\begin{tleftbar}
    \begin{proof}
        以下定理1.1と同様に$f(x + iy) = u + iv$と書く.
        このとき\hyperref[p166--167:1]{p166--167:1}より$\Laplacian u = \Laplacian v = 0$が($C^2$級の仮定なしに)成立することを用いて%
        \footnote{%
            $\Laplacian u, \Laplacian v$延いては$\Laplacian\abs{f}^2$が定義できることが保証されるのは,これに拠るものである.問題文において(不親切にも)言及されていないので注意.
        }%
        ,
        \begin{align*}
            \Laplacian\abs{f}^2 ={} & \Laplacian\abs{u + iv}^2                                                                                       \\
            ={}                     & \Laplacian (u^2 + v^2)                                                                                         \\
            ={}                     & u\Laplacian u + 2(u_x^2 + u_y^2) + v\Laplacian v + 2(v_x^2 + v_y^2)                                            \\
            ={}                     & 4(u_x^2 + v_y^2) \quad                                              & (\because\ \text{コーシー・リーマンの関係式(定理1.1)}).
        \end{align*}
        これと$\abs{f'(x)}^2 = \abs{u_x + iv_y}^2 = u_x^2 + v_y^2$(定理1.1)を合わせて,題意が示される.
    \end{proof}
\end{tleftbar}


\section*{p166--167:3}
\addcontentsline{toc}{section}{\texorpdfstring{p166--167:3}{p166--167:3}}

\begin{tleftbar}
    \begin{proof}
        以下$z = x + iy$と書く.まず$\abs{z}$について示す.実部を$u$,虚部を$v$と置けば$u = \abs{x + iy},\ v = 0$.
        ここで,$u$は$(x, y) = (0, 0)$で微分可能ではないから,$\abs{z}$も$0$において複素微分可能でない.
        以下$(x, y) \neq (0, 0)$であったとする.$x \neq 0$のとき,
        \begin{align*}
            u_x ={} & \frac{\partial}{\partial x}\sqrt{x^2 + y^2} \\
            ={}     & \frac{x}{\sqrt{x^2 + y^2}}                  \\
            \neq{}  & 0                                           \\
            ={}     & v_y(x, y).
        \end{align*}
        $y \neq 0$の時も同様に $u_y \neq -v_x$.
        従ってコーシー・リーマンの関係式(定理1.1)が満されないから,$\abs{z}$は$\mathbb{C}$上至る所複素微分可能でない.

        次に$\bar{z}$について示す.同様に実部を$u$,虚部を$v$と置けば$u = x,\ v = -y$.ここで
        \[
            u_x = 1,\quad v_y = -1.
        \]
        よって$u_x(x, y) \neq v_y(x, y)$よりコーシー・リーマンの関係式(定理1.1)が満されず,$\bar{z}$は$\mathbb{C}$上至る所複素微分可能でない.
    \end{proof}
\end{tleftbar}

別解として$\abs{z}^2$の微分可能性に帰着させることもできる.実際$z_0 \in \mathbb{C}$において$\abs{z}$あるいは$\bar{z}$の少なくとも一方が複素微分可能であったならば,命題1.2より$\abs{z}^2 = z\bar{z}$も$z_0$で複素微分可能となるが,$z \neq 0$のときこれは成立しない.ただし$z = 0$の場合のみ別で証明する必要がある%
\footnote{$\abs{z}$ の方については平方根を回避できるメリットはあるが,おそらく普通に示した方が楽であろう.}%
.


\section*{p166--167:4}
\addcontentsline{toc}{section}{\texorpdfstring{p166--167:4}{p166--167:4}}

\begin{tleftbar}
    \begin{proof}
        $U$上正則な函数$f$が実数値のみを取るとき,$f$が定数であることを示せば十分.以下定理1.1と同様に$f(x + iy) = u + iv$と書く.このとき$U$上至る所$v = 0$であるから$v_x = v_y = 0$.これとコーシー・リーマンの関係式(定理1.1)から$u_x = u_y = 0$.再び定理1.1より$f'(z) = u_x + iv_x = 0$.命題1.4と合わせて$f$が定数であることが導かれる.
    \end{proof}
\end{tleftbar}

\section*{p191--193:1}
\addcontentsline{toc}{section}{\texorpdfstring{p191--193:1}{p191--193:1}}


\begin{tleftbar}
    \begin{proof}
        $m,n \in \mathbb{N},~n >0$とし,$e=\frac{m}{n}$と表される,すなわち$e$が有理数だと仮定する.このとき,与えられた式を変形して,
        \[
            \frac{e^\theta}{(n+1)!} = e-\sum_{k=0}^{n} \frac{1}{k!} =\frac{m}{n}-\sum_{k=0}^{n} \frac{1}{k!}
        \]
        とする.これにより.
        \[
            \frac{e^{\theta}}{n+1} = m \cdot n! - \sum_{k=0}^{n} \frac{n!}{k!}
        \]
        であり,$m \cdot n! - \sum_{k=0}^{n} \frac{n!}{k!} \in \mathbb{Z}$であるから,$\frac{e^{\theta}}{n+1} \in \mathbb{Z}$である.
        よって,$0< \theta <1,~2<e<3$とあわせて,
        \[
            1 \le \frac{e^{\theta}}{n+1} < \frac{3}{n+1}
        \]
        であり,$\frac{e^{\theta}}{n+1} \in \mathbb{Z}$であるから$n=1$となる.ゆえに$e=m$となり,$e$は整数である.しかし$2<e<3$であるから,これは矛盾である.よって先の仮定が誤りであり,$e$は無理数である.
    \end{proof}
\end{tleftbar}



\section*{p191--193:2}
\addcontentsline{toc}{section}{\texorpdfstring{p191--193:2}{p191--193:2}}

\subsection*{p191--193:2-(\romannumeral1)}
\addcontentsline{toc}{subsection}{\texorpdfstring{p191--193:2-(\romannumeral1)}{p191--193:2-(\romannumeral1)}}

\begin{tleftbar}
    \begin{proof}
        $k = 0$のとき$f(0) = f(\pi) = 0$より明らかに成立.$1 \leq k \leq 2\pi$とする.
        $t = x(p - qx)$ と置くと,
        \[
            t' = p - 2qx, \quad t'' = -2q, \quad t^{(3)} = 0
        \]
        より$t^{(n)}$は$x=0,\pi$で整数.ここで$T_k = \frac{1}{n!}\frac{d^kt^n}{dt^k}$と置けば,$T_k' = t' \cdot T_{k+1}$である.従って$f^{(k)}(x)$は
        \begin{align*}
            f'(x) ={}      & t' \cdot T_1,                       \\
            f''(x) ={}     & t'' \cdot T_1 + (t')^2 \cdot T_2,   \\
            f^{(3)}(x) ={} & 3t't'' \cdot T_2 + (t')^3 \cdot T_3 \\
        \end{align*}
        のように$t', t''$と$T_k$の積の和で表わされる.よって$T_k$が$x = 0, \pi$において整数であることを示せば十分.$1 \leq k \leq n$ のとき
        \[
            T_k = \frac{n(n-1)\dotsm(n-k+1)}{n!}t^{n-k}
        \]
        である.特に$k < n$のとき$t$を因数に持つから,$t = 0$より$0$となり整数.
        また$k = n$のときも$T_k = \frac{n!}{n!} = 1$となり整数である.
        更に$k > n$のとき,$T_k$は多項式として$0$となるから整数.
        よって題意は示された.
    \end{proof}
\end{tleftbar}

\subsection*{p191--193:2-(\romannumeral2)}
\addcontentsline{toc}{subsection}{\texorpdfstring{p191--193:2-(\romannumeral2)}{p191--193:2-(\romannumeral2)}}

\begin{tleftbar}
    \begin{proof}
        定理IV.5.7の部分積分法を用いれば,
        \begin{align*}
                & \int_0^\pi f(x)\sin x\, dx                                                                             \\
            ={} & - \Bigl[f(x) \cos x \Bigr]_0^\pi + \int_0^\pi f'(x)\cos x\, dx                         & (\text{部分積分}) \\
            ={} & (f(0) + f(\pi)) + \Bigl[f'(x) \sin x \Bigr]_0^\pi - \int_0^\pi f''(x)\sin x\, dx \quad & (\text{部分積分}) \\
            ={} & (f(0) + f(\pi)) - \int_0^\pi f''(x)\sin x\, dx.
        \end{align*}
        これと同様の操作を $\int_0^\pi f^{(2n+2)}(x)\sin x\, dx = 0$ の項が出てくるまで繰り返すことにより,
        \[
            I = \int_0^\pi f(x)\sin x\, dx = \sum_{k = 0}^n (-1)^k (f^{2k}(0) + f^{(2k)}(\pi)).
        \]
        よって(\romannumeral1)と合わせて$I$は整数.
    \end{proof}
\end{tleftbar}

\subsection*{p191--193:2-(\romannumeral3)}
\addcontentsline{toc}{subsection}{\texorpdfstring{p191--193:2-(\romannumeral3)}{p191--193:2-(\romannumeral3)}}

\begin{tleftbar}
    \begin{proof}
        $0 < x < \pi = \frac{p}{q}$のとき$0 < p - qx < p$.この範囲で$0 < \sin x \leq 1$であることと合わせて,
        \[
            0 < \frac{x^n(p - qx)^n}{n!}\sin x \leq \frac{x^np^n}{n!}.
        \]
        従って,積分することにより
        \[
            0 < I \leq \int_{0}^{\pi} \frac{x^np^n}{n!}\, dx = \frac{(p\pi)^{n+1}}{p(n+1)!}
        \]
        を得る.$(\text{最右辺}) \to 0\ (n \to \infty)$であるから,特に十分大きな$n$に対して
        \[
            0 < I < \frac{(p\pi)^{n+1}}{p(n+1)!} < 1
        \]
        が導かれる.

        これは(\romannumeral2)の結果と矛盾するから,最初の仮定が誤っていたことが分かる.よって$\pi$は無理数であることが示され,証明は完結した.
    \end{proof}
\end{tleftbar}


\section*{p191--193:6}
\addcontentsline{toc}{section}{\texorpdfstring{p191--193:2}{p191--193:2}}

$M \times N$行列の列$\{A_i\}$が$A$に収束するとは$\norm{A_i - A} \to 0$で定義されるのであった.ここで$\norm{B}$は
\[
    \norm{B} = \sup_{x \in \mathbf{C}^n,\ \abs{x} \leqq 1}\abs{Bx}
\]
で定義されていたことを思い出そう(杉浦II.p106--107:命題4.2 を参照せよ).以下が行列のノルムの重要な性質である:
\[
    \norm{A + B} \leq \norm{A} + \norm{B}, \quad \norm{aA} = \abs{a}\norm{A}, \quad \norm{AB} \leq \norm{A}\norm{B}.
\]

行列の級数$\sum_{i=0}^\infty A_i$が絶対収束するとは$\sum_{i=0}^\infty \norm{A_i}$が有限の値に収束することを意味する.絶対収束する級数は収束する.まず題意の級数が絶対収束することから示そう.

\begin{tleftbar}
    \begin{proof}
        $\norm{A^n} \leq \norm{A}^n$より,$\sum_{n = 0}^\infty \norm{\frac{A^n}{n!}}$の各項は$\frac{\norm{A}^n}{n!}$で上から抑えられる.また,$\sum_{n=0}^\infty \frac{\norm{A}^n}{n!}$は$\exp \norm{A}$に収束する.従って$\sum_{n = 0} \frac{A^n}{n!}$は絶対収束する.
    \end{proof}
\end{tleftbar}


\subsection*{p191--193:6-(\romannumeral1)}
\addcontentsline{toc}{subsection}{\texorpdfstring{p191--193:6-(\romannumeral1)}{p191--193:6-(\romannumeral1)}}

$\exp(z + w) = \exp z \exp w$と全く同様に示される.まず定理I.5.8の行列に対する対応物を示そう.

\kakko{補題}

$\sum_{n=0}^\infty A_n, \sum_{n=0}^\infty B_n$ がそれぞれ$L \times M$行列,$M \times N$行列の級数で$A, B$に絶対収束するものとする.このとき
\[
    C_n = \sum_{k=0}^n A_kB_{n-k}
\]
と置けば$\sum_{n=0}^\infty C_n$は絶対収束し,その和を$C$と置けば$C = AB$が成立する.

\begin{proof}
    証明も定理I.5.8と全く同様である.

    まず
    \[
        \sum_{n = 0}^m\norm{C_n} \leq \sum_{n = 0}^m\sum_{p + q = n}\norm{A_p}\norm{B_q} \leq \left(\sum_{p=0}^m \norm{A_p}\right) \left(\sum_{q=0}^m \norm{B_q}\right)
    \]
    であり,右辺は$m \to \infty$で有限の値に収束するから,$\sum_{n=0}^\infty C_n$も絶対収束する.その和を$C$と置こう.
    ここで非負実数列$\{\alpha_n\}, \{\beta_n\}$を
    \[
        \alpha_n = \sum_{p = n}^\infty \norm{A_p},\quad \beta_n = \sum_{q = n}^\infty \norm{B_q}
    \]
    によって定義すれば,
    \begin{align*}
               & \norm{\sum_{n=0}^{2m} C_n - \sum_{p = 0}^m A_p \sum_{q = 0}^m B_q}                                                                                                  \\
        \leq{} & \sum_{q = m + 1}^{2m} \sum_{p = 0}^{2m - q} \norm{A_p}\norm{B_q} + \sum_{p = m+1}^{2m}\sum_{q = 0}^{2m - p} \norm{A_p}\norm{B_q}                                    \\
        \leq{} & \left(\sum_{p=0}^m \norm{A_p}\right) \left(\sum_{q=m+1}^\infty \norm{B_q}\right) + \left(\sum_{p=m+1}^\infty \norm{A_p}\right) \left(\sum_{q=0}^m \norm{B_q}\right) \\
        \leq{} & \alpha_0\beta_{m+1} + \alpha_{m+1}\beta_0.
    \end{align*}
    ここで最右辺は$m \to \infty$で$0$に収束するから,$\norm{C - AB} = 0$,つまり$C = AB$が成り立つ.
\end{proof}

\begin{tleftbar}
    \begin{proof}
        $\exp A = \sum_{n=0}^\infty \frac{A^n}{n!},\ \exp B = \sum_{n=0}^\infty \frac{B^n}{n!}$である.
        \[
            C_n = \sum_{k = 0}^n \frac{A^k}{k!} \frac{B^{n-k}}{(n-k)!}
            = \frac{1}{n!} \sum_{k=0}^n \binom{n}{k} A^k B^{n-k}
        \]
        と置けば,$AB = BA$と二項定理より$C_n = \frac{(A + B)^n}{n!}$.ここで補題を適用すれば,
        \[
            \exp A \exp B = \sum_{n=0}^\infty C_n = \exp(A + B).
        \]
    \end{proof}
\end{tleftbar}


\subsection*{p191--193:6-(\romannumeral2)}
\addcontentsline{toc}{subsection}{\texorpdfstring{p191--193:6-(\romannumeral2)}{p191--193:6-(\romannumeral2)}}
\begin{tleftbar}
    \begin{proof}
        \[
            A =
            \begin{pmatrix}
                0 & 1 \\
                0 & 0
            \end{pmatrix}, \quad
            B =
            \begin{pmatrix}
                0 & 0 \\
                1 & 0
            \end{pmatrix}
        \]
        と置けば$AB \neq BA$.この時単位行列を$I$と書けば
        \begin{align*}
            \exp A \exp B ={} & (I + A) (I + B) \\
            ={}               &
            \begin{pmatrix}
                2 & 1 \\
                1 & 1
            \end{pmatrix}.
        \end{align*}
        また,
        \begin{align*}
            \exp(A + B) ={} & \sum_{n=0}^\infty \frac{1}{n!} \begin{pmatrix} 0 & 1 \\ 1 & 0\end{pmatrix}^n \\
            ={}             &
            \begin{pmatrix}
                \cosh 1 & \sinh 1 \\
                \sinh 1 & \cosh 1 \\
            \end{pmatrix}.
        \end{align*}
        従って$\exp(A + B) \neq \exp A \exp B$.二次行列以外の場合も同様に反例が構成される.
    \end{proof}
\end{tleftbar}


\subsection*{p191--193:6-(\romannumeral3)}
\addcontentsline{toc}{subsection}{\texorpdfstring{p191--193:6-(\romannumeral3)}{p191--193:6-(\romannumeral3)}}

\begin{tleftbar}
    \begin{proof}
        \begin{align*}
            P (\exp A)P^{-1} ={} & P\left(\sum_{n = 0}^\infty \frac{A^n}{n!}\right) P^{-1} \\
            ={}                  & \sum_{n = 0}^\infty P \frac{A^n}{n!} P^{-1}             \\
            ={}                  & \sum_{n = 0}^\infty \frac{(PAP^{-1})^n}{n!}             \\
            ={}                  & \exp(PAP^{-1}).
        \end{align*}
    \end{proof}
\end{tleftbar}