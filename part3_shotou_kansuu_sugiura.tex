\part*{第3章:初等函数}
\addcontentsline{toc}{part}{\texorpdfstring{第3章:初等函数}{第3章:初等函数}}


\section*{p191--193:1}
\addcontentsline{toc}{section}{\texorpdfstring{p191--193:1}{p191--193:1}}


\begin{tleftbar}
    \begin{proof}
        $m,n \in \mathbb{N},~n >0$とし,$e=\frac{m}{n}$と表される,すなわち$e$が有理数だと仮定する.このとき,与えられた式を変形して,
        \[
            \frac{e^\theta}{(n+1)!} = e-\sum_{k=0}^{n} \frac{1}{k!} =\frac{m}{n}-\sum_{k=0}^{n} \frac{1}{k!}
        \]
        とする.これにより.
        \[
            \frac{e^{\theta}}{n+1} = m \cdot n! - \sum_{k=0}^{n} \frac{n!}{k!}
        \]
        であり,$m \cdot n! - \sum_{k=0}^{n} \frac{n!}{k!} \in \mathbb{Z}$であるから,$\frac{e^{\theta}}{n+1} \in \mathbb{Z}$である.
        よって,$0< \theta <1,~2<e<3$とあわせて,
        \[
            1 \le \frac{e^{\theta}}{n+1} < \frac{3}{n+1}
        \]
        であり,$\frac{e^{\theta}}{n+1} \in \mathbb{Z}$であるから$n=1$となる.ゆえに$e=m$となり,$e$は整数である.しかし$2<e<3$であるから,これは矛盾である.よって先の仮定が誤りであり,$e$は無理数である.
    \end{proof}
\end{tleftbar}