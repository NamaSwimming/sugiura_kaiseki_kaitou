
\part*{第2章:微分法}
\addpart{第2章:微分法}


\problem[subproblems = 6]{p90--91}{1}


\subproblem{1}


\begin{minipage}[c]{0.60\textwidth}
  楕円
  \[
    \left\{ (x,y) \in \mathbb{R}^2 \; \middle| \; \frac{x^2}{4} + y^2 = 1 \right\}
  \]
  を描写すればよく,図のようになる.
\end{minipage}
\hfill
\begin{minipage}[c]{0.35\textwidth}
  \begin{tikzpicture}[scale=0.8]
    \draw[->] (-3,0) -- (3,0) node[below] {$x$};
    \draw[->] (0,-3) -- (0,3) node[left]  {$y$};
    \draw (0,0) node[below left] {$\mathrm{O}$} coordinate (O);
    \draw (0,1) node [above left] {$1$};
    \draw (0,-1) node [below left] {$-1$};
    \draw (2,0) node [above left] { $2$};
    \draw (-2,0) node [above left] { $-2$};
    \draw plot[domain=0:{2*pi}, variable=\theta, smooth] ({2*cos (\theta r)},{sin (\theta r)});
  \end{tikzpicture}
\end{minipage}

\subproblem{2}

\begin{minipage}[c]{0.60\textwidth}
  \renewcommand{\arraystretch}{1.6}
  \begin{tabular}{|c||cccccccccc|}
    \hline
    $t$
     & $0$
     & $\cdots$
     & $\tfrac{\pi}{2}$
     & $\cdots$
     & $\pi$
     & $\cdots$
     & $\tfrac{3\pi}{2}$
     & $\cdots$
     & $2\pi$
     & $\cdots$
    \\ \hline
    $\frac{dx}{dt}$
     & $0$
     & $-$
     & $0$
     & $+$
     & $0$
     & $-$
     & $0$
     & $+$
     & $0$
     &
    \\ \hline
    $x(t)$
     & $1$
     & $\searrow$
     & $-1$
     & $\nearrow$
     & $1$
     & $\searrow$
     & $-1$
     & $\nearrow$
     & $1$
     &
    \\ \hline
    $\frac{dy}{dt}$
     & $+$
     & $+$
     & $0$
     & $-$
     & $-$
     & $-$
     & $0$
     & $+$
     & $+$
     &
    \\ \hline
    $ y(t)$
     & $0$
     & $\nearrow$
     & $1$
     & $\searrow$
     & $0$
     & $\searrow$
     & $-1$
     & $\nearrow$
     & $0$
     &
    \\ \hline
  \end{tabular}
\end{minipage}
\hfill
\begin{minipage}[c]{0.35\textwidth}
  \begin{tikzpicture}[scale=0.8]
    \draw[->] (-3,0) -- (3,0) node[below] {$x$};
    \draw[->] (0,-3) -- (0,3) node[left]  {$y$};
    \draw (0,0) node[below left] {$\mathrm{O}$} coordinate (O);
    \draw plot[domain=0:{2*pi}, variable=\theta, smooth] ({cos (2*\theta r)},{sin (\theta r)});
  \end{tikzpicture}
\end{minipage}


\subproblem{3}



\begin{tikzpicture}
  \draw[->] (-3,0) -- (3,0) node[below] {$x$};
  \draw[->] (0,-3) -- (0,3) node[left]  {$y$};
  \draw (0,0) node[below left] {$\mathrm{O}$} coordinate (O);
  \draw plot[domain=0:{2*pi}, variable=\theta, smooth] ({cos (3*\theta r)},{sin (\theta r)});
\end{tikzpicture}

\problemtodo{p90--91}{2}

\problem[subproblems = 9, label = p90--91:3]{p90--91}{3}

以下では,
\[
  \bm{a}=(a_{11},a_{12},a_{13}),\quad \bm{b}=(a_{21},a_{22},a_{23}),\quad \bm{c}=(a_{31},a_{32},a_{33})
\]とする.
必要に応じて,
\[
  \bm{a}=(a_1,a_2,a_3),\quad \bm{b}=(b_1,b_2,b_3) ,\quad \bm{c}=(c_1,c_2,c_3)
\]
とも表記する.

\subproblem[label = p90--91:3-(\romannumeral1)]{1}

\begin{tproof}
  \begin{align*}
    \det (a,b,c) & = \sum_{\sigma \in S_3} \sgn(\sigma) a_{1\sigma(1)} a_{2\sigma(2)} a_{3\sigma(3)}                                                         \\
                 & = a_{11} a_{22} a_{33} + a_{12} a_{23} a_{31} + a_{13} a_{21} a_{32} - a_{13} a_{22} a_{31} - a_{12} a_{21} a_{33} - a_{11} a_{23} a_{32} \\
                 & = (a_{12} a_{23}-a_{13} a_{22})  a_{31} + (a_{13} a_{21}-a_{11} a_{23}) a_{32} + (a_{11} a_{22}-a_{12} a_{21}) a_{33}                     \\
                 & = (a \times b \, | \,  c).
  \end{align*}
  以上より,$\det (a,b,c) = (a \times b \, | \,  c)$が成り立つ.
\end{tproof}

\begin{tproof}[別解]
  $\det(a,b,c)$ を第3行について余因子展開すると,
  \begin{align*}
    \det (a,b,c) & = a_{31} \begin{vmatrix} a_{12} & a_{13} \\ a_{22} & a_{23} \end{vmatrix}
    - a_{32} \begin{vmatrix} a_{11} & a_{13} \\ a_{21} & a_{23} \end{vmatrix}
    + a_{33} \begin{vmatrix} a_{11} & a_{12} \\ a_{21} & a_{22} \end{vmatrix}                                                        \\
                 & = a_{31}(a_{12}a_{23} - a_{13}a_{22}) + a_{32}(a_{13}a_{21} - a_{11}a_{23}) + a_{33}(a_{11}a_{22} - a_{12}a_{21}) \\
                 & = (a_{12}a_{23} - a_{13}a_{22})a_{31} + (a_{13}a_{21} - a_{11}a_{23})a_{32} + (a_{11}a_{22} - a_{12}a_{21})a_{33} \\
                 & = (a \times b \mid c).
  \end{align*}
  以上より,$\det (a,b,c) = (a \times b \mid c)$が成り立つ.
\end{tproof}

\subproblem[label = p90--91:3-(\romannumeral2)]{2}

\begin{tproof}
  \begin{align*}
    (a \, | \, a \times b) & = a_{11} (a_{12} a_{23}-a_{13} a_{22})+ a_{12} (a_{13} a_{21}-a_{11} a_{23}) + a_{13} (a_{11} a_{22}-a_{12} a_{21}) \\
                           & =0.
  \end{align*}
  また,
  \begin{align*}
    (b\, | \, a \times b) & = a_{21} (a_{12} a_{23}-a_{13} a_{22})+ a_{22} (a_{13} a_{21}-a_{11} a_{23}) + a_{23} (a_{11} a_{22}-a_{12} a_{21}) \\
                          & =0.
  \end{align*}
  ゆえに,$(a \, | \, a \times b) = (b \, | \, a \times b) = 0$が成り立つ.
\end{tproof}

\subproblem[label = p90--91:3-(\romannumeral3)]{3}

\begin{tproof}
  \begin{alignat*}{2}
    \det (a,b, a \times b) & = (a \times b \, | \, a \times b) & \quad & \text{(\hyperref[p90--91:3-(\romannumeral1)]{p90--91:3-(\romannumeral1)}より)} \\
                           & = \abs{a \times b}^2              &       &
  \end{alignat*}
\end{tproof}

\subproblem[label = p90--91:3-(\romannumeral4)]{4}

\begin{tproof}
  \begin{align*}
    a \times b + b \times a & = (a_2 b_3 - a_3 b_2, a_3 b_1 - a_1 b_3, a_1 b_2 - a_2 b_1) + (b_2 a_3 - b_3 a_2, b_3 a_1 - b_1 a_3, b_1 a_2 - b_2 a_1) \\
                            & = (0, 0, 0)                                                                                                             \\
                            & = 0.
  \end{align*}
  ゆえに
  \[
    a \times b = - b \times a.
  \]
\end{tproof}

\subproblem[label = p90--91:3-(\romannumeral5)]{5}

\begin{tproof}
  \begin{align*}
    (a + b) \times c & = ( (a_2 + b_2) c_3 - (a_3 + b_3) c_2, (a_3 + b_3) c_1 - (a_1 + b_1) c_3, (a_1 + b_1) c_2 - (a_2 + b_2) c_1 ) \\
                     & =(a_2 c_3-a_3 c_2, a_3 c_1-a_1 c_3, a_1 c_2-a_2 c_1) + (b_2 c_3-b_3 c_2, b_3 c_1-b_1 c_3, b_1 c_2-b_2 c_1)    \\
                     & = a \times c + b \times c.
  \end{align*}
\end{tproof}

\subproblem[label = p90--91:3-(\romannumeral6)]{6}

\begin{tproof}
  \begin{align*}
    a \times (b \times c)
     & = a \times (b_2 c_3 - b_3 c_2, \, b_3 c_1 - b_1 c_3, \, b_1 c_2 - b_2 c_1)      \\
     & = \begin{pmatrix}
           a_2 (b_1 c_2 - b_2 c_1) - a_3 (b_3 c_1 - b_1 c_3) \\
           a_3 (b_2 c_3 - b_3 c_2) - a_1 (b_1 c_2 - b_2 c_1) \\
           a_1 (b_3 c_1 - b_1 c_3) - a_2 (b_2 c_3 - b_3 c_2)
         \end{pmatrix}                             \\
     & = \begin{pmatrix}
           a_2 b_1 c_2 - a_2 b_2 c_1 - a_3 b_3 c_1 + a_3 b_1 c_3 \\
           a_3 b_2 c_3 - a_3 b_3 c_2 - a_1 b_1 c_2 + a_1 b_2 c_1 \\
           a_1 b_3 c_1 - a_1 b_1 c_3 - a_2 b_2 c_3 + a_2 b_3 c_2
         \end{pmatrix}                         \\
     & = \begin{pmatrix}
           (a_1 c_1 + a_2 c_2 + a_3 c_3)b_1 - (a_1 b_1 + a_2 b_2 + a_3 b_3)c_1 \\
           (a_1 c_1 + a_2 c_2 + a_3 c_3)b_2 - (a_1 b_1 + a_2 b_2 + a_3 b_3)c_2 \\
           (a_1 c_1 + a_2 c_2 + a_3 c_3)b_3 - (a_1 b_1 + a_2 b_2 + a_3 b_3)c_3
         \end{pmatrix}           \\
     & = (a_1 c_1 + a_2 c_2 + a_3 c_3) \begin{pmatrix} b_1 \\ b_2 \\ b_3 \end{pmatrix}
    - (a_1 b_1 + a_2 b_2 + a_3 b_3) \begin{pmatrix} c_1 \\ c_2 \\ c_3 \end{pmatrix}    \\
     & = (a \,| \, c) b - (a \,| \, b) c.
  \end{align*}
  以上より,$a \times (b \times c) = (a \,| \, c) b - (a \,| \, b) c$ が示された.
\end{tproof}

\subproblem[label = p90--91:3-(\romannumeral7)]{7}


\begin{tproof}
  \begin{alignat*}{2}
    (a \times b \,| \, c \times d)
     & = (a \,| \, b \times (c \times d))                    & \quad & (\text{\hyperref[p90--91:3-(\romannumeral2)]{p90--91:3-(\romannumeral2)}より}) \\
     & = (a \,| \, (b \,| \, d)c - (b \,| \, c)d)            & \quad & (\text{\hyperref[p90--91:3-(\romannumeral6)]{p90--91:3-(\romannumeral6)}より}) \\
     & = (b \,| \, d)(a \,| \, c) - (b \,| \, c)(a \,| \, d) & \quad &                                                                              \\
     & = (a \,| \, c)(b \,| \, d) - (a \,| \, d)(b \,| \, c) & \quad                                                                                \\
     & = \begin{vmatrix}
           (a \,| \, c) & (a \,| \, d) \\
           (b \,| \, c) & (b \,| \, d)
         \end{vmatrix}.
     &                                                       &
  \end{alignat*}
  以上より,この等式は示された.
\end{tproof}

\subproblem[label = p90--91:3-(\romannumeral8)]{8}

\begin{tproof}
  \[
    (a \times b \,| \, c \times d) = (a \,| \, c)(b \,| \, d) - (a \,| \, d)(b \,| \, c)
  \]
  において,$c=a$, $d=b$すると,
  \begin{align*}
    (a \times b \mid a \times b) & = (a \,| \, a)(b \,| \, b) - (a \,| \, b)(b \,| \, a) \\
    \abs{a \times b}^2           & = \abs{a}^2 \abs{b}^2 - (a \,| \, b)^2.
  \end{align*}
  ここで、$a$ と $b$ のなす角を $\theta$ とすると,内積の定義は $(a \,| \, b) = \abs{a}\abs{b}\cos\theta$ である.これを代入すると,
  \begin{align*}
    \abs{a \times b}^2 & = \abs{a}^2 \abs{b}^2 - (\abs{a}\abs{b}\cos\theta)^2     \\
                       & = \abs{a}^2 \abs{b}^2 - \abs{a}^2 \abs{b}^2 \cos^2\theta \\
                       & = \abs{a}^2 \abs{b}^2 (1-\cos^2\theta)                   \\
                       & = \abs{a}^2 \abs{b}^2 \sin^2\theta
  \end{align*}
  $\theta$が$a$と$b$のなす角であることに注意し,両辺の正の平方根をとると,
  \[
    \abs{a \times b} = \abs{a}\abs{b}\sin\theta
  \]
  が得られる.これが証明すべきことであった.
\end{tproof}

\subproblem[label = p90--91:3-(\romannumeral9)]{9}

\begin{tproof}
  スカラー値関数の積の微分法則$(d/dt)(fg)=(df/dt)g+f(dg/dt)$を各成分に適用し,
  \begin{align*}
    \frac{d}{dt} (a \times b)
     & = \frac{d}{dt} \begin{pmatrix} a_2 b_3 - a_3 b_2 \\ a_3 b_1 - a_1 b_3 \\ a_1 b_2 - a_2 b_1 \end{pmatrix}                                                                                                                                            \\
     & = \begin{pmatrix} (d/dt)(a_2 b_3) - (d/dt)(a_3 b_2) \\ (d/dt)(a_3 b_1) - (d/dt)(a_1 b_3) \\ (d/dt)(a_1 b_2) - (d/dt)(a_2 b_1) \end{pmatrix}                                                                                                         \\
     & = \begin{pmatrix} ( (da_2/dt)b_3 + a_2(db_3/dt) ) - ( (da_3/dt)b_2 + a_3(db_2/dt) ) \\ ( (da_3/dt)b_1 + a_3(db_1/dt) ) - ( (da_1/dt)b_3 + a_1(db_3/dt) ) \\ ( (da_1/dt)b_2 + a_1(db_2/dt) ) - ( (da_2/dt)b_1 + a_2(db_1/dt) ) \end{pmatrix}         \\
     & = \begin{pmatrix} ( (da_2/dt)b_3 - (da_3/dt)b_2 ) + ( a_2(db_3/dt) - a_3(db_2/dt) ) \\ ( (da_3/dt)b_1 - (da_1/dt)b_3 ) + ( a_3(db_1/dt) - a_1(db_3/dt) ) \\ ( (da_1/dt)b_2 - (da_2/dt)b_1 ) + ( a_1(db_2/dt) - a_2(db_1/dt) ) \end{pmatrix}         \\
     & = \begin{pmatrix} (da_2/dt)b_3 - (da_3/dt)b_2 \\ (da_3/dt)b_1 - (da_1/dt)b_3 \\ (da_1/dt)b_2 - (da_2/dt)b_1 \end{pmatrix} + \begin{pmatrix} a_2(db_3/dt) - a_3(db_2/dt) \\ a_3(db_1/dt) - a_1(db_3/dt) \\ a_1(db_2/dt) - a_2(db_1/dt) \end{pmatrix} \\
     & = \frac{da}{dt} \times b + a \times \frac{db}{dt}.
  \end{align*}
  これが証明すべきことであった.
\end{tproof}

\problemtodo[subproblems = 5]{p90--91}{4}
\problemtodo[subproblems = 2]{p90--91}{5}
\problemtodo[subproblems = 2]{p90--91}{6}
\problemtodo[subproblems = 5]{p90--91}{7}
\problemtodo[subproblems = 6]{p90--91}{8}

\problem[subproblems = 6]{p90--91}{9}

\subproblem{1}

\begin{tanswer}
  \[
    ( \log x )^{(1)}= 1/x , \quad (\log x)^{(2)} = - 1/x^2 , \quad (\log x)^{(3)} = 2/x^3,\quad (\log x)^{(4)} = - 6 /x^4
  \]
  であるから,
  \[
    (\log x)^{(n)} = \frac{(-1)^{n-1}  (n-1)!}{x^n}
  \]
  と推測できる.この推測が正しいことを数学的帰納法により証明する.
  \begin{enumerate}
    \item $n=1$のとき,$(\log x)^{(1)} = 1/x$であり,
          \[
            \frac{(-1)^{1-1}  (1-1)!}{x^1}=1/x
          \]
          であるから,この場合に推測は正しい.
    \item $n=k$のときに,この推測が正しいと仮定すると,
          \[
            (\log x)^{(k)} = \frac{(-1)^{k-1}  (k-1)!}{x^k}
          \]
          である.ここで.
          \begin{align*}
            (\log x)^{(k+1)} & = \left (\frac{(-1)^{k-1}  (k-1)!}{x^k} \right ) ' \\
                             & = \frac{(-1)^k  k!}{x^{k+1}}
          \end{align*}
          であるから,$n=k+1$のときも推測は正しい.
  \end{enumerate}
  (1),(2)より,
  \[
    (\log x)^{(n)} = \frac{(-1)^{n-1}  (n-1)!}{x^n}
  \]
  である.
\end{tanswer}

\begin{tanswer}
  \[
    \left(   \frac{1}{x^2+3x+2} \right)^{(n)} = (-1)^n n! \{ (x+1)^{-n-1} - (x+2)^{-n-1} \}
  \]
\end{tanswer}



\problem{p90--91}{10}

\begin{tproof}
  $u(x)= (x^2-1)^n$とおく,このとき,
  \[
    U'(x)= 2x n(x^2-1)^{n-1}
  \]
  だから,
  \[
    (x^2-1) u'(x)=2nx \cdot u(x)
  \]
  この両辺を$(n+1)$回微分して,
  \begin{align*}
     & (x^2-1)u^{(n+2)}(x)+2(n+1)x u^{(n+1)} x + \frac{(n+1)n}{2} \cdot u^{(n)} (x) = 2nx u^{(n+1)}(x) + 2(n+1) n u^{(n)}(x) \\
     & \therefore ~(x^2-1)u^{(n+2)}(x) + 2n u^{(n+1)}(x)-(n+1)n u^{(n)}(x)=0
  \end{align*}
  ここで,$ u^{(n)} (x)= 2^n n! P_n (x)$なので,
  \begin{align*}
     & (x^2 -1) \{ 2^n n! P_n ''(x) \} +2x \{ 2^n n! P_n (x) \} -n(n+1) \{ 2^n n P_n(x) \} =0 \\
     & \therefore ~ (x^2-1) P_n ''(x)+2x P_n '(x) -n(n+1) P_n (x)=0
  \end{align*}
\end{tproof}


\problem{p106--107}{1}

\begin{lemma}{}{}
  \[
    Q_k (x)= (n! 2^n)^{-1} \frac{d^k}{dx^k} (x^2-1)^n \quad (0 \leqq k \leqq n)
  \]
  とおく.$0 \leqq k <n$のとき,$Q_k (x)$は$(x^2-1)^{n-k}$で割り切られ.
  \[
    Q_k (1)= Q_k(-1)=0
  \]
  である.
\end{lemma}

\begin{proof}
  $k$に関する帰納法により示す.$k=0$のときは
  \[
    Q_0(x)= (n! 2^n)^{-1} (x^2-1)^n
  \]
  であり,$Q_o(X)$は$(x^2-1)^n$で割り切れ,$Q_0(1)=Q_0(-1)=0$である.

  $k>0$として,$k-1$の場合の主張の成立を仮定すると,$Q_{k-1}(x)$は$x$についての式$g(x)$を用いて,
  \[
    Q_{k-1} = g(x)(x^2-1)^{n-k+1}
  \]
  と表せ,
  \begin{align*}
    Q_k (x) & =Q_{k-1}' (x)                                        \\
            & = g'(x)(x^2-1)^{n-k+1} + g(x) (x^2-1)^{n-k} \cdot 2x \\
            & =(g'(x)(x^2-1)+g(x) \cdot 2x) (x^2-1)^{n-k}
  \end{align*}
  したがって,$Q_k(x)$は$(x^2-1)^{n-k}$で割り切れ,$n-k >0$より,
  \[
    Q_k (1)= Q_k(-1)=0
  \]
  となり,これが証明すべきことであった.
\end{proof}

\begin{tproof}
  $Q_k (x)$が$(-1,1)$上に$k$個の相異なる根を持つことを,$k$に関する数学的帰納法で示す.
  \begin{enumerate}[(i)]
    \item $k=0$の場合
          \[
            Q_0(x)= (n! 2^n)^{-1} (x^2-1)^n
          \]
          なので,$(-1,1)$に根を持たない.
    \item $k>0$として,$k-1$の場合の主張の成立を仮定すると,$Q_{k-1}(x)$は$(-1,1)$に$k-1$個の相異なる根を持つから,
          それらを
          \[
            x_1,x_2,\dots,x_{k-1}\quad (-1<x_1<x_2<\dots <x_{k-1}<1)
          \]とおく.
          ここで,$Q_{k-1} (x)$は閉区間$[-1,1]$で微分可能なので,
          各区間$[-1,x_1],[x_1,x_2],\dots,[x_{k-1},1]$においても微分可能であり,補題により,
          \[
            Q_{k-1} (-1) = Q_{k-1}(x_1)=\dots = Q_{k-1}(x_{k-1})=Q_{k-1}(1)=0
          \]
          である.したがって各小区間においてロルの定理が適用できて,$Q_{k-1}' (x)=Q_k(x)$は$(-1,x_1),(x_1,x_2),\dots,(x_{k-1},1)$において根をもつ.
  \end{enumerate}
  以上(\romannumeral1),(\romannumeral2)の議論により,$Q_k (x)$は$(-1,1)$において相異なる$k$個の根をもつ.
  特に$k=n$のとき,$Q_n (x)=P_n(x)$で$P_n (x)$は$(-1,1)$において相異なる$n$個の根をもつ.
\end{tproof}



% subproblem なし扱いとなっている
\problem{p106--107}{2}


\begin{lemma}{}{}
  $H_n(x)$は$n$次多項式で,$n$個の相異なる実根を持つとする.

  このとき,$H_n (x)$の隣り合う$2$根を$a < b$とすると,
  \[
    H_n'(a) H_n'(b)<0.
  \]
\end{lemma}

\begin{proof}
  $H_n (x)$は$n$次多項式だから,$x_1 < x_2 < \dots <x_n$として以下のようにかける:
  \[
    H_n (x) = c \prod_{k=1}^{n} (x-x_k).
  \]
  これを$x$について微分すると,
  \begin{align*}
    H_n '(x) & = c \sum_{j=1}^{n} \left[ \frac{d}{dx} (x-x_j) \times  \prod_{\substack{1 \leqq k \leqq n \\ k \ne j}} (x-x_k) \right] \\
             & = c \sum_{j=1}^{n} \prod_{\substack{1 \leqq k \leqq n                                     \\ k \ne j}} (x-x_k).
  \end{align*}
  とくに$x = x_j$とすると
  \[
    H_n '(x_j) = c \prod_{\substack{1 \leqq k \leqq n \\ k \ne j}} (x_j-x_k).
  \]
  したがって,$k <j$のとき,$x_k < x_j$であり,$k > j$のとき,$x_k > x_j$であるから,
  $\prod_{\substack{1 \leqq k \leqq n \\ k \ne j}} (x_j-x_k)$の符号は$(-1)^{n-j}$である.
  よって,隣り合う$2$根$a < b$について,
  \[
    H_n '(a) H_n ' (b)<0.
  \]
\end{proof}

補題をふまえた上で,いくつかの命題に分けて証明していく.

\begin{proposition}{}{}
  テキストの(i),(ii)が成り立つ.
\end{proposition}


\begin{tproof}
  (i)について,
  \begin{align*}
      & \frac{d}{dx} \left[ e^{x^2} \frac{d^n}{dx^n} (e^{-x^2} )\right]                      \\
    = & 2x e^{x^2} \frac{d^n}{dx^n} (e^{-x^2}) + e^{x^2} \frac{d^{n+1}}{dx^{n+1}} (e^{-x^2})
  \end{align*}
  であるから,この式に$(-1)^n$をかけて整理すると,
  \begin{align*}
    (-1)^n \frac{d}{dx} \left[ e^{x^2} \frac{d^n}{dx^n} (e^{-x^2} )\right] & = 2x(-1)^n e^{x^2} \frac{d^n}{dx^n} (e^{-x^2}) -(-1)^{n+1} e^{x^2} \frac{d^{n+1}}{dx^{n+1}} (e^{-x^2}) \\
    \therefore ~ H_n ' (x)                                                 & = 2x H_n(x) - H_{n+1}(x).
  \end{align*}
  整理すると
  \[
    H_{n+1}(x) =  2x H_n(x)-H_n'(x).
  \]
  (ii)について,
  $ (e^{-x^2})^{(n+1)} = \{ (-2x)e^{-x^2} \}^{(n)}$であるから,
  \begin{align*}
    (e^{-x^2})^{(n+1)} & = (-2xe^{-x^2})^{(n)}                                                                                                                                  \\
                       & = -2 (xe^{-x^2})^{(n)}                                                                                                                                 \\
                       & = -2 \sum_{r=0}^{n} \binom{n}{r} x^{(n)} (e^{-x^2})^{(n-r)}                                                                                            \\
                       & = -2 \left ( \binom{n}{0} x^{(0)} (e^{-x^2})^{(n-0)} + \binom{n}{1} x^{(1)} (e^{-x^2})^{(n-1)} + \dots + \binom{n}{n} x^{(n)} (e^{-x^2})^{(0)} \right) \\
                       & = -2 \left(  x (e^{-x^2})^{(n)} + n x^{(1)} (e^{-x^2})^{(n-1)} \right)                                                                                 \\
                       & = -2 \left ( x \frac{d^n}{dx^n} (e^{-x^2}) + n \frac{d^{n-1}}{dx^{n-1}} (e^{-x^2}) \right).
  \end{align*}
  よって,
  \begin{align*}
    (-1)^{n+1} e^{x^2} \frac{d^{n+1}}{d x^{n+1}} (e^{-x^2}) & = 2 (-1)^n e^{x^2} x \frac{d^n}{dx^n} (e^{-x^2}) -2 n (-1)^{n-1} e^{x^2}\frac{d^{n-1}}{dx^{n-1}} (e^{-x^2}) \\
    \therefore ~ H_{n+1} (x)                                & = 2x H_n(x) - 2nH_{n-1}(x).
  \end{align*}
  これと(i)の結果を比較すると,
  \[
    H '' (x) - 2x H_n' (x) + 2nH_n (x)=0.
  \]
\end{tproof}

\begin{proposition}{}{}
  $H_n (x)$は$n$次多項式である.
\end{proposition}

\begin{tproof}
  $n$についての数学的帰納法により示す.
  \begin{enumerate}[(I)]
    \item $n =0, 1$のとき,
          \begin{align*}
            H_0 (x) & = 1,  \\
            H_1 (x) & = 2x,
          \end{align*}
          であるから,$n=0, 1 $のとき成立する.
    \item $k \in \mathbb{N}$を任意に取り,$H_k (x)$が$k$次多項式であると仮定する.
          $ H_{k+1}(x) =  2x H_k(x)-H_k'(x)$であるから,
          帰納法の仮定により,$ 2 x H_k (x)$が$k+1$次多項式であることと,$H_k'(x)$が$k-1$次多項式であることが従い,$H_{k+1}(x)$も$k+1$次多項式である.
  \end{enumerate}
  よって数学的帰納法により,$H_n (x)$は任意の$n$について$n$次多項式である.
\end{tproof}

\begin{proposition}{}{}
  $H_n(x)$ は,任意の $n \in \mathbb{N}$ に対して $n$ 個の相異なる実根をもつ.
\end{proposition}

\begin{tproof}
  数学的帰納法およびRolleの定理を用いて示す.
  \noindent
  \begin{enumerate}[(I)]
    \item \mbox{}
          \[
            H_0(x) = 1, \quad H_1(x) = 2x, \quad H_2(x) = 2x^2-1.
          \]
          $H_0(x)$ は定数多項式で実根を持たない(ただし「$n=0$個」の根と解釈),
          $H_1(x) = 2x$ は $x=0$ に $1$つの実根をもつ.
          $H_2(x)$ は相異なる$2$つの実根をもつ.
          したがって,$n=0,1,2$のときに成立する.
    \item
          $k \geqq 2$ について,
          $H_k(x)$は$k$個の相異なる実根をもつ
          と仮定する.$H_k(x)$ の実根を
          \[
            \alpha_1 < \alpha_2 < \dots < \alpha_k
          \]
          と表す.

          Hermite多項式の基本的性質として,以下の関係式が成り立つ:
          \[
            H_{k+1}(x)  = 2xH_k(x)-H_k'(x).
          \]
          まず,Rolleの定理により,開区間 $(\alpha_i,\alpha_{i+1})$ に対して少なくとも1つの $\beta_i$ が存在し,
          \[
            \alpha_i < \beta_i < \alpha_{i+1},\quad
            H_k'(\beta_i) = 0
            \quad (i=1,2,\dots,k-1)
          \]
          が成り立つ.
          $H_k '(x)$は$k-1$次多項式であるから,そのような点は$\beta_i$のみである.
          よって$i = 0,1,\dots,k-1$に対し,$(\beta_i,\beta_{i+1})$においては$H_k '(x) \ne 0$であり,
          かつ$H_k '(x)$の符号は一定である(ただし,$ \beta_0 =-\infty$,$\beta_k = \infty$とする).

          特に,$(\beta_{k-1},\beta_k)$で$H_k (x)$は単調増加または単調減少であり,
          $\lim_{x \to \infty} H_k (x) =\infty$だから消去法で単調増加である.
          つまり$H_k ' (\alpha_k) > 0$である.

          (i)の結果により$H_{n+1} (\alpha_i) =-H_k'(\alpha_i)$であるから,
          補題により,$H_{n+1}(\alpha_i)$の符号は交互に変わり,
          かつ$H_{k+1}(\alpha_k) < 0$である.

          以上の議論から,$k$ が偶数のときは $\alpha_1$ での値が正となり,以降交互に符号が変化して最後 $\alpha_k$ では負になる.
          一方,$k$ が奇数のときは $\alpha_1$ での値が負となり,以降交互に符号が変化して最後 $\alpha_k$ ではやはり負になる.

          ここで中間値の定理を適用すると,
          \[
            (\alpha_1,\alpha_2),~(\alpha_2,\alpha_3),~\dots,~(\alpha_{k-1},\alpha_k)
          \]
          の各区間内で $H_{k+1}(x)$ は符号を変えるため,少なくとも1つずつ実根をもつ.
          よってここまでで合計 $(k-1)$ 個の相異なる根が存在する.

          最後に,$H_{k+1}(x)$ の最高次項は $2^{k+1} x^{k+1}$ であり,
          \[
            \lim_{x \to +\infty} H_{k+1}(x) =
            \begin{cases}
              +\infty & (k \text{が偶数}) \\
              +\infty & (k \text{が奇数})
            \end{cases},
            \quad
            \lim_{x \to -\infty} H_{k+1}(x) =
            \begin{cases}
              -\infty & (k \text{が偶数}) \\
              +\infty & (k \text{が奇数})
            \end{cases}.
          \]
          したがって,$\alpha_1$ より左側と $\alpha_k$ より右側に
          もう1つずつ根が必ず存在する.

          以上をまとめると,
          \[
            (\alpha_j,\alpha_{j+1}) \text{ に } (k-1)\text{ 個}, \quad
            (-\infty,\alpha_1) \text{ または } (\alpha_k,\infty) \text{ に } 2\text{ 個}
          \]
          の相異なる実根が存在するので,合計で $k+1$ 個の実根が存在する.

          以上により,$H_k(x)$ が $k$ 個の実根をもつとき,$H_{k+1}(x)$ は
          $k+1$ 個の相異なる実根をもつことが示された.
  \end{enumerate}
  よって数学的帰納法により,すべての $n \in \mathbb{N}$(あるいは $n \ge 0$)に対し
  $H_n(x)$ は $n$ 個の相異なる実根をもつことが証明された.
\end{tproof}

\begin{proposition}{}{}
  $H_n(x)$の隣り合う二根の間に$H_{n-1}(x)$の根が一つある.
\end{proposition}


\begin{tproof}
  $H_n(x)$の隣り合う$2$根を$a$,$b$とすると
  補題より,
  \[
    H_n '(a) H_n '(b)<0
  \]
  である.

  $ H_n '(x)=2n H_{n-1}(x)$より,
  \[
    H_{n-1}(a) H_{n-1}(b) < 0
  \]
  このことから,$a$と$b$の間に$H_{n-1}(x)$の根が存在することがわかった.
\end{tproof}
以上のことから,全ての事柄が証明された.\qed


\problem{p106--107}{3}

\begin{tproof}
  まず,$f(a)=0$, $g(a)=0$と定義しておく.$0<\delta <b-a$を満たす$\delta$をとると,$f$と$g$は$[a,a+\delta]$で連続であり,
  $(a,a+\delta)$で微分可能である.いま$g(x) \ne 0$~($ x \in (a,a+\delta)$)であるから,
  コーシーの平均値の定理により,
  \[
    \frac{f(x)-f(a)}{g(x)-g(a)} = \frac{f'(c)}{g'(c)}
  \]
  を満たす$c \in (a,x)$が存在する.いま$f(a)=0$,$g(a)=0$と定義したので,
  \[
    \frac{f(x)}{g(x)} = \frac{f'(c)}{g'(c)}
  \]
  である.$ x \to a+0$のとき,$c \to a+0$であるから,仮定により,
  \[
    \lim_{x \to a+0} \frac{f(x)}{g(x)} = \lim_{c \to a+0} \frac{f'(c)}{g'(c)} =l
  \]
  である.$ x \to a-0$のときは同様にすればよく,$ x \to a$の場合はこれら二つの考察により結果が従う.

  $ x \to +\infty$のときは$ t=1/x$として$f(t)$と$g(t)$の極限を考えれば同じ結論に帰着する.

  以上の議論により証明された.
\end{tproof}





\problemtodo{p106--107}{4}
\problemtodo[subproblems = 5]{p106--107}{5}
\problemtodo[subproblems = 3]{p106--107}{6}
\problemtodo{p106--107}{7}

\problem[subproblems = 2]{p106--107}{8}


\subproblem{1}

\begin{tanswer}
  \[
    f(x)=3x^4 -8x^3+6x^2
  \]
  とおくと,
  \begin{align*}
     & f'(x) =12x^3 -24x^2+12x = 12x(x-1)^2, \\
     & f''(x)=36x^2-48x+12 = 12(3x-1)(x-1)
  \end{align*}
  であるから,増減表は以下のようになる.

  \vspace{2mm}

  \begin{tabular}{|c||ccccccc|}
    \hline
    $x$      & $\cdots$ & $0$ & $\cdots$ & $1/3$ & $\cdots$ & $1$ & $\cdots$ \\
    \hline
    $f'(x)$  & $-$      & $0$ & $+$      & $+$   & $+$      & $0$ & $+$      \\
    \hline
    $f''(x)$ & $+$      & $+$ & $+$      & $0$   & $-$      & $0$ & $+$      \\
    \hline
    $f(x)$   & \ser     &     & \ner     &       & \nel     &     & \ner     \\
    \hline
  \end{tabular}

  \vspace{2mm}

  \begin{tikzpicture}[scale = 3]
    \draw[->,>=stealth,semithick](-0.3,0)--(1.3,0)node[above]{$x$};%x軸
    \draw[->,>=stealth,semithick](0,-0.2)--(0,1.1)node[right]{$y$};%y軸
    \draw(0,0)node[below right]{O};%原点
    \draw[domain=-0.2:1.3,samples=100]plot(\x,{3*pow(\x,4)-8*pow(\x,3)+6*pow(\x,2)})node[right]{$y=3x^4-8x^3+6x^2$};
  \end{tikzpicture}
\end{tanswer}



\subproblem{2}

\begin{tanswer}
  $x>0$のもとで,
  \[
    f(x)=x^\frac{1}{x}
  \]
  とおく.ここで,
  \[
    f(x)=\exp(\log x^{1/x}) = \exp (\log x/x)
  \]
  と変形できるので,
  \begin{align*}
     & f'(x) = \exp (\log x /x) \left (\frac{1-\log x}{x^2} \right) ,                  \\
     & f''(x) = \exp (\log x /x) \left (\frac{(\log x -1)^2 -3x+2x\log x}{x^4} \right)
  \end{align*}
  であり,増減表は以下のようになる.ただし$\alpha$,$\beta$は$(\log x -1)^2 -3x+2x\log x =0$の$2$解である.
  \vspace{2mm}

  \begin{tabular}{|c||cccccccc|}
    \hline
    $x$      & $0$ & $\cdots$ & $\alpha$ & $\cdots$ & $e$ & $\cdots$ & $\beta $ & $\cdots$ \\
    \hline
    $f'(x)$  &     & $+$      &          & $+$      & $0$ & $-$      &          & $-$      \\
    \hline
    $f''(x)$ &     & $+$      & $0$      & $-$      &     & $-$      & $0$      & $+$      \\
    \hline
    $f(x)$   &     & \ner     &          & \nel     &     & \sel     &          & \ser     \\
    \hline
  \end{tabular}

  \vspace{2mm}

  \begin{tikzpicture}[scale = 2]
    \draw[->,>=stealth,semithick](-0.1,0)--(3,0)node[above]{$x$};%x軸
    \draw[->,>=stealth,semithick](0,-0.1)--(0,3)node[right]{$y$};%y軸
    \draw(0,0)node[below right]{O};%原点
    \draw[domain=0.1:4,samples=100]plot(\x,{pow(\x,1/\x)}) node[right]{$y=x^{1/x}$};
  \end{tikzpicture}
\end{tanswer}


\problem[label = p106--107:9]{p106--107}{9}


\begin{lemma}{}{}
  $\sin  x$は$0$を中心として$I=[0,\pi/2]$でテイラー展開できる.
\end{lemma}

\begin{proof}
  \[
    (\sin x)^{(n)} =\sin(x + n\pi/2)
  \]
  であるから, $ \abs{\sin (x+n\pi/2)}  \leqq 1$であることと定理 2.11:2)より主張が従う.
\end{proof}

\begin{tproof}
  $m \geqq 1$のとき,
  \[
    S_{2m} = S_{2m-2} +a_{2m-1} + a_{2m} = S_{2m-2} + \frac{(-1)^{2m-1}}{(2(2m-1)+1)!} x^{2(2m-1)+1} + \frac{(-1)^{2m}}{(2(2m)+1)!} x^{2(2m)+1}.
  \]
  $ 0< x \leqq \pi /2$のとき,
  \[
    \frac{1}{(2(2m-1)+1)!} x^{2(2m-1)+1} > \frac{1}{(2(2m)+1)!} x^{2(2m)+1}
  \]
  であり,なおかつ$(-1)^{2m-1} =-1$,$(-1)^{2m} =1$であることをふまえると
  \[
    \frac{(-1)^{2m-1}}{(2(2m-1)+1)!} x^{2(2m-1)+1} + \frac{(-1)^{2m}}{(2(2m)+1)!} x^{2(2m)+1} <0.
  \]
  よって
  \[
    S_{2m} -S_{2m-2}  = \frac{(-1)^{2m-1}}{(2(2m-1)+1)!} x^{2(2m-1)+1} + \frac{(-1)^{2m}}{(2(2m)+1)!} x^{2(2m)+1} <0 \quad S_{2m}<S_{2m-2}.
  \]
  よって,$(S_{2m})_{m \in \mathbb{N}}$は$0 < x \leqq \pi /2$で狭義単調減少である.

  $S_{2m-1}$についても同様に考え,
  \[
    S_{2m+1}=S_{2m-1} + a_{2m-1} +a_{2m}= S_{2m-1} + \frac{(-1)^{2m-1}}{(2(2m-1)+1)!} x^{2(2m-1)+1} + \frac{(-1)^{2m}}{(2(2m)+1)!} x^{2(2m)+1}.
  \]
  などから,$ S_{2m+1}-S_{2m-1} >0$が成り立ち,$m \leqq 1$のとき$S_{2m+1}>S_{2m-1}$である.
  よって,$(S_{2m+1})_{m \in \mathbb{N}}$は$0 < x \leqq \pi /2$で狭義単調増加である.
  そして,補題により$\sin x $は$0 < x \leqq \pi /2$でテイラー展開できるので,そのことをふまえると
  \[
    \lim_{m \to \infty} S_{2m} =\lim_{m \to \infty} S_{2m+1} =\sin x
  \]
  であるから,
  \[
    S_0 > S_2 > S_4 > \cdots > S_{2m} > \sin x > S_{2m+1} > S_{2m-1} > \cdots > S_3 > S_1
  \]
  となり\footnote{$S_n=\sin x$となる$n \in \mathbb{N}$が存在すると,$ \sin x=S_n  > S_{n+1}\geqq \sin x$となり矛盾する.},
  \[
    S_{2m} > \sin x > S_{2m+1} \quad ( 0 < x \leqq \frac{\pi}{2} )
  \]
  が示された.
\end{tproof}


\problem{p106--107}{10}



\begin{tproof}
  3つのことを証明する.
  \begin{description}
    \item[a)とb)が同値であること] \mbox{} \par
          $a < x \leqq y$または$y \leqq x <a$に対して,$0 \leqq t < 1$を用いて,
          \[
            x=ta+(1-t)y
          \]
          とおく.$a < x \leqq y$とする.このとき,
          \begin{equation}
            \label{eq:p107 10) 1}
            t = \frac{x-y}{a-y}
          \end{equation}
          と表せることはよい.

          また,$f$は$I$で凸であり,これは
          \begin{equation}
            \label{eq:p107 10) 2}
            f(x)=tf(x)+(1-t)f(x) < tf(a)+(1-t)f(y)
          \end{equation}
          と同値である.$x<y$のとき\eqref{eq:p107 10) 2}に\eqref{eq:p107 10) 1}の$t$の値を代入すれば,
          \begin{equation}
            \label{eq:p107 10) 3}
            \frac{f(x)-f(a)}{x-a} < \frac{f(y)-f(x)}{y-x}
          \end{equation}
          を得る.$x=y$のときは明らか.
    \item [a)とc)が同値であること] \mbox{} \par
          c)を仮定する.\eqref{eq:p107 10) 3}の左辺について,平均値の定理により,
          \[
            \frac{f(x)-f(a)}{x-a}=f'(\xi)
          \]
          をみたす$\xi ~(a<\xi <x)$が存在する.同様に,\eqref{eq:p107 10) 3}の右辺について,平均値の定理により,
          \[
            \frac{f(y)-f(x)}{y-x}=f'(\eta)
          \]
          をみたす$\eta~ (x<\eta < y)$が存在する.仮定により,$f'(\xi)\leqq f'(\eta)$だから,\eqref{eq:p107 10) 3}が成り立ち,a)が従う.

          a),すなわち\eqref{eq:p107 10) 3}を仮定する.
          左辺について,$x \to + a$とすれば,これは$f'(a)$に収束し,右辺は,
          これは$(f(y)-f(a))/(y-a)$に収束する(これを$\alpha$とおく.).$f'(a) \leqq \alpha$となることはよい.
          また,$x \to - y$とすれば,左辺は$\alpha$に,右辺は$f'(y)$に収束する.$\alpha \leqq f'(y)$となることもよい.

          よって,a)とc)は同値である.

    \item[a)とd)が同値であること] \mbox{} \par
          d)を仮定する.これは$f'(x)$は$I$上で単調増加であることと同値である.
  \end{description}
  以上の議論により,示された.
\end{tproof}

\problem{p112--113}{1}

\subproblem{1}

\begin{tanswer}
\begin{align*}
    \frac{\partial f}{\partial x} &= \cos(x+y) \\
    \frac{\partial f}{\partial y} &= \cos(x+y)
\end{align*}
\begin{align*}
    \frac{\partial^2 f}{\partial x^2} &= -\sin(x+y) \\
    \frac{\partial^2 f}{\partial y^2} &= -\sin(x+y) \\
    \frac{\partial^2 f}{\partial y \partial x} &= \frac{\partial^2 f}{\partial x \partial y} = -\sin(x+y)
\end{align*}
\end{tanswer}


\subproblem{2}
\begin{tanswer}
\begin{align*}
    \frac{\partial f}{\partial x} &= 2xy^3\cos(x^2y^3) \\
    \frac{\partial f}{\partial y} &= 3x^2y^2\cos(x^2y^3)
\end{align*}
\begin{align*}
    \frac{\partial^2 f}{\partial x^2} &= 2y^3\cos(x^2y^3) - 4x^2y^6\sin(x^2y^3) \\
    \frac{\partial^2 f}{\partial y^2} &= 6x^2y\cos(x^2y^3) - 9x^4y^4\sin(x^2y^3) \\
    \frac{\partial^2 f}{\partial y \partial x} &= \frac{\partial^2 f}{\partial x \partial y} = 6xy^2\cos(x^2y^3) - 6x^3y^5\sin(x^2y^3)
\end{align*}
\end{tanswer}

\subproblem{3}

\begin{tanswer}
\begin{align*}
    \frac{\partial f}{\partial x} &= yx^{y-1} \\
    \frac{\partial f}{\partial y} &= x^y \log  x
\end{align*}
\begin{align*}
    \frac{\partial^2 f}{\partial x^2} &= y(y-1)x^{y-2} \\
    \frac{\partial^2 f}{\partial y^2} &= x^y (\log x)^2 \\
    \frac{\partial^2 f}{\partial y \partial x} &= \frac{\partial^2 f}{\partial x \partial y} = x^{y-1}(1 + y\log x)
\end{align*}
\end{tanswer}

\subproblem{4}

\begin{tanswer}
\begin{align*}
    \frac{\partial f}{\partial x} &= \frac{2x}{x^2+y^2} \\
    \frac{\partial f}{\partial y} &= \frac{2y}{x^2+y^2}
\end{align*}
\begin{align*}
    \frac{\partial^2 f}{\partial x^2} &= \frac{2(y^2-x^2)}{(x^2+y^2)^2} \\
    \frac{\partial^2 f}{\partial y^2} &= \frac{2(x^2-y^2)}{(x^2+y^2)^2} \\
    \frac{\partial^2 f}{\partial y \partial x} &= \frac{\partial^2 f}{\partial x \partial y} = -\frac{4xy}{(x^2+y^2)^2}
\end{align*}
\end{tanswer}

\subproblem{5}
\begin{tanswer}
\begin{align*}
    \frac{\partial f}{\partial x} &= \frac{x}{\sqrt{x^2+y^2+z^2}} \\
    \frac{\partial f}{\partial y} &= \frac{y}{\sqrt{x^2+y^2+z^2}} \\
    \frac{\partial f}{\partial z} &= \frac{z}{\sqrt{x^2+y^2+z^2}}
\end{align*}
\begin{align*}
    \frac{\partial^2 f}{\partial x^2} &= \frac{y^2+z^2}{(x^2+y^2+z^2)^{3/2}} \\
    \frac{\partial^2 f}{\partial y^2} &= \frac{x^2+z^2}{(x^2+y^2+z^2)^{3/2}} \\
    \frac{\partial^2 f}{\partial z^2} &= \frac{x^2+y^2}{(x^2+y^2+z^2)^{3/2}} \\
    \frac{\partial^2 f}{\partial y \partial x} &= \frac{\partial^2 f}{\partial x \partial y} = -\frac{xy}{(x^2+y^2+z^2)^{3/2}} \\
    \frac{\partial^2 f}{\partial z \partial x} &= \frac{\partial^2 f}{\partial x \partial z} = -\frac{xz}{(x^2+y^2+z^2)^{3/2}} \\
    \frac{\partial^2 f}{\partial z \partial y} &= \frac{\partial^2 f}{\partial y \partial z} = -\frac{yz}{(x^2+y^2+z^2)^{3/2}}
\end{align*}
\end{tanswer}

\problemtodo{p112--113}{2}
\problemtodo{p112--113}{3}

\problem{p112--113}{4}

\begin{tanswer}
原点 $(0,0)$ における$f(x,y)$の一階偏導関数を計算すると,次のようになる:
\[
\frac{\partial f}{\partial x}(0,0) = \lim_{\substack{h \to 0 \\ h \ne 0}}\frac{f(h,0)-f(0,0)}{h} = \lim_{\substack{h \to 0 \\ h \ne 0}}\frac{0-0}{h} = 0.
\]
\[
\frac{\partial f}{\partial y}(0,0) = \lim_{\substack{k \to 0 \\ k \ne 0}}\frac{f(0,k)-f(0,0)}{k} = \lim_{\substack{k \to 0 \\ k \ne 0}}\frac{0-0}{k} = 0.
\]

これをふまえると,$f(x,y)$の二階偏導関数は次のように計算できる:
\begin{align*}
\frac{\partial^2 f}{\partial y \partial x}(0,0) &= \frac{\partial}{\partial y}\left(\dfrac{\partial f}{\partial x}\right)\Biggr|_{(0,0)}\\
&= \lim_{\substack{k \to 0 \\ k \ne 0}}\frac{\dfrac{\partial f}{\partial x}(0,k) - \dfrac{\partial f}{\partial x}(0,0)}{k}.
\end{align*}
ここで,
\begin{align*}
\frac{\partial f}{\partial x}(0,k) &= \lim_{\substack{h \to 0 \\ h \ne 0}}\frac{f(h,k) - f(0,k)}{h} \\
&= \lim_{\substack{h \to 0 \\ h \ne 0}}\frac{\dfrac{hk(h^2-k^2)}{h^2+k^2} - 0}{h} \\
&= \lim_{\substack{h \to 0 \\ h \ne 0}} k \frac{h^2-k^2}{h^2+k^2} \\
&= k \frac{-k^2}{k^2} = -k.
\end{align*}
したがって,
\[
\frac{\partial^2 f}{\partial y \partial x}(0,0) = \lim_{\substack{k \to 0 \\ k \ne 0}}\frac{-k - 0}{k} = -1.
\]

同様に,
\begin{align*}
\frac{\partial^2 f}{\partial x \partial y}(0,0) &= \frac{\partial}{\partial x}\left(\dfrac{\partial f}{\partial y}\right)\Biggr|_{(0,0)} \\
&= \lim_{\substack{h \to 0 \\ h \ne 0}}\frac{\dfrac{\partial f}{\partial y}(h,0) - \frac{\partial f}{\partial y}(0,0)}{h}.
\end{align*}
ここで,
\begin{align*}
\frac{\partial f}{\partial y}(h,0) &= \lim_{\substack{k \to 0 \\ k \ne 0}}\frac{f(h,k) - f(h,0)}{k} \\
&= \lim_{\substack{k \to 0 \\ k \ne 0}}\frac{\dfrac{hk(h^2-k^2)}{h^2+k^2} - 0}{k} \\
&= \lim_{\substack{k \to 0 \\ k \ne 0}} h \frac{h^2-k^2}{h^2+k^2} \\
&= h \frac{h^2}{h^2} = h.
\end{align*}
したがって,
\[
\frac{\partial^2 f}{\partial x \partial y}(0,0) = \lim_{\substack{h \to 0 \\ h \ne 0}}\frac{h - 0}{h} = 1.
\]

以上のことをまとめると,原点における$f(x,y)$の二階偏導関数は
\[
\frac{\partial^2 f}{\partial y \partial x}(0,0) = -1  ,\quad    \frac{\partial^2 f}{\partial x \partial y}(0,0)=1
\]
である\footnote{偏微分の順序によって値が異なることに注意する.}.
\end{tanswer}

\problemtodo{p112--113}{5}
\problemtodo{p112--113}{6}
\problemtodo{p112--113}{7}

\problemtodo[subproblems = 4]{p117-118}{1}
\problemtodo[subproblems = 2]{p117-118}{2}
\problemtodo[subproblems = 3]{p117-118}{3}

\problemtodo[subproblems = 3]{p126}{1}

\problem{p126}{2}


\begin{tproof}
  \begin{description}
    \item[必要条件であること]
          $ f $ が $ x=0 $ で微分可能であると仮定する.この条件は次のように書ける:
          \[
            f(x) - f(0) = c \cdot x + \varepsilon(x)
          \]
          ここで,$ c = (c_1, \ldots, c_n) $ は $ f $ の $ x=0 $ における勾配ベクトルであり,$ \varepsilon(x) $ は
          $\lim_{x \to 0} \varepsilon(x)/\abs{x} = 0$をみたす関数である.このとき,各 $ i $ について次のように $ g_i(x) $ を定義する:
          \[
            g_i(x) =
            \begin{cases}
              c_i + (\varepsilon(x) x_i)/\abs{x}^2 & (x \ne 0) \\
              c_i                                  & (x = 0)
            \end{cases}
          \]

          この $ g_i(x) $ が $ x=0 $ で連続であることを示すために,$ x \to 0 $ の極限を考える.$ x \ne 0 $ のとき,
          \[
            \lim_{x \to 0} g_i(x) = \lim_{x \to 0} \left( c_i + \frac{\varepsilon(x) x_i}{|x|^2} \right) = c_i + \lim_{x \to 0} \frac{\varepsilon(x) x_i}{|x|^2}.
          \]
          ここで,$\varepsilon(x)/\abs{x} \to 0 $ なので,
          \[
            \abs{\frac{\varepsilon(x) x_i}{\abs{x}^2}} = \abs{\frac{\varepsilon(x)}{\abs{x}} \cdot \frac{x_i}{|x|}} \leqq  \abs{\frac{\varepsilon(x)}{\abs{x}} \cdot 1} = \abs{\frac{\varepsilon(x)}{\abs{x}}} \to 0 \quad (x \to 0).
          \]
          したがって,$\lim_{x \to 0} g_i(x) = c_i$なので,$ g_i(x) $ は $ x=0 $ で連続である.よって
          \[
            f(x) = f(0) + \sum_{i=1}^n x_i g_i(x).
          \]
    \item [十分条件であること]
          ある連続な関数 $ g_i \colon \mathbb{R}^n \to \mathbb{R} $ が存在して,次のように表されるとする:
          \[
            f(x) = f(0) + \sum_{i=1}^n x_i g_i(x)
          \]
          このとき,$ g_i(x) $ が $ x=0 $ で連続であるため,各 $ i $ について $ g_i(0) $ が存在し,次が成り立つ:
          \[
            f(x) - f(0) = \sum_{i=1}^n x_i g_i(0) + \sum_{i=1}^n x_i (g_i(x) - g_i(0)).
          \]
          ここで,$ \varepsilon(x) = \sum_{i=1}^n x_i (g_i(x) - g_i(0)) $ とおくと,
          \[
            \lim_{x \to 0} \frac{\varepsilon(x)}{\abs{x}} = \lim_{x \to 0} \frac{\sum_{i=1}^n x_i (g_i(x) - g_i(0))}{\abs{x}}.
          \]
          三角不等式を用いると,
          \[
            \abs{ \frac{\varepsilon(x)}{\abs{x}} } \leqq \sum_{i=1}^n \abs{ \frac{x_i(g_i(x) - g_i(0))}{\abs{x}} }.
          \]

          $ g_i(x) $ が $ x=0 $ で連続であるため,$ g_i(x) \to g_i(0) $ となる.したがって,各$ i $について
          \[
            \abs{ \frac{g_i(x) - g_i(0)}{\abs{x}} } \to 0 \quad (x \to 0).
          \]
          ゆえに,$\lim_{x \to 0} \varepsilon(x)/\abs{x} = 0$となり,$ f $ は $ x=0 $ で微分可能である.
  \end{description}
\end{tproof}

\problemtodo{p126}{3}
\problemtodo{p126}{4}
\problemtodo{p126}{5}

\problemtodo[subproblems = 2]{p144--145}{1}
\problemtodo{p144--145}{2}
\problemtodo{p144--145}{3}
\problemtodo{p144--145}{4}
\problemtodo{p144--145}{5}
\problemtodo{p144--145}{6}
\problemtodo{p144--145}{7}
\problemtodo{p144--145}{8}
\problemtodo{p144--145}{9}
\problemtodo{p144--145}{10}
\problemtodo{p144--145}{11}
\problemtodo{p144--145}{12}
\problemtodo[subproblems = 2]{p144--145}{13}

\problem[label = p148--149:1]{p148--149}{1}

\begin{tanswer}
  $x$を$(r,\theta)$の関数
  \[
    x = x(r,\theta) \colon \mathbb{R}^2 \ni (r,\theta) \mapsto r \cos \theta \in\mathbb{R}
  \]
  とみる.このとき (7.4) から$ z\in \mathbb{R}^2$であり,
  \begin{align*}
    dx_{(r,\theta)} \, (z) & = \frac{\partial x}{\partial r} (r,\theta) z_1 + \frac{\partial x}{\partial \theta} (r,\theta) z_2 \\
                           & = \cos \theta z_1 - r \sin \theta z_2.
  \end{align*}
  一方,$r$,$\theta$を射影
  \begin{align*}
     & r \colon \mathbb{R}^2 \ni (r,\theta) \mapsto r \in \mathbb{R} ,         \\
     & \theta \colon \mathbb{R}^2 \ni (r,\theta) \mapsto \theta \in \mathbb{R}
  \end{align*}
  とみると,
  \begin{align*}
    dr_{(r,\theta)} \, (z)      & = z_1, \\
    d\theta_{(r,\theta)} \, (z) & = z_2,
  \end{align*}
  であるから,
  \begin{align*}
    dx_{(r,\theta)} \, (z) & = \cos \theta \, d r_{(r,\theta)} \, (z) - r \sin \theta \, d \theta_{(r,\theta)} \, (z) \\
                           & = (\cos \theta \, d r_{(r,\theta)} - r \sin \theta \, d \theta_{(r,\theta)}) \, (z).
  \end{align*}
  よって,写像として
  \[
    dx_{(r,\theta)} = \cos \theta \,dr_{(r,\theta)} - r \sin \theta \,d \theta_{(r,\theta)}.
  \]
  略記すると
  \[
    dx = \cos \theta \,dr - r \sin \theta \,d \theta.
  \]
  同様にして,
  \[
    dy = \sin \theta \,dr + r \cos \theta \,d \theta.
  \]
  以上から,
  \begin{align*}
      & x \, dx +y \, dy                                                                                                          \\
    = & r \cos \theta(\cos \theta \, dr - r \sin \theta \, d \theta)+r \sin \theta(\sin \theta \, dr + r \cos \theta \, d \theta) \\
    = & r \, dr,                                                                                                                  \\
      & -y \, dx + x \, dy                                                                                                        \\
    = & -r \sin \theta (\cos \theta  \, dr -r\sin \theta \, d\theta)+r \cos \theta (\sin \theta \, dr + r \cos \theta \, d\theta) \\
    = & r^2 \, d \theta.
  \end{align*}
  以上より,$x \, dx +y \, dy = r \, dr$,$-y \, dx + x \, dy  = r^2 \, d \theta$が成り立つ.
\end{tanswer}

\problemtodo[subproblems = 3]{p148--149}{2}
\problemtodo[subproblems = 2]{p148--149}{3}

\problemtodo[subproblems = 4]{p161}{1}
\problemtodo{p161}{2}
\problemtodo{p161}{3}
\problemtodo{p161}{4}
\problemtodo{p161}{5}
\problemtodo{p161}{6}
\problemtodo{p161}{7}
\problemtodo{p161}{8}
\problemtodo{p161}{9}
\problemtodo{p161}{10}
