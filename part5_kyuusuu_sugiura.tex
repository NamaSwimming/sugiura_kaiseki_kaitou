
\part*{第5章:級数}
\addcontentsline{toc}{part}{\texorpdfstring{第5章:級数}{第5章:級数}}


\section*{p366:1}
\addcontentsline{toc}{section}{\texorpdfstring{p366:1}{p366:1}}

\begin{lemma}{}{}
    $a_{2n} \to \alpha$,$a_{2n+1} \to \beta$$\quad (\alpha,\beta \in [-\infty,\infty])$ならば,集積値全体の集合は$\{\alpha,\beta\}$である.
\end{lemma}
\begin{proof}
    $a_n$の収束部分列$a_{n(k)}$を任意にとる.
    $\{n(k):k \in \mathbb{N}\}$は偶数,奇数の少なくとも一方を無限個含む.
    偶数がそうであったとして,$n(k)$から奇数を取り除いて小さい順に並べた列の$l$番目の項を$a_{n(k(l))}$とする.
    $a_{n(k(l))}$は$a_{2n}$の部分列であるから,仮定より$a_{n(k(l))} \to \alpha \quad (l \to \infty)$である.
    $a_{n(k)}$は収束するから,$a_{n(k)} \to \alpha \quad (k \to \infty)$である.
    奇数が無限個含まれる場合も同様に考えて,$a_{n(k)} \to \beta \quad (k \to \infty)$である.
    $a_{n(k)}$の取り方は任意だったから,集積値は$\alpha$と$\beta$のみである.
\end{proof}

\subsection*{p366:1-(\romannumeral1)}
\addcontentsline{toc}{subsection}{\texorpdfstring{p366:1-(\romannumeral1)}{p366:1-(\romannumeral1)}}

\begin{tanswer}
    \[
        \lim_{m \to \infty} a_{2m} = 1 , \quad \lim_{m \to \infty} a_{2m-1} = -1
    \]
    であり,$(a_n)_{n \in \mathbb{N}}$の集積値全体の集合は補題により$\{ -1 , 1 \}$である.

    ゆえに,
    \[
        \varlimsup_{n \to \infty} a_n = 1 , \quad \varliminf_{n \to \infty} a_n =-1
    \]
    である.
\end{tanswer}


\subsection*{p366:1-(\romannumeral2)}
\addcontentsline{toc}{subsection}{\texorpdfstring{p366:1-(\romannumeral2)}{p366:1-(\romannumeral2)}}

\begin{tanswer}
    \[
        \lim_{m \to \infty} a_{2m} = \infty , \quad \lim_{m \to \infty} a_{2m-1} = -\infty
    \]
    であり,$(a_n)_{n \in \mathbb{N}}$の集積値全体の集合は補題により$\{ -\infty , \infty \}$である\footnote{補完数直線を考えているため,これでよい.}.

    ゆえに,
    \[
        \varlimsup_{n \to \infty} a_n = \infty , \quad \varliminf_{n \to \infty} a_n =-\infty
    \]
    である.
\end{tanswer}

\subsection*{p366:1-(\romannumeral3)}
\addcontentsline{toc}{subsection}{\texorpdfstring{p366:1-(\romannumeral3)}{p366:1-(\romannumeral3)}}

\begin{tanswer}
    $ a_{n} = a_{n+6}$であることを考慮すると,$(a_n)_{n \in \mathbb{N}}$の集積値全体の集合は$\{ 0 , \pm \frac{\sqrt{3}}{2} \}$である.
    ゆえに,
    \[
        \varlimsup_{n \to \infty} a_n = \frac{\sqrt{3}}{2} , \quad \varliminf_{n \to \infty} a_n =-\frac{\sqrt{3}}{2}
    \]
    である.
\end{tanswer}



証明の前に,$\pm \infty$について演算を定義しておく.
\begin{align*}
    a + \infty           & = \infty + a = \infty \qquad \text{if $a \in (-\infty,\infty]$},                   \\
    a - \infty           & = a + (-\infty) = -\infty + a = -\infty \qquad \text{if $a \in [-\infty,\infty)$}, \\
    a \cdot (\pm \infty) & = (\pm \infty) \cdot a =
    \begin{cases*}
        \pm \infty \quad & if $a \in (0,\infty]$,  \\
        \mp \infty \quad & if $a \in [-\infty,0)$.
    \end{cases*}
\end{align*}
$0 \cdot \infty$や$\infty - \infty$などは定義しない.
このとき,補完数直線において次の命題が成り立つ.
\begin{lemma}{}{}
    実数列$\{a_n\}$,$\{b_n\}$が$a_n \xrightarrow{n \to \infty} a$,$b_n \xrightarrow{n \to \infty} b$を満たしており,$a+b$,$ab$が定義されるとき,それぞれ$a_n + b_n \xrightarrow{n \to \infty} a+b$,$a_n b_n \xrightarrow{n \to \infty} ab$が成り立つ.
\end{lemma}
問題2),3)において,演算が定義されないような場合は除外したものを証明する.


\section*{p366:2}
\addcontentsline{toc}{section}{\texorpdfstring{p366:2}{p366:2}}

\begin{tproof}
    \[
        \limsup _{n \to \infty} a_n + \liminf _{n \to \infty} b_n \leq \limsup _{n \to \infty} (a_n + b_n) \leq \limsup _{n \to \infty} a_n + \limsup _{n \to \infty} b_n
    \]
    を示す.その他の不等式についても同様である.
    $n \in \mathbb{N}$を任意に取って固定する.上限の定義より,$m \geq n$に対して,
    \begin{align*}
        a_m & \leq \sup _{k \geq n} a_k, \\
        b_m & \leq \sup _{k \geq n} b_k
    \end{align*}
    であるから,
    \[
        a_m + b_m \leq \sup _{k \geq n} a_k + \sup _{k \geq n} b_k
    \]
    が成り立つ.左辺でmについての上限を取ると,
    \[
        \sup _{k \geq n} (a_k + b_k) \leq \sup _{k \geq n} a_k + \sup _{k \geq n} b_k
    \]
    となる.(右辺は$\{a_m + b_m; m \geq n\}$の上界であり,$\sup$は最小上界であるからこの操作は正当.)$n \to \infty$とすれば,
    \[
        \limsup _{n \to \infty} (a_n + b_n) \leq \limsup _{n \to \infty} a_n + \limsup _{n \to \infty} b_n
    \]
    となって右側の不等式が示される.
    また,$m \geq n$に対し,
    \[
        a_m + \inf _{k \geq n} b_k \leq a_m + b_m \leq \sup _{k \geq n} (a_k + b_k)
    \]
    であるから,左辺の上限を取って,
    \[
        \sup _{k \geq n} a_k + \inf _{k \geq n} b_k \leq \sup _{k \geq n} (a_k + b_k)
    \]
    となる.$n \to \infty$とすれば左側の不等式も示される.
\end{tproof}

\section*{p366:3}
\addcontentsline{toc}{section}{\texorpdfstring{p366:3}{p366:3}}

\begin{tproof}
    $\limsup$についてのみ示す.$\liminf$についても同様である.
    $\limsup_{n \to \infty} a_n = 0$の場合,$\limsup_{n \to \infty} b_n < \infty$であるから$\sup_{k \geq n} b_k < \infty$となり$\left(\sup _{k \geq n} a_k\right) \left(\sup _{k \geq n} b_k\right)$が定義されることに注意する.$a_n, b_n \geq 0$であるから,$m \geq n$に対して,
    \[
        a_m b_m \leq \left(\sup _{k \geq n} a_k\right) \left(\sup _{k \geq n} b_k\right)
    \]
    が成り立つ.左辺で$m$について上限を取ってから$n \to \infty$とすると,
    \[
        \limsup_{n \to \infty} (a_n b_n) \leq \left(\limsup _{n \to \infty} a_n\right) \left(\limsup_{n \to \infty} b_n\right)
    \]
    となる.また,第I章の命題1.4,1.6より,
    \begin{align*}
        \sup_{k \geq n} (c a_k) & = c\sup_{k \geq n} a_k \quad (c > 0),                             \\
        \sup_{k \geq n} (c a_k) & = -\inf_{k \geq n} (-c a_k) = c\inf_{k \geq n} a_k \quad (c < 0),
    \end{align*}
    が成り立つ.$n \to \infty$とすれば残りの等式が得られる.
\end{tproof}

\section*{p366:4}
\addcontentsline{toc}{section}{\texorpdfstring{p366:4}{p366:4}}

\begin{tproof}
    p16の問題4)の一般化である.右側の不等式のみ示す.
    $\alpha = \limsup _{n \to \infty} \frac{a_{n+1}}{a_{n}}$とする.$\alpha = \infty$ならば明らかであるから$\alpha \neq \infty$の場合を考える.$\varepsilon > 0$を任意に取ると,定理1.4より,
    \[
        \frac{a_{n+1}}{a_{n}} < \alpha + \varepsilon \quad (n \geq N)
    \]
    となるような$N$が取れる.この不等式を繰り返し用いて,
    \begin{align*}
        a_n
         & < (\alpha + \varepsilon) a_{n-1}   \\
         & < (\alpha + \varepsilon)^2 a_{n-2} \\
         & < \ldots                           \\
         & < (\alpha + \varepsilon)^{n-N} a_N
    \end{align*}
    がわかる.両側の式を$1/n$乗してから$\limsup$を取ると,
    \[
        \limsup _{n \to \infty} \sqrt[n]{a_n} < \lim _{n \to \infty} (\alpha + \varepsilon)^{1-N/n} \sqrt[n]{a_N} = \alpha + \varepsilon
    \]
    となる.$\varepsilon \to +0$とすると,
    \[
        \limsup _{n \to \infty} \sqrt[n]{a_n} \leq \alpha
    \]
    である.これが証明すべきことであった.
\end{tproof}


\section*{p371:1}
\addcontentsline{toc}{section}{\texorpdfstring{p371:1}{p371:1}}


\subsection*{p371:1-(\romannumeral1)}
\addcontentsline{toc}{subsection}{\texorpdfstring{p371:1-(\romannumeral1)}{p371:1-(\romannumeral1)}}

\begin{tanswer}
    $n =0$のとき,$\sqrt{1+n^2}-n =1-0= 1$であり,この項は有限なので級数の収束・発散に影響しない.

    $n \geq 1$のとき,
    \begin{align*}
        \sqrt{1+n^2}-n & = \frac{(\sqrt{1+n^2}+n)(\sqrt{1+n^2}-n)}{\sqrt{1+n^2}+n} \\
                       & = \frac{1}{\sqrt{1+n^2}+n}                                \\
                       & \geqq \frac{1}{\sqrt{n^2+n^2}+n}                          \\
                       & \geqq \frac{1}{(\sqrt{2}+1)n}
    \end{align*}
    である.ここで,
    \[
        \sum_{n=1}^{\infty} \frac{1}{(\sqrt{2}+1)n} =\infty
    \]
    であるから,比較判定法により,$ \sum_{n=0}^{\infty} ( \sqrt{1+n^2}-n )$は発散する.
\end{tanswer}


\subsection*{p371:1-(\romannumeral3)}
\addcontentsline{toc}{subsection}{\texorpdfstring{p371:1-(\romannumeral3)}{p371:1-(\romannumeral3)}}

\begin{tanswer}
    $n \in \mathbb{N}$に対して,$ \sqrt[n]{n} \leqq n$であるから,

    \[
        \frac{1}{\sqrt[n]{n}} \geqq \frac{1}{n}
    \]
    となる.

    $ \sum_{n=1}^{\infty} \frac{1}{n}$は発散するから,比較判定法により,$ \sum_{n=1}^{\infty} \frac{1}{\sqrt[n]{n}}$も発散する.
\end{tanswer}


\subsection*{p371:1-(\romannumeral4)}
\addcontentsline{toc}{subsection}{\texorpdfstring{p371:1-(\romannumeral4)}{p371:1-(\romannumeral4)}}

\hyperref[p49--50:2-(x)]{p49--50:2-(x)}と重複している.
