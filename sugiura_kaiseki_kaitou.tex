\documentclass[uplatex,dvipdfmx,a4paper,10pt,fleqn]{jsarticle}

% パッケージの読み込み
\usepackage{luatexja}
\usepackage{luatexja-fontspec}
\usepackage{auxhook}
\usepackage{graphicx}
\usepackage{adjustbox}

\usepackage{cite}


%括弧
\usepackage{delimseasy}

%二段組
\usepackage{multicol}
\setlength{\columnseprule}{.5pt} %中央の線

% ヘッダーフォントの設定
\renewcommand{\headfont}{\sffamily\bfseries}
\renewcommand{\familydefault}{\sfdefault}


% 色を定義

\AtEndPreamble{
  %ハイパーリンク用
  \usepackage{url}
  \usepackage{hyperref}
  \definecolor{BlueViolet}{RGB}{105,39,255}
  \definecolor{mylightgray}{HTML}{DDDDDD}
  \definecolor{mydarkgray}{HTML}{777777}
  \hypersetup{
    colorlinks=true,
    citecolor=BlueViolet,
    linkcolor=blue!50!black,
    urlcolor=blue!70!black,
  }
  \usepackage{bookmark}
}




%数式
\usepackage{nccmath,amsmath,amssymb}
\usepackage{mathtools}
\usepackage{empheq} %数式の囲いに使う
\usepackage{bm}
\usepackage[bbsets]{jkmath} %\Nなどをつかえる
\usepackage{amsthm}
\usepackage{color}


%箇条書き
\usepackage[shortlabels]{enumitem}
\setlist[description]{font={\bfseries\sffamily}}

\usepackage{xparse} % ラッパー環境作成のために追加

\usepackage[many]{tcolorbox}
\tcbuselibrary{breakable,skins,theorems}
\newtcolorbox{hosoibox}[1]{colframe=black,colback=white,coltitle=black,colbacktitle=white,boxrule=0.5pt,arc=0mm,enhanced,attach boxed title to top left={xshift=10mm,yshift=-3mm},boxed title style={frame hidden},title=#1}

\usepackage{framed}

\usepackage{etoolbox}
\usepackage{needspace}

%======================================================================
% ★★★ レイアウト問題解決・最終調整版 ★★★
%======================================================================

% --- 1. 必要なパッケージ ---
\usepackage{xcolor}
\usepackage[framemethod=default]{mdframed}

% --- 2. 基本的なレイアウト設定(変更なし) ---
\raggedbottom
\widowpenalty=10000
\clubpenalty=10000
\displaywidowpenalty=10000

%======================================================================
% ★★★ レイアウト問題解決・改訂版 ★★★
%======================================================================


%======================================================================
% ★★★ レイアウト問題解決・最終確定版 Ver.4 ★★★
% ユーザー様ご提供の情報を元にtrivlistの実装を完成
%======================================================================

% --- 1. leftbar の「見た目」を定義(変更なし) ---
\newmdenv[
  linecolor=gray!70,
  linewidth=3pt,
  topline=false,
  bottomline=false,
  rightline=false,
  skipabove=\smallskipamount,
  skipbelow=\smallskipamount,
  leftmargin=10pt,
  rightmargin=0pt,
  innerleftmargin=10pt,
  innerrightmargin=0pt,
  innertopmargin=0pt,
  innerbottommargin=0pt,
]{leftbarstyle}


% --- 2. 堅牢なtrivlistベースの環境定義 ---

\newenvironment{tproof}[1][証明]{%
  \begin{leftbarstyle} % 左の縦線を開始
    \begin{trivlist}
      % (1) \item[...] で見出しを描画
      \item[\hskip\labelsep{\setlength{\fboxsep}{3.5pt}\colorbox{gray!25}{\sffamily\bfseries #1}}]
      % (2) ★★★\leavevmode\par で強制的に改行し、新しい段落を開始する★★★
      \leavevmode\par
      % (3) \ignorespaces で、ユーザーが入力した不要なスペースを無視
      \ignorespaces
      }{%
      % 環境の終了処理
      \par\nopagebreak\hfill\qed % 証明終記号
    \end{trivlist}
  \end{leftbarstyle}
}

\newenvironment{tanswer}[1][解答]{%
  \begin{leftbarstyle}
    \begin{trivlist}
      \item[\hskip\labelsep{\setlength{\fboxsep}{3.5pt}\colorbox{gray!25}{\sffamily\bfseries #1}}]
      \leavevmode\par
      \ignorespaces
      }{%
    \end{trivlist}
  \end{leftbarstyle}
}

\definecolor{mypurple}{HTML}{9900FF}

\definecolor{applePaper}{HTML}{F5F5F7}
\definecolor{appleInk}{HTML}{1D1D1F}
\definecolor{appleLine}{HTML}{D1D1D6}
\definecolor{appleCard}{HTML}{FFFFFF}


\tcbset{
  appleMonoBase/.style={
      enhanced, breakable,
      colback=appleCard, colframe=appleLine, coltitle=appleInk,
      fonttitle=\sffamily\bfseries,
      boxrule=0.5pt, arc=3pt,
      left=8pt,right=8pt,top=6pt,bottom=6pt,
      attach boxed title to top left={xshift=8pt,yshift=-3pt},
      boxed title style={size=small,interior engine=empty},
      drop shadow={black!6!applePaper}
    }
}



\newtcbtheorem
[auto counter, number within=subsection]%
{theorem}{Theorem}% ←英文タイトル
{appleMonoBase,%
  title={\thetcbcounter.\, #2}}% ←タイトル行 1 行で書く
{th}
% 定理環境の定義% ─── 定理ボックス ───────────────────────────────────
\newtcbtheorem
[use counter from=theorem]%
{proposition}{Proposition}% ←英文タイトル
{appleMonoBase,%
  title={\thetcbcounter.\, #2}}% ←タイトル行 1 行で書く
{pr}
\newtcbtheorem
[use counter from=theorem]%
{corollary}{Corollary}% ←英文タイトル
{appleMonoBase,%
  title={\thetcbcounter.\, #2}}% ←タイトル行 1 行で書く
{co}
\newtcbtheorem
[use counter from=theorem]%
{definition}{Definition}% ←英文タイトル
{appleMonoBase,%
  title={\thetcbcounter.\, #2}}% ←タイトル行 1 行で書く
{de}
\newtcbtheorem
[use counter from=theorem]%
{lemma}{Lemma}% ←英文タイトル
{appleMonoBase,%
  title={\thetcbcounter.\, #2}}% ←タイトル行 1 行で書く
{le}
\newtcbtheorem
[use counter from=theorem]%
{example}{Example}% ←英文タイトル
{appleMonoBase,%
  title={\thetcbcounter.\, #2}}% ←タイトル行 1 行で書く
{ex}


% 参照用コマンド
\newcommand{\thref}[1]{{\sffamily\bfseries Theorem\,\ref{th:#1}}}
\newcommand{\prref}[1]{{\sffamily\bfseries Proposition\,\ref{pr:#1}}}
\newcommand{\coref}[1]{{\sffamily\bfseries Corollary\,\ref{co:#1}}}
\newcommand{\deref}[1]{{\sffamily\bfseries Definition\,\ref{de:#1}}}
\newcommand{\leref}[1]{{\sffamily\bfseries Lemma\,\ref{le:#1}}}
\newcommand{\exref}[1]{{\sffamily\bfseries Example\,\ref{ex:#1}}}


%コラム環境
\colorlet{colexam}{lightgray!60!black}

\newtcolorbox[auto counter,number format=\Roman]{column}{
  empty,
  title={\bfseries\sffamily Column \thetcbcounter}, % カウンタをローマ数字で表示
  attach boxed title to top left,
  boxed title style={
      empty,
      size=minimal,
      toprule=2pt,
      top=4pt,
      left=1cm, % タイトルを右に移動
      overlay={
          % タイトルの上の線を削除
          % \draw[colexam,line width=2pt]
          %   ([yshift=-1pt]frame.north west) -- ([yshift=-1pt]frame.north east);
        }
    },
  coltitle=colexam,
  fonttitle=\Large\bfseries,
  before=\par\medskip\noindent,
  parbox=false,
  boxsep=0pt,
  left=5mm, % 左のマージンを増やしてタイトルを右に移動
  right=3mm,
  top=4pt,
  breakable,
  pad at break*=0mm,
  vfill before first,
  overlay unbroken={
      \draw[colexam,line width=2pt]
      ([xshift=-0.5pt,yshift=10pt]frame.north east)
      -- ([xshift=-0.5pt]frame.south east);
      \draw[colexam,line width=2pt]
      ([xshift=-1pt,yshift=10pt]frame.north west)
      -- ([xshift=-1pt]frame.south west);
    },
  overlay first={
      \draw[colexam,line width=2pt]
      ([xshift=-0.5pt,yshift=10pt]frame.north east)
      -- ([xshift=-0.5pt]frame.south east);
      \draw[colexam,line width=2pt]
      ([xshift=-1pt,yshift=10pt]frame.north west)
      -- ([xshift=-1pt]frame.south west);
    },
  overlay middle={
      \draw[colexam,line width=2pt]
      ([xshift=-0.5pt,yshift=10pt]frame.north east)
      -- ([xshift=-0.5pt]frame.south east);
      \draw[colexam,line width=2pt]
      ([xshift=-1pt,yshift=10pt]frame.north west)
      -- ([xshift=-1pt]frame.south west);
    },
  overlay last={
      \draw[colexam,line width=2pt]
      ([xshift=-0.5pt,yshift=10pt]frame.north east)
      -- ([xshift=-0.5pt]frame.south east);
      \draw[colexam,line width=2pt]
      ([xshift=-1pt,yshift=10pt]frame.north west)
      -- ([xshift=-1pt]frame.south west);
    },%
}


%題名付き四角
\usepackage{ascmac}
\usepackage{fancybox}

%図に使うもの
\usepackage{tikz}
\usetikzlibrary{intersections,calc,arrows.meta}
\usepackage{tikz-3dplot}
\usepackage[
  % ---- 共通 ----
  marginparwidth = 0pt,
  % ---- 上下左右 ----
  top    = 25truemm,   %  ← ここは現状維持
  bottom = 25truemm,   %  ← 欲しい下余白に調整
  left   = 25truemm,
  right  = 25truemm
]{geometry}

\usepackage{bxpapersize}
\usepackage[absolute,overlay]{textpos} %図の配置を好きにする


%画像
\usepackage{wrapfig}
%footnoteの変更
\renewcommand\thefootnote{{\dag}\arabic{footnote}}
\renewcommand{\thempfootnote}{{\dag}\arabic{mpfootnote}}
\interfootnotelinepenalty=10000

\usepackage{oubraces} %overunderbraces

%underbraceの文字数が多いときのためのadunderbrace
\usepackage{ifthen}
\newlength{\wdTempA}
\newlength{\wdTempB}
\newcommand{\adunderbrace}[2]{%
  \settowidth{\wdTempA}{$#1$}%
  \settowidth{\wdTempB}{${\scriptstyle #2}$}%
  \ifthenelse{\wdTempA<\wdTempB}{%
    \hspace*{.5\wdTempA}\hspace*{-.5\wdTempB}%
    \underbrace{#1}_{#2}%
    \hspace*{.5\wdTempA}\hspace*{-.5\wdTempB}%
  }{%
    \underbrace{#1}_{#2}%
  }%
}%
%丸付き文字
\newcommand{\ctext}[1]{\raise0.2ex\hbox{\textcircled{\scriptsize{#1}}}}

\setlength{\abovedisplayskip}{5pt}
\setlength{\belowdisplayskip}{3pt}
%ユーザー定義
\newcommand{\dash}[1]{#1^\prime}
\newcommand{\ddash}[1]{#1^{\prime\prime}}
\newcommand{\dddash}[1]{#1^{\prime\prime\prime}}
\newcommand{\hodash}[2]{#2^{(#1)}}
\renewcommand{\labelenumi}{(\arabic{enumi})}%itemを(数字)に変更
\newcommand{\two}{I\hspace{-1.2pt}I}
\newcommand{\three}{I\hspace{-1.2pt}I\hspace{-1.2pt}I}
\renewcommand{\proofname}{証明}
\DeclareMathOperator{\Ker}{Ker}
\DeclareMathOperator{\sgn}{sgn}


\renewcommand{\leq}{\leqq}
\renewcommand{\geq}{\geqq}
\renewcommand{\le}{\leqq}
\renewcommand{\ge}{\geqq}

\newcommand{\Laplacian}{{\mathop{}\!\mathbin\bigtriangleup}}


\newcommand{\cmd}[1]{\texttt{\symbol{"5C}#1}}% 《》囲みコマンド(\kakko)をシンプル囲みに変更
\newcommand\kakko[1]{\noindent{\setlength{\fboxsep}{3.5pt}\colorbox{gray!25}{\textbf{#1}}}}

\newcommand{\pH}{\ensuremath{\mathrm{pH}}}
%%%〈amsthm 読み込み後〉%%%
\makeatletter
\newlength{\proofindent}
\setlength{\proofindent}{1\zw}   % 好きな字下げ幅

%増減表関連
\newcommand{\ner}{
  \begin{tikzpicture}[scale=0.3,baseline=0.3]
    \draw[->,>=stealth] (0,0) to[bend right=45] (1,1);
  \end{tikzpicture}
}

\newcommand{\nel}{
  \begin{tikzpicture}[scale=0.3,baseline=0.3]
    \draw[->,>=stealth] (0,0) to[bend left=45] (1.2,1);
  \end{tikzpicture}
}

\newcommand{\sel}{
  \begin{tikzpicture}[scale=0.3,baseline=0.3]
    \draw[->,>=stealth] (0,1) to[bend left=45] (1,0);
  \end{tikzpicture}
}

\newcommand{\ser}{
  \begin{tikzpicture}[scale=0.3,baseline=0.3]
    \draw[->,>=stealth] (0,1) to[bend right=45] (1.2,0);
  \end{tikzpicture}
}
\usepackage[pagecolor=white,nopagecolor={none}]{pagecolor} % 背景色を変更するためのパッケージ

\newcommand{\tituloum}[5]{\begin{titlepage}
    \begin{center}
      \pagecolor{white} % 背景色をBlueVioletに設定
      \color{black} % テキストカラーを白に設定

      \vspace*{2\baselineskip}

      \rule{\textwidth}{1.6pt}\vspace*{-\baselineskip}\vspace*{2pt}
      \rule{\textwidth}{0.4pt}

      \vspace{0.75\baselineskip}

      {\huge #1}

      \vspace{0.75\baselineskip}

      \rule{\textwidth}{0.4pt}\vspace*{-\baselineskip}\vspace{3.2pt}
      \rule{\textwidth}{1.6pt}

      \vspace{2\baselineskip}

      #3

      \vspace*{3\baselineskip}


      {\huge #2}

      \vspace{0.5\baselineskip}

      \textit{#4}

      \vfill

      \vspace{0.3\baselineskip}

      #5

    \end{center}
  \end{titlepage}}

\everymath{\displaystyle}

\newcommand{\HRule}[1]{\rule{\linewidth}{#1}}



% addpart, problem, problemtodo, subproblem マクロ

% [key = value] 型のオプション引数を使用するためのパッケージ
% `texdoc keyval` 参照
\usepackage{keyval}

% Lua の読みこみ,\luaprogressfalse の場合は代わりに \luadirect を何もしないコマンドとする
\ifluaprogress
    \usepackage{luacode}
\else
    \newcommand{\luadirect}[1]{}
\fi

% Lua の進捗集計の本体のコード(lib.lua)を読みこむ
\luadirect{
dofile("lib.lua")
}

\newcommand*{\addpart}[1]{%
  \addcontentsline{toc}{part}{\texorpdfstring{#1}{#1}}%
  % 章を Lua 側で記録
  \luadirect{
    chapter = Chapter.new(\luastring{#1})
    table.insert(chapters, chapter)
  }%
}

\makeatletter

\newif\ifkaitou@problem@done
\newif\ifkaitou@problem@countitself

\define@key{problem}{label}[abc]{\def\kaitou@problem@label{\label{#1}}}
\define@key{problem}{subproblems}[0]{\def\kaitou@problem@subproblems{#1}}
\define@key{problem}{done}[0]{\kaitou@problem@donetrue}
\define@key{problem}{undone}[0]{\kaitou@problem@donefalse}
\define@key{problem}{count-itself}[0]{\kaitou@problem@countitselftrue}
\def\KV@problem@label@default{\def\kaitou@problem@label{\relax}}

\newcommand*{\kaitou@problem@internal}[3][]{%
  \kaitou@problem@countitselffalse
  \setkeys{problem}{subproblems = 0, label}%
  \setkeys{problem}{#1}%
  \def\ProblemName{#2:#3}%
  % 問題を Lua 側で記録
  \luadirect{
    --[[ 小問の個数 --]]
    local subproblems = tonumber(\luastring{\kaitou@problem@subproblems})

    --[[ 小問を持つか --]]
    local haveSubproblems = false
    if subproblems ~= 0 then
      haveSubproblems = true
    end

    --[[ 自身をカウントするか判断 --]]
    local countitself = false
    if subproblems == 0 then
      countitself = true
    elseif \luastring{\ifkaitou@problem@countitself true\else false\fi} == "true" then
      countitself = true
    end
    if countitself then
      subproblems = subproblems + 1
    end

    --[[ problem インスタンスを作成 --]]
    local page = Page.fromString(\luastring{#2})
    local problemName = tostring(page) .. ":" .. \luastring{#3}
    problem = Problem.new(page, problemName, subproblems, haveSubproblems)
    chapter:addProblem(problem)

    --[[ 自身をカウントする場合,done ならば incrementDone() --]]
    if countitself == true and \luastring{\ifkaitou@problem@done true\else false\fi} == "true" then
      problem:incrementDone()
    end
  }%
}

\newcommand*{\problem}[3][]{%
  \kaitou@problem@donetrue% done がデフォルト
  \kaitou@problem@internal[#1]{#2}{#3}%
  \section*{\ProblemName} \kaitou@problem@label%
  \addcontentsline{toc}{section}{\texorpdfstring{#2:#3}{#2:#3}}%
}

\newcommand*{\problemtodo}[3][]{%
  \kaitou@problem@donefalse% undone がデフォルト
  \kaitou@problem@internal[#1]{#2}{#3}%
}


\newif\ifkaitou@subproblem@done

\define@key{subproblem}{label}{\def\kaitou@subproblem@label{\label{#1}}}
\define@key{subproblem}{done}[0]{\kaitou@subproblem@donetrue}
\define@key{subproblem}{undone}[0]{\kaitou@subproblem@donefalse}

\def\KV@subproblem@label@default{\def\kaitou@subproblem@label{\relax}}

\newcommand*{\subproblem}[2][]{%
  % default
  \kaitou@subproblem@donetrue% done がデフォルト
  \setkeys{subproblem}{label}%
  \setkeys{subproblem}{#1}%
  \subsection*{\ProblemName-(\romannumeral#2)}\kaitou@subproblem@label
  \addcontentsline{toc}{subsection}{\texorpdfstring{\ProblemName-(\romannumeral#2)}{\ProblemName-(\romannumeral#2)}}
  % 小問を Lua 側で記録
  \luadirect{
    if \luastring{\ifkaitou@subproblem@done true\else false\fi} == "true" then
      problem:incrementDone()
    end
  }%
}
\makeatother

\AtBeginDocument{\RenewCommandCopy\qty\SI}

% ドキュメントの最後で progress.md を出力する
\AtEndDocument{\luadirect{
  local file = io.open("progress.md", "w")
  outputProgress(file, chapters)
  }%
}


\usepackage{autobreak}

\usepackage{docmute}
%ヘッダー・フッダー
\usepackage{fancyhdr}
\pagestyle{fancy}
\lhead{}
\chead{杉浦・解析入門解答集}
\rhead{\thepage}
\cfoot{}
\begin{document}

\title{杉浦・解析入門解答集}
\author{なまちゃん}
\date{\today}
\maketitle
\begin{multicols*}{3}
    \tableofcontents
\end{multicols*}
\addcontentsline{toc}{section}{\texorpdfstring{目次}{目次}}
\newpage
\section*{第1章:実数と連続}
\addcontentsline{toc}{section}{\texorpdfstring{第1章:実数と連続}{第1章:実数と連続}}

\subsection*{p2:問1}
\addcontentsline{toc}{subsection}{\texorpdfstring{p2:問1}{p2:問1}}

\begin{itembox}[c]{(1)}
    \begin{proof}
$0,0' \in K$がともに加法単位元の性質を満たすとする.

このとき,$0$が加法単位元の性質をもつことから,
\[
    0'+0=0'
\]
同様に,$0'$が加法単位元の性質をもつことから,
\[
    0+0' = 0
\]
交換律より,$0+0'=0'+0$なので,
\[
    0'=0'+0 =0+0' =0
\]
これからただちに加法単位元の一意性が従う.
    \end{proof}
    \end{itembox}
    \begin{itembox}[c]{(2)}
        \begin{proof}
$a ,b \in K$とし,
\[
    a+b =0
\]
とする.このとき,
\[
    -a = -a+0 = -a +(a+b)=(-a+a)+b =0+b = b
\]
となり,加法逆元の一意性が従う.
        \end{proof}
    \end{itembox}

    \begin{itembox}[c]{(3)}
        \begin{proof}
$a \in K$のとき,
\begin{gather*}
    a+(-a)=0 \\
    \therefore (-a)+a =0
\end{gather*}
他方,$-(-a)$は$(-a)$の加法逆元であるから,
\[
    (-a)+(-(-a))=0
\]
これと逆元の一意性により,$a=-(-a)$が従う.
\end{proof}
\end{itembox}
\begin{itembox}[c]{(5)}
    \begin{proof}
    $a \in K$に対して,
    \[
        a+(-1)a=(1+(-1))a =0a =0
    \]
    であるから,$-a$が$a$の加法逆元であることと含めて主張が従う.
    \end{proof}
\end{itembox}
\newpage 
\begin{itembox}[c]{(6)}
    \begin{proof}
    (4)の結果を用いる.$a=-1$とすると,
    \[
        (-1)(-1)=-(-1)=1
    \]
    これが証明すべきことであった.
    \end{proof}
\end{itembox}
\begin{itembox}[c]{(7)}
    \begin{proof}
    $a,b \in K$に対して,
    \begin{align*}
     a(-b)+ab & = a((-b)+b) \\
     & = a0 \\
    & =0 \\
\therefore \quad & a(-b)=-ab 
    \end{align*}
$(-a)b = -ab$も同様にして示される.
\end{proof}
\end{itembox}
%
\newpage\subsection*{p3:問2}
\addcontentsline{toc}{subsection}{\texorpdfstring{p3:問2}{p3:問2}}

\begin{itembox}[c]{(i)}
    \begin{proof}
        $a \leqq b$において,(R15)より
        \[
            0 \leqq b-a
        \]
        を得る.

        逆に,$0 \leqq b-a$において,(R15)により
        \[
            a \leqq b 
        \]
        を得る.

        以上の考察により証明された.
    \end{proof}
    \end{itembox}

    \begin{itembox}[c]{(ii)}
        \begin{proof}
            $a \leqq b$において,(R15)により
            \[
                0 \leqq b-a
            \]
            となる.
            ここで,(R15)を適用して,
            \[
                -b \leqq -a
            \]
            を得る.

            逆に,$-b\leqq -a$について,(R15)より
            \[
                0 \leqq b -a
            \]
            となる.この両辺に$a$を加えて,
            \[
                a \leqq b 
            \]
            を得る.

            以上の考察により証明された.
        \end{proof}
        \end{itembox}

    \newpage 

    \begin{itembox}[c]{(iii)}
        \begin{proof}
            $c \leqq 0$から
            \begin{align*} 
                 c+(-c) &\leqq -c \\
                \therefore ~ 0 &\leqq -c 
            \end{align*} 
            である.ここで,$a \leqq b$,$-c \leqq 0$であることより
            \[
                a(-c) \leqq b (-c)
            \]
            である.これより$ -ac \leqq -bc$であるから,
            \begin{align*} 
                -ac + (ac+bc) &\leqq -bc +(ac+bc) \\
                \therefore ~ bc &\leqq ac 
            \end{align*} 
            を得て,これが証明すべきことであった.
        \end{proof}
    \end{itembox}

    \begin{itembox}[c]{(iv)}
        \begin{proof}
            $a<0$であると仮定する.このとき,
            \[
               \left ( \frac{1}{a} \right) a < 0a
            \]
            なので,
            \[
                1<0
            \]
            となるが,これは$1>0$に矛盾.

            よって,仮定したことが誤りであり,$a>0$のとき$a^{-1} >0$である.
            \end{proof}
        \end{itembox}

        \begin{itembox}[c]{(v)}
            \begin{proof}
                \[
                    a \leqq c
                \]
                において,(R15)より.
                \[
                    a+b \leqq b+c
                \]
                を得る.他方,
                \[
                    b \leqq d
                \]
                において,(R15)より
                \[
                    b + c \leqq c+d
                \]
                となる.ここで,推移律を適用すると,
                \[
                    a+b \leqq c+d
                \]
                が得られ,これが証明すべきことであった.
            \end{proof}
            \end{itembox}
%
\subsection*{p16-17:1-(i)}
\addcontentsline{toc}{subsection}{\texorpdfstring{p16-17:1-(i)}{p16-17:1-(i)}}

\begin{tleftbar}
    $a=0$のときは明らかに$0$に収束するので,$a \ne 0$とする.$2\abs{a} \le N$となる$N \in \mathbb{N}$をとる.このとき,
    \begin{align*}
         0 &< \abs{ \frac{a^n}{n!} } \\
         &\le \frac{\abs{a}^n}{n!} \\
         &= \frac{\abs{a}^{N}}{N!} \cdot \frac{\abs{a}}{N+1} \cdot \frac{\abs{a}}{N+2} \dotsm \frac{\abs{a}}{n} \\
         & \le  \frac{\abs{a}^{N}}{N!} \left(\frac{1}{2} \right)^{n-N}
    \end{align*}
    であるから,
    \[
        - \frac{\abs{a}^{N}}{N!} \left(\frac{1}{2} \right)^{n-N} \le  \frac{a^n}{n!} \le \frac{\abs{a}^{N}}{N!} \left(\frac{1}{2} \right)^{n-N}
    \]
    となり,はさみうちの原理により,
    \[
        \lim_{n \to \infty} \frac{a^n}{n!} =0
    \]
    である
\end{tleftbar}

\subsection*{p16-17:1-(ii)}
\addcontentsline{toc}{subsection}{\texorpdfstring{p16-17:1-(ii)}{p16-17:1-(ii)}}

\begin{tleftbar}
    $a=1$のときは明らかに$1$に収束するので,まず$a>1$のときを考える.$\delta_n >0$を用いて,
    \[
        \sqrt[n]{a} =1+\delta_n
    \]
    とおくことができる.両辺を$n$乗すると
    \begin{align*}
        a& = 1+ n \delta_n + \frac{1}{2}n(n-1) {\delta_n}^2 + \cdots + {\delta_n}^2 \\
        &>1+n \delta_n \\
        & >n \delta_n
    \end{align*}
    となり,$0<\delta_n <\frac{a}{n}$であるから,はさみうちの原理により,
    \[
        \lim_{n \to \infty} \delta_n =0
    \]
    となる.$a<1$のときは,$a^{\frac{1}{n}}=\left(\left(\frac{1}{a}\right)^{\frac{1}{n}}\right)^{-1}$を使えば同じ結果が得られ,以上の議論により,
    \[
        \lim_{n \to \infty} \sqrt[n]{a} =1
    \]
    となる.
\end{tleftbar}

\subsection*{p16-17:1-(iii)}
\addcontentsline{toc}{subsection}{\texorpdfstring{p16-17:1-(iii)}{p16-17:1-(iii)}}

\begin{tleftbar}
    $\frac{n}{2}$以下の最大の自然数を$m$とおく.与えられた式は,
    \[
       \left( 0  < \right) \frac{n!}{n^n}  = \frac{1 \cdot 2 \dotsm m \cdot (m+1) \dotsm n}{n^n}
    \]
    と表されるので,$\frac{(m+1) \cdot (m+2) \dotsm n}{n^{n-m}} <1$であることと,$m \le \frac{n}{2}$から$\frac{m}{n} \le \frac{1}{2}$であることを用いると,
    \[
        \frac{1 \cdot 2 \dotsm m \cdot (m+1) \dotsm n}{n^n} < \frac{1 \cdot 2 \dotsm m}{n^m} <\left(\frac{1}{2}\right)^m
    \]
    よって,
    \[
        0 < \frac{n!}{n^n} <\left(\frac{1}{2}\right)^m
    \]
    である.$n \to \infty$のとき$m \to \infty$なので,はさみうちの原理により,
    \[
        \lim_{n \to \infty}\frac{n!}{n^n} =0
    \]
    である.
\end{tleftbar}
\subsection*{p16-17:1-(iv)}
\addcontentsline{toc}{subsection}{\texorpdfstring{p16-17:1-(iv)}{p16-17:1-(iv)}}
\begin{tleftbar}
    のちに$n \to \infty$の極限を考えることを考慮すると,
    \begin{align*}
        2^n &= (1+1)^n \\
        =& 1+n +\frac{1}{2} n(n-1)+ \cdots +n+1 \\
        & > \frac{1}{2} n(n-1)
    \end{align*}
    となり,この不等式から,
    \[
        0< \frac{n}{2^n} < \frac{2}{n-1}
    \]
    を得る.ここで,はさみうちの原理により,
    \[
        \lim_{n \to \infty} \frac{n}{2^n}=0
    \]
    である.
\end{tleftbar}

\subsection*{p16-17:1-(v)}
\addcontentsline{toc}{subsection}{\texorpdfstring{p16-17:1-(v)}{p16-17:1-(v)}}
\begin{tleftbar}
    まず,
    \[
       0<\sqrt{n+1} - \sqrt{n} = \frac{1}{\sqrt{n+1} + \sqrt{n}}
    \]
    である.ここで,のちに$n \to \infty$の極限を考えることを考慮すると,
    \[
    \frac{1}{\sqrt{n+1} + \sqrt{n}} < \frac{1}{\sqrt{n}} 
    \]
    であり,
    \[
        0< \sqrt{n+1} - \sqrt{n} <\frac{1}{\sqrt{n}}
    \]
    を得る.ここで,はさみうちの原理を用いると,
    \[
        \lim_{n \to \infty} (\sqrt{n+1} - \sqrt{n} )=0
    \]
    である.
\end{tleftbar}

\subsection*{p16-17:2)}
\addcontentsline{toc}{subsection}{\texorpdfstring{p16-17:2)}{p16-17:2)}}

$n=1,2,\ldots$に対して,
\[
	f_{n} (x)=\lim_{m \to \infty} (\cos (n! \pi x)) ^{2m}
\]
とおく.
ここで,$n!x \in \mathbb{Z}$のとき,
\[
	\cos (n! \pi x)=\pm 1
\]
$n!x \notin \mathbb{Z}$のときは,
\[
	\abs{\cos (n! \pi x)}<1
\]
であるから,
\[
	f_{n} (x)=
	\begin{cases}
		1 &(n!x \in \mathbb{Z}) \\
		0 & (n!x \notin \mathbb{Z})
	\end{cases}
\]
となる.
さて,$x \in \mathbb{R} \setminus \mathbb{Q}$であるならば,どんな$n \in \mathbb{N}$に対しても,$n! x$が整数とならない.
また,$x \in \mathbb{Q}$のとき,$ x=\frac{p}{q}(p,q \in \mathbb{Z},q>0)$とすれば,$n$が$q$より十分大きいときに$n!x$は整数となる.
よって,
\[
	\lim_{n \to \infty} \left( \lim_{m \to \infty} (\cos (n! \pi x)) ^{2m} \right)=
	\begin{cases}
		1 &(x \in \mathbb{Q}) \\
		0 & (x \in \mathbb{R} \setminus \mathbb{Q})
	\end{cases}
\]


\subsection*{p16-17:3)}
\addcontentsline{toc}{subsection}{\texorpdfstring{p16-17:3)}{p16-17:3)}}

\kakko{補題}

    任意の$a_1 , a_2 , \dots a_n \in \mathbb{R}$について,
    \[
        \abs{a_1+a_2+\dots+a_n} \le \abs{a_1}+\abs{a_2}+\dots+\abs{a_n}
    \]
    が成り立つ.


\begin{proof}
    $n=2$のときは三角不等式そのものであるから,
    $n \ge 3$とし,$n-1$個の実数については補題の主張が成り立つものとする.

    いま,
    \[
        a_1 + a_2 + \dots + a_n = (a_1+a_2+\dots+a_{n-1})+a_n
    \]
    であるから,これに三角不等式を適用して,
    \[
        \abs{a_1+a_2+\dots+a_n} \le \abs{a_1+a_2+\dots+a_{n-1}} +\abs{a_n}
    \]
    を得る.ここで,数学的帰納法の仮定より,
    \[
        \abs{a_1+a_2+\dots+a_{n-1}} \le \abs{a_1}+\abs{a_2}+\dots+\abs{a_{n-1}}
    \]
    がいえるので,ここまでの議論で,
    \[
        \abs{a_1+a_2+\dots+a_n} \le \abs{a_1}+\abs{a_2}+\dots+\abs{a_n}
    \]
    は$n$の場合にも成り立つことが示され.以上の議論より補題の主張が従う.
\end{proof}

\begin{tleftbar}
    \begin{proof}
    $\lim_{n \to \infty} a_n =a$であるから,
    任意の$\varepsilon >0$に対して,ある$N_1 \in \mathbb{N}$が存在して,任意の$n \in \mathbb{N}$に対して,
    \[
        n \ge N_1 \Longrightarrow \abs{a_n -a}<\varepsilon 
    \]
    となる.

    また,
    \[
        \abs{\frac{a_1+a_2+\cdots+a_n}{n}-a}= \abs{\frac{(a_1-a)+(a_2-a)+\cdots+(a_n-a)}{n}}
    \]
    と変形する.この右辺に補題を適用し,
    \[
        \abs{\frac{(a_1-a)+(a_2-a)+\cdots+(a_n-a)}{n}} \le \frac{\abs{a_1-a}+\abs{a_2-a}+\cdots+\abs{a_n-a}}{n}
    \]
    を得る.これにより,
    \[
        n \ge N_1 \Longrightarrow \frac{\abs{a_1-a}+\abs{a_2-a}+\cdots+\abs{a_{n_1-1}-a}}{n} +\left( \frac{n-n_1+1}{n} \right ) \varepsilon 
    \]
    となる.ここで$N_2 \coloneqq N_1 +1$とすると,$n \ge N_2$であるとき$\left( \frac{n-n_1+1}{n} \right ) \varepsilon < \varepsilon$となることに注意する.
    
    さて,
    \[
        n \ge N_3 \Longrightarrow \frac{\abs{a_1-a}+\abs{a_2-a}+\cdots+\abs{a_{n_1-1}-a}}{n}<\varepsilon
    \]
    となるように$N_3 \in \mathbb{N}$をとる.$N \coloneqq \max \{ N_2 , N_3 \}$とすると,
    \[
        n \ge N \Longrightarrow \frac{\abs{a_1-a}+\abs{a_2-a}+\cdots+\abs{a_{n_1-1}-a}}{n} +\left( \frac{n-n_1+1}{n} \right ) \varepsilon < \varepsilon + \varepsilon = 2 \varepsilon
    \]
    であり,これより
    \[
        n \ge N \Longrightarrow \abs{\frac{a_1+a_2+\cdots+a_n}{n}-a} < 2 \varepsilon
    \]
    となる.書き換えると.
    \[
        \lim_{n \to \infty} \frac{a_1+a_2+\cdots+a_n}{n}=a
    \]
    となり,これが証明すべきことであった.
    \end{proof}
\end{tleftbar}

\subsection*{p31-33:1-(i)}
\addcontentsline{toc}{subsection}{\texorpdfstring{p31-33:1-(i)}{p31-33:1-(i)}}

\begin{tleftbar}
	\begin{align*}
		\frac{1^2+2^2+\cdots+n^2}{n^3} & = \frac{\dfrac{1}{6}n(n+1)(2n+1)}{n^3} \\
		& =\frac{1}{6} \left(1+\frac{1}{n} \right ) \left(2+\frac{1}{n} \right)
	\end{align*}
	$a_n = \frac{1}{6} \left(1+\frac{1}{n} \right ),~b_n = \left(2+\frac{1}{n} \right)$とおくと,$(a_n)_{n \in \mathbb{N}},~(b_n)_{n \in \mathbb{N}}$は明らかに収束するから,定理2.5(2)より,
	\[
		\lim_{n \to \infty} a_n b_n = \lim_{n \to \infty} a_n \cdot  \lim_{n \to \infty} b_n 
	\]
	である.また,アルキメデスの原理により, 任意の$\varepsilon >0$に対して,$n_0 \in \mathbb{N}$を$n_0 >\frac{1}{\varepsilon}$となるようにとることができ,このとき,任意の$n \in \mathbb{N}$に対して,
	\[
		n \ge n_0 \Longrightarrow 0<\frac{1}{n} \le \frac{1}{n_0} = \varepsilon
	\]
	となり,$\lim_{n \to \infty} \frac{1}{n} =0$である.これより,
	\begin{align*}
		\lim_{n \to \infty} \frac{1^2+2^2+\cdots+n^2}{n^3} & = \lim_{n \to \infty} \frac{1}{6} \left(1+\frac{1}{n} \right ) \left(2+\frac{1}{n} \right) \\
		& = \left \{\lim_{n \to \infty} \frac{1}{6} \left(1+\frac{1}{n} \right ) \right \} \cdot \left \{\lim_{n \to \infty} \left(2+\frac{1}{n} \right ) \right \} \\
		& = \frac{1}{6} (1+0) \cdot (2+0) =\frac{1}{3}
	\end{align*}
\end{tleftbar}


\subsection*{p31-33:1-(ii)}
\addcontentsline{toc}{subsection}{\texorpdfstring{p31-33:1-(ii)}{p31-33:1-(ii)}}

\kakko{補題}


    正数列$(a_n)_{n \in \mathbb{N}}$に対して,$\left(\frac{a_{n+1}}{a_n} \right)_{n \in \mathbb{N}}$が収束し,
    \[
    \lim_{n \to \infty} \frac{a_{n+1}}{a_n} <1
    \]
    となるとき,$\lim_{n \to \infty} a_n =0$である.

\begin{proof}
   $ \lim_{n \to \infty} \frac{a_{n+1}}{a_n} <1$であるから,$r~(0<r<1)$に対して,ある$N_1 \in \mathbb{N}$が存在して,任意の$n \in \mathbb{N}$に対して,
   \[
       n \ge N_1 \Longrightarrow \frac{a_{n+1}}{a_n}<r
   \]
   が成り立つ.このとき,
   \[
       a_n = a_{N_1} \cdot \frac{a_{N_1+1}}{a_{N_1}} \cdot \frac{a_{N_1 +2}}{a_{N_1 +1}} \dotsm \frac{a_{n-1}}{a_{n-2}} \frac{a_n}{a_{n-1}}< a_{N_1} r^{n-N_1}=\frac{a_{N_1}}{r^{N_1}} r^n
   \]
   となる.$0<r<1$より$\lim_{n \to \infty} \frac{a_{N_1}}{r^{N_1}} r^n =0$であるから,$\lim_{n \to \infty} a_n =0$である.
\end{proof}


\begin{tleftbar}
        $a_n = \frac{n^2}{a^n}$とおく.$0<a \le 1$のときは明らかに$\lim_{n \to \infty} a_n=\infty$となる.\par 
        $a>1$のとき,$\frac{a_{n+1}}{a_n} =\frac{\left(1+\dfrac{1}{n}\right)^2}{a}$となり,
        \[
            \lim_{n \to \infty} \frac{\left(1+\dfrac{1}{n}\right)^2}{a} = \frac{1}{a} <1
        \]
        であるから,補題により,$\lim_{n \to \infty} a_n =0$となる.以上の議論により,
        \begin{align*}
            \lim_{n \to \infty} \frac{n^2}{a^n}
            =
            \begin{cases}
                \infty & (0<a \le 1) \\
                0 & (a>1)
            \end{cases}
        \end{align*}
        となる.
	\end{tleftbar}
    \subsection*{p31-33:1-(iii)}
    \addcontentsline{toc}{subsection}{\texorpdfstring{p31-33:1-(iii)}{p31-33:1-(iii)}}
\begin{tleftbar}
    明らかに$\sqrt[n]{n} >1$なので,$\delta_n >0$を用いて,
    \[
        \sqrt[n]{n} = 1+ \delta_n
    \]
    とかける.両辺を$n$乗して,$n \to \infty$の極限を考えることを考慮すると,
    \begin{align*}
        n = (1+\delta_n)^n &=1 + n \delta_n + \frac{1}{2}n(n-1) {\delta_n}^2 + \cdots + (\delta_n)^n \\
        & > \frac{1}{2}n(n-1) {\delta_n}^2
    \end{align*}
    となり,この不等式から,
    \[
        0<\delta_n < \sqrt{\frac{2}{n-1}}
    \]
    を得る.ここで,はさみうちの原理により,$\lim_{n \to \infty} \delta_n =0$であるから,
    \[
        \lim_{n \to \infty} \sqrt[n]{n} =1
    \]
    である.
\end{tleftbar}

\subsection*{p31-33:1-(iv)}
\addcontentsline{toc}{subsection}{\texorpdfstring{p31-33:1-(iv)}{p31-33:1-(iv)}}
\begin{tleftbar}
      $a_n= n^k e^{-n}$とおく.このとき,
      \[
          \lim_{n \to \infty} \frac{a_{n+1}}{a_n} =  \lim_{n \to \infty} \frac{\left(1+\dfrac{1}{n}\right)^k}{e} =\frac{1}{e} <1
      \]
      ゆえに,補題により,
      \[
        \lim_{n \to \infty} n^k e^{-n}=0
      \]
      である.
\end{tleftbar}

\subsection*{p31-33:1-(v)}
\addcontentsline{toc}{subsection}{\texorpdfstring{p31-33:1-(v)}{p31-33:1-(v)}}
\begin{tleftbar}
      $a_n =\left (1-\frac{1}{n^2}\right)^n$とおく.
      \begin{align*}
        \lim_{n \to \infty} a_n & =\lim_{n \to \infty} \left (1-\frac{1}{n^2}\right)^n \\
        & = \lim_{n \to \infty} \left (1+\frac{1}{n}\right)^n \left (1-\frac{1}{n}\right)^n \\
        & = e \cdot \frac{1}{e} =1
      \end{align*}
      である.
\end{tleftbar}

\subsection*{p31-33:1-(vi)}
\addcontentsline{toc}{subsection}{\texorpdfstring{p31-33:1-(vi)}{p31-33:1-(vi)}}
\kakko{補題}


    $(a_n)_{n \in \mathbb{N}}$を実数列とし,$a_n > 0$とする.もし$\lim_{n \to \infty} a_n =0$であれば$\lim_{n \to \infty} \frac{1}{a_n}=\infty$である.
    また,もし$\lim_{n \to \infty} a_n =\infty$であるならば,$\lim_{n \to \infty} \frac{1}{a_n} =0$である,



\begin{proof}
    前半の主張のみ示せば後半の主張も同様に示せるので,前半のみ示す.与えられた条件により,任意の$\varepsilon>0$に対して,$n_0 \in \mathbb{N}$が存在して,任意の$n \in \mathbb{N}$に対して,
    \[
        n \ge n_0 \Longrightarrow |a_n-0|<\varepsilon
    \]
    が成り立つ,ここで$\frac{1}{\varepsilon}=M$とおくと,上の$n_0 \in \mathbb{N}$に対して,
    \[
        n \ge n_0 \Longrightarrow \frac{1}{a_n} >\frac{1}{\varepsilon}=M
    \]
    となり,これより$\lim_{n \to \infty} \frac{1}{a_n}=\infty$が示された.
\end{proof}

\kakko{補題}


    $c>1$のとき,$\lim_{n \to \infty} \frac{1}{c^n} = 0$である.

\begin{proof}
    $c>1$より,$\delta >0$を用いて$c=1+\delta$とおける.このとき,のちに$n \to \infty$の極限を考えることを考慮すると,
    \begin{align*}
        c^n &= (1+\delta)^n \\
        & = 1+n \delta +\frac{1}{2}n (n-1) \delta^2 + \cdots +\delta^n \\
        & > 1+n \delta
    \end{align*}
    このことから,$0<\frac{1}{c^n} <\frac{1}{1+n\delta}$であるから,はさみうちの原理により,
    \[
        \lim_{n \to \infty} \frac{1}{c^n} = 0~(c>1)
    \]
    である.
\end{proof}

\begin{tleftbar}
    $a_n = (c^n +c^{-n})^{-1}$とおく.$c=1$のときは明らかに$\lim_{n \to \infty} a_n =\frac{1}{2}$である.

    $c>1$のとき,2つの補題により,
    \[
        \lim_{n \to \infty} (c^n + c^{-n}) = \infty
    \]
    であるから,補題により$\lim_{n \to \infty} a_n = \lim_{n \to \infty} (c^n +c^{-n})^{-1} =0$である.

    $0<c<1$のときは$c$の逆数を考えることにより同じ結論に帰着する.以上の議論により,
    \begin{align*}
        \lim_{n \to \infty} (c^n +c^{-n})^{-1} =
        \begin{cases}
            \frac{1}{2}&(c=1)\\
            0 & (c \ne 1)
        \end{cases}
    \end{align*}
    である.
\end{tleftbar}


\subsection*{p31-33:2}
\addcontentsline{toc}{subsection}{\texorpdfstring{p31-33:2}{p31-33:2}}

\begin{tleftbar}
	\begin{proof}
		二項定理を用いて$(a_n)_{n \in \mathbb{N}}$の一般項を展開すると,
		\begin{align*}
			a_n & =  1 + n \cdot \frac{1}{n} + \frac{n(n-1)}{2!} \cdot \frac{1}{n^2} + \dots + \frac{n(n-1)\cdots(n-r+1)}{r!} \cdot \frac{1}{n^r} + \cdots \frac{n!}{n!} \cdot + \frac{1}{n^n} \\
            & = 1+ \frac{1}{1!} + \frac{1}{2!} \left(1- \frac{1}{n} \right) + \dots + \frac{1}{r!} \cdot  \left(1 - \frac{1}{n} \right) \dots \left (1-\frac{r-1}{n} \right) + \dots +  \frac{1}{n!} \left(1 - \frac{1}{n} \right) \dots \left(1- \frac{n-1}{n} \right)
        \end{align*}
		同様にして,$a_{n+1}$の展開式を得たとき,$ \frac{1}{n+1} < \frac{1}{n}$であることにより,$r\in \{ 1,2,\dots ,n\}$に対して,
		\[
			\frac{1}{r!} \cdot  \left(1 - \frac{1}{n} \right) \dots \left (1-\frac{r-1}{n} \right) < \frac{1}{r!} \cdot  \left(1 - \frac{1}{n+1} \right) \dots \left (1-\frac{r-1}{n+1} \right)
		\]
		が成立する.これと,$a_{n+1}$の展開式のほうが,正の項を一つ多く含むことから,任意の$n \in \mathbb{N}$に対して,
		\[
			a_{n} < a_{n+1}
		\]
		が成立し,$(a_n)_{n \in \mathbb{N}}$は単調増加数列である.また,
		\begin{align*}
			\label{p32 問2 2}
			a_n &< 1 + \frac{1}{1!} + \frac{1}{2!} +\frac{1}{3!} + \dots + \frac{1}{n!} \\
			& <  1 + \frac{1}{1!} + \frac{1}{2}+ \frac{1}{2^2} + \dots + \frac{1}{2^n} \\
			& <  2 + \frac{1}{2} \left( \frac{ 1 - 2^{-n} }{ 1- \dfrac{1}{2} } \right)\\
			& =  3-2^{-n} \\
			& <  3
		\end{align*}
		であるから,$a_n < 3$となる.
        
        また,$(a_n)_{n \in \mathbb{N}}$が単調増加数列であることから,任意の$n \in \mathbb{N} \setminus \{0,1\}$に対して,
		\[
			a_n > a_1 = \left(1+\frac{1}{1}\right)^1 =2
		\]
		であるから,
        \[
            2<e<3
        \]
        を得る.また,
		\[
			\lim_{n \to \infty} a_n \le e
		\]
		である.他方,
		\[
			\lim_{n \to \infty} a_n \ge a_n > 1 + \frac{1}{1!} + \frac{1}{2!} + \dots + \frac{1}{r!}
		\]
		であるから,ここで$r \to \infty$として,
		\[
			\lim_{n \to \infty} a_n \ge e
		\]
		となり,$\lim_{n \to \infty} a_n =e$を得る.
	\end{proof}
\end{tleftbar}


    \subsection*{p49-50:1}
    \addcontentsline{toc}{subsection}{\texorpdfstring{p49-50:1}{p49-50:1}}

\begin{tleftbar}
	\begin{proof}
		$\lim_{n \to \infty} \sqrt[n]{a_n} =r$であるから,$0<r<1$のとき,$r<k<1$に対して,ある$n_1 \in \mathbb{N}$が存在して,任意の$n \in \mathbb{N}$に対して,
		\[
			n \ge n_1 \Longrightarrow a_n<k^n
		\]
		となる.ここで,定理5.5(比較判定法)と$\lim_{n \to \infty} k^n =0$より,$\sum a_n$は収束する.

		$r>1$のとき,$1>0$に対して,ある$n_2 \in \mathbb{N}$が存在して,任意の$n \in \mathbb{N}$に対して,
		\[
			n \ge n_2 \Longrightarrow a_n >1
		\]
		が成り立ち,$\lim_{n \to \infty} a_n \ne 0$となる.よって定理5.1~系の対偶により$\sum a_n$は発散する.
	\end{proof}
\end{tleftbar}


\subsection*{p49-50:2-(i)}
\addcontentsline{toc}{subsection}{\texorpdfstring{p49-50:2-(i)}{p49-50:2-(i)}}

\begin{screen}
	\[
	\frac{2n^2}{n^3+1}=\frac{2/3}{n+1}+\frac{4n/3-2/3}{n^2-n+1}
	\]
	により
	\begin{align*}
	\sum \frac{2n^2}{n^3+1}&=\sum \frac{2/3}{n+1}+\sum \frac{4n/3-2/3}{n^2-n+1} \\
	&>\sum \frac{2/3}{n+1} \rightarrow \infty
	\end{align*}
	となる.よってこの級数は発散する
	\end{screen}
	

    \subsection*{p49-50:2-(ii)}
    \addcontentsline{toc}{subsection}{\texorpdfstring{p49-50:2-(ii)}{p49-50:2-(ii)}}

	\begin{screen}
	\[
	\sum ^{\infty}_{n=1}\frac{\sqrt{n}}{1+n^2}<\sum \frac{\sqrt{n}}{n^2}=\sum^{\infty}_{n=1}\frac{1}{n^{3/2}}~\left(<1+\int^{\infty}_{1}\frac{dx}{x^{3/2}}=3\right)
	\]
    であるから,この級数は収束する.
	また
	\[
	\sum \frac{1}{n^\alpha}
	\]
	が$\alpha >1$のときに収束することを用いることもできる.
	\end{screen}


    \subsection*{p49-50:2-(iii)}
    \addcontentsline{toc}{subsection}{\texorpdfstring{p49-50:2-(iii)}{p49-50:2-(iii)}}

	\begin{screen}
	$a=1$のとき,この級数は明らかに収束する.
	$a \neq 1$のとき,$a^x$の$x=0$のまわりでのTaylor展開
	\[
	a^x=1+(\log a)x+\frac{(\log a)^2}{2!}x^2+O(x^3)
	\]
	を用いて
	\begin{align*}
	\sum (a^{1/n}-1)&=\sum \left\{ \frac{\log a}{n}+\frac{(\log a)^2}{2!n^2}+O\left(\frac{1}{n^3}\right)\right\}\\
	&>\sum \frac{\log a}{n} \rightarrow \infty\\
	\end{align*}
	となる.よってこの級数は$a=1$のときは収束し,$a \neq 1$のときは発散する.
	\end{screen}


    \subsection*{p49-50:2-(v)}
    \addcontentsline{toc}{subsection}{\texorpdfstring{p49-50:2-(v)}{p49-50:2-(v)}}


	\begin{screen}
        定理5.7(ダランベールの収束判定)を用いる.$a_n=n/2^n$とおくと
        \[
        \lim_{n \to \infty}\frac{a_{n+1}}{a_n}=\lim_{n \to \infty}\frac{n+1}{n}\frac{2^n}{2^{n+1}}=\frac{1}{2}<1
        \]
        となることにより,この級数は収束する
        \end{screen}
        

    \subsection*{p49-50:2-(viii)}
    \addcontentsline{toc}{subsection}{\texorpdfstring{p49-50:2-(viii)}{p49-50:2-(viii)}}

        \begin{screen}
        \[
        \frac{(1+n)^n}{n^{n+1}}>\frac{n^n}{n^{n+1}}=\frac{1}{n}
        \]
        であることから
        \[
        \sum \frac{(1+n)^n}{n^{n+1}}>\sum \frac{1}{n} \rightarrow \infty
        \]
        となる.よってこの級数は発散する.
        \end{screen}
        

        \subsection*{p49-50:2-(x)}
        \addcontentsline{toc}{subsection}{\texorpdfstring{p49-50:2-(x)}{p49-50:2-(x)}}

        \begin{screen}
        定理5.7(ダランベールの収束判定)より$a_n=\left(\dfrac{n}{n+1}\right)^{n^2}$とすると
        \begin{align*}
        \frac{a_{n+1}}{a_n}&=\frac{(n+1)^{(n+1)^2}}{(n+2)^{(n+1)^2}}\frac{(n+1)^{n^2}}{n^{n^2}}=\frac{(n+1)^{n^2}}{n^{n^2}}\frac{(n+1)^{n^2}}{(n+2)^{n^2}}\frac{(n+1)^{2n+1}}{(n+2)^{2n+1}}\\
        &=\left(1+\frac{1}{n}\right)^{n^2}\left(1-\frac{1}{n+2}\right)^{n^2}\left(1-\frac{1}{n+2}\right)^{2n+1}
        \end{align*}
        ここで
        \[
        \left(1+\frac{1}{n}\right)^{n^2}\left(1-\frac{1}{n+2}\right)^{n^2}=\left(1+\frac{1}{n(n+2)}\right)^{n^2}=\left(1+\frac{1}{n(n+2)}\right)^{n(n+2)}\left(1+\frac{1}{n(n+2)}\right)^{-2n}
        \]
        とすることにより,右辺は$e \times 1=e$に収束する.
        
        さらに,
        \[
        \left(1-\frac{1}{n+2}\right)^{2n+1}=\left\{\left(1-\frac{1}{n+2}\right)^{-(n+2)}\right\}^{-2}\left(1-\frac{1}{n+2}\right)^{-3}
        \]
        とすることで,右辺は$\dfrac{1}{e^2} \times 1=\dfrac{1}{e}$に収束する.よって
        \[
        \frac{a_{n+1}}{a_n}=e \times \frac{1}{e^2}=\frac{1}{e}<1
        \]
        となる.ゆえにこの級数は収束する.
    \end{screen}


    \subsection*{p49-50:3}
    \addcontentsline{toc}{subsection}{\texorpdfstring{p49-50:3}{p49-50:3}}

    \begin{tleftbar}
        \begin{proof}
            $\sum a_n $が絶対収束するので,$\sum \abs{a_n}$も収束する.
            よって,定理5.1 系より,$\lim_{n \to \infty} \abs{a_n} =0$となる.
            このとき,$1>0$に対して,ある$n_1 \in \mathbb{N}$が存在して,任意の$n \in \mathbb{N}$に対して,
            \[
                n \ge n_1 \Longrightarrow \abs{\abs{a_n} -0}<1
            \]
            が成り立つ.また,$\abs{a_n}<1$のとき,
            \[
                0 \le \abs{{a_n}^2} \le {\abs{a_n}}^2 \le \abs{a_n}
            \]
            が成り立つ.ここで,$\sum \abs{a_n}$が収束し,各項は正なので, 定理5.5(比較判定法)により,$\sum \abs{{a_n}^2}$も収束する.ゆえに$\sum {a_n}^2$は絶対収束する.
        \end{proof}
    \end{tleftbar}


    \subsection*{p49-50:5}
    \addcontentsline{toc}{subsection}{\texorpdfstring{p49-50:5}{p49-50:5}}

    \begin{tleftbar}
        \begin{proof}
            与えられた条件により,$0 <r <c , r \ne \infty $をみたす$r \in \mathbb{R}$に対して,ある$N \in \mathbb{N}$が存在して,
            任意の$n \in \mathbb{N}$に対して,
            \[
                n \ge N \Longrightarrow \abs{\frac{a_n}{b_n}-c}<r
            \]
            が成り立つ.これにより,
            \begin{align*}
                & 0<-r +c < \frac{a_n}{b_n} < r+c \\
                &\therefore ~  (-r+c) b_n < a_n < (r+c) b_n
            \end{align*}
            となる.これと比較原理により,$\sum a_n$と$\sum b_n$は同時に収束,発散することが証明された.
        \end{proof}
    \end{tleftbar}

    \subsection*{p49-50:6}
    \addcontentsline{toc}{subsection}{\texorpdfstring{p49-50:6}{p49-50:6}}
    \begin{tleftbar}
        $\lim_{n \to \infty} \frac{1}{2n+1} =0$により,$\sum_{n=0}^{\infty} \frac{(-1)^n}{2n+1}$は収束する.ここで,
        \[
             1-x^2+x^4-\cdots+(-1)^n x^{2n} =\frac{1}{1+x^2} +\frac{(-1)^n x^{2n+2}}{x^2+1}
        \]
    の両辺を0から1まで$x$で積分すると,
    \begin{align*}
       \overbrace{1-\frac{1}{3}+\frac{1}{5}-\cdots+\frac{(-1)^n}{2n+1}}^{s_n} &=\int_{0}^{1} \frac{1}{1+x^2} \, dx +\int_{0}^{1}\frac{(-1)^n x^{2n+2}}{x^2+1}  \, dx \\
    & = \frac{\pi}{4} + R_n
    \end{align*}
    である.ただしここで$R_n =\int_{0}^{1}\frac{(-1)^n x^{2n+2}}{x^2+1} \, dx$とおいた.この式から,
    \begin{align*}
      \abs{s_n -\frac{\pi}{4}  } & = \abs{\int_{0}^{1}\frac{(-1)^n x^{2n+2}}{x^2+1} \, dx } \\
       & < \int_{0}^{1} x^{2n} \, dx \\
       & =\frac{1}{2n+1} \to 0 ~(n \to \infty)
    \end{align*}
    である.よって,
    \[
        \sum_{n=0}^{\infty} \frac{(-1)^n}{2n+1} =\frac{\pi}{4}
    \]
    である.
    \end{tleftbar}


    \subsection*{p63-64:1-(i)}
    \addcontentsline{toc}{subsection}{\texorpdfstring{p63-64:1-(i)}{p63-64:1-(i)}}

\begin{tleftbar}
    $f(a)$を考えるため,$a \ne 0$としてよい.$\delta \le \frac{\abs{a}}{2}$とすると,$\abs{x-a}<\delta$より
    \[
       \abs{x} > \abs{a}-\delta \ge \frac{\abs{a}}{2}
    \]
    である.これに留意すると,
    \[
        \abs{\frac{1}{x}-\frac{1}{a}}=\frac{\abs{a-x}}{\abs{ax}} <\frac{2 \delta}{\abs{a}^2}
    \]
    であるから,$\delta = \min \{ \abs{a}/2,\abs{a}^2\varepsilon/2 \}$でよい.
\end{tleftbar}



\subsection*{p63-64:1-(iii)}
\addcontentsline{toc}{subsection}{\texorpdfstring{p63-64:1-(iii)}{p63-64:1-(iii)}}


\begin{tleftbar}
    $t \coloneqq \min \{x,a\}$とする.このとき,指数法則により,
    \[
        \abs{e^x-e^a}=e^t (e^{\abs{x-a}}-1)
    \]
    が成立する.また,$t \le a$であるから,
    \begin{align*}
       & e^t \le e^a \\
        & \therefore ~ e^t(e^{\abs{x-a}}-1) \le e^a(e^{\abs{x-a}}-1)
    \end{align*}
    ゆえに,
    \[
        \varepsilon = e^a (e^\delta -1)
    \]
    となればよい.すなわち,
    \[
        \delta = \log (1+e^{-a}\varepsilon)
    \]
    である.
\end{tleftbar}

\subsection*{p63-64:2-(i)}
\addcontentsline{toc}{subsection}{\texorpdfstring{p63-64:2-(i)}{p63-64:2-(i)}}

\begin{tleftbar}
    $\abs{\sin \frac{1}{x}} \le 1$,$\abs{\sin \frac{1}{y}} \le 1$であるから,
    \[
        \abs{(x+y) \sin \frac{1}{x} \sin \frac{1}{y}} \le \abs{(x+y)} \le \abs{x}+\abs{y} \to 0 \quad  \Bigl( (x,y) \to 0 \Bigl)
    \]
    である.よって,
    \[
        \lim_{(x,y)\to 0} (x+y) \sin \frac{1}{x} \sin \frac{1}{y} =0
    \]
    となる.
\end{tleftbar}

\subsection*{p63-64:2-(iii)}
\addcontentsline{toc}{subsection}{\texorpdfstring{p63-64:2-(iii)}{p63-64:2-(iii)}}

\begin{tleftbar}
    まず,$\log x^x = x \log x$である.また,
    \[
        \lim_{x \to +0} x \log x  =\lim_{x \to +0} \frac{\log x}{1/x} 
    \]
    となる.ここで,$\lim_{x \to +0} \log x = -\infty$,$\lim_{x \to +0} 1/x =\infty$であるから,ロピタルの定理が適用でき,
    \begin{align*}
        \lim_{x \to +0} \frac{\log x}{1/x} & = \lim_{x \to +0} \frac{1/x}{-1/x^2} \\
        & = \lim_{x \to +0} (-x) \\
        & =0
    \end{align*}
    である.よって,
    \begin{align*}
    \lim_{x \to +0} x^x &= \lim_{x \to +0} e^{\log x^x} \\
    & =e^0 =1
    \end{align*}
    となり,これが答である.
\end{tleftbar}

\subsection*{p63-64:2-(iv)}
\addcontentsline{toc}{subsection}{\texorpdfstring{p63-64:2-(iv)}{p63-64:2-(iv)}}

\begin{tleftbar}
    $x=r \cos \theta,~y=r\sin \theta$とおくと,
    \begin{align*}
        \lim_{(x,y)\to 0} \frac{1-\cos (x^2+y^2)}{x^2+y^2} & = \lim_{r \to 0} \frac{1-\cos (r^2)}{r^2} \\
        & =\lim_{r \to 0} \frac{1-(-2\sin ^2 (r^2/2)+1)}{r^2} \\
        & =\lim_{r \to 0} \frac{\sin ^2 (r^2/2)}{(r^2/2)^2} \cdot (r^2/2) \\
        & = 1^2 \cdot 0 =0
    \end{align*}
    を得て,これが答えである.
\end{tleftbar}



\subsection*{p63-64:3-(i)}
\addcontentsline{toc}{subsection}{\texorpdfstring{p63-64:3-(i)}{p63-64:3-(i)}}

\begin{tleftbar}
    以下,$\mathbb{Q}$の閉包は$\mathbb{R}$であることを示す.

    $a \in \mathbb{R}$に対して,
    \[
    U(a,\varepsilon) \cap \mathbb{Q} \ne \varnothing
    \]
    すなわち,
    \[
        (a-\varepsilon,a+\varepsilon) \cap \mathbb{Q} \ne \varnothing
    \]
    となればよい.アルキメデスの原理により,任意の$\varepsilon >0$に対して,
    \[
        \frac{1}{n}< 2\varepsilon 
    \]
    となる$n \in \mathbb{N} \setminus \{0\}$が存在する.また,$n$を分母とする有理数は数直線上に幅$\frac{1}{n}$で並んでいるから,
    \[
        \frac{m}{n} \in (a-\varepsilon,a+\varepsilon)
    \]
    となる$ m \in \mathbb{N} \setminus \{0\}$が存在する.\par 
    したがって$a \in \mathbb{R}$と任意の$\varepsilon>0$に対して$(a-\varepsilon,a+\varepsilon) \cap \mathbb{Q} \ne \varnothing$となるため,
    \[
        \overline{\mathbb{Q}}=\mathbb{R}
    \]
    である.
\end{tleftbar}

\subsection*{p63-64:4-(i)}
\addcontentsline{toc}{subsection}{\texorpdfstring{p63-64:4-(i)}{p63-64:4-(i)}}

\begin{tleftbar}
    \begin{proof}
            $d(x)=0$は$\inf_{y \in A} \abs{x-y} =0$ともかける.
            \begin{align*}
                \inf_{y \in A} \abs{x-y} =0 & \iff (\forall \varepsilon>0) \ (\exists y \in A)\ ( \abs{x-y}<\varepsilon ) \\
                & \iff x \in \overline{A}
            \end{align*}
            これにより示された.
        \end{proof}
    \end{tleftbar}

\section*{第2章:微分法}
\addcontentsline{toc}{section}{\texorpdfstring{第2章:微分法}{第2章:微分法}}


\subsection*{p90-91:1-(i)}
\addcontentsline{toc}{subsection}{\texorpdfstring{p90-91:1-(i)}{p90-91:1-(i)}}


    \begin{tikzpicture}
    \draw[->] (-3,0) -- (3,0) node[below] {$x$};
    \draw[->] (0,-3) -- (0,3) node[left]  {$y$};
    \draw (0,0) node[below left] {$\mathrm{O}$} coordinate (O);
    \draw (0,1) node [above left] {$1$};
    \draw (0,-1) node [below left] {$-1$};
    \draw (2,0) node [above left] { $2$};
    \draw (-2,0) node [above left] { $-2$};
    \draw plot[domain=0:{2*pi}, variable=\theta, smooth] ({2*cos (\theta r)},{sin (\theta r)});
    \end{tikzpicture} 

    \subsection*{p90-91:1-(ii)}
\addcontentsline{toc}{subsection}{\texorpdfstring{p90-91:1-(ii)}{p90-91:1-(ii)}}


    \begin{tikzpicture}
    \draw[->] (-3,0) -- (3,0) node[below] {$x$};
    \draw[->] (0,-3) -- (0,3) node[left]  {$y$};
    \draw (0,0) node[below left] {$\mathrm{O}$} coordinate (O);
    \draw plot[domain=0:{2*pi}, variable=\theta, smooth] ({cos (2*\theta r)},{sin (\theta r)});
    \end{tikzpicture} 


    \subsection*{p90-91:1-(iii)}
\addcontentsline{toc}{subsection}{\texorpdfstring{p90-91:1-(iii)}{p90-91:1-(iii)}}



    \begin{tikzpicture}
    \draw[->] (-3,0) -- (3,0) node[below] {$x$};
    \draw[->] (0,-3) -- (0,3) node[left]  {$y$};
    \draw (0,0) node[below left] {$\mathrm{O}$} coordinate (O);
    \draw plot[domain=0:{2*pi}, variable=\theta, smooth] ({cos (3*\theta r)},{sin (\theta r)});
    \end{tikzpicture} 


\subsection*{p90-91:9-(i)}
\addcontentsline{toc}{subsection}{\texorpdfstring{p90-91:9-(i)}{p90-91:9-(i)}}

\begin{tleftbar}
    \[
       ( \log x )^{(1)}= 1/x , \quad (\log x)^{(2)} = - 1/x^2 , \quad (\log x)^{(3)} = 2/x^3,\quad (\log x)^{(4)} = - 6 /x^4
    \]
    であるから,
    \[
        (\log x)^{(n)} = \frac{(-1)^{n-1}  (n-1)!}{x^n}
    \]
    と推測できる.この推測が正しいことを数学的帰納法により証明する.
    \begin{enumerate}
        \item $n=1$のとき,$(\log x)^{(1)} = 1/x$であり,
        \[
            \frac{(-1)^{1-1}  (1-1)!}{x^1}=1/x
        \]
        であるから,この場合に推測は正しい.
        \item $n=k$のときに,この推測が正しいと仮定すると,
        \[
            (\log x)^{(k)} = \frac{(-1)^{k-1}  (k-1)!}{x^k}
        \]
        である.ここで.
        \begin{align*} 
            (\log x)^{(k+1)} &= \left (\frac{(-1)^{k-1}  (k-1)!}{x^k} \right ) ' \\
            & = \frac{(-1)^k  k!}{x^{k+1}}
        \end{align*} 
        であるから,$n=k+1$のときも推測は正しい.
    \end{enumerate}
    (1),(2)より,
    \[
        (\log x)^{(n)} = \frac{(-1)^{n-1}  (n-1)!}{x^n}
     \]
     である.
\end{tleftbar}

\begin{tleftbar}
    \[
      \left(   \frac{1}{x^2+3x+2} \right)^{(n)} = (-1)^n n! \{ (x+1)^{-n-1} - (x+2)^{-n-1} \} 
    \]
\end{tleftbar}



\subsection*{p90-91:10}
\addcontentsline{toc}{subsection}{\texorpdfstring{p90-91:10}{p90-91:10}}

\begin{tleftbar}
    \begin{proof}
    $u(x)= (x^2-1)^n$とおく,このとき,
    \[
     U'(x)= 2x n(x^2-1)^{n-1}
    \]
    だから,
    \[
        (x^2-1) u'(x)=2nx \cdot u(x)
    \]
    この両辺を$(n+1)$回微分して,
    \begin{align*}
       & (x^2-1)u^{(n+2)}(x)+2(n+1)x u^{(n+1)} x + \frac{(n+1)n}{2} \cdot u^{(n)} (x) = 2nx u^{(n+1)}(x) + 2(n+1) n u^{(n)}(x) \\
       & \therefore ~(x^2-1)u^{(n+2)}(x) + 2n u^{(n+1)}(x)-(n+1)n u^{(n)}(x)=0
    \end{align*}
    ここで,$ u^{(n)} (x)= 2^n n! P_n (x)$なので,
    \begin{align*} 
        & (x^2 -1) \{ 2^n n! P_n ''(x) \} +2x \{ 2^n n! P_n (x) \} -n(n+1) \{ 2^n n P_n(x) \} =0 \\
      &   \therefore ~ (x^2-1) P_n ''(x)+2x P_n '(x) -n(n+1) P_n (x)=0
    \end{align*}
\end{proof}
\end{tleftbar}


\subsection*{p106-107:10}
\addcontentsline{toc}{subsection}{\texorpdfstring{p106-107:10}{p106-107:10}}



\begin{tleftbar}
    \begin{proof}
        3つのことを証明する.
        \begin{description}
            \item[a)とb)が同値であること] \mbox{} \par 
            $a < x \leqq y$または$y \leqq x <a$に対して,$0 \leqq t < 1$を用いて,
            \[
                x=ta+(1-t)y
            \]
            とおく.$a < x \leqq y$とする.このとき,
            \begin{equation}
                \label{eq:p107 10) 1}
            t = \frac{x-y}{a-y}
            \end{equation}
            と表せることはよい.

            また,$f$は$I$で凸であり,これは
            \begin{equation}
                \label{eq:p107 10) 2}
                f(x)=tf(x)+(1-t)f(x) < tf(a)+(1-t)f(y)
            \end{equation}
            と同値である.$x<y$のとき\eqref{eq:p107 10) 2}に\eqref{eq:p107 10) 1}の$t$の値を代入すれば,
            \begin{equation}
                \label{eq:p107 10) 3}
                \frac{f(x)-f(a)}{x-a} < \frac{f(y)-f(x)}{y-x}
            \end{equation}
            を得る.$x=y$のときは明らか.
            \item [a)とc)が同値であること] \mbox{} \par
            c)を仮定する.\eqref{eq:p107 10) 3}の左辺について,平均値の定理により,
            \[
                \frac{f(x)-f(a)}{x-a}=f'(\xi)
            \]
            をみたす$\xi ~(a<\xi <x)$が存在する.同様に,\eqref{eq:p107 10) 3}の右辺について,平均値の定理により,
            \[
                \frac{f(y)-f(x)}{y-x}=f'(\eta)
            \]
            をみたす$\eta~ (x<\eta < y)$が存在する.仮定により,$f'(\xi)\leqq f'(\eta)$だから,\eqref{eq:p107 10) 3}が成り立ち,a)が従う.

            a),すなわち\eqref{eq:p107 10) 3}を仮定する.
            左辺について,$x \to + a$とすれば,これは$f'(a)$に収束し,右辺は,
            これは$(f(y)-f(a))/(y-a)$に収束する(これを$\alpha$とおく.).$f'(a) \leqq \alpha$となることはよい.
            また,$x \to - y$とすれば,左辺は$\alpha$に,右辺は$f'(y)$に収束する.$\alpha \leqq f'(y)$となることもよい.

            よって,a)とc)は同値である.

            \item[a)とd)が同値であること] \mbox{} \par
            d)を仮定する.これは$f'(x)$は$I$上で単調増加であることと同値である. 
        \end{description}
        以上の議論により,示された.
    \end{proof}
\end{tleftbar}


\section*{第3章:初等函数}
\addcontentsline{toc}{section}{\texorpdfstring{第3章:初等函数}{第3章:初等函数}}


\subsection*{p191-193:1}
\addcontentsline{toc}{subsection}{\texorpdfstring{p191-193:1}{p191-193:1}}


\begin{tleftbar}
    \begin{proof}
        $m,n \in \mathbb{N},~n >0$とし,$e=\frac{m}{n}$と表される,すなわち$e$が有理数だと仮定する.このとき,与えられた式を変形して,
        \[
            \frac{e^\theta}{(n+1)!} = e-\sum_{k=0}^{n} \frac{1}{k!} =\frac{m}{n}-\sum_{k=0}^{n} \frac{1}{k!}
        \]
        とする.これにより.
        \[
            \frac{e^{\theta}}{n+1} = m \cdot n! - \sum_{k=0}^{n} \frac{n!}{k!}
        \]
        であり,$m \cdot n! - \sum_{k=0}^{n} \frac{n!}{k!} \in \mathbb{Z}$であるから,$\frac{e^{\theta}}{n+1} \in \mathbb{Z}$である.
        よって,$0< \theta <1,~2<e<3$とあわせて,
        \[
            1 \le \frac{e^{\theta}}{n+1} < \frac{3}{n+1}
        \]
        であり,$\frac{e^{\theta}}{n+1} \in \mathbb{Z}$であるから$n=1$となる.ゆえに$e=m$となり,$e$は整数である.しかし$2<e<3$であるから,これは矛盾である.よって先の仮定が誤りであり,$e$は無理数である.
    \end{proof}
\end{tleftbar}

\section*{第4章:積分法}
\addcontentsline{toc}{section}{\texorpdfstring{第4章:積分法}{第4章:積分法}}

\subsection*{p239:1-(i)}
\addcontentsline{toc}{subsection}{\texorpdfstring{p239:1-(i)}{p239:1-(i)}}

\begin{screen}
	 $\tan \frac{x}{2}=t$とおくと,
	\begin{align*}
		\int_{0}^{\frac{\pi}{2}} \cfrac{\sin x}{1+\cos x} \, dx & = \int_{0}^{1} \cfrac{\cfrac{2t}{1+t^2}}{1+\cfrac{1-t^2}{1+t^2}} \cdot \cfrac{2}{1+t^2} \, dt \\
		& = \int_{0}^{1} \frac{2t}{1+t^2} \, dt \\
		& = \Bigl[\log (t^2+1)\Bigl]_{0}^{1} \\
		& = \log 2-0 = \log 2
	\end{align*}
\end{screen}


\subsection*{p239:1-(ii)}
\addcontentsline{toc}{subsection}{\texorpdfstring{p239:1-(ii)}{p239:1-(ii)}}

\begin{screen}
	  $x-a=a \sin \theta$ ($-\pi \le \theta < \pi$)とする置換を用いる.
	\begin{align*}
		\int_{0}^{a} \sqrt{2ax-x^2} \, dx & = \int_{0}^{a} \sqrt{-(x-a)^2+a^2} \, dx \\
		& = \int_{-\frac{\pi}{2}}^{0} a \sqrt{1-\sin ^2 \theta } \cdot a\cos \theta \, d \theta \\
		& = \int_{-\frac{\pi}{2}}^{0} a \abs{\cos \theta} \cdot a\cos \theta \, d \theta \\
		& = \int_{-\frac{\pi}{2}}^{0} a^2 \cos^2 \theta \, d \theta \\
		& = \int_{-\frac{\pi}{2}}^{0} a^2 \left (\frac{1+\cos 2 \theta }{2}\right) \, d \theta \\
		& = \frac{1}{2} a^2 \left [\theta + \frac{1}{2}\sin 2 \theta \right ]_{-\frac{\pi}{2}}^{0} \\
		&= \frac{\pi a^2}{4}
	\end{align*}
\end{screen}


\subsection*{p239:1-(iii)}
\addcontentsline{toc}{subsection}{\texorpdfstring{p239:1-(iii)}{p239:1-(iii)}}


\begin{screen}
	\begin{align*}
		\abs{\sin 2 \theta} =
		\begin{cases}
			\sin 2 \theta & (0 \le \theta < \frac{\pi}{2} のとき)\\
			- \sin 2 \theta & (\frac{\pi}{2}\le \theta \le \pi のとき)
		\end{cases}
	\end{align*}
		なので,
		\begin{align*}
			\int_{0}^{\pi} \abs{\sin 2 \theta} \, d \theta & = \int_{0}^{\frac{\pi}{2}} \sin 2 \theta \, d \theta +\int_{\frac{\pi}{2}}^{\pi} (-\sin 2 \theta) \, d \theta \\
			&= \left [-\frac{\cos 2 \theta}{2}\right]_{0}^{\frac{\pi}{2}} + \left [\frac{\cos 2 \theta}{2}\right]_{\frac{\pi}{2}}^{\pi} \\
			& = -\frac{(-1-1)}{2} + \frac{1+1}{2} = 2
		\end{align*}
	\end{screen}


\subsection*{p239:1-(iv)}
\addcontentsline{toc}{subsection}{\texorpdfstring{p239:1-(iv)}{p239:1-(iv)}}
	\begin{screen}
		\begin{align*}
			\int_{0}^{\pi} e^{inx} \, dx  =
			\begin{cases}
				2 \pi & (n=0 のとき) \\
				0 & (n \in \mathbb{Z}\setminus \{0\} のとき)
			\end{cases}
		\end{align*}
		である.$n=0$のときは
		\begin{align*}
			\int_{0}^{2 \pi} e^{inx} \, dx & = \int_{0}^{2\pi} \, dx \\
			& = \Bigl[x\Bigl]_{0}^{2\pi} = 2\pi
		\end{align*}
		となり,$n \ne 0$のときは
		\begin{align*}
			\int_{0}^{2\pi} e^{inx} \, dx & = \left [\frac{e^{inx}}{in} \right ]_{0}^{2\pi} \\
			& = \frac{1}{in} (1-1)=0
		\end{align*}
		となる.
	\end{screen}


\subsection*{p239:1-(v)}
\addcontentsline{toc}{subsection}{\texorpdfstring{p239:1-(v)}{p239:1-(v)}}

	\begin{screen}
		$m=n$のとき,
		\begin{align*}
			\int_{0}^{2\pi} \cos m x \sin nx \, dx & = \int_{0}^{2\pi} \cos mx \sin mx \, dx \\
			& = \int_{0}^{2\pi} \left (\frac{\sin 2mx + \sin 0}{2}\right ) \, dx \\
			& = \left [-\frac{\cos 2mx}{2m}\right ]_{0}^{2\pi} =0
		\end{align*}
		となる.$m \ne n$のとき,
		\begin{align*}
			\int_{0}^{2\pi} \cos mx \sin nx \, dx & = \int_{0}^{2\pi} \left (\frac{\sin (m+n)x + \sin (n-m)x}{2}\right) \, dx \\
			& = \frac{1}{2}\left [\frac{\sin (m+n)x}{m+n}+\frac{\sin (n-m)x}{n-m} \right]_{0}^{2\pi} =0
		\end{align*}
		である,ここまでの議論と,$m$と$n$の対称性により,
			\[
				\int_{0}^{2\pi} \cos mx \sin nx \, dx =\int_{0}^{2\pi} \sin mx \cos nx \, dx =0 ~(n,m \in \mathbb{N})
			\]
		となる.
	\end{screen}


\subsection*{p239:1-(vi)}
\addcontentsline{toc}{subsection}{\texorpdfstring{p239:1-(vi)}{p239:1-(vi)}}
	\begin{screen}
		$m \ne n$のとき,
		\begin{align*}
			\int_{0}^{2\pi} \cos mx \cos nx \, dx & = \int_{0}^{2\pi} \left (\frac{\cos(m+n)x+\cos(n-m)x}{2}\right) \, dx\\
			& = \frac{1}{2} \left [\frac{\sin (m+n)x}{m+n}+\frac{\sin(n-m)x}{n-m}\right]_{0}^{2\pi} \\
			& = 0-0 =0
		\end{align*}
		となる.$m =n \ne 0$のとき,
		\begin{align*}
			\int_{0}^{2\pi} \cos mx \cos nx \, dx & = \int_{0}^{2\pi} \cos^2 mx \, dx \\
			& = \int_{0}^{2\pi} \left (\frac{1+\cos 2mx}{2}\right) \, dx  \\
			& = \left [\frac{x}{2}+\frac{\sin 2mx}{4m}\right]_{0}^{2\pi} = \pi
		\end{align*}
		となる.$m =n =0$のとき,
		\begin{align*}
			\int_{0}^{2\pi} \cos mx \cos nx \, dx & = \int_{0}^{2\pi} dx \\
			& = [x]_{0}^{2\pi} = 2\pi
		\end{align*}
		となる,以上をまとめて,
		\begin{align*}
			\int_{0}^{2\pi} \cos mx \cos nx \, dx =
			\begin{cases}
				0 & (m \ne n のとき)\\
				\pi & (m = n\ne 0のとき)\\
				2 \pi & (m=n=0 のとき)
			\end{cases}
		\end{align*}
		である.
	\end{screen}

    \subsection*{p239:3-(i)}
    \addcontentsline{toc}{subsection}{\texorpdfstring{p239:3-(i)}{p239:3-(i)}}

\begin{tleftbar}
    部分積分法を用いると,
    \begin{align*}
        \int x^\alpha \log x \, dx & = \frac{x^{\alpha +1} \log x}{\alpha +1} - \int \frac{x^{\alpha +1}}{\alpha+1} \cdot \frac{1}{x} \, dx \\
        & = \frac{x^{\alpha +1} \log x}{\alpha +1}- \frac{1}{\alpha +1} \int x^{\alpha} \, dx \\
        & = \frac{x^{\alpha+1} \log x}{\alpha +1} - \frac{x^{\alpha +1}}{(\alpha +1)^2}+ C
    \end{align*}
    となり,これが答えである.
\end{tleftbar}

\subsection*{p239:3-(ii)}
\addcontentsline{toc}{subsection}{\texorpdfstring{p239:3-(ii)}{p239:3-(ii)}}

\begin{tleftbar}
    部分積分法を用いると,
    \begin{align*}
        \int x^n e^x \, dx & = x^n e^x - n \int x^{n-1} e^x \, dx \\
        & = x^n e^x - n x^{n-1} e^x + (n-1)\int x^{n-2} e^x \, dx \\
        & = \cdots = x^n e^x - n x^{n-1} e^x + \cdots + (-1)^n n! \int e^x \, dx \\
        & = e^x (x^n -n x^{n-1}+ \cdots +(-1)^n n!) + C
    \end{align*}
    となる.
\end{tleftbar}

\subsection*{p239:3-(iii)}
\addcontentsline{toc}{subsection}{\texorpdfstring{p239:3-(iii)}{p239:3-(iii)}}

\begin{tleftbar}
    部分積分法を用いると,
    \begin{align*}
        \int (\log x)^n \, dx & = \int (\log x)^n  (x)' \, dx \\
        & = x (\log x)^n - n \int  \frac{(\log x)^{n-1}}{x} \cdot  x \, dx \\
        & =  x (\log x)^n - n x(\log x)^{n-1} + (n-1) \int \frac{(\log x)^{n-2}}{x} \cdot x \, dx \\
        & = \cdots = x (\log x)^n - n x(\log x)^{n-1} + \cdots + x(-1)^n n!+C
    \end{align*}
    となり,これが答えである.
\end{tleftbar}

\subsection*{p239:3-(iv)}
\addcontentsline{toc}{subsection}{\texorpdfstring{p239:3-(iv)}{p239:3-(iv)}}
\begin{leftbar}
    部分積分法を用いると,
    \begin{align*}
        \int \arcsin x \, dx & = \int (x)' \arcsin x \, dx \\
        & = x \arcsin x  - \int \frac{x}{\sqrt{1-x^2}} \, dx \\
        & = x \arcsin x + \sqrt{1-x^2} + C
    \end{align*}
    であり,これが答えである.
\end{leftbar}

\subsection*{p239:3-(v)}
\addcontentsline{toc}{subsection}{\texorpdfstring{p239:3-(v)}{p239:3-(v)}}

\begin{tleftbar}
    $\cos ^2 x = \frac{1+\cos 2x}{2}$なので,
    \begin{align*}
        \int \cos ^2 x \, dx & = \int \frac{1+\cos 2x}{2} \, dx \\
        & = \frac{x}{2}+\frac{1}{4} \sin 2x + C
    \end{align*}
    である.
\end{tleftbar}

\newpage 

\subsection*{p239:3-(vii)}
\addcontentsline{toc}{subsection}{\texorpdfstring{p239:3-(vii)}{p239:3-(vii)}}

\begin{tleftbar}
    $x+1 =t$とおくと,$dt=dx$であり,
    \begin{align*}
        \int \frac{x^2+2}{(x+1)^3} \, dx & = \int \frac{(t-1)^2+2}{t^3} \, dt \\
        & = \int \left (\frac{1}{t}-\frac{2}{t^2}+\frac{3}{t^3}\right) \, dt \\
        & = \log \abs{x+1}+\frac{2}{x+1}-\frac{3}{2(x+1)^2}+C
    \end{align*}
    となり,これが答えである.
\end{tleftbar}

\subsection*{p239:3-(vi)}
\addcontentsline{toc}{subsection}{\texorpdfstring{p239:3-(vi)}{p239:3-(vi)}}

\begin{tleftbar}
    $(\log x)' = \frac{1}{x}$であることを用いて,
    \begin{align*}
        \int \frac{(\log x)^2}{x} \, dx & = \int (\log x)^2 (\log x)' \, dx \\
        & = \frac{(\log x)^3}{3} + C
    \end{align*}
    となり,これが答えである.
\end{tleftbar}

\subsection*{p239:3-(viii)}
\addcontentsline{toc}{subsection}{\texorpdfstring{p239:3-(viii)}{p239:3-(viii)}}

\begin{tleftbar}
    $\cos ^2 x = 1- \sin ^2 x$なので,
    \begin{align*}
        \int \sin ^3 x \cos ^2 x \, dx & = \int \sin ^3 x (1-\sin ^2 x) \, dx = \int (\sin ^3 x - \sin ^5 x ) \, dx \\
        & = \int (\sin ^2 x - \sin ^4 x) \sin x \, dx \\
        & = \int \{ (1-t^2)- (1-t^2)^2 \} (-1) \, dt \quad (\cos x =t) \\
        & = \int (t^4 -t^2) \, dt \\
        & = \frac{\cos ^5 x}{5}-\frac{\cos ^3 x}{3}+C
    \end{align*}
    である.
\end{tleftbar}

\subsection*{p239:3-(ix)}
\addcontentsline{toc}{subsection}{\texorpdfstring{p239:3-(ix)}{p239:3-(ix)}}

\begin{tleftbar}
    $\sqrt[6]{x}=t$とおくと,$x=t^6$であるから,$\frac{dx}{dt}=6t^5$である.これらを用いると,
    \begin{align*}
        \int \frac{1}{\sqrt{x}-\sqrt[3]{x}} \, dx & = \int \frac{1}{t^3-t^2} \cdot 6t^5 \, dt \\
        & = \int \frac{6t^3}{t-1} \, dt \\
        & = \int \frac{(t-1)(6t^2+6t+1)+6}{t-1} \, dt \\
        & = \int \left (6t^2+6t+6 + \frac{6}{t-1}\right) \, dt \\
        & = 2\sqrt{x}+3 \sqrt[3]{x} + 6 \sqrt[6]{x} + 6 \log \abs{\sqrt[6]{x}-1}+ C
    \end{align*}
    である
\end{tleftbar}


\subsection*{p239-240:4-(i)}
\addcontentsline{toc}{subsection}{\texorpdfstring{p239-240:4-(i)}{p239-240:4-(i)}}

\begin{tleftbar}
    \begin{align*} 
        \frac{1}{n+1}+ \frac{1}{n+2}+\dots + \frac{1}{2n} & = \sum_{k=1}^{n} \frac{1}{n+k} \\
        & = \frac{1}{n} \sum_{k=1}^{n} \frac{1}{1+k/n} 
    \end{align*}
    であるから,
    \begin{align*} 
        \lim_{n \to \infty} a_n &= \lim_{n \to \infty} \frac{1}{n} \sum_{k=1}^{n} \frac{1}{1+k/n} \\
        & = \int_{0}^{1} \frac{1}{1+x} \, dx \\
        & = \Bigl [ \log (1+x) \Bigl]_{0}^{1} = \log 2
    \end{align*}
    である.
\end{tleftbar}



\subsection*{p239-240:4-(ii)}
\addcontentsline{toc}{subsection}{\texorpdfstring{p239-240:4-(i)}{p239-240:4-(ii)}}

\begin{tleftbar}
    \begin{align*} 
        \frac{1}{\sqrt{n^2+n}}+\frac{1}{\sqrt{n^2+2n}}+\dots+\frac{1}{\sqrt{n^2+n^2}}& = \sum_{k=1}^{n} \frac{1}{\sqrt{n^2+kn}} \\
        & = \frac{1}{n} \sum_{k=1}^{n} \frac{1}{\sqrt{1+k/n}}
    \end{align*}
    なので,
    \begin{align*} 
        \lim_{n \to \infty} a_n &= \lim_{n \to \infty} \frac{1}{n} \sum_{k=1}^{n} \frac{1}{\sqrt{1+k/n}} \\
        & =\int_{0}^{1} \frac{1}{\sqrt{1+x}} \, dx \\
        & =\int_{1}^{\sqrt{2}} \frac{1}{t} \cdot 2t \, dt \\
        & = \Bigl[2t \Bigl ]_{1}^{\sqrt{2}} =2(\sqrt{2}-1)
    \end{align*}
\end{tleftbar}



\subsection*{p247:1-(i)}
\addcontentsline{toc}{subsection}{\texorpdfstring{p247:1-(i)}{p247:1-(i)}}

\begin{tleftbar}
    計算すると,
    \begin{align*}
        \int \frac{1}{x^3-x} \, dx & = \frac{1}{2} \int \left \{ \frac{-(x-1)+(x+1)}{(x-1)x(x+1)} \right \} \, dx \\
        & = \frac{1}{2} \int \left \{ -\frac{1}{(x+1)x}+\frac{1}{x(x-1)} \right \} \, dx \\
        & = \frac{1}{2} \int \left \{ \frac{-(x+1)+x}{(x+1)x} + \frac{x-(x-1)}{x(x-1)} \right \} \, dx \\
        & = \frac{1}{2} \int \left ( -\frac{2}{x}+\frac{1}{x+1}+\frac{1}{x-1} \right) \, dx \\
        & = -\log \abs{x} + \frac{1}{2} \log \abs{x^2-1}+ C
    \end{align*}
    であり,これが答えである.
\end{tleftbar}


\subsection*{p247:1-(ii)}
\addcontentsline{toc}{subsection}{\texorpdfstring{p247:1-(ii)}{p247:1-(ii)}}

\begin{tleftbar}
    $(x-1)(x-2)(x-3)=x^3 -6x^2+11x-6$であるから,
    \[
        \int \frac{x^3}{(x-1)(x-2)(x-3)} \, dx  = \int \left (1+ \frac{6x^2-11x+6}{(x-1)(x-2)(x-3)}\right) \, dx
    \]
    である.ここで,$A,B,C$を定数として,
    \[
        \frac{6x^2-11x+6}{(x-1)(x-2)(x-3)} = \frac{A}{(x-1)}+\frac{B}{(x-2)}+\frac{C}{(x-3)}
    \]
    とおく.これより,
    \begin{gather*}
        6x^2-11x+6 = A(x-2)(x-3)+B (x-1)(x-3)+C(x-1)(x-2) \\
        \therefore ~ A = \frac{1}{2}, \quad B = -\frac{1}{8},\quad C= \frac{27}{2}
    \end{gather*}
    となる.これより,
    \begin{align*}
        \int \frac{x^3}{(x-1)(x-2)(x-3)} \, dx & = \int \left \{ 1+ \frac{1}{2(x-1)} - \frac{1}{8(x-2)}+\frac{27}{2(x-3)} \right \} \, dx  \\
        & = x + \frac{1}{2} \log \abs{\frac{(x-1)(x-3)^{27}}{(x-2)^{16}}}+C
    \end{align*}
    である.
\end{tleftbar}

\subsection*{p247:1-(iii)}
\addcontentsline{toc}{subsection}{\texorpdfstring{p247:1-(iii)}{p247:1-(iii)}}

\begin{tleftbar}
    $A,B,C,D$を定数として,
    \[
        \frac{x^3+1}{x(x-1)^3} = \frac{A}{x}+\frac{B}{x-1}+\frac{C}{(x-1)^2}+\frac{D}{(x-1)^3}
    \]
    とおくと,簡単な計算により,$A=-1,~B=2,~C=1,~D=2$とわかるので,
    \begin{align*}
        \int \frac{x^3+1}{x(x-1)^3} \, dx & = \int \left (-\frac{1}{x}+\frac{2}{x-1}+\frac{1}{(x-1)^2}+\frac{2}{(x-1)^3} \right ) \, dx \\
        & = \log \abs*{\frac{(x-1)^2}{x}} -\frac{1}{(x-1)}-\frac{1}{(x-1)^2}+ C
    \end{align*}
    を得る.
\end{tleftbar}

\subsection*{p247:1-(v)}
\addcontentsline{toc}{subsection}{\texorpdfstring{p247:1-(v)}{p247:1-(v)}}


\begin{tleftbar}
    $\tan \frac{x}{2}=t$とおくと,$\frac{dx}{dt}=\frac{2}{1+t^2}$,$\cos x = \frac{1-t^2}{1+t^2}$であるから,
    \[
        \int \frac{1}{a+b \cos x} \, dx  = \int \frac{1}{(a-b)t^2 +a+b} \, dt
    \]
    となる.よって,
    \[
        \int \frac{1}{a+b\cos x} \, dx =
        \begin{cases}
            \frac{1}{a} \tan \frac{x}{2}+ C &(\text{$a=b \ne 0$のとき})\\
            -\frac{1}{a \tan \frac{x}{2}}+ C&(\text{$a=-b\ne 0$のとき})
        \end{cases}
    \]
    \end{tleftbar}


\subsection*{p247:1-(viii)}
\addcontentsline{toc}{subsection}{\texorpdfstring{p247:1-(viii)}{p247:1-(viii)}}



\begin{tleftbar}
    $\tan x = t$とおくと,$\frac{dt}{dx}= \frac{1}{\cos ^2 x}$であるから,求める不定積分は,
    \begin{align*} 
        \int \frac{1/\cos ^2 x}{a^2 + b^2 \tan ^2 x} \, dx & = \int \frac{1}{a^2+b^2 t^2} \, dt \\
        & = \frac{1}{ab} \arctan \left (\frac{b}{a} \tan x \right)+ C
    \end{align*}
    である.
\end{tleftbar}


\subsection*{p247:1-(xiii)}
\addcontentsline{toc}{subsection}{\texorpdfstring{p247:1-(xiii)}{p247:1-(xiii)}}


\begin{tleftbar}
    $x+\frac{1}{x}=t$とおくと,$\frac{dt}{dx}=1-\frac{1}{x^2}$であり,求める不定積分は,
    \begin{align*} 
        \int \frac{1-x^2}{1+x^2\sqrt{1+x^4}} \, dx & = \int \frac{1-x^2}{x^2 (x+1/x)\sqrt{x^2 + 1/x^2}} \, dx \\
        & =\int \frac{1}{t\sqrt{t^2-2}} \, dt
    \end{align*}
    となる.ここで,$s=\frac{1}{t}$とすると,$\frac{dt}{ds}=-\frac{1}{s^2}$であり,
    \begin{align*} 
        \int \frac{1}{t \sqrt{t^2-2}} \, dt & = \int \frac{1}{\sqrt{1-2s^2}} \, ds \\
        & = \frac{1}{\sqrt{2}} \arcsin (\sqrt{2}s)+C \\
        & = \frac{1}{\sqrt{2}} \arcsin \left (\frac{\sqrt{2}x}{x^2+1} \right)+C
    \end{align*}
    となる.
\end{tleftbar}


\subsection*{p247:1-(xi))}
\addcontentsline{toc}{subsection}{\texorpdfstring{p247:1-(xi)}{p247:1-(xi)}}


\begin{tleftbar}
    $x=\frac{1}{t}$とおくと,$\frac{dx}{dt}=-\frac{1}{t^2}$である.また,$x^2+1 = \frac{1}{t^2} +1$となる.よって,
    \begin{align*} 
        \int \frac{1}{x\sqrt{x^2+1}} \, dx & = \int \frac{t}{\sqrt{1/t^2+1}} \cdot (-1/t^2) \, dt \\
        & = -\int \frac{1}{\sqrt{t^2+1}} \, dt \\
        & = \log \abs{t+\sqrt{t^2+1}}+ C \\
        & = \log \abs{\frac{1+\sqrt{x^2+1}}{x}}+C
    \end{align*}
    である.
\end{tleftbar}


\subsection*{p247:1-(xii)}
\addcontentsline{toc}{subsection}{\texorpdfstring{p247:1-(xii)}{p247:1-(xii)}}


\begin{tleftbar}
    \begin{align*} 
        \int \frac{1}{\sqrt{(x-a)(b-x)}} \, dx & = \int \frac{1}{\sqrt{\left (\dfrac{b-a}{2} \right )^2 - \left (x-\dfrac{a+b}{2} \right )^2 }} \\
        & = \arcsin \left (\frac{2x-a-b}{b-a} \right)+C
    \end{align*}
\end{tleftbar}

\newpage 


\subsection*{p290:3}
\addcontentsline{toc}{subsection}{\texorpdfstring{p290:3}{p290:3}}

\begin{leftbar}
    求める面積を $S$とすると,第1象限の面積を$4$倍すればよいので,
    \begin{align*} 
        S & =4 \cdot  \frac{1}{2}\int_{0}^{\frac{\pi}{4}} r^2 \, d \theta \\
        & = 2 \int_{0}^{\frac{\pi}{4}} (2a^2 \cos 2\theta) \, d \theta \\
        & =2\Bigl [ a^2 \sin 2\theta \Bigl ]_{0}^{\frac{\pi}{4}} \\
        & = 2\left ( a^2 \sin (\pi/2)-a^2 \sin (0) \right) \\
        & = 2a^2
    \end{align*} 
    となる.
\end{leftbar}

\begin{tikzpicture}[scale=1.8]
    % パラメータ
    \def\a{1}
  
    % 領域を塗りつぶす
    \fill[gray!20, domain=-pi/4:pi/4, variable=\t]
      plot ({\a*sqrt(2*cos(2*\t r)) * cos(\t r)}, {\a*sqrt(2*cos(2*\t r)) * sin(\t r)})
      -- (0,0) -- cycle;


    \fill[gray!20, domain=-pi/4:pi/4, variable=\t]
    plot ({-\a*sqrt(2*cos(2*\t r)) * cos(-\t r)}, {-\a*sqrt(2*cos(2*\t r)) * sin(-\t r)})
    -- (0,0) -- cycle;
  
    % 軸の描画
    \draw[->] (-1.7,0) -- (1.7,0) node[below] {$x$};
    \draw[->] (0,-1.7) -- (0,1.7) node[left] {$y$};
  
    % 方程式の描画
    \draw[black, thick, domain=-pi/4:pi/4, variable=\t]
    plot ({\a*sqrt(2*cos(2*\t r)) * cos(\t r)}, {\a*sqrt(2*cos(2*\t r)) * sin(\t r)});


    \draw[black, thick, domain=-pi/4:pi/4, variable=\t]
    plot ({-\a*sqrt(2*cos(2*\t r)) * cos(-\t r)}, {-\a*sqrt(2*cos(2*\t r)) * sin(-\t r)});


  \node at (-1.5,1.5) {$r^2 = 2a^2 \cos(2\theta)$};
  \end{tikzpicture}

\newpage 

\section*{第5章:級数}
\addcontentsline{toc}{section}{\texorpdfstring{第5章:級数}{第5章:級数}}

\subsection*{p366:1-(i)}
\addcontentsline{toc}{subsection}{\texorpdfstring{p366:1-(i)}{p366:1-(i)}}

\begin{tleftbar}
    \[
        \lim_{m \to \infty} a_{2m} = 1 , \quad \lim_{m \to \infty} a_{2m-1} = -1
    \]
    であり,$(a_n)_{n \in \mathbb{N}}$の集積値全体の集合は$\{ -1 , 1 \}$である.

    ゆえに,
    \[
        \varlimsup_{n \to \infty} a_n = 1 , \quad \varliminf_{n \to \infty} a_n =-1
    \]
    である.
\end{tleftbar}


\subsection*{p366:1-(ii)}
\addcontentsline{toc}{subsection}{\texorpdfstring{p366:1-(ii)}{p366:1-(ii)}}

\begin{tleftbar}
    \[
        \lim_{m \to \infty} a_{2m} = \infty , \quad \lim_{m \to \infty} a_{2m-1} = -\infty
    \]
    であり,$(a_n)_{n \in \mathbb{N}}$の集積値全体の集合は$\{ -\infty , \infty \}$である\footnote{補完数直線を考えているため,これでよい.}.

    ゆえに,
    \[
        \varlimsup_{n \to \infty} a_n = \infty , \quad \varliminf_{n \to \infty} a_n =-\infty 
    \]
    である.
\end{tleftbar}

\subsection*{p366:1-(iii)}
\addcontentsline{toc}{subsection}{\texorpdfstring{p366:1-(iii)}{p366:1-(iii)}}

\begin{tleftbar}
    $(a_n)_{n \in \mathbb{N}}$の集積値全体の集合は$\{ -\frac{\sqrt{3}}{2} , \frac{\sqrt{3}}{2} \}$である.
    ゆえに,
    \[
        \varlimsup_{n \to \infty} a_n = \frac{\sqrt{3}}{2} , \quad \varliminf_{n \to \infty} a_n =-\frac{\sqrt{3}}{2}
    \]
    である.
\end{tleftbar}



\begin{thebibliography}{9}
	\bibitem{sugiura} 杉浦光夫『解析入門I』,東京大学出版会,1980
\end{thebibliography}

\end{document}