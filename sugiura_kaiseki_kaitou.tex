\documentclass[a4paper,10pt,fleqn]{ltjsarticle}

\usepackage[hiragino-pron]{luatexja-preset}

% パッケージの読み込み
\usepackage{luatexja}
\usepackage{luatexja-fontspec}
\usepackage{auxhook}
\usepackage{graphicx}
\usepackage{adjustbox}

\usepackage{cite}


%括弧
\usepackage{delimseasy}

%二段組
\usepackage{multicol}
\setlength{\columnseprule}{.5pt} %中央の線

% ヘッダーフォントの設定
\renewcommand{\headfont}{\sffamily\bfseries}
\renewcommand{\familydefault}{\sfdefault}


% 色を定義

\AtEndPreamble{
  %ハイパーリンク用
  \usepackage{url}
  \usepackage{hyperref}
  \definecolor{BlueViolet}{RGB}{105,39,255}
  \definecolor{mylightgray}{HTML}{DDDDDD}
  \definecolor{mydarkgray}{HTML}{777777}
  \hypersetup{
    colorlinks=true,
    citecolor=BlueViolet,
    linkcolor=blue!50!black,
    urlcolor=blue!70!black,
  }
  \usepackage{bookmark}
}




%数式
\usepackage{nccmath,amsmath,amssymb}
\usepackage{mathtools}
\usepackage{empheq} %数式の囲いに使う
\usepackage{bm}
\usepackage[bbsets]{jkmath} %\Nなどをつかえる
\usepackage{amsthm}
\usepackage{color}


%箇条書き
\usepackage[shortlabels]{enumitem}
\setlist[description]{font={\bfseries\sffamily}}

\usepackage{xparse} % ラッパー環境作成のために追加

\usepackage[many]{tcolorbox}
\tcbuselibrary{breakable,skins,theorems}
\newtcolorbox{hosoibox}[1]{colframe=black,colback=white,coltitle=black,colbacktitle=white,boxrule=0.5pt,arc=0mm,enhanced,attach boxed title to top left={xshift=10mm,yshift=-3mm},boxed title style={frame hidden},title=#1}

\usepackage{framed}

\usepackage{etoolbox}
\usepackage{needspace}

%======================================================================
% ★★★ レイアウト問題解決・最終調整版 ★★★
%======================================================================

% --- 1. 必要なパッケージ ---
\usepackage{xcolor}
\usepackage[framemethod=default]{mdframed}

% --- 2. 基本的なレイアウト設定(変更なし) ---
\raggedbottom
\widowpenalty=10000
\clubpenalty=10000
\displaywidowpenalty=10000

%======================================================================
% ★★★ レイアウト問題解決・改訂版 ★★★
%======================================================================


%======================================================================
% ★★★ レイアウト問題解決・最終確定版 Ver.4 ★★★
% ユーザー様ご提供の情報を元にtrivlistの実装を完成
%======================================================================

% --- 1. leftbar の「見た目」を定義(変更なし) ---
\newmdenv[
  linecolor=gray!70,
  linewidth=3pt,
  topline=false,
  bottomline=false,
  rightline=false,
  skipabove=\smallskipamount,
  skipbelow=\smallskipamount,
  leftmargin=10pt,
  rightmargin=0pt,
  innerleftmargin=10pt,
  innerrightmargin=0pt,
  innertopmargin=0pt,
  innerbottommargin=0pt,
]{leftbarstyle}


% --- 2. 堅牢なtrivlistベースの環境定義 ---

\newenvironment{tproof}[1][証明]{%
  \begin{leftbarstyle} % 左の縦線を開始
    \begin{trivlist}
      % (1) \item[...] で見出しを描画
      \item[\hskip\labelsep{\setlength{\fboxsep}{3.5pt}\colorbox{gray!25}{\sffamily\bfseries #1}}]
      % (2) ★★★\leavevmode\par で強制的に改行し、新しい段落を開始する★★★
      \leavevmode\par
      % (3) \ignorespaces で、ユーザーが入力した不要なスペースを無視
      \ignorespaces
      }{%
      % 環境の終了処理
      \par\nopagebreak\hfill\qed % 証明終記号
    \end{trivlist}
  \end{leftbarstyle}
}

\newenvironment{tanswer}[1][解答]{%
  \begin{leftbarstyle}
    \begin{trivlist}
      \item[\hskip\labelsep{\setlength{\fboxsep}{3.5pt}\colorbox{gray!25}{\sffamily\bfseries #1}}]
      \leavevmode\par
      \ignorespaces
      }{%
    \end{trivlist}
  \end{leftbarstyle}
}

\definecolor{mypurple}{HTML}{9900FF}

\definecolor{applePaper}{HTML}{F5F5F7}
\definecolor{appleInk}{HTML}{1D1D1F}
\definecolor{appleLine}{HTML}{D1D1D6}
\definecolor{appleCard}{HTML}{FFFFFF}


\tcbset{
  appleMonoBase/.style={
      enhanced, breakable,
      colback=appleCard, colframe=appleLine, coltitle=appleInk,
      fonttitle=\sffamily\bfseries,
      boxrule=0.5pt, arc=3pt,
      left=8pt,right=8pt,top=6pt,bottom=6pt,
      attach boxed title to top left={xshift=8pt,yshift=-3pt},
      boxed title style={size=small,interior engine=empty},
      drop shadow={black!6!applePaper}
    }
}



\newtcbtheorem
[auto counter, number within=subsection]%
{theorem}{Theorem}% ←英文タイトル
{appleMonoBase,%
  title={\thetcbcounter.\, #2}}% ←タイトル行 1 行で書く
{th}
% 定理環境の定義% ─── 定理ボックス ───────────────────────────────────
\newtcbtheorem
[use counter from=theorem]%
{proposition}{Proposition}% ←英文タイトル
{appleMonoBase,%
  title={\thetcbcounter.\, #2}}% ←タイトル行 1 行で書く
{pr}
\newtcbtheorem
[use counter from=theorem]%
{corollary}{Corollary}% ←英文タイトル
{appleMonoBase,%
  title={\thetcbcounter.\, #2}}% ←タイトル行 1 行で書く
{co}
\newtcbtheorem
[use counter from=theorem]%
{definition}{Definition}% ←英文タイトル
{appleMonoBase,%
  title={\thetcbcounter.\, #2}}% ←タイトル行 1 行で書く
{de}
\newtcbtheorem
[use counter from=theorem]%
{lemma}{Lemma}% ←英文タイトル
{appleMonoBase,%
  title={\thetcbcounter.\, #2}}% ←タイトル行 1 行で書く
{le}
\newtcbtheorem
[use counter from=theorem]%
{example}{Example}% ←英文タイトル
{appleMonoBase,%
  title={\thetcbcounter.\, #2}}% ←タイトル行 1 行で書く
{ex}


% 参照用コマンド
\newcommand{\thref}[1]{{\sffamily\bfseries Theorem\,\ref{th:#1}}}
\newcommand{\prref}[1]{{\sffamily\bfseries Proposition\,\ref{pr:#1}}}
\newcommand{\coref}[1]{{\sffamily\bfseries Corollary\,\ref{co:#1}}}
\newcommand{\deref}[1]{{\sffamily\bfseries Definition\,\ref{de:#1}}}
\newcommand{\leref}[1]{{\sffamily\bfseries Lemma\,\ref{le:#1}}}
\newcommand{\exref}[1]{{\sffamily\bfseries Example\,\ref{ex:#1}}}


%コラム環境
\colorlet{colexam}{lightgray!60!black}

\newtcolorbox[auto counter,number format=\Roman]{column}{
  empty,
  title={\bfseries\sffamily Column \thetcbcounter}, % カウンタをローマ数字で表示
  attach boxed title to top left,
  boxed title style={
      empty,
      size=minimal,
      toprule=2pt,
      top=4pt,
      left=1cm, % タイトルを右に移動
      overlay={
          % タイトルの上の線を削除
          % \draw[colexam,line width=2pt]
          %   ([yshift=-1pt]frame.north west) -- ([yshift=-1pt]frame.north east);
        }
    },
  coltitle=colexam,
  fonttitle=\Large\bfseries,
  before=\par\medskip\noindent,
  parbox=false,
  boxsep=0pt,
  left=5mm, % 左のマージンを増やしてタイトルを右に移動
  right=3mm,
  top=4pt,
  breakable,
  pad at break*=0mm,
  vfill before first,
  overlay unbroken={
      \draw[colexam,line width=2pt]
      ([xshift=-0.5pt,yshift=10pt]frame.north east)
      -- ([xshift=-0.5pt]frame.south east);
      \draw[colexam,line width=2pt]
      ([xshift=-1pt,yshift=10pt]frame.north west)
      -- ([xshift=-1pt]frame.south west);
    },
  overlay first={
      \draw[colexam,line width=2pt]
      ([xshift=-0.5pt,yshift=10pt]frame.north east)
      -- ([xshift=-0.5pt]frame.south east);
      \draw[colexam,line width=2pt]
      ([xshift=-1pt,yshift=10pt]frame.north west)
      -- ([xshift=-1pt]frame.south west);
    },
  overlay middle={
      \draw[colexam,line width=2pt]
      ([xshift=-0.5pt,yshift=10pt]frame.north east)
      -- ([xshift=-0.5pt]frame.south east);
      \draw[colexam,line width=2pt]
      ([xshift=-1pt,yshift=10pt]frame.north west)
      -- ([xshift=-1pt]frame.south west);
    },
  overlay last={
      \draw[colexam,line width=2pt]
      ([xshift=-0.5pt,yshift=10pt]frame.north east)
      -- ([xshift=-0.5pt]frame.south east);
      \draw[colexam,line width=2pt]
      ([xshift=-1pt,yshift=10pt]frame.north west)
      -- ([xshift=-1pt]frame.south west);
    },%
}


%題名付き四角
\usepackage{ascmac}
\usepackage{fancybox}

%図に使うもの
\usepackage{tikz}
\usetikzlibrary{intersections,calc,arrows.meta}
\usepackage{tikz-3dplot}
\usepackage[
  % ---- 共通 ----
  marginparwidth = 0pt,
  % ---- 上下左右 ----
  top    = 25truemm,   %  ← ここは現状維持
  bottom = 25truemm,   %  ← 欲しい下余白に調整
  left   = 25truemm,
  right  = 25truemm
]{geometry}

\usepackage{bxpapersize}
\usepackage[absolute,overlay]{textpos} %図の配置を好きにする


%画像
\usepackage{wrapfig}
%footnoteの変更
\renewcommand\thefootnote{{\dag}\arabic{footnote}}
\renewcommand{\thempfootnote}{{\dag}\arabic{mpfootnote}}
\interfootnotelinepenalty=10000

\usepackage{oubraces} %overunderbraces

%underbraceの文字数が多いときのためのadunderbrace
\usepackage{ifthen}
\newlength{\wdTempA}
\newlength{\wdTempB}
\newcommand{\adunderbrace}[2]{%
  \settowidth{\wdTempA}{$#1$}%
  \settowidth{\wdTempB}{${\scriptstyle #2}$}%
  \ifthenelse{\wdTempA<\wdTempB}{%
    \hspace*{.5\wdTempA}\hspace*{-.5\wdTempB}%
    \underbrace{#1}_{#2}%
    \hspace*{.5\wdTempA}\hspace*{-.5\wdTempB}%
  }{%
    \underbrace{#1}_{#2}%
  }%
}%
%丸付き文字
\newcommand{\ctext}[1]{\raise0.2ex\hbox{\textcircled{\scriptsize{#1}}}}

\setlength{\abovedisplayskip}{5pt}
\setlength{\belowdisplayskip}{3pt}
%ユーザー定義
\newcommand{\dash}[1]{#1^\prime}
\newcommand{\ddash}[1]{#1^{\prime\prime}}
\newcommand{\dddash}[1]{#1^{\prime\prime\prime}}
\newcommand{\hodash}[2]{#2^{(#1)}}
\renewcommand{\labelenumi}{(\arabic{enumi})}%itemを(数字)に変更
\newcommand{\two}{I\hspace{-1.2pt}I}
\newcommand{\three}{I\hspace{-1.2pt}I\hspace{-1.2pt}I}
\renewcommand{\proofname}{証明}
\DeclareMathOperator{\Ker}{Ker}
\DeclareMathOperator{\sgn}{sgn}


\renewcommand{\leq}{\leqq}
\renewcommand{\geq}{\geqq}
\renewcommand{\le}{\leqq}
\renewcommand{\ge}{\geqq}

\newcommand{\Laplacian}{{\mathop{}\!\mathbin\bigtriangleup}}


\newcommand{\cmd}[1]{\texttt{\symbol{"5C}#1}}% 《》囲みコマンド(\kakko)をシンプル囲みに変更
\newcommand\kakko[1]{\noindent{\setlength{\fboxsep}{3.5pt}\colorbox{gray!25}{\textbf{#1}}}}

\newcommand{\pH}{\ensuremath{\mathrm{pH}}}
%%%〈amsthm 読み込み後〉%%%
\makeatletter
\newlength{\proofindent}
\setlength{\proofindent}{1\zw}   % 好きな字下げ幅

%増減表関連
\newcommand{\ner}{
  \begin{tikzpicture}[scale=0.3,baseline=0.3]
    \draw[->,>=stealth] (0,0) to[bend right=45] (1,1);
  \end{tikzpicture}
}

\newcommand{\nel}{
  \begin{tikzpicture}[scale=0.3,baseline=0.3]
    \draw[->,>=stealth] (0,0) to[bend left=45] (1.2,1);
  \end{tikzpicture}
}

\newcommand{\sel}{
  \begin{tikzpicture}[scale=0.3,baseline=0.3]
    \draw[->,>=stealth] (0,1) to[bend left=45] (1,0);
  \end{tikzpicture}
}

\newcommand{\ser}{
  \begin{tikzpicture}[scale=0.3,baseline=0.3]
    \draw[->,>=stealth] (0,1) to[bend right=45] (1.2,0);
  \end{tikzpicture}
}
\usepackage[pagecolor=white,nopagecolor={none}]{pagecolor} % 背景色を変更するためのパッケージ

\newcommand{\tituloum}[5]{\begin{titlepage}
    \begin{center}
      \pagecolor{white} % 背景色をBlueVioletに設定
      \color{black} % テキストカラーを白に設定

      \vspace*{2\baselineskip}

      \rule{\textwidth}{1.6pt}\vspace*{-\baselineskip}\vspace*{2pt}
      \rule{\textwidth}{0.4pt}

      \vspace{0.75\baselineskip}

      {\huge #1}

      \vspace{0.75\baselineskip}

      \rule{\textwidth}{0.4pt}\vspace*{-\baselineskip}\vspace{3.2pt}
      \rule{\textwidth}{1.6pt}

      \vspace{2\baselineskip}

      #3

      \vspace*{3\baselineskip}


      {\huge #2}

      \vspace{0.5\baselineskip}

      \textit{#4}

      \vfill

      \vspace{0.3\baselineskip}

      #5

    \end{center}
  \end{titlepage}}

\everymath{\displaystyle}

\newcommand{\HRule}[1]{\rule{\linewidth}{#1}}



% addpart, problem, problemtodo, subproblem マクロ

% [key = value] 型のオプション引数を使用するためのパッケージ
% `texdoc keyval` 参照
\usepackage{keyval}

% Lua の読みこみ,\luaprogressfalse の場合は代わりに \luadirect を何もしないコマンドとする
\ifluaprogress
    \usepackage{luacode}
\else
    \newcommand{\luadirect}[1]{}
\fi

% Lua の進捗集計の本体のコード(lib.lua)を読みこむ
\luadirect{
dofile("lib.lua")
}

\newcommand*{\addpart}[1]{%
  \addcontentsline{toc}{part}{\texorpdfstring{#1}{#1}}%
  % 章を Lua 側で記録
  \luadirect{
    chapter = Chapter.new(\luastring{#1})
    table.insert(chapters, chapter)
  }%
}

\makeatletter

\newif\ifkaitou@problem@done
\newif\ifkaitou@problem@countitself

\define@key{problem}{label}[abc]{\def\kaitou@problem@label{\label{#1}}}
\define@key{problem}{subproblems}[0]{\def\kaitou@problem@subproblems{#1}}
\define@key{problem}{done}[0]{\kaitou@problem@donetrue}
\define@key{problem}{undone}[0]{\kaitou@problem@donefalse}
\define@key{problem}{count-itself}[0]{\kaitou@problem@countitselftrue}
\def\KV@problem@label@default{\def\kaitou@problem@label{\relax}}

\newcommand*{\kaitou@problem@internal}[3][]{%
  \kaitou@problem@countitselffalse
  \setkeys{problem}{subproblems = 0, label}%
  \setkeys{problem}{#1}%
  \def\ProblemName{#2:#3}%
  % 問題を Lua 側で記録
  \luadirect{
    --[[ 小問の個数 --]]
    local subproblems = tonumber(\luastring{\kaitou@problem@subproblems})

    --[[ 小問を持つか --]]
    local haveSubproblems = false
    if subproblems ~= 0 then
      haveSubproblems = true
    end

    --[[ 自身をカウントするか判断 --]]
    local countitself = false
    if subproblems == 0 then
      countitself = true
    elseif \luastring{\ifkaitou@problem@countitself true\else false\fi} == "true" then
      countitself = true
    end
    if countitself then
      subproblems = subproblems + 1
    end

    --[[ problem インスタンスを作成 --]]
    local page = Page.fromString(\luastring{#2})
    local problemName = tostring(page) .. ":" .. \luastring{#3}
    problem = Problem.new(page, problemName, subproblems, haveSubproblems)
    chapter:addProblem(problem)

    --[[ 自身をカウントする場合,done ならば incrementDone() --]]
    if countitself == true and \luastring{\ifkaitou@problem@done true\else false\fi} == "true" then
      problem:incrementDone()
    end
  }%
}

\newcommand*{\problem}[3][]{%
  \kaitou@problem@donetrue% done がデフォルト
  \kaitou@problem@internal[#1]{#2}{#3}%
  \section*{\ProblemName} \kaitou@problem@label%
  \addcontentsline{toc}{section}{\texorpdfstring{#2:#3}{#2:#3}}%
}

\newcommand*{\problemtodo}[3][]{%
  \kaitou@problem@donefalse% undone がデフォルト
  \kaitou@problem@internal[#1]{#2}{#3}%
}


\newif\ifkaitou@subproblem@done

\define@key{subproblem}{label}{\def\kaitou@subproblem@label{\label{#1}}}
\define@key{subproblem}{done}[0]{\kaitou@subproblem@donetrue}
\define@key{subproblem}{undone}[0]{\kaitou@subproblem@donefalse}

\def\KV@subproblem@label@default{\def\kaitou@subproblem@label{\relax}}

\newcommand*{\subproblem}[2][]{%
  % default
  \kaitou@subproblem@donetrue% done がデフォルト
  \setkeys{subproblem}{label}%
  \setkeys{subproblem}{#1}%
  \subsection*{\ProblemName-(\romannumeral#2)}\kaitou@subproblem@label
  \addcontentsline{toc}{subsection}{\texorpdfstring{\ProblemName-(\romannumeral#2)}{\ProblemName-(\romannumeral#2)}}
  % 小問を Lua 側で記録
  \luadirect{
    if \luastring{\ifkaitou@subproblem@done true\else false\fi} == "true" then
      problem:incrementDone()
    end
  }%
}
\makeatother

\AtBeginDocument{\RenewCommandCopy\qty\SI}

% ドキュメントの最後で progress.md を出力する
\AtEndDocument{\luadirect{
  local file = io.open("progress.md", "w")
  outputProgress(file, chapters)
  }%
}


\usepackage{autobreak}

\usepackage{docmute}
%ヘッダー・フッダー
\usepackage{fancyhdr}

\renewcommand{\headrulewidth}{0.4pt} % 線の太さ
\renewcommand{\headrule}{\color{black}\hrule width\headwidth height\headrulewidth \vskip-\headrulewidth} % 線の色

\pagestyle{fancy}
\fancyhf{} % 既存の設定をクリア
\fancyhead[C]{\textcolor{black}{杉浦『解析入門』解答集}} % ヘッダーの中央にテキストを挿入
\fancyhead[R]{\textcolor{black}{\thepage}} % ヘッダーの右側にページ番号を挿入

\setcounter{secnumdepth}{2}

\begin{document}
% タイトルページの呼び出し
\tituloum{杉浦『解析入門』解答集}{数学書解答集作成班}{}{}{\today}


\thispagestyle{empty}

\newpage
\pagenumbering{arabic}
\pagecolor{white}

\begin{multicols}{3}
    \tableofcontents
\end{multicols}

\newpage

\part*{はじめに}
\addcontentsline{toc}{part}{\texorpdfstring{はじめに}{はじめに}}


\section*{概要}
\addcontentsline{toc}{section}{\texorpdfstring{概要}{概要}}

この文書は,なまちゃんが運営する「数学書解答集作成班」が制作した,杉浦光夫著『解析入門』(東京大学出版会)の解答集である.

未完ではあるものの,編集の際の利便性を考慮して,オープンソースでの公開となった.それゆえ,数学的な誤りや誤植,改善案の提案などがあればぜひIssueに書き込んだり,Pull Requestを送っていただきたい\footnote{永遠の工事中}.


\section*{Special Thanks}
\addcontentsline{toc}{section}{\texorpdfstring{Special Thanks}{Special Thanks}}

掲載許可を得た方のみ\footnote{掲載されていないという方は「ニックネーム」を記入のもと,なまちゃんへ連絡していただきたい.}を敬称略で掲載する.
\begin{itemize}
    \item ねたんほ(解答の提供)
    \item まっちゃん(解答の提供)
    \item Daiki(解答の提供)
    \item qwer(解答の提供)
    \item やたろう(Git管理)
    \item 不自然対数(\LaTeX 関連)
\end{itemize}

その他,多くの方々.

\newpage
\part*{第1章:実数と連続}
\addcontentsline{toc}{part}{\texorpdfstring{第1章:実数と連続}{第1章:実数と連続}}

\section*{p2:1}
\addcontentsline{toc}{section}{\texorpdfstring{p2:1}{p2:1}}


\subsection*{p2:1-(i)}
\addcontentsline{toc}{subsection}{\texorpdfstring{p2:1-(i)}{p2:1-(i)}}

\begin{leftbar}
    \begin{proof}
        $0,0' \in K$がともに加法単位元の性質を満たすとする.

        このとき,$0$が加法単位元の性質をもつことから,
        \[
            0'+0=0'
        \]
        同様に,$0'$が加法単位元の性質をもつことから,
        \[
            0+0' = 0
        \]
        交換律より,$0+0'=0'+0$なので,
        \[
            0'=0'+0 =0+0' =0
        \]
        これからただちに加法単位元の一意性が従う.
    \end{proof}
\end{leftbar}

\subsection*{p2:1-(ii)}
\addcontentsline{toc}{subsection}{\texorpdfstring{p2:1-(ii)}{p2:問1-(ii)}}

\begin{leftbar}
    \begin{proof}
        $a ,b \in K$とし,
        \[
            a+b =0
        \]
        とする.このとき,
        \[
            -a = -a+0 = -a +(a+b)=(-a+a)+b =0+b = b
        \]
        となり,加法逆元の一意性が従う.
    \end{proof}
\end{leftbar}

\subsection*{p2:1-(iii)}
\addcontentsline{toc}{subsection}{\texorpdfstring{p2:1-(iii)}{p2:1-(iii)}}

\begin{leftbar}
    \begin{proof}
        $a \in K$のとき,
        \begin{gather*}
            a+(-a)=0 \\
            \therefore (-a)+a =0
        \end{gather*}
        他方,$-(-a)$は$(-a)$の加法逆元であるから,
        \[
            (-a)+(-(-a))=0
        \]
        これと逆元の一意性により,$a=-(-a)$が従う.
    \end{proof}
\end{leftbar}

\subsection*{p2:1-(v)}
\addcontentsline{toc}{subsection}{\texorpdfstring{p2:1-(v)}{p2:1-(v)}}
\begin{leftbar}
    \begin{proof}
        $a \in K$に対して,
        \[
            a+(-1)a=(1+(-1))a =0a =0
        \]
        であるから,$-a$が$a$の加法逆元であることと含めて主張が従う.
    \end{proof}
\end{leftbar}

\newpage

\subsection*{p2:1-(vi)}
\addcontentsline{toc}{subsection}{\texorpdfstring{p2:1-(vi)}{p2:1-(vi)}}
\begin{leftbar}
    \begin{proof}
        (4)の結果を用いる.$a=-1$とすると,
        \[
            (-1)(-1)=-(-1)=1
        \]
        これが証明すべきことであった.
    \end{proof}
\end{leftbar}
\subsection*{p2:1-(vii)}
\addcontentsline{toc}{subsection}{\texorpdfstring{p2:1-(vii)}{p2:1-(vii)}}

\begin{leftbar}
    \begin{proof}
        $a,b \in K$に対して,
        \begin{align*}
            a(-b)+ab         & = a((-b)+b) \\
                             & = a0        \\
                             & =0          \\
            \therefore \quad & a(-b)=-ab
        \end{align*}
        $(-a)b = -ab$も同様にして示される.
    \end{proof}
\end{leftbar}

\subsection*{p2:1-(viii)}
\addcontentsline{toc}{subsection}{\texorpdfstring{p2:1-(viii)}{p2:1-(viii)}}

\begin{leftbar}
    \begin{proof}
        (7)の前半の式において,$a$を$-a$に置き換えると,
        \[
            (-a)(-b) = -(-a)b.
        \]
        ここで,(7)の後半の式を用いると,
        \begin{align*}
            -(-a)b & = -(-ab) \\
                   & = ab.
        \end{align*}
        これが証明すべきことであった.
    \end{proof}
\end{leftbar}


\subsection*{p2:1-(ix)}
\addcontentsline{toc}{subsection}{\texorpdfstring{p2:1-(ix)}{p2:1-(ix)}}

\begin{leftbar}
    \begin{proof}
        $ a, b \in K$をとり,$ab =0$を仮定すると,
        \begin{align*}
             & ab + ab = ab            \\
             & \therefore ~ a(b+b)=ab.
        \end{align*}
        ここで,$a \ne 0$を仮定すると.$a^{-1}$が存在するので,
        $a^{-1}$を上記の式の両辺に左から掛けると
        \begin{align*}
             & a^{-1}a(b+b)=a^{-1}ab \\
             & \therefore ~ b+b=b    \\
             & \therefore ~ b=0.
        \end{align*}
        よって$a \ne 0$のとき$b=0$である.

        また,$b \ne 0$を仮定して同様に議論を進めると,
        $b \ne 0$であるとき$ a=0$であることがわかる.

        以上の議論から,$ab=0$ならば$a=0$または$b=0$であることが示された.
    \end{proof}
\end{leftbar}

\subsection*{p2:1-(x)}
\addcontentsline{toc}{subsection}{\texorpdfstring{p2:1-(x)}{p2:1-(x)}}

\begin{leftbar}
    \begin{proof}
        $ a \in K \setminus \{ 0 \} $に関して,
        \begin{align*}
            (-a) \{ -(a)^{-1} \} & = aa^{-1} \\
                                 & =1.
        \end{align*}
        よって,$-(a)^{-1}$は$-a$の逆元である.
        このことから,
        \[
            (-a)^{-1} = -(a)^{-1}.
        \]
        これが証明すべきことであった.
    \end{proof}
\end{leftbar}
\subsection*{p2:1-(xi)}
\addcontentsline{toc}{subsection}{\texorpdfstring{p2:1-(xi)}{p2:1-(xi)}}
\begin{leftbar}
    \begin{proof}
        $ a ,b\in K \setminus \{ 0 \} $に関して,
        \begin{align*}
            (ab) (b^{-1} a^{-1}) & = a (bb^{-1}) a^{-1} \\
                                 & = a1a^{-1}           \\
                                 & = aa^{-1}            \\
                                 & = 1
        \end{align*}
        となり,$(ab)^{-1} = b^{-1} a^{-1}$である.
        これが証明すべきことであった.
    \end{proof}
\end{leftbar}
%
\newpage
\section*{p3:2}
\addcontentsline{toc}{section}{\texorpdfstring{p3:2}{p3:2}}

\subsection*{p2:2-(i)}
\addcontentsline{toc}{subsection}{\texorpdfstring{p3:2-(i)}{p3:2-(i)}}

\begin{leftbar}
    \begin{proof}
        $a \leqq b$において,(R15)より
        \[
            0 \leqq b-a
        \]
        を得る.

        逆に,$0 \leqq b-a$において,(R15)により
        \[
            a \leqq b
        \]
        を得る.

        以上の考察により証明された.
    \end{proof}
\end{leftbar}

\subsection*{p2:2-(ii)}
\addcontentsline{toc}{subsection}{\texorpdfstring{p3:2-(ii)}{p3:2-(ii)}}

\begin{leftbar}
    \begin{proof}
        $a \leqq b$において,(R15)により
        \[
            0 \leqq b-a
        \]
        となる.
        ここで,(R15)を適用して,
        \[
            -b \leqq -a
        \]
        を得る.

        逆に,$-b\leqq -a$について,(R15)より
        \[
            0 \leqq b -a
        \]
        となる.この両辺に$a$を加えて,
        \[
            a \leqq b
        \]
        を得る.

        以上の考察により証明された.
    \end{proof}
\end{leftbar}

\newpage

\subsection*{p2:2-(iii)}
\addcontentsline{toc}{subsection}{\texorpdfstring{p3:2-(iii)}{p3:2-(iii)}}

\begin{leftbar}
    \begin{proof}
        $c \leqq 0$から
        \begin{align*}
            c+(-c)         & \leqq -c \\
            \therefore ~ 0 & \leqq -c
        \end{align*}
        である.ここで,$a \leqq b$,$-c \geqq 0$であることより
        \[
            a(-c) \leqq b (-c)
        \]
        である.これより$ -ac \leqq -bc$であるから,
        \begin{align*}
            -ac + (ac+bc)   & \leqq -bc +(ac+bc) \\
            \therefore ~ bc & \leqq ac
        \end{align*}
        を得て,これが証明すべきことであった.
    \end{proof}
\end{leftbar}

\subsection*{p2:2-(iv)}
\addcontentsline{toc}{subsection}{\texorpdfstring{p3:2-(iv)}{p3:2-(iv)}}

\begin{leftbar}
    \begin{proof}
        $a<0$かつ$a^{-1} \leqq 0$であると仮定する.このとき,
        \[
            \left ( \frac{1}{a} \right) a < 0a
        \]
        なので,
        \[
            1<0
        \]
        となるが,これは$1>0$に矛盾.

        よって,仮定したことが誤りであり,$a>0$のとき$a^{-1} >0$である.
    \end{proof}
\end{leftbar}

\subsection*{p2:2-(v)}
\addcontentsline{toc}{subsection}{\texorpdfstring{p3:2-(v)}{p3:2-(v)}}

\begin{leftbar}
    \begin{proof}
        \[
            a \leqq c
        \]
        において,(R15)より.
        \[
            a+b \leqq b+c
        \]
        を得る.他方,
        \[
            b \leqq d
        \]
        において,(R15)より
        \[
            b + c \leqq c+d
        \]
        となる.ここで,推移律を適用すると,
        \[
            a+b \leqq c+d
        \]
        が得られ,これが証明すべきことであった.
    \end{proof}
\end{leftbar}
%
\section*{p16--17:1}
\addcontentsline{toc}{section}{\texorpdfstring{p16--17:1}{p16--17:1}}

\subsection*{p16--17:1-(i)}
\addcontentsline{toc}{subsection}{\texorpdfstring{p16--17:1-(i)}{p16--17:1-(i)}}

\begin{tleftbar}
    $a=0$のときは明らかに$0$に収束するので,$a \ne 0$とする.$2\abs{a} \le N$となる$N \in \mathbb{N}$をとる.このとき,
    \begin{align*}
        0 & < \abs{ \frac{a^n}{n!} }                                                                              \\
          & \le \frac{\abs{a}^n}{n!}                                                                              \\
          & = \frac{\abs{a}^{N}}{N!} \cdot \frac{\abs{a}}{N+1} \cdot \frac{\abs{a}}{N+2} \dotsm \frac{\abs{a}}{n} \\
          & \le  \frac{\abs{a}^{N}}{N!} \left(\frac{1}{2} \right)^{n-N}
    \end{align*}
    であるから,
    \[
        - \frac{\abs{a}^{N}}{N!} \left(\frac{1}{2} \right)^{n-N} \le  \frac{a^n}{n!} \le \frac{\abs{a}^{N}}{N!} \left(\frac{1}{2} \right)^{n-N}
    \]
    となり,はさみうちの原理により,
    \[
        \lim_{n \to \infty} \frac{a^n}{n!} =0
    \]
    である
\end{tleftbar}

\subsection*{p16--17:1-(ii)}
\addcontentsline{toc}{subsection}{\texorpdfstring{p16--17:1-(ii)}{p16--17:1-(ii)}}

\begin{tleftbar}
    $a=1$のときは明らかに$1$に収束するので,まず$a>1$のときを考える.$\delta_n >0$を用いて,
    \[
        \sqrt[n]{a} =1+\delta_n
    \]
    とおくことができる.両辺を$n$乗すると
    \begin{align*}
        a & = 1+ n \delta_n + \frac{1}{2}n(n-1) {\delta_n}^2 + \cdots + {\delta_n}^2 \\
          & >1+n \delta_n                                                            \\
          & >n \delta_n
    \end{align*}
    となり,$0<\delta_n <\frac{a}{n}$であるから,はさみうちの原理により,
    \[
        \lim_{n \to \infty} \delta_n =0
    \]
    となる.$a<1$のときは,$a^{\frac{1}{n}}=\left(\left(\frac{1}{a}\right)^{\frac{1}{n}}\right)^{-1}$を使えば同じ結果が得られ,以上の議論により,
    \[
        \lim_{n \to \infty} \sqrt[n]{a} =1
    \]
    となる.
\end{tleftbar}

\subsection*{p16--17:1-(iii)}
\addcontentsline{toc}{subsection}{\texorpdfstring{p16--17:1-(iii)}{p16--17:1-(iii)}}

\begin{tleftbar}
    $\frac{n}{2}$以下の最大の自然数を$m$とおく.与えられた式は,
    \[
        \left( 0  < \right) \frac{n!}{n^n}  = \frac{1 \cdot 2 \dotsm m \cdot (m+1) \dotsm n}{n^n}
    \]
    と表されるので,$\frac{(m+1) \cdot (m+2) \dotsm n}{n^{n-m}} <1$であることと,$m \le \frac{n}{2}$から$\frac{m}{n} \le \frac{1}{2}$であることを用いると,
    \[
        \frac{1 \cdot 2 \dotsm m \cdot (m+1) \dotsm n}{n^n} < \frac{1 \cdot 2 \dotsm m}{n^m} <\left(\frac{1}{2}\right)^m
    \]
    よって,
    \[
        0 < \frac{n!}{n^n} <\left(\frac{1}{2}\right)^m
    \]
    である.$n \to \infty$のとき$m \to \infty$なので,はさみうちの原理により,
    \[
        \lim_{n \to \infty}\frac{n!}{n^n} =0
    \]
    である.
\end{tleftbar}
\subsection*{p16--17:1-(iv)}
\addcontentsline{toc}{subsection}{\texorpdfstring{p16--17:1-(iv)}{p16--17:1-(iv)}}
\begin{tleftbar}
    のちに$n \to \infty$の極限を考えることを考慮すると,
    \begin{align*}
        2^n & = (1+1)^n                            \\
        =   & 1+n +\frac{1}{2} n(n-1)+ \cdots +n+1 \\
            & > \frac{1}{2} n(n-1)
    \end{align*}
    となり,この不等式から,
    \[
        0< \frac{n}{2^n} < \frac{2}{n-1}
    \]
    を得る.ここで,はさみうちの原理により,
    \[
        \lim_{n \to \infty} \frac{n}{2^n}=0
    \]
    である.
\end{tleftbar}

\subsection*{p16--17:1-(v)}
\addcontentsline{toc}{subsection}{\texorpdfstring{p16--17:1-(v)}{p16--17:1-(v)}}
\begin{tleftbar}
    まず,
    \[
        0<\sqrt{n+1} - \sqrt{n} = \frac{1}{\sqrt{n+1} + \sqrt{n}}
    \]
    である.ここで,のちに$n \to \infty$の極限を考えることを考慮すると,
    \[
        \frac{1}{\sqrt{n+1} + \sqrt{n}} < \frac{1}{\sqrt{n}}
    \]
    であり,
    \[
        0< \sqrt{n+1} - \sqrt{n} <\frac{1}{\sqrt{n}}
    \]
    を得る.ここで,はさみうちの原理を用いると,
    \[
        \lim_{n \to \infty} (\sqrt{n+1} - \sqrt{n} )=0
    \]
    である.
\end{tleftbar}

\section*{p16--17:2}
\addcontentsline{toc}{section}{\texorpdfstring{p16--17:2}{p16--17:2}}

$n=1,2,\ldots$に対して,
\[
    f_{n} (x)=\lim_{m \to \infty} (\cos (n! \pi x)) ^{2m}
\]
とおく.
ここで,$n!x \in \mathbb{Z}$のとき,
\[
    \cos (n! \pi x)=\pm 1
\]
$n!x \notin \mathbb{Z}$のときは,
\[
    \abs{\cos (n! \pi x)}<1
\]
であるから,
\[
    f_{n} (x)=
    \begin{cases}
        1 & (n!x \in \mathbb{Z})    \\
        0 & (n!x \notin \mathbb{Z})
    \end{cases}
\]
となる.
さて,$x \in \mathbb{R} \setminus \mathbb{Q}$であるならば,どんな$n \in \mathbb{N}$に対しても,$n! x$が整数とならない.
また,$x \in \mathbb{Q}$のとき,$ x=\frac{p}{q}(p,q \in \mathbb{Z},q>0)$とすれば,$n$が$q$より十分大きいときに$n!x$は整数となる.
よって,
\[
    \lim_{n \to \infty} \left( \lim_{m \to \infty} (\cos (n! \pi x)) ^{2m} \right)=
    \begin{cases}
        1 & (x \in \mathbb{Q})                      \\
        0 & (x \in \mathbb{R} \setminus \mathbb{Q})
    \end{cases}
\]


\section*{p16--17:3}
\addcontentsline{toc}{section}{\texorpdfstring{p16--17:3}{p16--17:3}}

\kakko{補題}

任意の$a_1 , a_2 , \dots, a_n \in \mathbb{R}$について,
\[
    \abs{a_1+a_2+\dots+a_n} \leqq \abs{a_1}+\abs{a_2}+\dots+\abs{a_n}
\]
が成り立つ.


\begin{proof}
    $n=2$のときは三角不等式そのものであるから,
    $n \geqq 3$とし,$n-1$個の実数については補題の主張が成り立つものとする.

    いま,
    \[
        a_1 + a_2 + \dots + a_n = (a_1+a_2+\dots+a_{n-1})+a_n
    \]
    であるから,これに三角不等式を適用して,
    \[
        \abs{a_1+a_2+\dots+a_n} \leqq \abs{a_1+a_2+\dots+a_{n-1}} +\abs{a_n}
    \]
    を得る.ここで,数学的帰納法の仮定より,
    \[
        \abs{a_1+a_2+\dots+a_{n-1}} \leqq \abs{a_1}+\abs{a_2}+\dots+\abs{a_{n-1}}
    \]
    がいえるので,ここまでの議論で,
    \[
        \abs{a_1+a_2+\dots+a_n} \leqq \abs{a_1}+\abs{a_2}+\dots+\abs{a_n}
    \]
    は$n$の場合にも成り立つことが示され.以上の議論より補題の主張が従う.
\end{proof}

\begin{tleftbar}
    \begin{proof}
        $\lim_{n \to \infty} a_n =a$であるから,
        任意の$\varepsilon >0$に対して,ある$N_1 \in \mathbb{N}$が存在して,任意の$n \in \mathbb{N}$に対して,
        \[
            n \ge N_1 \Longrightarrow \abs{a_n -a}<\varepsilon
        \]
        となる.

        また,
        \[
            \abs{\frac{a_1+a_2+\cdots+a_n}{n}-a}= \abs{\frac{(a_1-a)+(a_2-a)+\cdots+(a_n-a)}{n}}
        \]
        と変形する.この右辺に補題を適用し,
        \[
            \abs{\frac{(a_1-a)+(a_2-a)+\cdots+(a_n-a)}{n}} \leqq \frac{\abs{a_1-a}+\abs{a_2-a}+\cdots+\abs{a_n-a}}{n}
        \]
        を得る.これにより,
        \[
            n \ge N_1 \Longrightarrow \frac{\abs{a_1-a}+\abs{a_2-a}+\cdots+\abs{a_{N_1-1}-a}}{n} +\left( \frac{n-N_1+1}{n} \right ) \varepsilon
        \]
        となる.ここで$N_2 \coloneqq N_1 +1$とすると,$n \ge N_2$であるとき$\left( \frac{n-N_1+1}{n} \right ) \varepsilon < \varepsilon$となることに注意する.

        さて,
        \[
            n \ge N_3 \Longrightarrow \frac{\abs{a_1-a}+\abs{a_2-a}+\cdots+\abs{a_{N_1-1}-a}}{n}<\varepsilon
        \]
        となるように$N_3 \in \mathbb{N}$をとる.$N \coloneqq \max \{ N_2 , N_3 \}$とすると,
        \[
            n \ge N \Longrightarrow \frac{\abs{a_1-a}+\abs{a_2-a}+\cdots+\abs{a_{N_1-1}-a}}{n} +\left( \frac{n-N_1+1}{n} \right ) \varepsilon < \varepsilon + \varepsilon = 2 \varepsilon
        \]
        であり,これより
        \[
            n \ge N \Longrightarrow \abs{\frac{a_1+a_2+\cdots+a_n}{n}-a} < 2 \varepsilon
        \]
        となる.書き換えると.
        \[
            \lim_{n \to \infty} \frac{a_1+a_2+\cdots+a_n}{n}=a
        \]
        となり,これが証明すべきことであった.
    \end{proof}
\end{tleftbar}

\newpage

\section*{p16--17:5}
\addcontentsline{toc}{section}{\texorpdfstring{p16--17:5}{p16--17:5}}

\begin{tleftbar}
    \begin{proof}
        イ)の条件により,
        \[
            A \subset \{n \in \mathbb{N} \mid n \geqq m\}
        \]
        は明らかである.ここで,
        \[
            H=\{0,1,\dots,m-1\} \cup A
        \]
        とおく.このとき,$H$の定義から$0 \in H$である.

        次に,$n \in H$であることを仮定する.このとき,
        \begin{enumerate}[(i)]
            \item  $n<m-1$であれば,$n+1 \in \{0,1,\dots,m-1\}$より,$n+1 \in H$となる
            \item $n=m-1$であれば,$n+1=m \in A$より,$n+1 \in H$となる
            \item $n \geqq m$であれば,ロ)より$n+1 \in A$であり,$A \subset H $により$n + 1\in H$となる
        \end{enumerate}
        よって,いずれの場合でも
        \[
            n+1 \in H
        \]
        したがって,$H$は継承的である.$\mathbb{N}$は最小の継承的な集合であるから,$\mathbb{N} \subset H$である.

        つまり,$\mathbb{N} =  \{0,1,\dots,m-1\} \cup \{n \in \mathbb{N} \mid n \geqq m \}$であることと,
        $H$の定義により,
        \[
            \{0,1,\dots,m-1\} \cup \{n \in \mathbb{N} \mid n \geqq m \} \subset \{0,1,,\dots,m-1\} \cup A
        \]
        となる.よって,
        \[
            \{n \in \mathbb{N}\mid n \geqq m \} \subset A
        \]
        となる.これと$ A \subset \{n \in \mathbb{N} \mid n \geqq m\}$であることを併せると,
        \[
            A=\{n \in \mathbb{N} \mid n \geqq m \}
        \]
        となり,これが証明すべきことであった.
    \end{proof}
\end{tleftbar}

\newpage

\section*{p16--17:6}
\addcontentsline{toc}{section}{\texorpdfstring{p16--17:6}{p16--17:6}}

\kakko{$m+n$について}

\begin{tleftbar}
    \begin{proof}
        $n \in \mathbb{N}$についての命題$p(n)$を
        \[
            p(n) ~{:}~ \forall m \in \mathbb{N} \colon  m+n \in \mathbb{N}
        \]
        とする.
        \begin{enumerate}[(i)]
            \item $n=0$のとき,$m+0=m$であり,$m \in \mathbb{N}$であるから,$p(0)$は真である.
            \item 任意の$k \in \mathbb{N}$に対して,$p(k)$が真であると仮定する.このとき,$m+k \in \mathbb{N}$である.
                  ここで,
                  \[
                      m+(k+1)=(m+k)+1
                  \] であり,$m+k \in \mathbb{N}$に注意すると,$\mathbb{N}$が最小の継承的集合であることから,
                  \[
                      m+(k+1)=(m+k)+1  \in \mathbb{N}
                  \]
                  である.よって,このとき$p(k+1)$も真である.
        \end{enumerate}
        以上(i),(ii)より,任意の$n \in \mathbb{N}$に対して$p(n)$が真である.これが証明すべきことであった.
    \end{proof}
\end{tleftbar}


\kakko{$mn$について}

\begin{tleftbar}
    \begin{proof}
        $n \in \mathbb{N}$についての命題$q(n)$を
        \[
            q(n) ~{:}~ \forall m \in \mathbb{N} \colon  mn \in \mathbb{N}
        \]
        とする.
        \begin{enumerate}[(i)]
            \item $n=0$のとき,$m \cdot 0=0$であり,$0 \in \mathbb{N}$であるから,$p(0)$は真である.
            \item 任意の$k \in \mathbb{N}$に対して,$p(k)$が真であると仮定する.このとき,$mk \in \mathbb{N}$である.
                  ここで,
                  \[
                      m(k+1)=mk + k
                  \] であり,$mk \in \mathbb{N}$,$k \in \mathbb{N}$に注意すると,証明したことから
                  \[
                      mk + k   \in \mathbb{N}
                  \]
                  である.よって,このとき$p(k+1)$も真である.
        \end{enumerate}
        以上(i),(ii)より,任意の$n \in \mathbb{N}$に対して$p(n)$が真である.これが証明すべきことであった.
    \end{proof}
\end{tleftbar}

\newpage

\kakko{$m-n$について}

\begin{tleftbar}
    \begin{proof}
        $n \in \mathbb{N}$についての命題$r(n)$を
        \[
            r(n) ~{:}~ \forall m \in \mathbb{N} \colon  m<n \lor m-n  \in \mathbb{N}
        \]
        とする.
        \begin{enumerate}[(i)]
            \item $n=0$のとき,$m - 0 =m$であり,$m \in \mathbb{N}$であるから,$r(0)$は真である.
            \item 任意の$k \in \mathbb{N}$に対して,$r(k)$が真であると仮定する.このとき,$k<n \lor k-n  \in \mathbb{N}$である.
                  \begin{enumerate}[(I)]
                      \item $m \leqq   k$のとき,$m <k+1$であり,$r(k+1)$も真である.
                      \item $m > k$のとき,まず
                            \[
                                m-(k+1)=m-k-1
                            \] であり,$m-k \in \mathbb{N}$,$1 \in \mathbb{N}$に注意すると,証明したことから
                            \[
                                m-k-1  \in \mathbb{N}
                            \]
                            である.
                  \end{enumerate}
                  よって,このとき$r(k+1)$も真である.
        \end{enumerate}
        以上(i),(ii)より,任意の$n \in \mathbb{N}$に対して$r(n)$が真である.これが証明すべきことであった.
    \end{proof}
\end{tleftbar}

\newpage

\section*{p16--17:7}
\addcontentsline{toc}{section}{\texorpdfstring{p16--17:7}{p16--17:7}}

\begin{tleftbar}
    \begin{proof}
        $n < k < n+1$となる自然数$k$が存在すると仮定する.
        この不等式から,辺々$n$を引くと
        \[
            0 < k - n < 1
        \]
        を得る.$n$と$k$は自然数としたから,問6により,$k-n$は自然数である.

        よって,$0<a<1$となる自然数$a$が存在しないことを示せばよい.
        \[
            H=\{0\} \cup \{n \in \mathbb{N} \mid n \geqq 1 \}
        \]
        とおくと,
        \begin{align*}
             & 0 \in H   \\
             & 0+1 \in H
        \end{align*}
        である.ゆえに$ 0 \in H$となる.

        また,$k \geqq 1$となる$k \in \mathbb{N}$に対しては,$k +1 \geqq 1$
        であり,
        \[
            k+1 \in \{n \in \mathbb{N} \mid n \geqq 1 \}
        \]
        となるため,$k+1 \in H$である.

        ここまでの考察から,
        \begin{enumerate}
            \item $0 \in H$
            \item $n \in H \Longrightarrow n+1 \in H$
        \end{enumerate}
        であるから$H$は継承的である.

        したがって,$\mathbb{N}$が最小の継承的集合であることから,
        $\mathbb{N} \subset H$となる.

        $H$の定義により,$a \notin H$なので,$a \notin \mathbb{N}$となり,$0<a<1$となる自然数$a$は存在しない.
        これが証明すべきことであった.
    \end{proof}
\end{tleftbar}

\newpage

\section*{p31--33:1}
\addcontentsline{toc}{section}{\texorpdfstring{p31--33:1}{p31--33:1}}


\subsection*{p31--33:1-(i)}
\addcontentsline{toc}{subsection}{\texorpdfstring{p31--33:1-(i)}{p31--33:1-(i)}}

\begin{tleftbar}
    \begin{align*}
        \frac{1^2+2^2+\cdots+n^2}{n^3} & = \frac{\dfrac{1}{6}n(n+1)(2n+1)}{n^3}                                \\
                                       & =\frac{1}{6} \left(1+\frac{1}{n} \right ) \left(2+\frac{1}{n} \right)
    \end{align*}
    $a_n = \frac{1}{6} \left(1+\frac{1}{n} \right ),~b_n = \left(2+\frac{1}{n} \right)$とおくと,$(a_n)_{n \in \mathbb{N}},~(b_n)_{n \in \mathbb{N}}$は明らかに収束するから,定理2.5(2)より,
    \[
        \lim_{n \to \infty} a_n b_n = \lim_{n \to \infty} a_n \cdot  \lim_{n \to \infty} b_n
    \]
    である.これより,
    \begin{align*}
        \lim_{n \to \infty} \frac{1^2+2^2+\cdots+n^2}{n^3} & = \lim_{n \to \infty} \frac{1}{6} \left(1+\frac{1}{n} \right ) \left(2+\frac{1}{n} \right)                                                                \\
                                                           & = \left \{\lim_{n \to \infty} \frac{1}{6} \left(1+\frac{1}{n} \right ) \right \} \cdot \left \{\lim_{n \to \infty} \left(2+\frac{1}{n} \right ) \right \} \\
                                                           & = \frac{1}{6} (1+0) \cdot (2+0) =\frac{1}{3}
    \end{align*}
\end{tleftbar}


\subsection*{p31--33:1-(ii)}
\addcontentsline{toc}{subsection}{\texorpdfstring{p31--33:1-(ii)}{p31--33:1-(ii)}}

\kakko{補題}


正数列$(a_n)_{n \in \mathbb{N}}$に対して,$\left(\frac{a_{n+1}}{a_n} \right)_{n \in \mathbb{N}}$が収束し,
\[
    \lim_{n \to \infty} \frac{a_{n+1}}{a_n} <1
\]
となるとき,$\lim_{n \to \infty} a_n =0$である.

\begin{proof}
    $ \lim_{n \to \infty} \frac{a_{n+1}}{a_n} <1$であるから,この左辺を$r_0$とおくと,$r_0<1$である.
    このとき,$r~(r_0<r<1)$に対して,ある$N_1 \in \mathbb{N}$が存在して,任意の$n \in \mathbb{N}$に対して,
    \[
        n \ge N_1 \Longrightarrow \frac{a_{n+1}}{a_n}<r
    \]
    が成り立つ.このとき,
    \[
        a_n = a_{N_1} \cdot \frac{a_{N_1+1}}{a_{N_1}} \cdot \frac{a_{N_1 +2}}{a_{N_1 +1}} \dotsm \frac{a_{n-1}}{a_{n-2}} \frac{a_n}{a_{n-1}}< a_{N_1} r^{n-N_1}=\frac{a_{N_1}}{r^{N_1}} r^n
    \]
    となる.$0<r<1$より$\lim_{n \to \infty} \frac{a_{N_1}}{r^{N_1}} r^n =0$であるから,$\lim_{n \to \infty} a_n =0$である.
\end{proof}


\begin{tleftbar}
    $a_n = \frac{n^2}{a^n}$とおく.$0<a \le 1$のときは明らかに$\lim_{n \to \infty} a_n=\infty$となる.\par
    $a>1$のとき,$\frac{a_{n+1}}{a_n} =\frac{\left(1+\dfrac{1}{n}\right)^2}{a}$となり,
    \[
        \lim_{n \to \infty} \frac{\left(1+\dfrac{1}{n}\right)^2}{a} = \frac{1}{a} <1
    \]
    であるから,補題により,$\lim_{n \to \infty} a_n =0$となる.以上の議論により,
    \begin{align*}
        \lim_{n \to \infty} \frac{n^2}{a^n}
        =
        \begin{cases}
            \infty & (0<a \le 1) \\
            0      & (a>1)
        \end{cases}
    \end{align*}
    となる.
\end{tleftbar}
\subsection*{p31--33:1-(iii)}
\addcontentsline{toc}{subsection}{\texorpdfstring{p31--33:1-(iii)}{p31--33:1-(iii)}}
\begin{tleftbar}
    明らかに$\sqrt[n]{n} >1$なので,$\delta_n >0$を用いて,
    \[
        \sqrt[n]{n} = 1+ \delta_n
    \]
    とかける.両辺を$n$乗して,$n \to \infty$の極限を考えることを考慮すると,
    \begin{align*}
        n = (1+\delta_n)^n & =1 + n \delta_n + \frac{1}{2}n(n-1) {\delta_n}^2 + \cdots + (\delta_n)^n \\
                           & > \frac{1}{2}n(n-1) {\delta_n}^2
    \end{align*}
    となり,この不等式から,
    \[
        0<\delta_n < \sqrt{\frac{2}{n-1}}
    \]
    を得る.ここで,はさみうちの原理により,$\lim_{n \to \infty} \delta_n =0$であるから,
    \[
        \lim_{n \to \infty} \sqrt[n]{n} =1
    \]
    である.
\end{tleftbar}

\subsection*{p31--33:1-(iv)}
\addcontentsline{toc}{subsection}{\texorpdfstring{p31--33:1-(iv)}{p31--33:1-(iv)}}
\begin{tleftbar}
    $a_n= n^k e^{-n}$とおく.このとき,
    \[
        \lim_{n \to \infty} \frac{a_{n+1}}{a_n} =  \lim_{n \to \infty} \frac{\left(1+\dfrac{1}{n}\right)^k}{e} =\frac{1}{e} <1
    \]
    ゆえに,補題により,
    \[
        \lim_{n \to \infty} n^k e^{-n}=0
    \]
    である.
\end{tleftbar}

\subsection*{p31--33:1-(v)}
\addcontentsline{toc}{subsection}{\texorpdfstring{p31--33:1-(v)}{p31--33:1-(v)}}
\begin{tleftbar}
    $a_n =\left (1-\frac{1}{n^2}\right)^n$とおく.
    \begin{align*}
        \lim_{n \to \infty} a_n & =\lim_{n \to \infty} \left (1-\frac{1}{n^2}\right)^n                              \\
                                & = \lim_{n \to \infty} \left (1+\frac{1}{n}\right)^n \left (1-\frac{1}{n}\right)^n \\
                                & = e \cdot \frac{1}{e} =1
    \end{align*}
    である.
\end{tleftbar}

\subsection*{p31--33:1-(vi)}
\addcontentsline{toc}{subsection}{\texorpdfstring{p31--33:1-(vi)}{p31--33:1-(vi)}}
\kakko{補題}


$(a_n)_{n \in \mathbb{N}}$を実数列とし,$a_n > 0$とする.もし$\lim_{n \to \infty} a_n =0$であれば$\lim_{n \to \infty} \frac{1}{a_n}=\infty$である.
また,もし$\lim_{n \to \infty} a_n =\infty$であるならば,$\lim_{n \to \infty} \frac{1}{a_n} =0$である,



\begin{proof}
    前半の主張のみ示せば後半の主張も同様に示せるので,前半のみ示す.

    $M>0$を任意にとる.$1/M = \varepsilon$とする.仮定により,$n_0 \in \mathbb{N}$が存在して,任意の$n \in \mathbb{N}$に対して,
    \[
        n \geqq n_0 \Longrightarrow \abs{a_n-0}<\varepsilon
    \]
    が成り立つ.このとき,上の$n_0 \in \mathbb{N}$に対して,
    \[
        n \geqq n_0 \Longrightarrow \frac{1}{a_n} >\frac{1}{\varepsilon}=M
    \]
    となる.$M$は任意なので,これより$\lim_{n \to \infty} \frac{1}{a_n}=\infty$が示された.
\end{proof}

\kakko{補題}


$c>1$のとき,$\lim_{n \to \infty} \frac{1}{c^n} = 0$である.

\begin{proof}
    $c>1$より,$\delta >0$を用いて$c=1+\delta$とおける.このとき,のちに$n \to \infty$の極限を考えることを考慮すると,
    \begin{align*}
        c^n & = (1+\delta)^n                                               \\
            & = 1+n \delta +\frac{1}{2}n (n-1) \delta^2 + \cdots +\delta^n \\
            & > 1+n \delta
    \end{align*}
    このことから,$0<\frac{1}{c^n} <\frac{1}{1+n\delta}$であるから,はさみうちの原理により,
    \[
        \lim_{n \to \infty} \frac{1}{c^n} = 0~(c>1)
    \]
    である.
\end{proof}

\begin{tleftbar}
    $a_n = (c^n +c^{-n})^{-1}$とおく.$c=1$のときは明らかに$\lim_{n \to \infty} a_n =\frac{1}{2}$である.

    $c>1$のとき,2つの補題により,
    \[
        \lim_{n \to \infty} (c^n + c^{-n}) = \infty
    \]
    であるから,補題により$\lim_{n \to \infty} a_n = \lim_{n \to \infty} (c^n +c^{-n})^{-1} =0$である.

    $0<c<1$のときは$c$の逆数を考えることにより同じ結論に帰着する.以上の議論により,
    \begin{align*}
        \lim_{n \to \infty} (c^n +c^{-n})^{-1} =
        \begin{cases}
            \frac{1}{2} & (c=1)     \\
            0           & (c \ne 1)
        \end{cases}
    \end{align*}
    である.
\end{tleftbar}


\section*{p31--33:2}
\addcontentsline{toc}{section}{\texorpdfstring{p31--33:2}{p31--33:2}}

\begin{tleftbar}
    \begin{proof}
        二項定理を用いて$(a_n)_{n \in \mathbb{N}}$の一般項を展開すると,
        \begin{align*}
            a_n & =  1 + n \cdot \frac{1}{n} + \frac{n(n-1)}{2!} \cdot \frac{1}{n^2} + \dots + \frac{n(n-1)\cdots(n-r+1)}{r!} \cdot \frac{1}{n^r} + \cdots \frac{n!}{n!} \cdot + \frac{1}{n^n}                                                                              \\
                & = 1+ \frac{1}{1!} + \frac{1}{2!} \left(1- \frac{1}{n} \right) + \dots + \frac{1}{r!} \cdot  \left(1 - \frac{1}{n} \right) \dots \left (1-\frac{r-1}{n} \right) + \dots +  \frac{1}{n!} \left(1 - \frac{1}{n} \right) \dots \left(1- \frac{n-1}{n} \right)
        \end{align*}
        同様にして,$a_{n+1}$の展開式を得たとき,$ \frac{1}{n+1} < \frac{1}{n}$であることにより,$r\in \{ 1,2,\dots ,n\}$に対して,
        \[
            \frac{1}{r!} \cdot  \left(1 - \frac{1}{n} \right) \dots \left (1-\frac{r-1}{n} \right) < \frac{1}{r!} \cdot  \left(1 - \frac{1}{n+1} \right) \dots \left (1-\frac{r-1}{n+1} \right)
        \]
        が成立する.これと,$a_{n+1}$の展開式のほうが,正の項を一つ多く含むことから,任意の$n \in \mathbb{N}$に対して,
        \[
            a_{n} < a_{n+1}
        \]
        が成立し,$(a_n)_{n \in \mathbb{N}}$は単調増加数列である.また,
        \begin{align*}
            a_n
             & < 1 + \frac{1}{1!} + \frac{1}{2!} + \cdots + \frac{1}{n!}                                                         \\
             & = \frac{5}{2} + \frac{1}{2}\left(\frac{1}{3} + \frac{1}{3 \cdot 4} + \cdots + \frac{1}{3 \cdot 4 \cdots n}\right) \\
             & < \frac{5}{2} + \frac{1}{2}\left(\frac{1}{3} + \frac{1}{3^2} + \cdots + \frac{1}{3^{n-2}}\right)                  \\
             & = \frac{5}{2} + \frac{1}{4}\left(1 - \frac{1}{3^{n-2}}\right)                                                     \\
             & < \frac{11}{4}.
        \end{align*}
        よって,$a_n$は単調増加かつ上に有界であるから収束する.
        $e = \lim_{n \to \infty} a_n$と定義すると,前述した評価により$e \leqq \frac{11}{4} < 3$である.

        また,$(a_n)_{n \in \mathbb{N}}$が単調増加数列であることから,任意の$n \in \mathbb{N} \setminus \{0,1\}$に対して,
        \[
            a_n > a_1 = \left(1+\frac{1}{1}\right)^1 =2
        \]
        であるから,
        \[
            2<e<3
        \]
        を得る.これが証明すべきことであった.
    \end{proof}
\end{tleftbar}
\newpage


\section*{p31--33:11}
\addcontentsline{toc}{section}{\texorpdfstring{p31--33:11}{p31--33:11}}


\begin{leftbar}
    \begin{proof}
        背理法で示す.

        ある$ a\in \mathbb{Q}$と$\varepsilon >0$に対して,$\abs{b-a}<\varepsilon$となるような$b \in \mathbb{R} \setminus \mathbb{Q}$が存在しない,つまり,
        ある$a \in \mathbb{Q}$と$\varepsilon >0$に対して,区間$(a-\varepsilon , a+ \varepsilon)$が有理数のみから成るとする.
        このとき
        \[
            (a-\varepsilon , a+ \varepsilon) \subset \mathbb{Q}
        \]
        である.

        さて,任意の$ c \in \mathbb{Q}$に対して,$ (c-\varepsilon , c+ \varepsilon)$が不可算集合であることから,とくに$(a-\varepsilon , a+ \varepsilon)$も不可算集合である.
        しかし,$(a-\varepsilon , a+ \varepsilon) \subset \mathbb{Q}$であるから,可算集合の部分集合が不可算集合であることになり矛盾である.

        ゆえに,先の仮定が誤りであるから,任意の$ a\in \mathbb{Q}$と任意の$\varepsilon >0$に対して,$\abs{b-a}<\varepsilon$となるような$b \in \mathbb{R} \setminus \mathbb{Q}$が存在する.
    \end{proof}
\end{leftbar}

\newpage
\section*{p42--43:3}
\addcontentsline{toc}{section}{\texorpdfstring{p42--43:3}{p42--43:3}}


\begin{tleftbar}
    \begin{proof}
        正規性と直交性をそれぞれ証明する.
        \begin{description}
            \item[正規性] $(u_1, \ldots, u_n)$の各ベクトルのノルムが$1$であることを示す.
                  \[
                      u_i = \frac{y_i}{\abs{y_i}} \quad (i = 1, \ldots, n)
                  \]
                  であるから,正規性は明らかである.
            \item[直交性] $(u_1, \ldots, u_n)$の各ベクトルが互いに直交することを数学的帰納法で証明する.
                  \begin{enumerate}[(i)]
                      \item $u_1$と$u_2$について,
                            \begin{align*}
                                (u_2 | u_1) & = \left(\left. \frac{x_2 - (x_2 | u_1) u_1}{\abs{ x_2 - (x_2 | u_1) u_1 }} \right|u_1 \right) \\
                                            & = \frac{(x_2 | u_1) - (x_2 | u_1) \|u_1\|^2}{\abs{ x_2 - (x_2 | u_1) u_1 }}                   \\
                                            & = \frac{(x_2 | u_1) - (x_2 | u_1) \cdot 1}{\abs{ x_2 - (x_2 | u_1) u_1 }}                     \\
                                            & = 0.
                            \end{align*}
                            したがって.$u_1$と$u_2$は直交する.

                      \item $u_1, u_2, \ldots, u_i$が互いに直交するベクトルであると仮定する.すなわち,
                            \[
                                (u_i | u_j) = 0 \quad (i \ne j)
                            \]
                            を仮定する.このとき,$u_j$と$u_{i+1}$\footnote{$u_{i+1}$は問題文で述べられている定義からわかる.}について,
                            \begin{align*}
                                (u_j | u_{i+1}) & =\left  ( u_j \left| \frac{x_{i+1} - \sum_{k=1}^{i} (x_{i+1} | u_k) u_k}{\abs{| x_{i+1} - \sum_{k=1}^{i} (x_{i+1} | u_k) u_k }} \right. \right) \\
                                                & = \frac{(u_j | x_{i+1}) - \sum_{k=1}^{i} (x_{i+1} | u_k) (u_j | u_k)}{\abs{| x_{i+1} - \sum_{k=1}^{i} (x_{i+1} | u_k) u_k }}                    \\
                                                & = \frac{(u_j | x_{i+1}) - (x_{i+1} | u_j) \abs{u_j}^2}{\abs{| x_{i+1} - \sum_{k=1}^{i} (x_{i+1} | u_k) u_k }}                                   \\
                                                & = \frac{(u_j | x_{i+1}) - (x_{i+1} | u_j) \cdot 1}{\abs{| x_{i+1} - \sum_{k=1}^{i} (x_{i+1} | u_k) u_k }}                                       \\
                                                & = 0.
                            \end{align*}
                  \end{enumerate}
                  (i)と(ii)より,直交性が示される.
        \end{description}
        以上の考察により,$(u_1, \ldots, u_n)$は$\mathbb{R}^n$の正規直交基底である.
    \end{proof}
\end{tleftbar}

\newpage

\section*{p42--43:11}
\addcontentsline{toc}{section}{\texorpdfstring{p42--43:11}{p42--43:11}}

\begin{leftbar}
    \begin{proof}
        四元数体 $H$ を以下のように定義する.

        \[
            H = \{ (a_1, a_2, a_3, a_4) \mid a_i \in \mathbb{R}\}
        \]

        また,

        \[
            p=(a_1, a_2, a_3, a_4),\quad q=(b_1, b_2, b_3, b_4)
        \]
        とする.このとき,$ p ,q \in H$である.
        \begin{description}
            \item [(R1)] \mbox{} \\
                  任意の $q = (a_i)$,$p = (b_i) \in H$ に対して,

                  \[
                      q + p = (a_1 + b_1, a_2 + b_2, a_3 + b_3, a_4 + b_4) = (b_1 + a_1, b_2 + a_2, b_3 + a_3, b_4 + a_4) = p + q
                  \]
            \item [(R2)] \mbox{} \\
                  任意の $q = (a_i)$,$p = (b_i)$,$r = (c_i) \in H$ に対して,実数の加法の結合律から,

                  \[
                      (q + p) + r = q + (p + r)
                  \]
            \item [(R3)] \mbox{} \\
                  加法の単位元を $0 = (0, 0, 0, 0)$ とすると,任意の $q = (a_i) \in H$ に対して,
                  \[
                      q + 0 = (a_1 + 0, a_2 + 0, a_3 + 0, a_4 + 0) = q
                  \]

            \item [(R4)] \mbox{} \\
                  任意の $q = (a_1, a_2, a_3, a_4) \in H$ に対して,加法の逆元を $-q = (-a_1, -a_2, -a_3, -a_4)$ と定義すると,
                  \[
                      q + (-q) = (a_1 - a_1, a_2 - a_2, a_3 - a_3, a_4 - a_4) = (0, 0, 0, 0) = 0
                  \]
            \item [(R6)] \mbox{} \\
                  任意の $q, p, r \in H$ に対して,
                  \[
                      (q \cdot p) \cdot r = q \cdot (p \cdot r)
                  \]
                  これは四元数の乗法の定義から成り立つ.
            \item[(R7)] \mbox{} \\
                  任意の $q, p, r \in H$ に対して,

                  \[
                      q \cdot (p + r) = q \cdot p + q \cdot r
                  \]

                  \[
                      (p + q) \cdot r = p \cdot r + q \cdot r
                  \]
                  が成立する.これは乗法の定義と実数の分配律から成り立つ.
            \item [(R8)] \mbox{} \\
                  乗法の単位元を $1 = (1, 0, 0, 0)$ とすると,任意の $q = (a_1, a_2, a_3, a_4) \in H$ に対して,

                  \[
                      q \cdot 1 = (a_1, a_2, a_3, a_4)
                  \]
                  なぜなら,
                  \begin{align*}
                      c_1 & = a_1 \cdot 1 - a_2 \cdot 0 - a_3 \cdot 0 - a_4 \cdot 0 = a_1, \\
                      c_2 & = a_1 \cdot 0 + a_2 \cdot 1 + a_3 \cdot 0 - a_4 \cdot 0 = a_2, \\
                      c_3 & = a_1 \cdot 0 - a_2 \cdot 0 + a_3 \cdot 1 + a_4 \cdot 0 = a_3, \\
                      c_4 & = a_1 \cdot 0 + a_2 \cdot 0 - a_3 \cdot 0 + a_4 \cdot 1 = a_4.
                  \end{align*}
            \item [(R9)] \mbox{} \\
                  任意の $q = (a_1, a_2, a_3, a_4) \in H$ で $q \ne 0$(つまり、${a_1}^2 + {a_2}^2 + {a_3}^2 + {a_4}^2 > 0$)の場合,ノルムを

                  \[
                      N(q) = {a_1}^2 + {a_2}^2 + {a_3}^2 + {a_4}^2 > 0
                  \]

                  と定義する.共役四元数を

                  \[
                      q^* = (a_1, -a_2, -a_3, -a_4)
                  \]

                  と定義すると,

                  \[
                      q \cdot q^* = q^* \cdot q = (N(q), 0, 0, 0)
                  \]

                  よって,乗法の逆元を

                  \[
                      q^{-1} = \frac{q^*}{N(q)} = \left( \frac{a_1}{N(q)},\, -\frac{a_2}{N(q)},\, -\frac{a_3}{N(q)},\, -\frac{a_4}{N(q)} \right)
                  \]
                  と定義すると,
                  \[
                      q \cdot q^{-1} = \left( \frac{N(q)}{N(q)}, 0, 0, 0 \right) = (1, 0, 0, 0) = 1
                  \]
            \item [(R10)]
                  加法の単位元は $0 = (0, 0, 0, 0)$ であり,乗法の単位元は $1 = (1, 0, 0, 0)$ である.ゆえに
                  \[
                      0 \ne 1.
                  \]
        \end{description}
        以上の考察により,四元数全体の集合 $H$ は(R5)を除く乗法の交換律を除くすべての体の公理をみたす.
    \end{proof}
\end{leftbar}
\newpage

\section*{p49--50:1}
\addcontentsline{toc}{section}{\texorpdfstring{p49--50:1}{p49--50:1}}

\begin{tleftbar}
    \begin{proof}
        $\lim_{n \to \infty} \sqrt[n]{a_n} =r$であるから,$0<r<1$であるとき,$r<k<1$となるような$k$をひとつ取ると,ある$n_1 \in \mathbb{N}$が存在して,任意の$n \in \mathbb{N}$に対して,
        \[
            n \ge n_1 \Longrightarrow a_n<k^n
        \]
        となる.ここで,定理5.5(比較判定法)により,$\sum a_n$は収束する.

        $r>1$のとき,$1>0$に対して,ある$n_2 \in \mathbb{N}$が存在して,任意の$n \in \mathbb{N}$に対して,
        \[
            n \ge n_2 \Longrightarrow a_n >1
        \]
        が成り立ち,$\lim_{n \to \infty} a_n \ne 0$となる.よって定理5.1~系の対偶により$\sum a_n$は発散する.
    \end{proof}
\end{tleftbar}


\section*{p49--50:2}
\addcontentsline{toc}{section}{\texorpdfstring{p49--50:2}{p49--50:2}}


\subsection*{p49--50:2-(i)}
\addcontentsline{toc}{subsection}{\texorpdfstring{p49--50:2-(i)}{p49--50:2-(i)}}

\begin{screen}
    \[
        \frac{2n^2}{n^3+1}=\frac{2/3}{n+1}+\frac{4n/3-2/3}{n^2-n+1}
    \]
    により
    \begin{align*}
        \sum \frac{2n^2}{n^3+1} & =\sum \frac{2/3}{n+1}+\sum \frac{4n/3-2/3}{n^2-n+1} \\
                                & >\sum \frac{2/3}{n+1} \rightarrow \infty
    \end{align*}
    となる.よってこの級数は発散する
\end{screen}


\subsection*{p49--50:2-(ii)}
\addcontentsline{toc}{subsection}{\texorpdfstring{p49--50:2-(ii)}{p49--50:2-(ii)}}

\begin{screen}
    \[
        \sum ^{\infty}_{n=1}\frac{\sqrt{n}}{1+n^2}<\sum \frac{\sqrt{n}}{n^2}=\sum^{\infty}_{n=1}\frac{1}{n^{3/2}}~\left(<1+\int^{\infty}_{1}\frac{dx}{x^{3/2}}=3\right)
    \]
    であるから,この級数は収束する.
    また
    \[
        \sum \frac{1}{n^\alpha}
    \]
    が$\alpha >1$のときに収束することを用いることもできる.
\end{screen}


\subsection*{p49--50:2-(iii)}
\addcontentsline{toc}{subsection}{\texorpdfstring{p49--50:2-(iii)}{p49--50:2-(iii)}}

\begin{screen}
    $a=1$のときは明らかに収束する.

    $a>1$の場合を考える.$\lim_{x \to 0} \frac{a^x-1}{x} = \log a$であるから
    \[
        \lim_{n \to \infty} n(a^{\frac{1}{n}}-1) = \log a
    \]
    が成り立つ.
    ここで$0 < \varepsilon <\log a$であるような$\varepsilon$をとる.極限の定義より,
    \[
        n \geqq N \Longrightarrow \log a - \varepsilon < n (a^\frac{1}{n}-1)
    \]
    となるような自然数$N$が取れる.よって,$n \geqq N$において
    \[
        \frac{\log a - \varepsilon}{n} < a^\frac{1}{n}-1
    \]
    となる.$\sum \frac{\log a - \varepsilon}{n}$は発散するから,比較判定法により$\sum (a^{\frac{1}{n}}-1)$も発散する.

    次に$ a<1$の場合を考える.$\log a <0$に注意して,$0 < \varepsilon <-\log a $となるような$\varepsilon$をとる.このとき,十分大きな$n$に対して
    \[
        n (a^\frac{1}{n}-1) < \log a + \varepsilon
    \]
    すなわち
    \[
        \frac{-\log a - \varepsilon}{n} < -(a^\frac{1}{n}-1)
    \]
    が成り立つ.$\sum \frac{-\log a - \varepsilon}{n}=\infty$であるから,比較判定法により$\sum -(a^{\frac{1}{n}}-1)=\infty$である.
    よって$\sum (a^{\frac{1}{n}}-1)=-\infty$である.
\end{screen}


\subsection*{p49--50:2-(iv)}
\addcontentsline{toc}{subsection}{\texorpdfstring{p49--50:2-(iv)}{p49--50:2-(iv)}}

\kakko{補題}

$\lim_{n \to \infty} S_{2n}=S$,$\lim_{n \to \infty} S_{2n+1}=S$のとき,
\[
    \lim_{n \to \infty} S_n = S
\]
である.

\begin{proof}
    $\varepsilon >0$を任意にとる.仮定より
    \begin{align*}
         & n \geqq N_1 \Longrightarrow \abs{S_{2n}-S} < \varepsilon,  \\
         & n \geqq N_2 \Longrightarrow \abs{S_{2n+1}-S} < \varepsilon
    \end{align*}
    となるような自然数$N_1$,$N_2$が取れる.$N=\max\{N_1, N_2\}$と定義する.$n \geqq 2N +1$となる$n$を任意にとる.

    $n$が偶数ならば$n =2k$となるような自然数$k$が取れる.このとき$k \geqq N$であるから,
    \[
        \abs{S_n -S}=\abs{S_{2k}-S}<\varepsilon
    \]
    となる.同様に,$n$が奇数なら$n =2k+1$となる自然数$k$がとれる.$k \geqq N$であるから,
    \[
        \abs{S_n -S}=\abs{S_{2k+1}-S}<\varepsilon
    \]
    が成り立つ.よって$\lim_{n \to \infty} S_n =S$である.
\end{proof}

\begin{screen}
    $S_{n} = \sum ^{n}_{k=2} a_k$ とすると,
    \begin{align*}
        S_{2n} & = \sum ^{2n}_{k=2} a_{k}                                              \\
               & = \sum ^{n-1}_{k=1} a_{2k+1} + \sum ^{n}_{k=1} a_{2k}                 \\
               & = \sum ^{n-1}_{k=1} \frac{(-1)^k}{k} + \sum ^{n}_{k=1} \frac{1}{k^2}. \\
    \end{align*}
    となるから,
    \begin{align*}
        \lim_{n \to \infty} S_{2n} & =\lim_{n \to \infty} \left (\sum ^{n-1}_{k=1} \frac{(-1)^k}{k} + \sum ^{n}_{k=1} \frac{1}{k^2} \right ) \\
                                   & = -\log 2 + \frac{\pi^2}{6}.
    \end{align*}
    同様に,
    \begin{align*}
        S_{2n+1} = \sum ^{n}_{k=1} \frac{(-1)^k}{k} + \sum ^{n}_{k=1} \frac{1}{k^2}.
    \end{align*}
    であり,
    \[
        \lim_{n \to \infty} S_{2n+1} = -\log 2 + \frac{\pi^2}{6}.
    \]
    したがって$S_{2n}$,$S_{2n+1}$は同じ値に収束する.これは$S_{n}$が収束することを意味するため,この級数は収束する.
\end{screen}


\subsection*{p49--50:2-(v)}
\addcontentsline{toc}{subsection}{\texorpdfstring{p49--50:2-(v)}{p49--50:2-(v)}}


\begin{screen}
    定理5.7(ダランベールの収束判定)を用いる.$a_n=n/2^n$とおくと
    \[
        \lim_{n \to \infty}\frac{a_{n+1}}{a_n}=\lim_{n \to \infty}\frac{n+1}{n}\frac{2^n}{2^{n+1}}=\frac{1}{2}<1
    \]
    となることにより,この級数は収束する
\end{screen}

\subsection*{p49--50:2-(vi)}
\addcontentsline{toc}{subsection}{\texorpdfstring{p49--50:2-(vi)}{p49--50:2-(vi)}}

\begin{screen}
    部分和の数列を$S_{n}$として$S_{2n}$が発散することを示す. 二項ずつ括弧に入れることで,
    \[
        S_{2n} = \sum ^{n}_{k=1} \left (\frac{1}{(2k-1)!} - \frac{1}{2k} \right )
    \]
    と変形できる.ここで
    \[
        \lim_{k \to \infty} \frac{2k}{(2k-1)!} = 0
    \]
    であるから,ある$N \in \mathbb{N}$が存在して,$k \geq N$となるすべての$k$に対し
    \[
        \frac{1}{(2k-1)!} \leq \frac{1}{4k}
    \]
    となる.さて,
    \[
        \lim_{n \to \infty} \sum ^{n}_{k=N} \frac{1}{4k} = \infty
    \]
    であるから,$\sum ^{\infty}_{k=N} \frac{1}{4k}$は発散する.これをふまえると,定理5.5により,
    \[
        \sum^{\infty}_{k=N} \left (\frac{1}{2k} - \frac{1}{(2k-1)!}\right) =\infty
    \]
    であるからこの級数は発散する.
\end{screen}



\subsection*{p49--50:2-(vii)}
\addcontentsline{toc}{subsection}{\texorpdfstring{p49--50:2-(vii)}{p49--50:2-(vii)}}
\begin{screen}
    $\left (\frac{\log x}{\sqrt x}\right)' = \frac{2-\log x}{2x^{3/2}}$であり,右辺は十分大きな$x$に対して$0$以下となる.さらに,$\lim _{n \to \infty} \frac{\log n}{\sqrt n} = 0$であるから,定理 V. 4.1 よりこの交代級数は収束する.
\end{screen}

\subsection*{p49--50:2-(viii)}
\addcontentsline{toc}{subsection}{\texorpdfstring{p49--50:2-(viii)}{p49--50:2-(viii)}}

\begin{screen}
    \[
        \frac{(1+n)^n}{n^{n+1}}>\frac{n^n}{n^{n+1}}=\frac{1}{n}
    \]
    であることから
    \[
        \sum \frac{(1+n)^n}{n^{n+1}}>\sum \frac{1}{n} \rightarrow \infty
    \]
    となる.よってこの級数は発散する.
\end{screen}


\subsection*{p49--50:2-(ix)}
\addcontentsline{toc}{subsection}{\texorpdfstring{p49--50:2-(ix)}{p49--50:2-(ix)}}

\begin{screen}
    定理 5.7 (Ratio Test)より収束する.
\end{screen}

\subsection*{p49--50:2-(x)}
\addcontentsline{toc}{subsection}{\texorpdfstring{p49--50:2-(x)}{p49--50:2-(x)}}

\begin{screen}
    定理5.7(ダランベールの収束判定)より$a_n=\left(\dfrac{n}{n+1}\right)^{n^2}$とすると
    \begin{align*}
        \frac{a_{n+1}}{a_n} & =\frac{(n+1)^{(n+1)^2}}{(n+2)^{(n+1)^2}}\frac{(n+1)^{n^2}}{n^{n^2}}=\frac{(n+1)^{n^2}}{n^{n^2}}\frac{(n+1)^{n^2}}{(n+2)^{n^2}}\frac{(n+1)^{2n+1}}{(n+2)^{2n+1}} \\
                            & =\left(1+\frac{1}{n}\right)^{n^2}\left(1-\frac{1}{n+2}\right)^{n^2}\left(1-\frac{1}{n+2}\right)^{2n+1}
    \end{align*}
    ここで
    \[
        \left(1+\frac{1}{n}\right)^{n^2}\left(1-\frac{1}{n+2}\right)^{n^2}=\left(1+\frac{1}{n(n+2)}\right)^{n^2}=\left(1+\frac{1}{n(n+2)}\right)^{n(n+2)}\left(1+\frac{1}{n(n+2)}\right)^{-2n}
    \]
    とすることにより,右辺は$e \times 1=e$に収束する.

    さらに,
    \[
        \left(1-\frac{1}{n+2}\right)^{2n+1}=\left\{\left(1-\frac{1}{n+2}\right)^{-(n+2)}\right\}^{-2}\left(1-\frac{1}{n+2}\right)^{-3}
    \]
    とすることで,右辺は$\dfrac{1}{e^2} \times 1=\dfrac{1}{e}$に収束する.よって
    \[
        \frac{a_{n+1}}{a_n}=e \times \frac{1}{e^2}=\frac{1}{e}<1
    \]
    となる.ゆえにこの級数は収束する.
\end{screen}

\kakko{(x)の別解}

\begin{screen}
    1)で証明した命題を用いると,
    \[
        \lim _{n \to \infty} \cfrac{1}{\Bigg(1+\cfrac{1}{n}\Bigg)^n} = \frac{1}{e} < 1
    \]
    であるから収束する.
\end{screen}


\subsection*{p49--50:2-(xi)}
\addcontentsline{toc}{subsection}{\texorpdfstring{p49--50:2-(xi)}{p49--50:2-(xi)}}


\begin{screen}
    十分大きな$n$に対し$\log n \geq 2$であるから,$\frac{1}{n^2}$との比較により収束する.
\end{screen}

\newpage

\section*{p49--50:3}
\addcontentsline{toc}{section}{\texorpdfstring{p49--50:3}{p49--50:3}}

\begin{tleftbar}
    \begin{proof}
        $\sum a_n $が絶対収束するので,$\sum \abs{a_n}$も収束する.
        よって,定理5.1 系より,$\lim_{n \to \infty} \abs{a_n} =0$となる.
        このとき,$1>0$に対して,ある$n_1 \in \mathbb{N}$が存在して,任意の$n \in \mathbb{N}$に対して,
        \[
            n \ge n_1 \Longrightarrow \abs{\abs{a_n} -0}<1
        \]
        が成り立つ.また,$\abs{a_n}<1$のとき,
        \[
            0 \le \abs{{a_n}^2} \le {\abs{a_n}}^2 \le \abs{a_n}
        \]
        が成り立つ.ここで,$\sum \abs{a_n}$が収束し,各項は正なので, 定理5.5(比較判定法)により,$\sum \abs{{a_n}^2}$も収束する.ゆえに$\sum {a_n}^2$は絶対収束する.
    \end{proof}
\end{tleftbar}


\section*{p49--50:5}
\addcontentsline{toc}{section}{\texorpdfstring{p49--50:5}{p49--50:5}}

\begin{tleftbar}
    \begin{proof}
        与えられた条件により,$0 <r <c , r \ne \infty $をみたす$r \in \mathbb{R}$に対して,ある$N \in \mathbb{N}$が存在して,
        任意の$n \in \mathbb{N}$に対して,
        \[
            n \ge N \Longrightarrow \abs{\frac{a_n}{b_n}-c}<r
        \]
        が成り立つ.これにより,
        \begin{align*}
             & 0<-r +c < \frac{a_n}{b_n} < r+c            \\
             & \therefore ~  (-r+c) b_n < a_n < (r+c) b_n
        \end{align*}
        となる.これと比較原理により,$\sum a_n$と$\sum b_n$は同時に収束,発散することが証明された.
    \end{proof}
\end{tleftbar}

\section*{p49--50:6}
\addcontentsline{toc}{section}{\texorpdfstring{p49--50:6}{p49--50:6}}
\begin{tleftbar}
    まず,
    \[
        1-x^2+x^4-\cdots+(-1)^n x^{2n} =\frac{1}{1+x^2} +\frac{(-1)^n x^{2n+2}}{x^2+1}
    \]
    の両辺を0から1まで$x$で積分すると,
    \begin{align*}
        \overbrace{1-\frac{1}{3}+\frac{1}{5}-\cdots+\frac{(-1)^n}{2n+1}}^{s_n} & =\int_{0}^{1} \frac{1}{1+x^2} \, dx +\int_{0}^{1}\frac{(-1)^n x^{2n+2}}{x^2+1}  \, dx \\
                                                                               & = \frac{\pi}{4} + R_n
    \end{align*}
    である.ただしここで$R_n =\int_{0}^{1}\frac{(-1)^n x^{2n+2}}{x^2+1} \, dx$とおいた.この式から,
    \begin{align*}
        \abs{s_n -\frac{\pi}{4}  } & = \abs{\int_{0}^{1}\frac{(-1)^n x^{2n+2}}{x^2+1} \, dx } \\
                                   & < \int_{0}^{1} x^{2n} \, dx                              \\
                                   & =\frac{1}{2n+1} \to 0 ~(n \to \infty)
    \end{align*}
    である.よって,
    \[
        \sum_{n=0}^{\infty} \frac{(-1)^n}{2n+1} =\frac{\pi}{4}
    \]
    である.
\end{tleftbar}

\newpage

\section*{p49--50:8}
\addcontentsline{toc}{section}{\texorpdfstring{p49--50:8}{p49--50:8}}

\begin{tleftbar}
    \begin{proof}
        この級数\footnote{この級数をケンプナー級数という.}の第$n$部分和を$s_n$,分母が$n$桁である項の末尾までの和を$t_n$とする.このとき$s_n < t_n$である.

        分母が$k$桁である項の和を考える.このような項は,$8 \cdot 9^{k-1}$個ある.

        また,このような項の分母の最小値は$10^{k-1}$であるから,
        \begin{align*}
            t_n = \sum_{k=1}^{n} \frac{8 \cdot 9^{k-1}}{10^{k-1}} & = 8 \sum_{k=1}^{n} \left(\frac{9}{10}\right)^{k-1}               \\
                                                                  & = 8 \cdot \frac{1-\left(\dfrac{9}{10}\right)^n}{1-\dfrac{9}{10}} \\
                                                                  & = 80 \left(1-\left(\frac{9}{10}\right)^n\right)
        \end{align*}
        となり,$t_n$は単調増加かつ上に有界である

        よって,
        \[
            0 < s_n < t_n = 80 \left(1-\left(\frac{9}{10}\right)^n\right)
        \]
        であるから,
        \[
            \lim_{n \to \infty} s_n \leqq 80
        \]
        となり,これが証明すべきことであった.
    \end{proof}
\end{tleftbar}


\newpage


\section*{p63--64:1}
\addcontentsline{toc}{section}{\texorpdfstring{p63--64:1}{p63--64:1}}

\subsection*{p63--64:1-(i)}
\addcontentsline{toc}{subsection}{\texorpdfstring{p63--64:1-(i)}{p63--64:1-(i)}}

\begin{tleftbar}
    $f(a)$を考えるため,$a \ne 0$としてよい.$\delta \le \frac{\abs{a}}{2}$とすると,$\abs{x-a}<\delta$より
    \[
        \abs{x} > \abs{a}-\delta \ge \frac{\abs{a}}{2}
    \]
    である.これに留意すると,
    \[
        \abs{\frac{1}{x}-\frac{1}{a}}=\frac{\abs{a-x}}{\abs{ax}} <\frac{2 \delta}{\abs{a}^2}
    \]
    であるから,$\delta = \min \{ \abs{a}/2,\abs{a}^2\varepsilon/2 \}$でよい.
\end{tleftbar}



\subsection*{p63--64:1-(iii)}
\addcontentsline{toc}{subsection}{\texorpdfstring{p63--64:1-(iii)}{p63--64:1-(iii)}}


\begin{tleftbar}
    $t \coloneqq \min \{x,a\}$とする.このとき,指数法則により,
    \[
        \abs{e^x-e^a}=e^t (e^{\abs{x-a}}-1)
    \]
    が成立する.また,$t \le a$であるから,
    \begin{align*}
         & e^t \le e^a                                                \\
         & \therefore ~ e^t(e^{\abs{x-a}}-1) \le e^a(e^{\abs{x-a}}-1)
    \end{align*}
    ゆえに,
    \[
        \varepsilon = e^a (e^\delta -1)
    \]
    となればよい.すなわち,
    \[
        \delta = \log (1+e^{-a}\varepsilon)
    \]
    である.
\end{tleftbar}


\subsection*{p63--64:1-(v)}
\addcontentsline{toc}{subsection}{\texorpdfstring{p63--64:1-(v)}{p63--64:1-(v)}}


\begin{leftbar}
    任意の$\varepsilon > 0$に対し$\delta = \sqrt{\varepsilon}$とすると,$\sqrt{x^2 + y^2} < \delta$において
    \[
        \abs{x^2-y^2} \leq \abs{x}^2 + \abs{y}^2 < \delta^2 = \varepsilon
    \]
    が成り立つ.
\end{leftbar}



\section*{p63--64:2}
\addcontentsline{toc}{section}{\texorpdfstring{p63--64:2}{p63--64:2}}


\subsection*{p63--64:2-(i)}
\addcontentsline{toc}{subsection}{\texorpdfstring{p63--64:2-(i)}{p63--64:2-(i)}}

\begin{tleftbar}
    $\abs{\sin \frac{1}{x}} \le 1$,$\abs{\sin \frac{1}{y}} \le 1$であるから,
    \[
        \abs{(x+y) \sin \frac{1}{x} \sin \frac{1}{y}} \le \abs{(x+y)} \le \abs{x}+\abs{y} \to 0 \quad  \Bigl( (x,y) \to 0 \Bigl)
    \]
    である.よって,
    \[
        \lim_{(x,y)\to 0} (x+y) \sin \frac{1}{x} \sin \frac{1}{y} =0
    \]
    となる.
\end{tleftbar}

\subsection*{p63--64:2-(ii)}
\addcontentsline{toc}{subsection}{\texorpdfstring{p63--64:2-(ii)}{p63--64:2-(ii)}}

\begin{tleftbar}
    $x^2,y^2 \leq x^2+y^2$であるから,
    \[
        1 \leq (1 + x^2 y^2)^{1/(x^2+y^2)} \leq (1 + (x^2+y^2)^2)^{1/(x^2+y^2)}
    \]
    が成り立つ.ここで$(x_n, y_n) \to 0 \quad (n \to \infty)$となるような点列を任意に取ると,$x_n \to 0, y_n \to 0 \quad (n \to \infty)$である.すると$\lim_{x \to 0} (1+x^2)^{1/x} = 1$であるから$\lim_{n \to \infty} (1 + (x_n^2+y_n^2)^2)^{1/(x_n^2+y_n^2)} = 1$も成り立つ.したがって,
    \[
        (1 + (x^2+y^2)^2)^{1/(x^2+y^2)} \to 1 \quad ((x,y) \to 0)
    \]
    となるから,求めるべき極限値は$1$である.
\end{tleftbar}


\subsection*{p63--64:2-(iii)}
\addcontentsline{toc}{subsection}{\texorpdfstring{p63--64:2-(iii)}{p63--64:2-(iii)}}

\begin{tleftbar}
    まず,$\log x^x = x \log x$である.また,
    \[
        \lim_{x \to +0} x \log x  =\lim_{x \to +0} \frac{\log x}{1/x}
    \]
    となる.ここで,$\lim_{x \to +0} \log x = -\infty$,$\lim_{x \to +0} 1/x =\infty$であるから,ロピタルの定理が適用でき,
    \begin{align*}
        \lim_{x \to +0} \frac{\log x}{1/x} & = \lim_{x \to +0} \frac{1/x}{-1/x^2} \\
                                           & = \lim_{x \to +0} (-x)               \\
                                           & =0
    \end{align*}
    である.よって,
    \begin{align*}
        \lim_{x \to +0} x^x & = \lim_{x \to +0} e^{\log x^x} \\
                            & =e^0 =1
    \end{align*}
    となり,これが答である.
\end{tleftbar}

\subsection*{p63--64:2-(iv)}
\addcontentsline{toc}{subsection}{\texorpdfstring{p63--64:2-(iv)}{p63--64:2-(iv)}}

\begin{tleftbar}
    $x=r \cos \theta,~y=r\sin \theta$とおくと,
    \begin{align*}
        \lim_{(x,y)\to 0} \frac{1-\cos (x^2+y^2)}{x^2+y^2} & = \lim_{r \to 0} \frac{1-\cos (r^2)}{r^2}                       \\
                                                           & =\lim_{r \to 0} \frac{1-(-2\sin ^2 (r^2/2)+1)}{r^2}             \\
                                                           & =\lim_{r \to 0} \frac{\sin ^2 (r^2/2)}{(r^2/2)^2} \cdot (r^2/2) \\
                                                           & = 1^2 \cdot 0 =0
    \end{align*}
    を得て,これが答えである.
\end{tleftbar}


\section*{p63--64:3}
\addcontentsline{toc}{section}{\texorpdfstring{p63--64:3}{p63--64:3}}


\subsection*{p63--64:3-(i)}
\addcontentsline{toc}{subsection}{\texorpdfstring{p63--64:3-(i)}{p63--64:3-(i)}}

\begin{tleftbar}
    以下,$\mathbb{Q}$の閉包は$\mathbb{R}$であることを示す.

    $a \in \mathbb{R}$に対して,
    \[
        U(a,\varepsilon) \cap \mathbb{Q} \ne \varnothing
    \]
    すなわち,
    \[
        (a-\varepsilon,a+\varepsilon) \cap \mathbb{Q} \ne \varnothing
    \]
    となればよい.アルキメデスの原理により,任意の$\varepsilon >0$に対して,
    \[
        \frac{1}{n}< 2\varepsilon
    \]
    となる$n \in \mathbb{N} \setminus \{0\}$が存在する.また,$n$を分母とする有理数は数直線上に幅$\frac{1}{n}$で並んでいるから,
    \[
        \frac{m}{n} \in (a-\varepsilon,a+\varepsilon)
    \]
    となる$ m \in \mathbb{N} \setminus \{0\}$が存在する.\par
    したがって$a \in \mathbb{R}$と任意の$\varepsilon>0$に対して$(a-\varepsilon,a+\varepsilon) \cap \mathbb{Q} \ne \varnothing$となるため,
    \[
        \overline{\mathbb{Q}}=\mathbb{R}
    \]
    である.
\end{tleftbar}


\section*{p63--64:4}
\addcontentsline{toc}{section}{\texorpdfstring{p63--64:4}{p63--64:4}}


\subsection*{p63--64:4-(i)}
\addcontentsline{toc}{subsection}{\texorpdfstring{p63--64:4-(i)}{p63--64:4-(i)}}

\begin{tleftbar}
    \begin{proof}
        $d(x)=0$は$\inf_{y \in A} \abs{x-y} =0$ともかける.
        \begin{align*}
            \inf_{y \in A} \abs{x-y} =0 & \iff (\forall \varepsilon>0) \ (\exists y \in A)\ ( \abs{x-y}<\varepsilon ) \\
                                        & \iff x \in \overline{A}
        \end{align*}
        これにより示された.
    \end{proof}
\end{tleftbar}


\subsection*{p63--64:4-(ii)}
\addcontentsline{toc}{subsection}{\texorpdfstring{p63--64:4-(ii)}{p63--64:4-(ii)}}


\begin{leftbar}
    $d(x)$の定義より,任意の$y \in \mathbb{R} ^n$に対して,
    \[
        d(x) \leq \abs{x-y} \leq \abs{x-a} + \abs{a-y}
    \]
    が成り立つ.右辺で$y$について下限を取ると,
    \[
        d(x) - d(a) \leq \abs{x-a}.
    \]
    同様に$d(a) - d(x) \leq \abs{x-a}$も言えるので$\abs{d(x) - d(a)} \leq \abs{x-a}$である.これで示せた.
\end{leftbar}

\newpage
\section*{p63--64:5}
\addcontentsline{toc}{section}{\texorpdfstring{p63--64:5}{p63--64:5}}
\begin{leftbar}
    \begin{proof} \mbox{ }
        \begin{enumerate}[(I)]
            \item \mbox{} \\
                  $f$が$U$上連続のとき,$\mathbb{R}^n$の開集合$W$をひとつとる.
                  \begin{enumerate}[(i)]
                      \item \mbox{} \\
                            $ W \cap f(U) = \varnothing$のとき,$f(W) \ne \varnothing$より,$\mathbb{R}^n$上の開集合となる.
                      \item \mbox{} \\
                            $ W \cap f(U) \ne \varnothing$のとき,$f^{-1} (W)$の元$x$をひとつとる.

                            $ f(x)=y$とすると,$U$は開集合なので,ある$\delta_0 >0$に対して$U(x,\delta_0) \subset U$となる.
                            $f$の連続性より,任意の$\varnothing >0$に対して,ある$\delta >0$~($\delta_0 >\delta >0$)が存在して,
                            $f(U(x,\delta)) \subset U' (y,\varepsilon)$となる.

                            また, $W$は開集合なので,$\varepsilon$を十分小さくとると
                            \[
                                U'(y,\varepsilon) \subset W,\quad f(U(x,\delta)) \subset U (x,\varepsilon) \subset W
                            \]
                            となる.
                            ここで,$f^{-1} (f(U(x,\delta))) \supset U(x,\delta)$より,
                            \[
                                U(x,\delta) \subset f^{-1} (f(U(x,\delta))) \subset f^{-1} (W)
                            \]
                            となり,$f^{-1} (W)$は$\mathbb{R}^n$の開集合である.
                  \end{enumerate}
            \item \mbox{} \\
                  $ \mathbb{R}^n$の開集合$W$に対して,$f^{-1} (W)$が$\mathbb{R}^n$の開集合となるとき,
                  $U$の元$x$を1つとり,$f(x)=y$とする.

                  任意の$\varepsilon >0$に対して,$U(y,\varepsilon)$は$\mathbb{R}^n$の開集合なので,
                  $f^{-1} (U'(y,\varepsilon))$も$\mathbb{R}^n$の開集合となる.

                  $x \in f^{-1} (U'(y,\varepsilon))$より,
                  ある$\delta >0$が存在して,$U(x,\delta) \subset  f^{-1}(U'(y,\varepsilon)) \subset U$となり,
                  $f(U(x,\delta)) \subset f(f^{-1}(U(y,\varepsilon))) \subset U'(y,\varepsilon)$であるから,$f$は$x$で連続となる.
                  $x \in U$で任意に取れるので,$U$上連続である.
        \end{enumerate}
    \end{proof}
\end{leftbar}
\newpage


\section*{p63--64:6}
\addcontentsline{toc}{section}{\texorpdfstring{p63--64:6}{p63--64:6}}

\begin{leftbar}
    \begin{proof} \mbox{ }
        \begin{enumerate}[(I)]
            \item \mbox{} \\
                  $f$が$U$上連続であるとき,$\mathbb{R}^n$の開集合$W$をひとつとる.
                  \begin{enumerate}[(i)]
                      \item \mbox{} \\
                            $W \cap f(U) = \varnothing$のとき,$f^{-1} (W) = \varnothing$より,$\mathbb{R}^n$上の開集合となり,成立する.
                      \item \mbox{} \\
                            $W \cap f(U) \ne \varnothing$のとき,$f^{-1} (W)$の元$x$をひとつとり,$y=f(x)$とする.

                            $W$は開集合なので,$U'(y,\varepsilon) \subset W$となる$\varepsilon >0$がとれる.
                            $f$の連続性から,$\delta_0 >0$が存在して,
                            \[
                                f(U(x,\delta_0) \cap U) \subset U'(y,\varepsilon) \subset W.
                            \]
                            $f^{-1} (f(U(x,\delta_0) \cap U)) \supset U(x,\delta_0) \cap U$より,
                            \[
                                U(x,\delta_0) \cap U \subset f^{-1} (W).
                            \]
                            $V = \bigcup_{x \in f^{-1}(W)} U(x,\delta_0) \cap U$とすると,$V$は$\mathbb{R}^n$の開集合であり,
                            \[
                                V \cap U = \bigcup_{x \in f^{-1}(W)} (U(x,\delta_0) \cap U)  \subset f^{-1} (W).
                            \]
                            定義より$V \cap U \supset f^{-1} (W)$であるから,
                            \[
                                V \cap U = f^{-1} (W)
                            \]
                            となる.
                  \end{enumerate}
            \item \mbox{} \\
                  $\mathbb{R}^n$の開集合$W$に対して,$f^{-1}(W)=V \cap U$となる開集合$V$が存在するとき,
                  $U$の元$x$を任意にひとつとり,$y=f(x)$とする.

                  任意の$\varepsilon >0$に対して,$U'(y,\varepsilon)$は$\mathbb{R}^n$の開集合なので,
                  条件より,開集合$V$が存在して,
                  \[
                      f^{-1} (U'(y,\varepsilon)) = V \cap U
                  \]
                  となる.
                  $V$は開集合であるから,$\delta >0$が存在して,
                  \begin{align*}
                       & U(x,\delta)  \subset V,                          \\
                       & f(U(x,\delta) \cap U) \subset U'(y,\varepsilon).
                  \end{align*}
                  $x$は$U$上で任意にとれるので,$f$は$U$上連続である.
        \end{enumerate}
    \end{proof}
\end{leftbar}

\section*{p63--64:7}
\addcontentsline{toc}{section}{\texorpdfstring{p63--64:7}{p63--64:7}}
\begin{leftbar}
    \begin{proof} $f$が単調増加のときを示せば十分である.

        任意の$ x_0 \in (a,b)$に対して,$f$は単調増加であるから,
        \begin{align*}
             & \lim_{x \to x_0} f(x)=f_{+} (x_0) , \\
             & \lim_{x \to x_0} f(x)=f_{-} (x_0)
        \end{align*}
        がそれぞれ存在する.

        \begin{align*}
                 & \text{$f(x)$が$x_0$において不連続}  \\
            \iff & f_{+} (x_0) \ne f_{-} (x_0) \\
            \iff & f_{+} (x_0) - f_{-} (x_0)>0
        \end{align*}
        である.
        \[
            A_n =  \left \{ x \in (a,b) \mid f_{+} (x) - f_{-} (x) > \frac{f(b)-f(a)}{n} \right \}
        \]
        として.$\# (A_n) \geqq n$を仮定する.
        $A_n$から適当に$a_1,a_2,\ldots,a_n$を取り出すと($a<a_1 < a_2 < \cdots < a_n<b$),$f$の単調性により,
        \[
            f_{+} (a_i) \leqq f_{-} (a_{i+1}) \quad (1 \leqq \forall i \leqq n-1)
        \]
        なので,
        \begin{align*}
            f(b)-f(a) & = \sum_{i=1}^{n} \frac{f(b)-f(a)}{n}                                                         \\
                      & < f_{+} (a_1) - f_{-} (a_1) + f_{+} (a_2) - f_{-} (a_2) + \cdots + f_{+} (a_n) - f_{-} (a_n) \\
                      & \leqq f_{+} (a_n) - f_{-} (a_1)                                                              \\
                      & \leqq f(b)-f(a)
        \end{align*}
        となり,矛盾する.
        よって,$\# (A_n) < n$である.

        $f$の不連続点全体の集合を$A$とすれば,
        \[
            A \subset \left ( \bigcup_{n \in \mathbb{N}}  A_n \right ) \cup \{ a, b \}
        \]
        右辺は高々加算なので,$A$も高々可算である.
    \end{proof}
\end{leftbar}

\section*{p63--64:8}
\addcontentsline{toc}{section}{\texorpdfstring{p63--64:8}{p63--64:8}}
\begin{leftbar}
    $ [0,1]$に対し,
    \[
        f(x)= \begin{cases}
            0   & (x=0)                   \\
            1/n & (1/(n+1) < x \leqq 1/n) \\
        \end{cases}
        ,\quad \left ( \because [0,1] = \{ 0 \} \cup \bigcup_{n=1}^{\infty} (1/(n+1),1/n] \right )
    \]
    とすれば,広義単調増加で$x=1/n$($n \in \mathbb{N}$)で不連続である.
\end{leftbar}
\newpage

\section*{p63--64:9}
\addcontentsline{toc}{section}{\texorpdfstring{p63--64:9}{p63--64:9}}


\begin{tleftbar}
    \begin{proof}[\textup{\textbf{存在性の証明}}]
        $\overline{[a,b] \cap \mathbb{Q}}=[a,b]$であるから,$I=[a,b] \cap \mathbb{Q}$として,
        $[a,b] \cap (\mathbb{R}-\mathbb{Q})$の元$x$に対して,Th6.1(1)より,
        $x$に収束する$I$の数列$(x_n)_{n \in \mathbb{N}}$が存在する.

        その$(x_n)_{n \in \mathbb{N}}$に対して,
        \[
            F(x)\coloneqq \lim_{n \to \infty} f(x_n)
        \]
        とする\footnote{$x \in I$のときも同様に,$x$に収束する$I$上の数列$(x_n)_{n \in \mathbb{N}}$をとって$F(x)\coloneqq \lim_{n \to \infty} f(x_n)$とする.これはTh6.2により well--definedで$F(x)=f(x)$である.}.

        次に,これがwell-definedであることを示す\footnote{$\lim_{n \to \infty} f(x_n)$が存在し,$(x_n)_{n \in \mathbb{N}}$の取り方に依らないこと.}.
        $\varepsilon >0$に対して,$f$は一様連続であるから,ある$\delta >0$が存在して,
        \[
            \abs{x-y}<\delta \land x,y \in I \Rightarrow \abs{f(x)-f(y)}<\varepsilon
        \]
        となる.

        また,$x_n \to x$($n \to \infty$)であるので,ある$N \in \mathbb{N}$が存在して,
        \begin{align*}
            n, m > N & \Rightarrow \abs{x_n-x_m}<\delta \land x_n,x_m \in I \\
                     & \abs{f(x_n)-f(x_m)}<\varepsilon
        \end{align*}
        となり,$(f(x_n))_{n \in \mathbb{N}}$はコーシー列となるので収束する.

        $x$に収束する2つの数列$(x_n)_{n \in \mathbb{N}}$,$(y_n)_{n \in \mathbb{N}}$に対して,
        $x_n$,$y_n$を交互にとった数列を$(z_n)_{n \in \mathbb{N}}$とすれば,
        $z_n \to x$($n \to \infty$)となり,$(f(z_n))_{n \in \mathbb{N}}$は収束する.
        $(f(x_n))_{n \in \mathbb{N}}$,$(f(y_n))_{n \in \mathbb{N}}$は上の部分列であるので,
        同じ収束値に収束し,$(x_n)_{n \in \mathbb{N}}$の取り方に依らない.
    \end{proof}
\end{tleftbar}


\begin{leftbar}
    \begin{proof}[\textup{\textbf{連続性の証明1}}]
        $F$の$[a,b]$における一様連続性を示す.(一様連続なら特に連続)
        $\varepsilon >0$を任意に固定する.
        $f$の一様連続性を使って,任意の$x,y \in [a,b] \cap \mathbb{Q}$に対し
        \begin{align*}
            \abs*{x-y} < \delta \implies \abs*{f(x) - f(y)} < \varepsilon/3
        \end{align*}
        となるように,$\delta > 0$を取る.
        $\abs*{x-y} < \delta/3$,$x,y \in [a,b]$とし,$\abs*{F(x)-F(y)} < \varepsilon$であることを示す.
        $x,y$にそれぞれ収束する$[a,b] \cap \mathbb{Q}$の数列$x_n,y_n$を取る.
        $n$を十分大きく取って
        \begin{align*}
            \abs*{x_n-x}       & < \delta/3, \quad \abs*{y_n-y} < \delta/3,                \\
            \abs*{F(x)-f(x_n)} & < \varepsilon/3, \quad \abs*{F(y)-f(y_n)} < \varepsilon/3
        \end{align*}
        となるようにする.
        このとき
        \begin{align*}
            \abs*{x_n-y_n} \le \abs*{x_n-x} + \abs*{x-y} + \abs*{y-y_n} < \delta/3 + \delta/3 + \delta/3 = \delta
        \end{align*}
        が成り立つ.
        すると$\delta$の取り方より,
        \begin{align*}
            \abs*{f(x_n)-f(y_n)} < \varepsilon/3
        \end{align*}
        となる.
        したがって,
        \begin{align*}
            \abs*{F(x)-F(y)} \le \abs*{F(x)-f(x_n)} + \abs*{f(x_n)-f(y_n)} + \abs*{f(y_n)-F(y)}
            < \varepsilon/3 + \varepsilon/3 + \varepsilon/3 = \varepsilon.
        \end{align*}
    \end{proof}
\end{leftbar}

\begin{leftbar}
    \begin{proof}[\textup{\textbf{連続性の証明2}}]
        $\varepsilon>0$を任意に取り,$\delta$は上の証明と同じように取る.
        $\abs*{x-y} < \delta$,$x,y \in [a,b]$とする.
        $(x,y) \cap \mathbb{Q}$の数列$x_n,y_n$で$x,y$に収束するものを取る.
        明らかに$\abs*{x_n-y_n}<\delta$.
        $\delta$の取り方より$\abs*{f(x_n)-f(y_n)}<\varepsilon$.
        $n \to \infty$として$\abs*{F(x)-F(y)} \le \varepsilon$を得る.
    \end{proof}
\end{leftbar}


\begin{leftbar}
    \begin{proof}[\textup{\textbf{一意性の証明}}]
        条件をみたす2つの関数$F_1 (x)$と$F_2(x)$に対して,ある$x$で$F_1(x) \ne F_2(x)$とすると,
        $\varepsilon = \abs{F_1(x) - F_2(x)}/2$に対して,$F_1(x)$と$F_2(x)$の連続性により,
        ある$\delta >0$が存在して,
        \[
            \abs{x-y}<\delta \land y \in [a,b] \Rightarrow \abs{F_1(y)-F_1(x)}<\varepsilon \land  \abs{F_2(y)-F_2(x)}<\varepsilon.
        \]
        特に$ U (x,\delta) \cap [a,b] \cap \mathbb{Q} \ne \varnothing$より,その元$y$に対して,
        $F_1 (y)=F_2(y)~(=f(y))$となるが,
        \begin{align*}
            \abs{F_1(x)-F_2(x)} & \leqq \abs{F_1(x)-F_1(y)} + \abs{F_2(y)-F_2(x)}          \\
                                & < \frac{\varepsilon}{2} \cdot 2  = \abs{F_1(x) - F_2(x)}
        \end{align*}
        となり矛盾する.よって示された.
    \end{proof}
\end{leftbar}

\newpage

\section*{p72--74:1}
\addcontentsline{toc}{section}{\texorpdfstring{p72--74:1}{p72--74:1}}


\subsection*{p72--74:1-(i)}
\addcontentsline{toc}{subsection}{\texorpdfstring{p72--74:1-(i)}{p72--74:1-(i)}}

\begin{tleftbar}
    $ f\colon \mathbb{R}^3 \to \mathbb{R}$を$f(x,y,z)=3x^2 + 2y^2 + 5z^2 -1$とする.
    $f$は連続であり,$\mathbb{R} -\{ 0 \}$は$\mathbb{R}$の開集合であるので,\S 6 :問3より,
    その逆像である$A^c$も開集合である.
    よって,$A$は閉集合である.

    また,$(x,y,z) \in A$に対して,
    \begin{align*}
        1 & = 3x^2 + 2y^2 + 5z^2 \\
          & \geqq 3x^2
    \end{align*}
    であるから,
    \[
        \abs{x} \leqq \frac{1}{\sqrt{3}}.
    \]
    同様に,$ \abs{y} \leqq 1/\sqrt{2}$,$\abs{z} \leqq 1/\sqrt{5}$となり,$A$は有界である.

    以上より,$A$はコンパクトである.
\end{tleftbar}



\subsection*{p72--74:1-(ii)}
\addcontentsline{toc}{subsection}{\texorpdfstring{p72--74:1-(ii)}{p72--74:1-(ii)}}

\begin{tleftbar}
    $ x \in \mathbb{R}$に対して,$ (x,-x,1) \in A$であり,
    \[
        \norm{{}^t (x,-x,1)}=\sqrt{2x^2+1} \to \infty ~(x \to +\infty).
    \]
    このことから,$B$は有界でない.

    よって,$B$はコンパクトでない.
\end{tleftbar}


\subsection*{p72--74:1-(iii)}
\addcontentsline{toc}{subsection}{\texorpdfstring{p72--74:1-(iii)}{p72--74:1-(iii)}}

\begin{tleftbar}
    $(x,y,z) \in C$に対して,
    \[
        \varepsilon =\frac{1}{2}(1-\sqrt{x^2+y^2+z^2})
    \]
    とすれば,$U(x,\varepsilon)\subset C$となり,$C$は開集合である.

    よって,$C$はコンパクトでない.
\end{tleftbar}



\subsection*{p72--74:1-(iv)}
\addcontentsline{toc}{subsection}{\texorpdfstring{p72--74:1-(iv)}{p72--74:1-(iv)}}

\begin{tleftbar}
    $t>\sqrt{3}$に対して,$t^3 -3t >0$で,$(t,\sqrt{t^3-3t}/2) \in D$である.
    \begin{align*}
        \norm{{}^t (t,\sqrt{t^3-3t}/2)} & = \frac{1}{2}\sqrt{t^3+4t^2-3t} \\
                                        & \to  \infty ~(t \to \infty).
    \end{align*}
    より, $D$は有界でなく,コンパクトでない.
\end{tleftbar}


\subsection*{p72--74:1-(v)}
\addcontentsline{toc}{subsection}{\texorpdfstring{p72--74:1-(v)}{p72--74:1-(v)}}

\begin{tleftbar}
    $(0,0)~(\notin E)$が触点であることを示す.

    任意の$\varepsilon >0$に対して,アルキメデスの原理より,
    \[
        0 < \frac{1}{n} < \varepsilon
    \]
    となる$n \in \mathbb{N}$が存在するので,原点の任意の$\varepsilon$-近傍において$E$と交わりを持つ.

    よって,閉集合でなくコンパクトでない.
\end{tleftbar}


\newpage
\section*{p72--74:3}
\addcontentsline{toc}{section}{\texorpdfstring{p72--74:3}{p72--74:3}}

\begin{leftbar}
    \begin{proof}
        対偶を示す.$\bigcap_{n \in \mathbb{N}}F_n = \varnothing$とする.
        $\bigcup_{n \in \mathbb{N}}F_n^c = \mathbb{R}^m$であり,$F_n^c$は開集合であるから,$(F_n^c)_{n \in \mathbb{N}}$は$K$の開被覆である.
        $K$はコンパクトであるから,$(F_n^c)_{n \in \mathbb{N}}$の有限個の開集合$F_{n_1}^c,\ldots,F_{n_k}^c$によって
        \[
            K \subset F_{n_1}^c \cup \ldots \cup F_{n_k}^c
        \]
        とできる.したがって,
        \[
            K \cap F_{n_1} \cap \ldots \cap F_{n_k} = \varnothing
        \]
        である.$K$は$F_{n_1},\ldots,F_{n_k}$をすべて含んでいるから,
        \[
            F_{n_1} \cap \ldots \cap F_{n_k} = \varnothing
        \]
        である.
    \end{proof}
\end{leftbar}

\newpage

\section*{p72--74:4}
\addcontentsline{toc}{section}{\texorpdfstring{p72--74:4}{p72--74:4}}

\kakko{補題1}

ノルムに関して, $\abs{\norm{ \bm{x}  }- \norm{ \bm{y} } } \leq \norm{ \bm{x} - \bm{y} }$ が成り立つ.


\begin{proof}
    絶対値の定義 ( $\abs{a} \coloneqq \max \{ a , -a \}$ ) に立ち返ると,
    %		
    \[
        \norm{ \bm{x}  }- \norm{ \bm{y} } \leq \norm{ \bm{x} - \bm{y} } ,\quad  - \norm{ \bm{x} } + \norm{ \bm{y} } \leq \norm{ \bm{x} - \bm{y} }
    \]
    %		
    なる二つの不等式を示せばよい. これは,

    \[
        \norm{ \bm{x} } = \norm{ \bm{x} - \bm{y} + \bm{y} } \leq \norm{ \bm x - \bm y } + \norm{ \bm{y} }
    \]

    より示される.
\end{proof}

\kakko{補題2}

$\norm{ \cdot }_1 : \mathbb{R}^n \ni ( x_1 , x_2 , \cdots , x_n ) \mapsto \sum_{j=1} ^ n \abs{x_j} \in \mathbb{R}$ とすると,
%		
\[
    \exists M \in \mathbb{R} \ \mathrm{s.t.} \ \forall \bm x \in \mathbb{R}^n \text {に対して, } \norm{ \bm{x} }_1 \leq M \abs{ \bm{x} }
\]
%		
が成り立つ.


\begin{proof}
    $M \coloneqq n$ とおく. $\bm{x} \in \mathbb{R}^n$ とする. 各 $j ( 1 \leq j \leq n )$ において,
    %		
    \[
        \abs{ x_j } \leq \abs{ \bm{x} }
    \]
    %		
    が成り立つ. そこで, $j$ について足し合わせると,
    %		
    \[
        \abs{ x_1 } + \abs{ x_2 } + \cdots \abs{ x_n } \leq \abs{ \bm{x} } + \abs{ \bm{x} } + \cdots \abs{ \bm{x} } = n \abs{ \bm{x} } = M \abs{ \bm{x} }
    \]
    %		
    が成り立つ.
\end{proof}

\begin{leftbar}
    \begin{proof}
        まず,
        %		
        \[
            \exists P \in \mathbb{R} \ \mathrm{s.t.} \ \forall \bm{x} \in \mathbb{R}^n \text {に対して, } \norm{ \bm{x} } \leq P \abs{ \bm{x} } \ \ \ \cdots ( \ast )
        \]
        %		
        となることを示す. $\mathbb{R}^n$ の正規直交基底 を $\bm{e}_1 , \bm{e}_2 , \cdots , \bm{e}_n$ とする. $T \coloneqq  \max \{ \norm{ \bm{e}_j } \mid 1 \leq j \leq n \}$ とおき, $M$ を補題のものとして, $P \coloneqq TM$ とおく. . $\bm{x} \in \mathbb{R}^n$ を,
        %		
        \[
            \bm{x} = ( x_1 , x_2 , \cdots , x_n )
        \]
        %		
        とすると, $\bm{x} = \sum_{j=1} ^ n x_j \bm{e}_j$ とかける. これと, ノルムのi) , ii) の条件から,
        %		
        \[
            \norm{ \bm{x} } = \norm{ \sum_{j=1} ^ n x_j \bm{e}_j } \leq \sum \abs{ x_j } \norm{\bm{e}_j } \leq T \sum \abs{ x_j } \leq T M \norm{ \bm{x} } = P \abs{ \bm{x} }
        \]
        %		
        が成り立つ. 以上より $( \ast )$ は成り立つ. これより, 函数 $\norm{ \cdot }$ が 連続函数になる事を示す. $\bm{\alpha} \in \mathbb{R}^n$ とする. $\varepsilon > 0$ とする. $\delta < \frac{\varepsilon}{P}$ を満たすようにとる. このとき,
        $\bm{x} \in \mathbb{R}^n$ かつ $\abs{ \bm{x} - \bm{\alpha} } < \delta$ とする.
        %		
        \[
            \abs{ \norm{ \bm x } -\norm{ \bm{\alpha} } } \leq \norm{ \bm{x} - \bm{\alpha} } \leq P \abs{ \bm{x} - \bm{\alpha} } < \varepsilon
        \]
        %		
        よって, 連続である.  $\mathbb{S}^{n-1} \coloneqq \{ \bm{x} \in \mathbb{R}^n \mid \abs{ \bm{x} } = 1 \}$ なるコンパクト集合上で, $\norm{ \cdot }$ を考える. ノルムが連続であったことから, コンパクト集合上で最小値をとる. これを $m$ とする. この
        $m$ を用いると,
        %		
        \[
            \forall x \in \mathbb{R}^n \text {に対して, } m \abs{ \bm{x} } \leq \norm{ \bm{x} }
        \]
        %		
        が成り立つ事を示す. $\bm{x} \in \mathbb{R}^n$ とする. このとき, $\bm{x} / \abs{ \bm{x} } \in \mathbb{S}^{n-1}$ であることに注意すると, ( $\because \abs{ \bm{x} / \abs{ \bm{x} } } = \abs{ \bm{x} } / \abs{ \bm{x}} = 1 $ ) $m$ が最小値であることから,
        %		
        \[
            m \leq  \norm{ \frac {\bm{x}} {\abs{ \bm{x} }}  }
        \]
        %		
        を満たす.
        %		
        \[
            \norm{ \frac {\bm{x}} {\abs{\bm{x} }} } = \frac {1} {\abs{ \bm{x} }} \norm{ \bm{x}}
        \]
        %		
        に注意すると,
        %		
        \[
            m \abs{ \bm{x} } \leq \norm{ \bm{x} }
        \]
        %		
        となる. 以上より, 示すべき題意は満たされた.
    \end{proof}
\end{leftbar}

\begin{column}
    これは, なかなか背景を語るには, 奥深い問題です. まず, 注意しておくことは, この性質は有限次元の線型空間だからできる話だということです. 無限次元であれば, もう少し条件がないと無理です. ( そもそも一般にはユークリッド距離が入りません. ) 有限次元であれば, ノルムが定める位相というのが一意に定まるという事を言っています. それだけ有限次元の線型空間は「硬い」という風に定義づけることができるでしょう.

    次に証明の中で用いたテクニックです.  $\mathbb{S}^{n-1}$ を用いたところ. これは, 無限次元になっても用いられる手法です. 線型性がある操作や空間の中では, 綺麗な ( 空間全体に一様に広げていけそうな? ) 図形の上でだけ考えておいて, あとは線型に伸ばすということで全体の性質をみるということがあります. ここではそれを行なっています. 函数解析学などでは, 無限次元上に定まる理論上重要な写像( いわゆる有界線型作用素 )に対してノルムを定めますが, それらのノルムは, 無限次元の球面だけで見れば良いという性質があったりします.

    最後に証明の中で出てきた $\norm{ \cdot } _ 1$ というノルムですが, これはマンハッタン距離と呼ばれる距離です. 解析などでは計算が楽に済むので利用されます. ( 他にも応用はいくつかあると思いますが, 僕は知りません笑)
\end{column}
\newpage

\section*{p72--74:5}
\addcontentsline{toc}{section}{\texorpdfstring{p72--74:5}{p72--74:5}}

\begin{leftbar}
    \begin{proof}
        連続函数の定数倍, 連続函数の和が再び連続函数になることから, $C ( K )$ は線型空間になる. また, $K$ がcompactであるから, $f ( K )$ はcompactである. また, $\abs{ \cdot }$ も連続函数であるから, $\abs{ f ( K ) }$ も compactである. 従って最
        大値を持つから, $\norm{ f }$ なる値が存在していることがわかる. ノルムの条件を満たすことを示す.
        \begin{enumerate}
            \item $\norm{ f } \geq 0$ であること

                  \parindent=1\zw 各 $x \in K$ に対して, $\abs{f ( x ) } \geq 0$ である. 従って, $\norm{ f } = \max \{ \abs{ f ( x )} \mid x \in K \} \geq 0$

            \item $\norm{ f } = 0 \iff  f = 0$ であること ( $f = 0$ は写像として定数函数 $0$ に等しいという意味です. )

                  $f = 0$ であれば, 各 $x \in K$ に対して $\abs{ f ( x ) } = 0$ であるから, $\norm{ f } = 0$ は明らか. 逆を示す. $\norm{ f } = 0$ とする. 各 $x \in K$ に対して, $\abs{ f ( x ) } = 0$ である. つまり $f ( x ) = 0$ . よって $f = 0$

            \item $\norm{ f + g } \leq \norm{ f } + \norm{ g }$ であること

                  \begin{align*}
                      \norm{ f + g } & =  \max \{ \abs{ f ( x ) + g ( x )} \mid x \in K \}                                                                   \\
                                     & \leq  \max \{ \abs{ f ( x ) } + \abs{ g ( x ) } \mid x \in K \}                                                       \\
                                     & \leqq   \max \{ \abs{ f ( x ) }  \mid x \in K \} + \max \{ \abs{ g ( x ) }  \mid x \in K \} = \norm{ f } + \norm{ g }
                  \end{align*}
        \end{enumerate}
        よって, 確かにノルムである.
    \end{proof}
\end{leftbar}

\begin{column}
    皆さんには, 冗長な指摘かもしれませんが, これ,
    \begin{itemize}
        \item $C ( K )$ が線型空間であること
        \item $\max \{ \abs{ f ( x )} \mid x \in K \}$ が存在していること
    \end{itemize}
    を確認しなければそもそもノルムを定義したことにならないです. 後輩とのゼミなどでこの辺りうっかりミスが多かったのでコメントしておきました.
\end{column}

\newpage

\section*{p72--74:6}
\addcontentsline{toc}{section}{\texorpdfstring{p72--74:6}{p72--74:6}}

\begin{leftbar}
    \begin{proof}
        収束先となる函数 $f$ を構成し, i) 収束先であること ii) 連続であること  を示す.

        各 $x \in K$ において, 実数列 $( f_n ( x ) ) _ {n \in \mathbb{N}}$ はCauchy列である. 実際,
        %		
        \[
            \abs{ f _n ( x ) - f _m ( x ) } \leq \max \{ \abs{ f_n ( x ) - f_m ( x ) } \mid x \in K \} = \norm{f_n - f_m} \to 0 \ ( n , m \to N )
        \]
        %		
        よりわかる. 実数の完備性から列 $( f_n ( x ) )$ は収束する. そこで,
        %		
        \[
            f \colon K \ni x \mapsto \lim_{n \to \infty} f_n ( x ) \in \mathbb{R}
        \]
        %		
        と定める. この $f$ が i ) , ii ) を満たすことを示す. \\
        i) を満たすこと. すなわち,
        %		
        \[
            \forall \varepsilon > 0 \text {に対して, } \exists N \in \mathbb{N} \ \mathrm{s.t.} \ \forall n > N \text {に対して, } \norm{ f_n - f } < \varepsilon
        \]
        %
        を示す. ただし, $\norm{ \cdot }$ の定義から, $\norm{ f_n - f } < \varepsilon$ は,
        %		
        \[
            \forall t \in K \text {に対して, } \abs{ f_n ( t ) - f ( t ) } < \varepsilon
        \]
        %		
        を示せばよいことに注意する. $( f_n )$ が一様ノルムに関してCauchyの収束条件を満たすことに $\varepsilon / 2 > 0$ を適用すると, ある自然数 $N_1$ が存在して,
        %		
        \[
            \forall n , m > N _1 \text {に対して, } \norm{ f_n - f_m } < \varepsilon / 2
        \]
        %		
        を満たす. $N \coloneqq  N_1$ とおく. $n > N$ , $t \in K$ とする. 列 $( f_n ( t ) )$ は $f (t)$ に収束するから, ある自然数 $N_2$ が存在して,
        %		
        \[
            \forall m > N_2 \text {に対して, }\abs{ f_{m} ( t ) - f ( t ) } < \varepsilon / 2
        \]
        %		
        を満たす. そこで, $m > \max \{ N_1 , N_2 \}$ をとれば,
        %		
        \[
            \abs{ f_n ( t ) - f ( t ) } \leq \abs{ f_n ( t ) - f_m ( t ) } + \abs{ f _m ( t ) - f ( t ) } \leq \norm{ f_n - f_m } + \abs{ f_m ( t ) - f ( t ) } < \varepsilon
        \]
        %		
        となる. よって収束先である.

        ii) を満たすこと. すなわち, 各 $\alpha \in K$ において
        %		
        \[
            \varepsilon > 0 \text {に対して, } \exists \delta > 0 \ \mathrm{s.t.} \ \forall x \in K \ \mathrm{s.t.} \ d ( \alpha , x ) < \delta \text {に対して, } \abs{ f ( x ) - f ( \alpha ) } < \varepsilon
        \]
        %		
        を満たすことを示す. $\alpha \in K$ とする. $\varepsilon > 0$ とする. 列 $( f_n ( \alpha ) )$ が $f ( \alpha )$ に収束することから, ある自然数 $N_1$ が存在して,
        %		
        \[
            \forall n > N_1 \text {に対して, } \abs{ f_n ( \alpha ) - f ( \alpha ) } < \varepsilon / 3
        \]
        %	
        が成り立つ. また, $( f_n )$ がCauchyの収束条件を満たすことから, ある自然数 $N_2$ が存在して,
        %		
        \[
            \forall n , m > N_2 \text {に対して, } \norm{ f_n - f_m } < \varepsilon / 3
        \]
        %		
        が成り立つ. $N \coloneqq  \max \{ N_1 , N_2 \}$ とおく. $f_N$ は $K$ 上で連続であるから, $\alpha$ での連続性より, ある正数 $\delta'$ が存在して,
        %		
        \[
            \forall x \in K \ \mathrm{s.t.} \ d ( \alpha , x ) < \delta' \text {に対して, } \abs{ f_N ( x ) - f_ N( \alpha ) } < \varepsilon / 3
        \]
        %	
        が成り立つ. $\delta \coloneqq  \delta'$ とおく. $x \in K$ かつ $d ( \alpha , x ) < \delta$ とする.
        %		
        \[
            \abs{ f ( x ) - f ( \alpha ) } \leq \abs{ f ( x ) - f_N ( x ) } + \abs{ f_N ( x ) - f_N ( \alpha ) } + \abs{ f_N ( \alpha ) - f ( \alpha ) } \leq \norm{ f - f_N } + \varepsilon / 3 + \norm{ f - f_N } < \varepsilon
        \]
        %		
        より $f$ は連続である. つまり $f \in C ( K )$ である.

    \end{proof}
\end{leftbar}

\begin{column}
    これ, Compact上の連続函数であることはほとんど本質的なことに影響しません. 定義域がCompactである条件を外す代わりに, 有界連続函数に対して適用すれば同様の議論が成り立ちます. ( ただし, その場合, 最大値ではなく上限でノルムを定義することになります. ) もしくは, 定義域上でCompact - support ( あるCompact集合上で値をもち, それ以外では恒等的に0 ) など他にも少し条件を変えて修正することでいくつかの応用が存在します. また, 詳しく調べてないのでわかりませんが ,定義域が距離空間であることもあまり影響しなかったと思います.

    函数解析における函数空間の一例です. このような $C(K)$ に相当する空間として, 微分可能函数空間などもあげられます. ここは, 深入りすると, それだけで大学一年間分の解析の授業ができるぐらいですので, ここで止めておきます.


    \kakko{参考文献}

    \begin{enumerate}
        \item 宮寺功 関数解析 ( ちくま学芸出版 )
        \item 洲之内治男 関数解析入門 ( 近代ライブラリ社 )
    \end{enumerate}
\end{column}

\section*{p72--74:7}
\addcontentsline{toc}{section}{\texorpdfstring{p72--74:7}{p72--74:7}}

\begin{tleftbar}
    \begin{proof}
        部分列$(f_{n(k)})_{k \in \mathbb{N}}$を任意にひとつとる.
        \[
            f(x)= \lim_{k \to \infty} f_{n(k)}(x)
            \begin{cases}
                0 \quad  & ( 0 \leqq x < 1) \\
                1  \quad & ( x=1)
            \end{cases}
        \]
        であり,$f$に各点収束する.

        $(f_{n(k)})_{k \in \mathbb{N}}$は一様収束するとすれば$f$に収束するが,$f$は連続でなく
        「連続関数列$(f_n)_{n \in \mathbb{N}}$が$f$に各点収束すれば,$f$は連続」の対偶を考えることにより矛盾する\footnote{IV.~Thm 13.2,Thm 13.3より.}.よって一様収束しない.
    \end{proof}
\end{tleftbar}

\newpage


\section*{p72--74:8}
\addcontentsline{toc}{section}{\texorpdfstring{p72--74:8}{p72--74:8}}

\kakko{補題}


コンパクト集合$S$上の連続写像$g$について,その像$g(S)$はコンパクトである.

\begin{proof}[\textup{\textbf{補題の証明}}]
    $g(S)$の開被覆$(U_{\lambda})_{\lambda \in \Lambda}$を任意にとる.

    任意の$\lambda \in \Lambda$に対して,$g$は連続であるから,\S 6 :問5より,
    \[
        g^{-1}(U_{\lambda}) =  U_{\lambda}'  \cap S
    \]
    となる開集合$U_{\lambda}'$がとれる.特に$(U_{\lambda}')_{\lambda \in \Lambda}$は$S$の開被覆となる.

    $S$はコンパクトなので,$\lambda_1, \lambda_2, \ldots , \lambda_n \in \Lambda$が存在して
    \[
        S \subset \bigcup_{i=1}^{n} U_{\lambda_i}
    \]
    となり, $S$もコンパクトである.
\end{proof}

\begin{tleftbar}
    \begin{proof}
        $f$は単射なので,$f^{-1} \colon f(K) \mapsto K$がとれる.$\mathbb{R}^n$の開集合$W$をひとつとる.

        $W \cup K^c$は開集合であるから
        \[
            (W \cup K^c)^c = W^c \cap K~(\subset K)
        \]
        は開集合で特にコンパクトである.

        補題より,$f(W^c \cap K)$もコンパクトであり,
        $W$の$f^{-1}$による逆像$W$は
        \[
            W' = (f(W^c \cap K))^c \cap f(K)
        \]
        であり$(f(W^c \cap K))^c$は開集合であるから
        \S 6 :問6より$f^{-1}$は$f(K)$上で連続である.
    \end{proof}
\end{tleftbar}

\newpage

\section*{p72--74:9}
\addcontentsline{toc}{section}{\texorpdfstring{p72--74:9}{p72--74:9}}

\begin{tleftbar}
    \begin{proof}
        $g(x)$が下半連続であるとき,$-g(x)$は上半連続であるので,
        $f(x)$が上半連続であることを示せば十分である.

        $\varepsilon >0$をひとつとる.$ x\in K$に対して,
        \[
            \abs{y-x} < \delta_{(x)} \land y \in K \Rightarrow f(y) < f(x) + \varepsilon
        \]
        となる$\delta_{(x)}$がとれる.

        $(U(x,\delta_{(x)})_{x\in K}$は$K$の開被覆であるので,
        $K$のコンパクト性より,
        \[
            x_1 , x_2, \ldots , x_m \in K , \quad x_1 < x_2 < \cdots < x_m
        \]
        となるようにうまくとると,
        \[
            K \subset \bigcup_{i=1}^m U(x_i,\delta_i) ,\quad f(K) \subset \bigcup_{i=1}^m f(U(x_i,\delta_i))
        \]
        となる.

        $f(x_i)$の$\mathrm{Max}$をとれば,$f(x_i)+\varepsilon$は$f(K)$の上界となるので,
        $f(K)$は上に有界であり,特に$M = \sup f(K)$が存在する.

        $n \geqq 1$に対して,$ M - 1/n < f(x_n) \leqq M$となる$x_n \in K$がとれて,
        $\lim_{n \to \infty} f(x_n) = M$となる.

        $K$は点列コンパクトなので,$(x_n)_{n \in \mathbb{N}}$の収束する部分列$(x_{n_k})_{k \in \mathbb{N}}$をとって
        その極限値を$ x\in K$とする.
        $\lim_{n \to \infty} f(x_{n(k)})=m$,$f$は上半連続であるから
        $\varepsilon >0$に対して,十分大きい$k$で,
        \[
            \abs{M-f(x_{n(k)})} < \varepsilon,\quad f(x_{n(k)})<f(x)+\varepsilon
        \]
        となり,$M-2 \varepsilon < f(x) \leqq M $となる.
        $\varepsilon$は任意にとれるので,$f(x) = M$となり,この$x$で$\mathrm{Max}$に達する.
    \end{proof}
\end{tleftbar}


\section*{p72--74:10}
\addcontentsline{toc}{section}{\texorpdfstring{p72--74:10}{p72--74:10}}

\begin{tleftbar}
    \begin{proof}
        定義域上の任意の点$x$に対して,ある$ \lambda_x \in \Lambda$が存在して,
        \[
            f(x) \leqq f_{\lambda_x} (x) < f(x) + \varepsilon
        \]
        となる.

        定義域上の点$a$を任意に一つ取ると,
        $f_{\lambda_a}$は上半連続なので,適当な$\delta >0$に対して
        \begin{align*}
            \abs{x-a}<\delta & \Rightarrow f_{\lambda_a}(x) < f_{\lambda_a}(a)     \\
                             & \Rightarrow  f_{\lambda_a}(x) < f (a) + \varepsilon \\
                             & \Rightarrow f(x) < f(a) + \varepsilon
        \end{align*}
        となり,$f(x)$は$x=a$で上半連続である.
    \end{proof}
\end{tleftbar}


\section*{p72--74:11}
\addcontentsline{toc}{section}{\texorpdfstring{p72--74:11}{p72--74:11}}

\begin{tleftbar}
    \begin{proof}
        $K$,$L$がコンパクトであり,$f$の像は有界であるので,
        $f(x,y)$を$L$を添字集合とする開集合$f_y (x)$とみなすと,
        \[
            f(x) =\inf_{y \in L} f_y (x)=\min_{y \in L} f(x,y)
        \]
        が定義できる.

        各$f_y$は上半連続であり,問9より特に$\max $を持つ.
        よって,
        \[
            \max_{x \in K} f(x) = \max_{x \in K} \min_{y \in L} f(x,y)
        \]
        が存在する.同様に,$\min_{y \in L} \max_{x \in K} f(x,y)$も存在する.

        また,任意の$(x,y) \in K \times L$に対して,
        \[
            \min_{y \in L} \leqq f(x,y) \leqq \max_{x \in K} f(x,y)
        \]
        が成立するので,特に最左辺の$\max$を与える$x$と最右辺の$\min$を与える$y$に対して
        \[
            \max_{x \in K} \min_{y \in L} f(x,y) \leqq f(x,y) \leqq \min_{y \in L} \max_{x \in K} f(x,y).
        \]
    \end{proof}
\end{tleftbar}

\newpage


\section*{p72--74:12}
\addcontentsline{toc}{section}{\texorpdfstring{p72--74:12}{p72--74:12}}

\begin{tleftbar}
    \begin{proof}
        $(a,b) \in K \times L$が$f$の鞍点であるとき,(i),(ii)より,
        \[
            f(a,b) = \max_{x \in K} f(x,b) = \min_{y \in L} f(a,y)
        \]
        が成立する.

        また特に,
        \[
            \max_{x \in K} \min_{y \in L} f(x,y) \geqq \min_{y \in L} f(a,y) =f(a,b)= \max_{x \in K} f(x,b) \geqq \min_{y \in L} \max_{x \in K} f(x,y)
        \]
        であるので,問11 と併せて
        \[
            \max_{x \in K} \min_{y \in L} f(x,y) = \min_{y \in L} \max_{x \in K} f(x,y)
        \]
        を得る.

        逆に上をみたすとき,
        $\min_{y \in L} f(x,y)$の$K$上での$\max$を与える点を$a$,
        $\max_{x \in K} f(x,y)$の$L$上での$\min$を与える点を$b$とすると,$(a,b)$は(i),(ii)をみたし鞍点となる.
    \end{proof}
\end{tleftbar}

\newpage

\part*{第2章:微分法}
\addcontentsline{toc}{part}{\texorpdfstring{第2章:微分法}{第2章:微分法}}


\section*{p90--91:1}
\addcontentsline{toc}{section}{\texorpdfstring{p90--91:1}{p90--91:1}}


\subsection*{p90--91:1-(i)}
\addcontentsline{toc}{subsection}{\texorpdfstring{p90--91:1-(i)}{p90--91:1-(i)}}


\begin{tikzpicture}
    \draw[->] (-3,0) -- (3,0) node[below] {$x$};
    \draw[->] (0,-3) -- (0,3) node[left]  {$y$};
    \draw (0,0) node[below left] {$\mathrm{O}$} coordinate (O);
    \draw (0,1) node [above left] {$1$};
    \draw (0,-1) node [below left] {$-1$};
    \draw (2,0) node [above left] { $2$};
    \draw (-2,0) node [above left] { $-2$};
    \draw plot[domain=0:{2*pi}, variable=\theta, smooth] ({2*cos (\theta r)},{sin (\theta r)});
\end{tikzpicture}

\subsection*{p90--91:1-(ii)}
\addcontentsline{toc}{subsection}{\texorpdfstring{p90--91:1-(ii)}{p90--91:1-(ii)}}


\begin{tikzpicture}
    \draw[->] (-3,0) -- (3,0) node[below] {$x$};
    \draw[->] (0,-3) -- (0,3) node[left]  {$y$};
    \draw (0,0) node[below left] {$\mathrm{O}$} coordinate (O);
    \draw plot[domain=0:{2*pi}, variable=\theta, smooth] ({cos (2*\theta r)},{sin (\theta r)});
\end{tikzpicture}


\subsection*{p90--91:1-(iii)}
\addcontentsline{toc}{subsection}{\texorpdfstring{p90--91:1-(iii)}{p90--91:1-(iii)}}



\begin{tikzpicture}
    \draw[->] (-3,0) -- (3,0) node[below] {$x$};
    \draw[->] (0,-3) -- (0,3) node[left]  {$y$};
    \draw (0,0) node[below left] {$\mathrm{O}$} coordinate (O);
    \draw plot[domain=0:{2*pi}, variable=\theta, smooth] ({cos (3*\theta r)},{sin (\theta r)});
\end{tikzpicture}


\section*{p90--91:9}
\addcontentsline{toc}{section}{\texorpdfstring{p90--91:9}{p90--91:9}}

\subsection*{p90--91:9-(i)}
\addcontentsline{toc}{subsection}{\texorpdfstring{p90--91:9-(i)}{p90--91:9-(i)}}

\begin{tleftbar}
    \[
        ( \log x )^{(1)}= 1/x , \quad (\log x)^{(2)} = - 1/x^2 , \quad (\log x)^{(3)} = 2/x^3,\quad (\log x)^{(4)} = - 6 /x^4
    \]
    であるから,
    \[
        (\log x)^{(n)} = \frac{(-1)^{n-1}  (n-1)!}{x^n}
    \]
    と推測できる.この推測が正しいことを数学的帰納法により証明する.
    \begin{enumerate}
        \item $n=1$のとき,$(\log x)^{(1)} = 1/x$であり,
              \[
                  \frac{(-1)^{1-1}  (1-1)!}{x^1}=1/x
              \]
              であるから,この場合に推測は正しい.
        \item $n=k$のときに,この推測が正しいと仮定すると,
              \[
                  (\log x)^{(k)} = \frac{(-1)^{k-1}  (k-1)!}{x^k}
              \]
              である.ここで.
              \begin{align*}
                  (\log x)^{(k+1)} & = \left (\frac{(-1)^{k-1}  (k-1)!}{x^k} \right ) ' \\
                                   & = \frac{(-1)^k  k!}{x^{k+1}}
              \end{align*}
              であるから,$n=k+1$のときも推測は正しい.
    \end{enumerate}
    (1),(2)より,
    \[
        (\log x)^{(n)} = \frac{(-1)^{n-1}  (n-1)!}{x^n}
    \]
    である.
\end{tleftbar}

\begin{tleftbar}
    \[
        \left(   \frac{1}{x^2+3x+2} \right)^{(n)} = (-1)^n n! \{ (x+1)^{-n-1} - (x+2)^{-n-1} \}
    \]
\end{tleftbar}



\section*{p90--91:10}
\addcontentsline{toc}{section}{\texorpdfstring{p90--91:10}{p90--91:10}}

\begin{tleftbar}
    \begin{proof}
        $u(x)= (x^2-1)^n$とおく,このとき,
        \[
            U'(x)= 2x n(x^2-1)^{n-1}
        \]
        だから,
        \[
            (x^2-1) u'(x)=2nx \cdot u(x)
        \]
        この両辺を$(n+1)$回微分して,
        \begin{align*}
             & (x^2-1)u^{(n+2)}(x)+2(n+1)x u^{(n+1)} x + \frac{(n+1)n}{2} \cdot u^{(n)} (x) = 2nx u^{(n+1)}(x) + 2(n+1) n u^{(n)}(x) \\
             & \therefore ~(x^2-1)u^{(n+2)}(x) + 2n u^{(n+1)}(x)-(n+1)n u^{(n)}(x)=0
        \end{align*}
        ここで,$ u^{(n)} (x)= 2^n n! P_n (x)$なので,
        \begin{align*}
             & (x^2 -1) \{ 2^n n! P_n ''(x) \} +2x \{ 2^n n! P_n (x) \} -n(n+1) \{ 2^n n P_n(x) \} =0 \\
             & \therefore ~ (x^2-1) P_n ''(x)+2x P_n '(x) -n(n+1) P_n (x)=0
        \end{align*}
    \end{proof}
\end{tleftbar}

\newpage
\section*{p106--107:1}
\addcontentsline{toc}{section}{\texorpdfstring{p106--107:1}{p106--107:1}}

\kakko{補題}

\[
    Q_k (x)= (n! 2^n)^{-1} \frac{d^k}{dx^k} (x^2-1)^n \quad (0 \leqq k \leqq n)
\]
とおく.$0 \leqq k <n$のとき,$Q_k (x)$は$(x^2-1)^{n-k}$で割り切られ.
\[
    Q_k (1)= Q_k(-1)=0
\]
である.

\begin{proof}
    $k$に関する帰納法により示す.$k=0$のときは
    \[
        Q_0(x)= (n! 2^n)^{-1} (x^2-1)^n
    \]
    であり,$Q_o(X)$は$(x^2-1)^n$で割り切れ,$Q_0(1)=Q_0(-1)=0$である.

    $k>0$として,$k-1$の場合の主張の成立を仮定すると,$Q_{k-1}(x)$は$x$についての式$g(x)$を用いて,
    \[
        Q_{k-1} = g(x)(x^2-1)^{n-k+1}
    \]
    と表せ,
    \begin{align*}
        Q_k (x) & =Q_{k-1}' (x)                                        \\
                & = g'(x)(x^2-1)^{n-k+1} + g(x) (x^2-1)^{n-k} \cdot 2x \\
                & =(g'(x)(x^2-1)+g(x) \cdot 2x) (x^2-1)^{n-k}
    \end{align*}
    したがって,$Q_k(x)$は$(x^2-1)^{n-k}$で割り切れ,$n-k >0$より,
    \[
        Q_k (1)= Q_k(-1)=0
    \]
    となり,これが証明すべきことであった.
\end{proof}

\begin{leftbar}
    \begin{proof}
        $Q_k (x)$が$(-1,1)$上に$k$個の相異なる根を持つことを,$k$に関する数学的帰納法で示す.
        \begin{enumerate}[(I)]
            \item $k=0$の場合
                  \[
                      Q_0(x)= (n! 2^n)^{-1} (x^2-1)^n
                  \]
                  なので,$(-1,1)$に根を持たない.
            \item $k>0$として,$k-1$の場合の主張の成立を仮定すると,$Q_{k-1}(x)$は$(-1,1)$に$k-1$個の相異なる根を持つから,
                  それらを
                  \[
                      x_1,x_2,\dots,x_{k-1}\quad (-1<x_1<x_2<\dots <x_{k-1}<1)
                  \]とおく.
                  ここで,$Q_{k-1} (x)$は閉区間$[-1,1]$で微分可能なので,
                  各区間$[-1,x_1],[x_1,x_2],\dots,[x_{k-1},1]$においても微分可能であり,補題により,
                  \[
                      Q_{k-1} (-1) = Q_{k-1}(x_1)=\dots = Q_{k-1}(x_{k-1})=Q_{k-1}(1)=0
                  \]
                  である.したがって各小区間においてロルの定理が適用できて,$Q_{k-1}' (x)=Q_k(x)$は$(-1,x_1),(x_1,x_2),\dots,(x_{k-1},1)$において根をもつ.
        \end{enumerate}
        以上(I),(II)の議論により,$Q_k (x)$は$(-1,1)$において相異なる$k$個の根をもつ.
        特に$k=n$のとき,$Q_n (x)=P_n(x)$で$P_n (x)$は$(-1,1)$において相異なる$n$個の根をもつ.
    \end{proof}
\end{leftbar}

\newpage

\section*{p106--107:3}
\addcontentsline{toc}{section}{\texorpdfstring{p106--107:3}{p106--107:3}}

\begin{tleftbar}
    \begin{proof}
        まず,$f(a)=0$, $g(a)=0$と定義しておく.$0<\delta <b-a$を満たす$\delta$をとると,$f$と$g$は$[a,a+\delta]$で連続であり,
        $(a,a+\delta)$で微分可能である.いま$g(x) \ne 0$~($ x \in (a,a+\delta)$)であるから,
        コーシーの平均値の定理により,
        \[
            \frac{f(x)-f(a)}{g(x)-g(a)} = \frac{f'(c)}{g'(c)}
        \]
        を満たす$c \in (a,x)$が存在する.いま$f(a)=0$,$g(a)=0$と定義したので,
        \[
            \frac{f(x)}{g(x)} = \frac{f'(c)}{g'(c)}
        \]
        である.$ x \to a+0$のとき,$c \to a+0$であるから,仮定により,
        \[
            \lim_{x \to a+0} \frac{f(x)}{g(x)} = \lim_{c \to a+0} \frac{f'(c)}{g'(c)} =l
        \]
        である.$ x \to a-0$のときは同様にすればよく,$ x \to a$の場合はこれら二つの考察により結果が従う.

        $ x \to +\infty$のときは$ t=1/x$として$f(t)$と$g(t)$の極限を考えれば同じ結論に帰着する.

        以上の議論により証明された.
    \end{proof}
\end{tleftbar}

\newpage



\section*{p106--107:8}
\addcontentsline{toc}{section}{\texorpdfstring{p106--107:8}{p106--107:8}}


\subsection*{p106--107:8-(i)}
\addcontentsline{toc}{subsection}{\texorpdfstring{p106--107:8-(i)}{p106--107:8-(i)}}

\begin{tleftbar}
    \[
        f(x)=3x^4 -8x^3+6x^2
    \]
    とおくと,
    \begin{align*}
         & f'(x) =12x^3 -24x^2+12x = 12x(x-1)^2, \\
         & f''(x)=36x^2-48x+12 = 12(3x-1)(x-1)
    \end{align*}
    であるから,増減表は以下のようになる.

    \vspace{2mm}

    \begin{tabular}{|c||ccccccc|}
        \hline
        $x$      & $\cdots$ & $0$ & $\cdots$ & $1/3$ & $\cdots$ & $1$ & $\cdots$ \\
        \hline
        $f'(x)$  & $-$      & $0$ & $+$      & $+$   & $+$      & $0$ & $+$      \\
        \hline
        $f''(x)$ & $+$      & $+$ & $+$      & $0$   & $-$      & $0$ & $+$      \\
        \hline
        $f(x)$   & \ser     &     & \ner     &       & \nel     &     & \ner     \\
        \hline
    \end{tabular}

    \vspace{2mm}

    \begin{tikzpicture}[scale = 3]
        \draw[->,>=stealth,semithick](-0.3,0)--(1.3,0)node[above]{$x$};%x軸
        \draw[->,>=stealth,semithick](0,-0.2)--(0,1.1)node[right]{$y$};%y軸
        \draw(0,0)node[below right]{O};%原点
        \draw[domain=-0.2:1.3,samples=100]plot(\x,{3*pow(\x,4)-8*pow(\x,3)+6*pow(\x,2)})node[right]{$y=3x^4-8x^3+6x^2$};
    \end{tikzpicture}
\end{tleftbar}

\newpage

\subsection*{p106--107:8-(ii)}
\addcontentsline{toc}{subsection}{\texorpdfstring{p106--107:8-(ii)}{p106--107:8-(ii)}}

\begin{tleftbar}
    $x>0$のもとで,
    \[
        f(x)=x^\frac{1}{x}
    \]
    とおく.ここで,
    \[
        f(x)=\exp(\log x^{1/x}) = \exp (\log x/x)
    \]
    と変形できるので,
    \begin{align*}
         & f'(x) = \exp (\log x /x) \left (\frac{1-\log x}{x^2} \right) ,                  \\
         & f''(x) = \exp (\log x /x) \left (\frac{(\log x -1)^2 -3x+2x\log x}{x^4} \right)
    \end{align*}
    であり,増減表は以下のようになる.ただし$\alpha$,$\beta$は$(\log x -1)^2 -3x+2x\log x =0$の$2$解である.
    \vspace{2mm}

    \begin{tabular}{|c||cccccccc|}
        \hline
        $x$      & $0$ & $\cdots$ & $\alpha$ & $\cdots$ & $e$ & $\cdots$ & $\beta $ & $\cdots$ \\
        \hline
        $f'(x)$  &     & $+$      &          & $+$      & $0$ & $-$      &          & $-$      \\
        \hline
        $f''(x)$ &     & $+$      & $0$      & $-$      &     & $-$      & $0$      & $+$      \\
        \hline
        $f(x)$   &     & \ner     &          & \nel     &     & \sel     &          & \ser     \\
        \hline
    \end{tabular}

    \vspace{2mm}

    \begin{tikzpicture}[scale = 2]
        \draw[->,>=stealth,semithick](-0.1,0)--(3,0)node[above]{$x$};%x軸
        \draw[->,>=stealth,semithick](0,-0.1)--(0,3)node[right]{$y$};%y軸
        \draw(0,0)node[below right]{O};%原点
        \draw[domain=0.1:4,samples=100]plot(\x,{pow(\x,1/\x)}) node[right]{$y=x^{1/x}$};
    \end{tikzpicture}

\end{tleftbar}

\newpage

\section*{p106--107:10}
\addcontentsline{toc}{section}{\texorpdfstring{p106--107:10}{p106--107:10}}



\begin{tleftbar}
    \begin{proof}
        3つのことを証明する.
        \begin{description}
            \item[a)とb)が同値であること] \mbox{} \par
                  $a < x \leqq y$または$y \leqq x <a$に対して,$0 \leqq t < 1$を用いて,
                  \[
                      x=ta+(1-t)y
                  \]
                  とおく.$a < x \leqq y$とする.このとき,
                  \begin{equation}
                      \label{eq:p107 10) 1}
                      t = \frac{x-y}{a-y}
                  \end{equation}
                  と表せることはよい.

                  また,$f$は$I$で凸であり,これは
                  \begin{equation}
                      \label{eq:p107 10) 2}
                      f(x)=tf(x)+(1-t)f(x) < tf(a)+(1-t)f(y)
                  \end{equation}
                  と同値である.$x<y$のとき\eqref{eq:p107 10) 2}に\eqref{eq:p107 10) 1}の$t$の値を代入すれば,
                  \begin{equation}
                      \label{eq:p107 10) 3}
                      \frac{f(x)-f(a)}{x-a} < \frac{f(y)-f(x)}{y-x}
                  \end{equation}
                  を得る.$x=y$のときは明らか.
            \item [a)とc)が同値であること] \mbox{} \par
                  c)を仮定する.\eqref{eq:p107 10) 3}の左辺について,平均値の定理により,
                  \[
                      \frac{f(x)-f(a)}{x-a}=f'(\xi)
                  \]
                  をみたす$\xi ~(a<\xi <x)$が存在する.同様に,\eqref{eq:p107 10) 3}の右辺について,平均値の定理により,
                  \[
                      \frac{f(y)-f(x)}{y-x}=f'(\eta)
                  \]
                  をみたす$\eta~ (x<\eta < y)$が存在する.仮定により,$f'(\xi)\leqq f'(\eta)$だから,\eqref{eq:p107 10) 3}が成り立ち,a)が従う.

                  a),すなわち\eqref{eq:p107 10) 3}を仮定する.
                  左辺について,$x \to + a$とすれば,これは$f'(a)$に収束し,右辺は,
                  これは$(f(y)-f(a))/(y-a)$に収束する(これを$\alpha$とおく.).$f'(a) \leqq \alpha$となることはよい.
                  また,$x \to - y$とすれば,左辺は$\alpha$に,右辺は$f'(y)$に収束する.$\alpha \leqq f'(y)$となることもよい.

                  よって,a)とc)は同値である.

            \item[a)とd)が同値であること] \mbox{} \par
                  d)を仮定する.これは$f'(x)$は$I$上で単調増加であることと同値である.
        \end{description}
        以上の議論により,示された.
    \end{proof}
\end{tleftbar}


\section*{p126:2}
\addcontentsline{toc}{section}{\texorpdfstring{p126:2}{p126:2}}


\begin{tleftbar}
    \begin{proof}
        \begin{description}
            \item[必要条件であること]
                  $ f $ が $ x=0 $ で微分可能であると仮定する.この条件は次のように書ける:
                  \[
                      f(x) - f(0) = c \cdot x + \varepsilon(x)
                  \]
                  ここで,$ c = (c_1, \ldots, c_n) $ は $ f $ の $ x=0 $ における勾配ベクトルであり,$ \varepsilon(x) $ は
                  $\lim_{x \to 0} \varepsilon(x)/\abs{x} = 0$をみたす関数である.このとき,各 $ i $ について次のように $ g_i(x) $ を定義する:
                  \[
                      g_i(x) =
                      \begin{cases}
                          c_i + (\varepsilon(x) x_i)/\abs{x}^2 & (x \ne 0) \\
                          c_i                                  & (x = 0)
                      \end{cases}
                  \]

                  この $ g_i(x) $ が $ x=0 $ で連続であることを示すために,$ x \to 0 $ の極限を考える.$ x \ne 0 $ のとき,
                  \[
                      \lim_{x \to 0} g_i(x) = \lim_{x \to 0} \left( c_i + \frac{\varepsilon(x) x_i}{|x|^2} \right) = c_i + \lim_{x \to 0} \frac{\varepsilon(x) x_i}{|x|^2}.
                  \]
                  ここで,$\varepsilon(x)/\abs{x} \to 0 $ なので,
                  \[
                      \abs{\frac{\varepsilon(x) x_i}{\abs{x}^2}} = \abs{\frac{\varepsilon(x)}{\abs{x}} \cdot \frac{x_i}{|x|}} \leqq  \abs{\frac{\varepsilon(x)}{\abs{x}} \cdot 1} = \abs{\frac{\varepsilon(x)}{\abs{x}}} \to 0 \quad (x \to 0).
                  \]
                  したがって,$\lim_{x \to 0} g_i(x) = c_i$なので,$ g_i(x) $ は $ x=0 $ で連続である.よって
                  \[
                      f(x) = f(0) + \sum_{i=1}^n x_i g_i(x).
                  \]
            \item [十分条件であること]
                  ある連続な関数 $ g_i \colon \mathbb{R}^n \to \mathbb{R} $ が存在して,次のように表されるとする:
                  \[
                      f(x) = f(0) + \sum_{i=1}^n x_i g_i(x)
                  \]
                  このとき,$ g_i(x) $ が $ x=0 $ で連続であるため,各 $ i $ について $ g_i(0) $ が存在し,次が成り立つ:
                  \[
                      f(x) - f(0) = \sum_{i=1}^n x_i g_i(0) + \sum_{i=1}^n x_i (g_i(x) - g_i(0)).
                  \]
                  ここで,$ \varepsilon(x) = \sum_{i=1}^n x_i (g_i(x) - g_i(0)) $ とおくと,
                  \[
                      \lim_{x \to 0} \frac{\varepsilon(x)}{\abs{x}} = \lim_{x \to 0} \frac{\sum_{i=1}^n x_i (g_i(x) - g_i(0))}{\abs{x}}.
                  \]
                  三角不等式を用いると,
                  \[
                      \abs{ \frac{\varepsilon(x)}{\abs{x}} } \leqq \sum_{i=1}^n \abs{ \frac{x_i(g_i(x) - g_i(0))}{\abs{x}} }.
                  \]

                  $ g_i(x) $ が $ x=0 $ で連続であるため,$ g_i(x) \to g_i(0) $ となる.したがって,各$ i $について
                  \[
                      \abs{ \frac{g_i(x) - g_i(0)}{\abs{x}} } \to 0 \quad (x \to 0).
                  \]
                  ゆえに,$\lim_{x \to 0} \varepsilon(x)/\abs{x} = 0$となり,$ f $ は $ x=0 $ で微分可能である.
        \end{description}
    \end{proof}
\end{tleftbar}

\part*{第3章:初等函数}
\addcontentsline{toc}{part}{\texorpdfstring{第3章:初等函数}{第3章:初等函数}}


\section*{p191--193:1}
\addcontentsline{toc}{section}{\texorpdfstring{p191--193:1}{p191--193:1}}


\begin{tleftbar}
    \begin{proof}
        $m,n \in \mathbb{N},~n >0$とし,$e=\frac{m}{n}$と表される,すなわち$e$が有理数だと仮定する.このとき,与えられた式を変形して,
        \[
            \frac{e^\theta}{(n+1)!} = e-\sum_{k=0}^{n} \frac{1}{k!} =\frac{m}{n}-\sum_{k=0}^{n} \frac{1}{k!}
        \]
        とする.これにより.
        \[
            \frac{e^{\theta}}{n+1} = m \cdot n! - \sum_{k=0}^{n} \frac{n!}{k!}
        \]
        であり,$m \cdot n! - \sum_{k=0}^{n} \frac{n!}{k!} \in \mathbb{Z}$であるから,$\frac{e^{\theta}}{n+1} \in \mathbb{Z}$である.
        よって,$0< \theta <1,~2<e<3$とあわせて,
        \[
            1 \le \frac{e^{\theta}}{n+1} < \frac{3}{n+1}
        \]
        であり,$\frac{e^{\theta}}{n+1} \in \mathbb{Z}$であるから$n=1$となる.ゆえに$e=m$となり,$e$は整数である.しかし$2<e<3$であるから,これは矛盾である.よって先の仮定が誤りであり,$e$は無理数である.
    \end{proof}
\end{tleftbar}
\newpage

\part*{第4章:積分法}
\addcontentsline{toc}{part}{\texorpdfstring{第4章:積分法}{第4章:積分法}}



\section*{p211-212:1}
\addcontentsline{toc}{section}{\texorpdfstring{p211-212:1}{p211-212:1}}


\begin{tleftbar}
    \begin{proof}
        $f$は区間$I$内の$m$個の点$x = a_1, a_2, \ldots, a_m$でのみ$f(x) \ne 0$であるとする.
        この条件のもとで,
        \[
            M = \max_{1 \leqq k \leqq m} \abs{f(a_k)}
        \]
        と定義する.また任意の分割 $\Delta$ に対して,
        \[
            s(f; \Delta; \xi) = \sum_{k \in K(\Delta)} f(\xi_k) v(I_k)
        \]
        である.ここで、$f(\xi_k) \ne 0$である項は高々$m$個であり,その個数を$m_0$とする.
        このような項のみを考慮すると,リーマン和は次のように表せる:
        \[
            s(f; \Delta; \xi) = \sum_{i=1}^{m_0} f(\xi_{k_i}) v(I_{k_i})
        \]
        ここで,$i = 1, 2, \ldots, m_0$は$f(\xi_k) \ne 0$である区間$ I_{k_i} $を走るものとする.

        次に,先に定義した$M$を用いると,
        \[
            \abs{s(f; \Delta; \xi)} = \abs{\sum_{i=1}^{m_0} f(\xi_{k_i}) v(I_{k_i})} \leqq M \sum_{i=1}^{m_0} v(I_{k_i})
        \]
        を得る.さて,$d(\Delta)$を分割$\Delta$の直径とすると,各区間の体積に関して$ v(I_{k_i}) \leqq d(\Delta)^n $だから,
        \[
            \abs{s(f; \Delta; \xi)} \leqq M \sum_{i=1}^{m_0} d(\Delta)^n = m_0 M d(\Delta)^n
        \]
        となる.ここで,リーマン和に寄与する項は$m_0$個であることを用いた.

        ここで,$d(\Delta) \to 0$ とすると,$d(\Delta)^n \to 0$ であるため,
        \[
            \abs{s(f; \Delta; \xi)} \to 0
        \]
        となる.したがって,$f$は区間$I$上でリーマン積分可能であり,
        \[
            \int_{I} f(x)\, dx = 0.
        \]
    \end{proof}
\end{tleftbar}

\section*{p211-212:2}
\addcontentsline{toc}{section}{\texorpdfstring{p211-212:2}{p211-212:2}}


\begin{tleftbar}
    \begin{proof}
        まず,$f$は$I$上で可積分であることから,任意の$\varepsilon >0$に対して,分割$\Delta$が存在して
        \[
            \abs{s(f; \Delta; \xi) - J} < \varepsilon.
        \]
        が成立する.ここで$J = \int_I f$とおいた.ここでとくに,$1 >0$に対して,ある$\Delta$が存在し,
        \[
            \abs{ s(f; \Delta; \xi) - J } < 1
        \]
        が成立する.

        次に,分割$\Delta$のある区間$I_k$に対して代表点$\xi_k$を固定し,$k \ne m \in K(\Delta)$を満たす他の代表点を考え,さらに
        \[
            B = \sum_{k \ne m , m \in K(\Delta)} f(\xi_m) v(I_m)
        \]
        と定義する.このとき
        \[
            S(f; \Delta; \xi) = f(\xi_k) v(I_k) + B.
        \]
        ここで,先ほどの不等式$\abs{ s(f; \Delta; \xi) - J } < 1$を適用すると,
        \[
            \abs{ f(\xi_k) v(I_k) + B - J } < 1
        \]
        が得られ,これを整理すると
        \[
            -1 < f(\xi_k) v(I_k) + B - J < 1.
        \]
        つまり,
        \[
            J - 1 - B < f(\xi_k) v(I_k) < J + 1 - B.
        \]
        ここで,$v(I_k)$が$0$ではないことを考慮して両辺を$v(I_k)$で割ると,
        \[
            \frac{J - 1 - B}{v(I_k)} < f(\xi_k) < \frac{J + 1 - B}{v(I_k)}.
        \]
        したがって,$f(\xi_k)$は区間$I_k$上で有界であることがわかる.

        任意の区間$I_k$に対して同様の議論が適用できるため,関数$f$は区間$I$全体で有界であることが結論づけられる.

        以上により,$f$が$I$上で有界であることが証明された.
    \end{proof}
\end{tleftbar}


\section*{p211-212:3}
\addcontentsline{toc}{section}{\texorpdfstring{p211-212:3}{p211-212:3}}


\subsection*{p211-212:3-(i)}
\addcontentsline{toc}{subsection}{\texorpdfstring{p211-212:3-(i)}{p211-212:3-(i)}}

\begin{tleftbar}
    $f$は$I$で単調に増加するので,
    \[
        \zeta_k = x_{k-1} \leqq \xi_k \leqq x_k=\eta_k
    \]
    とすると,リーマン和に関して
    \[
        s(f;\Delta ; \zeta_k ) \leqq s(f;\Delta;\xi_k) \leqq s(f;\Delta;\eta_k).
    \]

    また,$\xi_k' = (x_k+x_{k-1} )/2 $とおくと,

    \begin{align*}
        s(f;\Delta,\xi_k' ) & = \sum_{k=1}^{n} f(\xi_k ') v(I_k)                          \\
                            & =  \sum_{k=1}^{n} \frac{1}{2} (x_k +x_{k-1})(x_k - x_{k-1}) \\
                            & = \frac{a^2}{2}
    \end{align*}
    であり,これを$J$とおくと,
    \begin{align*}
        0 & \leqq \abs{s(f;\Delta; \xi_k)-J }                 \\
          & \leqq  s(f;\Delta;\eta_k) -s(f;\Delta ; \zeta_k ) \\
          & = \sum_{k=1}^{n} (x_k-x_{k-1} ) v(I_k)            \\
          & = d(\Delta) \sum_{k=1}^{n} (x_k-x_{k-1} )         \\
          & = (a-0) d(\Delta)=a d(\Delta).
    \end{align*}
    ゆえに$d (\Delta) \to 0$としたときに$a d(\Delta) \to 0$となり,
    \[
        \lim_{d(\Delta) \to 0} s(f;\Delta; \xi_k)=J.
    \]
    よって$f$は$I$でリーマン積分可能となり,$\int_{I} f = \frac{a^2}{2}$である.
\end{tleftbar}


\subsection*{p211-212:3-(ii)}
\addcontentsline{toc}{subsection}{\texorpdfstring{p211-212:3-(ii)}{p211-212:3-(ii)}}
\begin{tleftbar}
    $f$は$I$で単調に増加するので,
    \[
        \zeta_k = x_{k-1} \leqq \xi_k \leqq x_k=\eta_k
    \]
    とすると,リーマン和に関して
    \[
        s(f;\Delta ; \zeta_k ) \leqq s(f;\Delta;\xi_k) \leqq s(f;\Delta;\eta_k).
    \]
    また,$\xi_k' = \sqrt{(x_k ^2 + x_k x_{k-1}+x_{k-1} ^2)/3} $とおくと,
    \begin{align*}
        s(f;\Delta,\xi_k' ) & = \sum_{k=1}^{n} f(\xi_k') v(I_k)                                                         \\
                            & = \sum_{k=1}^{n} (\xi_k')^2 (x_k - x_{k-1})                                               \\
                            & = \sum_{k=1}^{n} \left( \frac{x_k^2 + x_k x_{k-1} + x_{k-1}^2}{3} \right) (x_k - x_{k-1}) \\
                            & = \frac{1}{3} \sum_{k=1}^{n} (x_k^2 + x_k x_{k-1} + x_{k-1}^2) (x_k - x_{k-1})            \\
                            & = \frac{a^3}{3}.
    \end{align*}
    これを$J$とおくと,
    \begin{align*}
        0 & \leqq \abs{ s(f;\Delta; \xi_k) - J }                                                \\
          & \leqq s(f;\Delta;\eta_k) - s(f;\Delta ; \zeta_k )                                   \\
          & = \sum_{k=1}^{n} f(x_k) (x_k - x_{k-1}) - \sum_{k=1}^{n} f(x_{k-1}) (x_k - x_{k-1}) \\
          & = \sum_{k=1}^{n} x_k^2 (x_k - x_{k-1}) - \sum_{k=1}^{n} x_{k-1}^2 (x_k - x_{k-1})   \\
          & = \sum_{k=1}^{n} (x_k^2 - x_{k-1}^2) (x_k - x_{k-1})                                \\
          & = \sum_{k=1}^{n} (x_k - x_{k-1})^2 (x_k + x_{k-1})                                  \\
          & \leqq d(\Delta) \sum_{k=1}^{n} (x_k - x_{k-1}) (x_k + x_{k-1}).
    \end{align*}
    各区間$(x_{k-1},x_k)$において,$x_k + x_{k+1}$の最大値は$2a$であるから,
    \begin{align*}
        d(\Delta) \sum_{k=1}^{n} (x_k - x_{k-1}) (x_k + x_{k-1}) & \leqq d(\Delta) \sum_{k=1}^{n} 2a (x_k - x_{k-1}) \\
                                                                 & = 2a d(\Delta) \sum_{k=1}^{n} (x_k - x_{k-1})     \\
                                                                 & = 2a d(\Delta) (a - 0) = 2a^2 d(\Delta).
    \end{align*}
    ゆえに$d (\Delta) \to 0$としたときに$2a^2 d(\Delta) \to 0$となり,
    \[
        \lim_{d(\Delta) \to 0} s(f;\Delta; \xi_k)=J.
    \]
    よって$f$は$I$でリーマン積分可能となり,$\int_{I} f = \frac{a^3}{3}$である.
\end{tleftbar}


\section*{p239:1}
\addcontentsline{toc}{section}{\texorpdfstring{p239:1}{p239:1}}

\subsection*{p239:1-(i)}
\addcontentsline{toc}{subsection}{\texorpdfstring{p239:1-(i)}{p239:1-(i)}}

\begin{screen}
    $\tan \frac{x}{2}=t$とおくと,
    \begin{align*}
        \int_{0}^{\frac{\pi}{2}} \cfrac{\sin x}{1+\cos x} \, dx & = \int_{0}^{1} \cfrac{\cfrac{2t}{1+t^2}}{1+\cfrac{1-t^2}{1+t^2}} \cdot \cfrac{2}{1+t^2} \, dt \\
                                                                & = \int_{0}^{1} \frac{2t}{1+t^2} \, dt                                                         \\
                                                                & = \Bigl[\log (t^2+1)\Bigl]_{0}^{1}                                                            \\
                                                                & = \log 2-0 = \log 2
    \end{align*}
\end{screen}


\subsection*{p239:1-(ii)}
\addcontentsline{toc}{subsection}{\texorpdfstring{p239:1-(ii)}{p239:1-(ii)}}

\begin{screen}
    $x-a=a \sin \theta$ ($-\pi \le \theta < \pi$)とする置換を用いる.
    \begin{align*}
        \int_{0}^{a} \sqrt{2ax-x^2} \, dx & = \int_{0}^{a} \sqrt{-(x-a)^2+a^2} \, dx                                                 \\
                                          & = \int_{-\frac{\pi}{2}}^{0} a \sqrt{1-\sin ^2 \theta } \cdot a\cos \theta \, d \theta    \\
                                          & = \int_{-\frac{\pi}{2}}^{0} a \abs{\cos \theta} \cdot a\cos \theta \, d \theta           \\
                                          & = \int_{-\frac{\pi}{2}}^{0} a^2 \cos^2 \theta \, d \theta                                \\
                                          & = \int_{-\frac{\pi}{2}}^{0} a^2 \left (\frac{1+\cos 2 \theta }{2}\right) \, d \theta     \\
                                          & = \frac{1}{2} a^2 \left [\theta + \frac{1}{2}\sin 2 \theta \right ]_{-\frac{\pi}{2}}^{0} \\
                                          & = \frac{\pi a^2}{4}
    \end{align*}
\end{screen}


\subsection*{p239:1-(iii)}
\addcontentsline{toc}{subsection}{\texorpdfstring{p239:1-(iii)}{p239:1-(iii)}}


\begin{screen}
    \begin{align*}
        \abs{\sin 2 \theta} =
        \begin{cases}
            \sin 2 \theta   & (0 \le \theta < \frac{\pi}{2} のとき)    \\
            - \sin 2 \theta & (\frac{\pi}{2}\le \theta \le \pi のとき)
        \end{cases}
    \end{align*}
    なので,
    \begin{align*}
        \int_{0}^{\pi} \abs{\sin 2 \theta} \, d \theta & = \int_{0}^{\frac{\pi}{2}} \sin 2 \theta \, d \theta +\int_{\frac{\pi}{2}}^{\pi} (-\sin 2 \theta) \, d \theta              \\
                                                       & = \left [-\frac{\cos 2 \theta}{2}\right]_{0}^{\frac{\pi}{2}} + \left [\frac{\cos 2 \theta}{2}\right]_{\frac{\pi}{2}}^{\pi} \\
                                                       & = -\frac{(-1-1)}{2} + \frac{1+1}{2} = 2
    \end{align*}
\end{screen}


\subsection*{p239:1-(iv)}
\addcontentsline{toc}{subsection}{\texorpdfstring{p239:1-(iv)}{p239:1-(iv)}}
\begin{screen}
    \begin{align*}
        \int_{0}^{\pi} e^{inx} \, dx  =
        \begin{cases}
            2 \pi & (n=0 のとき)                             \\
            0     & (n \in \mathbb{Z}\setminus \{0\} のとき)
        \end{cases}
    \end{align*}
    である.$n=0$のときは
    \begin{align*}
        \int_{0}^{2 \pi} e^{inx} \, dx & = \int_{0}^{2\pi} \, dx           \\
                                       & = \Bigl[x\Bigl]_{0}^{2\pi} = 2\pi
    \end{align*}
    となり,$n \ne 0$のときは
    \begin{align*}
        \int_{0}^{2\pi} e^{inx} \, dx & = \left [\frac{e^{inx}}{in} \right ]_{0}^{2\pi} \\
                                      & = \frac{1}{in} (1-1)=0
    \end{align*}
    となる.
\end{screen}


\subsection*{p239:1-(v)}
\addcontentsline{toc}{subsection}{\texorpdfstring{p239:1-(v)}{p239:1-(v)}}

\begin{screen}
    $m=n$のとき,
    \begin{align*}
        \int_{0}^{2\pi} \cos m x \sin nx \, dx & = \int_{0}^{2\pi} \cos mx \sin mx \, dx                            \\
                                               & = \int_{0}^{2\pi} \left (\frac{\sin 2mx + \sin 0}{2}\right ) \, dx \\
                                               & = \left [-\frac{\cos 2mx}{2m}\right ]_{0}^{2\pi} =0
    \end{align*}
    となる.$m \ne n$のとき,
    \begin{align*}
        \int_{0}^{2\pi} \cos mx \sin nx \, dx & = \int_{0}^{2\pi} \left (\frac{\sin (m+n)x + \sin (n-m)x}{2}\right) \, dx                 \\
                                              & = \frac{1}{2}\left [\frac{\sin (m+n)x}{m+n}+\frac{\sin (n-m)x}{n-m} \right]_{0}^{2\pi} =0
    \end{align*}
    である,ここまでの議論と,$m$と$n$の対称性により,
    \[
        \int_{0}^{2\pi} \cos mx \sin nx \, dx =\int_{0}^{2\pi} \sin mx \cos nx \, dx =0 ~(n,m \in \mathbb{N})
    \]
    となる.
\end{screen}


\subsection*{p239:1-(vi)}
\addcontentsline{toc}{subsection}{\texorpdfstring{p239:1-(vi)}{p239:1-(vi)}}
\begin{screen}
    $m \ne n$のとき,
    \begin{align*}
        \int_{0}^{2\pi} \cos mx \cos nx \, dx & = \int_{0}^{2\pi} \left (\frac{\cos(m+n)x+\cos(n-m)x}{2}\right) \, dx                 \\
                                              & = \frac{1}{2} \left [\frac{\sin (m+n)x}{m+n}+\frac{\sin(n-m)x}{n-m}\right]_{0}^{2\pi} \\
                                              & = 0-0 =0
    \end{align*}
    となる.$m =n \ne 0$のとき,
    \begin{align*}
        \int_{0}^{2\pi} \cos mx \cos nx \, dx & = \int_{0}^{2\pi} \cos^2 mx \, dx                                \\
                                              & = \int_{0}^{2\pi} \left (\frac{1+\cos 2mx}{2}\right) \, dx       \\
                                              & = \left [\frac{x}{2}+\frac{\sin 2mx}{4m}\right]_{0}^{2\pi} = \pi
    \end{align*}
    となる.$m =n =0$のとき,
    \begin{align*}
        \int_{0}^{2\pi} \cos mx \cos nx \, dx & = \int_{0}^{2\pi} dx    \\
                                              & = [x]_{0}^{2\pi} = 2\pi
    \end{align*}
    となる,以上をまとめて,
    \begin{align*}
        \int_{0}^{2\pi} \cos mx \cos nx \, dx =
        \begin{cases}
            0     & (m \ne n のとき)   \\
            \pi   & (m = n\ne 0のとき) \\
            2 \pi & (m=n=0 のとき)
        \end{cases}
    \end{align*}
    である.
\end{screen}

\section*{p239:3}
\addcontentsline{toc}{section}{\texorpdfstring{p239:3}{p239:3}}

\subsection*{p239:3-(i)}
\addcontentsline{toc}{subsection}{\texorpdfstring{p239:3-(i)}{p239:3-(i)}}

\begin{tleftbar}
    部分積分法を用いると,
    \begin{align*}
        \int x^\alpha \log x \, dx & = \frac{x^{\alpha +1} \log x}{\alpha +1} - \int \frac{x^{\alpha +1}}{\alpha+1} \cdot \frac{1}{x} \, dx \\
                                   & = \frac{x^{\alpha +1} \log x}{\alpha +1}- \frac{1}{\alpha +1} \int x^{\alpha} \, dx                    \\
                                   & = \frac{x^{\alpha+1} \log x}{\alpha +1} - \frac{x^{\alpha +1}}{(\alpha +1)^2}+ C
    \end{align*}
    となり,これが答えである.
\end{tleftbar}

\subsection*{p239:3-(ii)}
\addcontentsline{toc}{subsection}{\texorpdfstring{p239:3-(ii)}{p239:3-(ii)}}

\begin{tleftbar}
    部分積分法を用いると,
    \begin{align*}
        \int x^n e^x \, dx & = x^n e^x - n \int x^{n-1} e^x \, dx                                   \\
                           & = x^n e^x - n x^{n-1} e^x + (n-1)\int x^{n-2} e^x \, dx                \\
                           & = \cdots = x^n e^x - n x^{n-1} e^x + \cdots + (-1)^n n! \int e^x \, dx \\
                           & = e^x (x^n -n x^{n-1}+ \cdots +(-1)^n n!) + C
    \end{align*}
    となる.
\end{tleftbar}

\subsection*{p239:3-(iii)}
\addcontentsline{toc}{subsection}{\texorpdfstring{p239:3-(iii)}{p239:3-(iii)}}

\begin{tleftbar}
    部分積分法を用いると,
    \begin{align*}
        \int (\log x)^n \, dx & = \int (\log x)^n  (x)' \, dx                                                           \\
                              & = x (\log x)^n - n \int  \frac{(\log x)^{n-1}}{x} \cdot  x \, dx                        \\
                              & =  x (\log x)^n - n x(\log x)^{n-1} + (n-1) \int \frac{(\log x)^{n-2}}{x} \cdot x \, dx \\
                              & = \cdots = x (\log x)^n - n x(\log x)^{n-1} + \cdots + x(-1)^n n!+C
    \end{align*}
    となり,これが答えである.
\end{tleftbar}

\subsection*{p239:3-(iv)}
\addcontentsline{toc}{subsection}{\texorpdfstring{p239:3-(iv)}{p239:3-(iv)}}
\begin{leftbar}
    部分積分法を用いると,
    \begin{align*}
        \int \arcsin x \, dx & = \int (x)' \arcsin x \, dx                        \\
                             & = x \arcsin x  - \int \frac{x}{\sqrt{1-x^2}} \, dx \\
                             & = x \arcsin x + \sqrt{1-x^2} + C
    \end{align*}
    であり,これが答えである.
\end{leftbar}

\subsection*{p239:3-(v)}
\addcontentsline{toc}{subsection}{\texorpdfstring{p239:3-(v)}{p239:3-(v)}}

\begin{tleftbar}
    $\cos ^2 x = \frac{1+\cos 2x}{2}$なので,
    \begin{align*}
        \int \cos ^2 x \, dx & = \int \frac{1+\cos 2x}{2} \, dx      \\
                             & = \frac{x}{2}+\frac{1}{4} \sin 2x + C
    \end{align*}
    である.
\end{tleftbar}

\newpage

\subsection*{p239:3-(vi)}
\addcontentsline{toc}{subsection}{\texorpdfstring{p239:3-(vi)}{p239:3-(vi)}}

\begin{tleftbar}
    $(\log x)' = \frac{1}{x}$であることを用いて,
    \begin{align*}
        \int \frac{(\log x)^2}{x} \, dx & = \int (\log x)^2 (\log x)' \, dx \\
                                        & = \frac{(\log x)^3}{3} + C
    \end{align*}
    となり,これが答えである.
\end{tleftbar}

\subsection*{p239:3-(vii)}
\addcontentsline{toc}{subsection}{\texorpdfstring{p239:3-(vii)}{p239:3-(vii)}}

\begin{tleftbar}
    $x+1 =t$とおくと,$dt=dx$であり,
    \begin{align*}
        \int \frac{x^2+2}{(x+1)^3} \, dx & = \int \frac{(t-1)^2+2}{t^3} \, dt                                 \\
                                         & = \int \left (\frac{1}{t}-\frac{2}{t^2}+\frac{3}{t^3}\right) \, dt \\
                                         & = \log \abs{x+1}+\frac{2}{x+1}-\frac{3}{2(x+1)^2}+C
    \end{align*}
    となり,これが答えである.
\end{tleftbar}

\subsection*{p239:3-(viii)}
\addcontentsline{toc}{subsection}{\texorpdfstring{p239:3-(viii)}{p239:3-(viii)}}

\begin{tleftbar}
    $\cos ^2 x = 1- \sin ^2 x$なので,
    \begin{align*}
        \int \sin ^3 x \cos ^2 x \, dx & = \int \sin ^3 x (1-\sin ^2 x) \, dx = \int (\sin ^3 x - \sin ^5 x ) \, dx \\
                                       & = \int (\sin ^2 x - \sin ^4 x) \sin x \, dx                                \\
                                       & = \int \{ (1-t^2)- (1-t^2)^2 \} (-1) \, dt \quad (\cos x =t)               \\
                                       & = \int (t^4 -t^2) \, dt                                                    \\
                                       & = \frac{\cos ^5 x}{5}-\frac{\cos ^3 x}{3}+C
    \end{align*}
    である.
\end{tleftbar}

\subsection*{p239:3-(ix)}
\addcontentsline{toc}{subsection}{\texorpdfstring{p239:3-(ix)}{p239:3-(ix)}}

\begin{tleftbar}
    $\sqrt[6]{x}=t$とおくと,$x=t^6$であるから,$\frac{dx}{dt}=6t^5$である.これらを用いると,
    \begin{align*}
        \int \frac{1}{\sqrt{x}-\sqrt[3]{x}} \, dx & = \int \frac{1}{t^3-t^2} \cdot 6t^5 \, dt                                 \\
                                                  & = \int \frac{6t^3}{t-1} \, dt                                             \\
                                                  & = \int \frac{(t-1)(6t^2+6t+1)+6}{t-1} \, dt                               \\
                                                  & = \int \left (6t^2+6t+6 + \frac{6}{t-1}\right) \, dt                      \\
                                                  & = 2\sqrt{x}+3 \sqrt[3]{x} + 6 \sqrt[6]{x} + 6 \log \abs{\sqrt[6]{x}-1}+ C
    \end{align*}
    である
\end{tleftbar}


\section*{p239--240:4}
\addcontentsline{toc}{section}{\texorpdfstring{p239--240:4}{p239--240:4}}


\subsection*{p239--240:4-(i)}
\addcontentsline{toc}{subsection}{\texorpdfstring{p239--240:4-(i)}{p239--240:4-(i)}}

\begin{tleftbar}
    \begin{align*}
        \frac{1}{n+1}+ \frac{1}{n+2}+\dots + \frac{1}{2n} & = \sum_{k=1}^{n} \frac{1}{n+k}               \\
                                                          & = \frac{1}{n} \sum_{k=1}^{n} \frac{1}{1+k/n}
    \end{align*}
    であるから,
    \begin{align*}
        \lim_{n \to \infty} a_n & = \lim_{n \to \infty} \frac{1}{n} \sum_{k=1}^{n} \frac{1}{1+k/n} \\
                                & = \int_{0}^{1} \frac{1}{1+x} \, dx                               \\
                                & = \Bigl [ \log (1+x) \Bigl]_{0}^{1} = \log 2
    \end{align*}
    である.
\end{tleftbar}



\subsection*{p239--240:4-(ii)}
\addcontentsline{toc}{subsection}{\texorpdfstring{p239--240:4-(i)}{p239--240:4-(ii)}}

\begin{tleftbar}
    \begin{align*}
        \frac{1}{\sqrt{n^2+n}}+\frac{1}{\sqrt{n^2+2n}}+\dots+\frac{1}{\sqrt{n^2+n^2}} & = \sum_{k=1}^{n} \frac{1}{\sqrt{n^2+kn}}            \\
                                                                                      & = \frac{1}{n} \sum_{k=1}^{n} \frac{1}{\sqrt{1+k/n}}
    \end{align*}
    なので,
    \begin{align*}
        \lim_{n \to \infty} a_n & = \lim_{n \to \infty} \frac{1}{n} \sum_{k=1}^{n} \frac{1}{\sqrt{1+k/n}} \\
                                & =\int_{0}^{1} \frac{1}{\sqrt{1+x}} \, dx                                \\
                                & =\int_{1}^{\sqrt{2}} \frac{1}{t} \cdot 2t \, dt                         \\
                                & = \Bigl[2t \Bigl ]_{1}^{\sqrt{2}} =2(\sqrt{2}-1)
    \end{align*}
\end{tleftbar}


\section*{p247:1}
\addcontentsline{toc}{section}{\texorpdfstring{p247:1}{p247:1}}


\subsection*{p247:1-(i)}
\addcontentsline{toc}{subsection}{\texorpdfstring{p247:1-(i)}{p247:1-(i)}}

\begin{tleftbar}
    計算すると,
    \begin{align*}
        \int \frac{1}{x^3-x} \, dx & = \frac{1}{2} \int \left \{ \frac{-(x-1)+(x+1)}{(x-1)x(x+1)} \right \} \, dx                 \\
                                   & = \frac{1}{2} \int \left \{ -\frac{1}{(x+1)x}+\frac{1}{x(x-1)} \right \} \, dx               \\
                                   & = \frac{1}{2} \int \left \{ \frac{-(x+1)+x}{(x+1)x} + \frac{x-(x-1)}{x(x-1)} \right \} \, dx \\
                                   & = \frac{1}{2} \int \left ( -\frac{2}{x}+\frac{1}{x+1}+\frac{1}{x-1} \right) \, dx            \\
                                   & = -\log \abs{x} + \frac{1}{2} \log \abs{x^2-1}+ C
    \end{align*}
    であり,これが答えである.
\end{tleftbar}


\subsection*{p247:1-(ii)}
\addcontentsline{toc}{subsection}{\texorpdfstring{p247:1-(ii)}{p247:1-(ii)}}

\begin{tleftbar}
    $(x-1)(x-2)(x-3)=x^3 -6x^2+11x-6$であるから,
    \[
        \int \frac{x^3}{(x-1)(x-2)(x-3)} \, dx  = \int \left (1+ \frac{6x^2-11x+6}{(x-1)(x-2)(x-3)}\right) \, dx
    \]
    である.ここで,$A,B,C$を定数として,
    \[
        \frac{6x^2-11x+6}{(x-1)(x-2)(x-3)} = \frac{A}{(x-1)}+\frac{B}{(x-2)}+\frac{C}{(x-3)}
    \]
    とおく.これより,
    \begin{gather*}
        6x^2-11x+6 = A(x-2)(x-3)+B (x-1)(x-3)+C(x-1)(x-2) \\
        \therefore ~ A = \frac{1}{2}, \quad B = -\frac{1}{8},\quad C= \frac{27}{2}
    \end{gather*}
    となる.これより,
    \begin{align*}
        \int \frac{x^3}{(x-1)(x-2)(x-3)} \, dx & = \int \left \{ 1+ \frac{1}{2(x-1)} - \frac{1}{8(x-2)}+\frac{27}{2(x-3)} \right \} \, dx \\
                                               & = x + \frac{1}{2} \log \abs{\frac{(x-1)(x-3)^{27}}{(x-2)^{16}}}+C
    \end{align*}
    である.
\end{tleftbar}

\subsection*{p247:1-(iii)}
\addcontentsline{toc}{subsection}{\texorpdfstring{p247:1-(iii)}{p247:1-(iii)}}

\begin{tleftbar}
    $A,B,C,D$を定数として,
    \[
        \frac{x^3+1}{x(x-1)^3} = \frac{A}{x}+\frac{B}{x-1}+\frac{C}{(x-1)^2}+\frac{D}{(x-1)^3}
    \]
    とおくと,簡単な計算により,$A=-1,~B=2,~C=1,~D=2$とわかるので,
    \begin{align*}
        \int \frac{x^3+1}{x(x-1)^3} \, dx & = \int \left (-\frac{1}{x}+\frac{2}{x-1}+\frac{1}{(x-1)^2}+\frac{2}{(x-1)^3} \right ) \, dx \\
                                          & = \log \abs*{\frac{(x-1)^2}{x}} -\frac{1}{(x-1)}-\frac{1}{(x-1)^2}+ C
    \end{align*}
    を得る.
\end{tleftbar}


\subsection*{p247:1-(iv)}
\addcontentsline{toc}{subsection}{\texorpdfstring{p247:1-(iv)}{p247:1-(iv)}}

\begin{leftbar}
    命題 6.2 3)と例 3の結果を用いると,
    \begin{align*}
        \int \frac{dx}{(x^2 + 1)^3}
         & = \frac{x}{4(x^2 + 1)^2} + \frac{3}{8}(\frac{x}{x^2 + 1} + \arctan x) \\
         & = \frac{3x^3 + 5x}{8(x^2+1)^2} + \frac{3}{8} \arctan x.
    \end{align*}
\end{leftbar}

\subsection*{p247:1-(v)}
\addcontentsline{toc}{subsection}{\texorpdfstring{p247:1-(v)}{p247:1-(v)}}


\begin{tleftbar}
    $\tan \frac{x}{2} = t$と置くと,
    \begin{align*}
        \int \frac{dx}{a+b \cos x}
         & = 2 \int \frac{dt}{(a-b)t^2 + a+b} \\
         & =
        \begin{cases*}
            \frac{2}{a-b} \int \frac{\displaystyle dt}{\displaystyle t^2 + \frac{a+b}{a-b}}
                                 & if $a \neq b$, \\
            2 \int \frac{dt}{2a} & if $a = b$.
        \end{cases*}
    \end{align*}
    ここで,$a \neq b$の場合はさらに$a = -b$,$\frac{a+b}{a-b} > 0$,$\frac{a+b}{a-b} < 0$の三通りに分ける.p234の原始関数表より,
    \begin{align*}
        \int \frac{dx}{a+b \cos x}
         & =
        \begin{cases*}
            \frac{2}{a-b} \sqrt{\frac{a-b}{a+b}}
            \arctan \left({\sqrt{\frac{a-b}{a+b}} \tan \frac{x}{2}}\right) & if $a^2 > b^2$, \\
            \frac{1}{a-b} \sqrt{\frac{b-a}{b+a}}
            \log \abs{\frac{\displaystyle\tan \frac{x}{2}
                    - \sqrt{\frac{b+a}{b-a}}}{\displaystyle\tan \frac{x}{2} + \sqrt{\frac{b+a}{b-a}}}}
                                                                           & if $a^2 < b^2$, \\
            \left(-a \tan \frac{x}{2}\right)^{-1}                          & if $a = -b$,    \\
            \frac{1}{a} \tan {\frac{x}{2}}                                 & if $a = b$.
        \end{cases*}
    \end{align*}
\end{tleftbar}

\subsection*{p247:1-(vi)}
\addcontentsline{toc}{subsection}{\texorpdfstring{p247:1-(vi)}{p247:1-(vi)}}


\begin{leftbar}
    $\frac{1}{\cos^2 x} = (\tan x)'$と見て部分積分する.
    \begin{align*}
        \int \frac{dx}{\sin^3 x \cos^2 x}
         & = \int \frac{(\tan x)'}{\sin^3 x} \,dx                                                                \\
         & = \frac{\tan x}{\sin^3 x} - \int \tan x \left(\frac{1}{\sin^3 x}\right)' \,dx                         \\
         & = \frac{1}{\sin^2 x \cos x} + 3 \int \frac{dx}{\sin^3 x}                                              \\
         & = \frac{1}{\sin^2 x \cos x} - \frac{3 \cos x}{2 \sin^2 x} + \frac{3}{2} \log \abs*{\tan \frac{x}{2}}.
    \end{align*}
\end{leftbar}


\subsection*{p247:1-(vii)}
\addcontentsline{toc}{subsection}{\texorpdfstring{p247:1-(vii)}{p247:1-(vii)}}

\begin{leftbar}
    分母分子に$1-(\sin x + \cos x)$を掛ける.
    \begin{align*}
        \int \frac{\sin x}{1 + \sin x + \cos x} \,dx
         & = \int \frac{\sin x - \sin^2 x - \sin x\cos x}{1-(\sin x + \cos x)^2} \,dx \\
         & = -\frac{1}{2} \int (\sec x - \tan x - 1) \,dx                             \\
         & = -\frac{1}{2}(\log \abs{\sec x + \tan x} + \log \abs{\cos x} - x)         \\
         & = \frac{1}{2}(x - \log \abs{1 + \sin x}).
    \end{align*}
\end{leftbar}

\subsection*{p247:1-(viii)}
\addcontentsline{toc}{subsection}{\texorpdfstring{p247:1-(viii)}{p247:1-(viii)}}



\begin{tleftbar}
    $\tan x = t$とおくと,$\frac{dt}{dx}= \frac{1}{\cos ^2 x}$であるから,求める不定積分は,
    \begin{align*}
        \int \frac{1/\cos ^2 x}{a^2 + b^2 \tan ^2 x} \, dx & = \int \frac{1}{a^2+b^2 t^2} \, dt                          \\
                                                           & = \frac{1}{ab} \arctan \left (\frac{b}{a} \tan x \right)+ C
    \end{align*}
    である.
\end{tleftbar}



\subsection*{p247:1-(ix)}
\addcontentsline{toc}{subsection}{\texorpdfstring{p247:1-(ix)}{p247:1-(ix)}}
\begin{leftbar}
    $\log x = t$と変換して部分積分すると,
    \begin{align*}
        \int \sin \log x \,dx
         & = \int e^t \sin t \,dt                            \\
         & = e^t \sin t - \int e^t \cos t \,dt               \\
         & = e^t \sin t - e^t \cos t - \int e^t \sin t \,dt.
    \end{align*}
    よって,
    \begin{align*}
        \int \sin \log x \,dx
         & = \frac{1}{2}(\sin t - \cos t)            \\
         & = \frac{1}{2}(\sin \log x - \cos \log x).
    \end{align*}
\end{leftbar}


\subsection*{p247:1-(x)}
\addcontentsline{toc}{subsection}{\texorpdfstring{p247:1-(x)}{p247:1-(x)}}

\begin{leftbar}
    $x = \sin t$と置く.
    \begin{align*}
        \int \sqrt {\frac{1-x}{1+x}} \,dx
         & = \int \frac{\sqrt{1-x^2}}{1+x} \,dx                              \\
         & = \int \frac{\cos^2 t}{1 + \sin t} \,dt                           \\
         & = \int \frac{\cos^2 t(1 - \sin t)}{(1 + \sin t)(1 - \sin t)} \,dt \\
         & = \int (1 - \sin t) \,dt                                          \\
         & = t + \cos t                                                      \\
         & = \arcsin x  + \cos \arcsin x                                     \\
         & = \arcsin x + \sqrt{1-x^2}.
    \end{align*}
\end{leftbar}

\subsection*{p247:1-(xi)}
\addcontentsline{toc}{subsection}{\texorpdfstring{p247:1-(xi)}{p247:1-(xi)}}


\begin{tleftbar}
    $x=\frac{1}{t}$とおくと,$\frac{dx}{dt}=-\frac{1}{t^2}$である.また,$x^2+1 = \frac{1}{t^2} +1$となる.よって,
    \begin{align*}
        \int \frac{1}{x\sqrt{x^2+1}} \, dx & = \int \frac{t}{\sqrt{1/t^2+1}} \cdot (-1/t^2) \, dt \\
                                           & = -\int \frac{1}{\sqrt{t^2+1}} \, dt                 \\
                                           & = \log \abs{t+\sqrt{t^2+1}}+ C                       \\
                                           & = \log \abs{\frac{1+\sqrt{x^2+1}}{x}}+C
    \end{align*}
    である.
\end{tleftbar}


\subsection*{p247:1-(xii)}
\addcontentsline{toc}{subsection}{\texorpdfstring{p247:1-(xii)}{p247:1-(xii)}}


\begin{tleftbar}
    \begin{align*}
        \int \frac{1}{\sqrt{(x-a)(b-x)}} \, dx & = \int \frac{1}{\sqrt{\left (\dfrac{b-a}{2} \right )^2 - \left (x-\dfrac{a+b}{2} \right )^2 }} \\
                                               & = \arcsin \left(\frac{2x-a-b}{b-a} \right)+C
    \end{align*}
\end{tleftbar}

\subsection*{p247:1-(xiii)}
\addcontentsline{toc}{subsection}{\texorpdfstring{p247:1-(xiii)}{p247:1-(xiii)}}


\begin{tleftbar}
    $x+\frac{1}{x}=t$とおくと,$\frac{dt}{dx}=1-\frac{1}{x^2}$であり,求める不定積分は,
    \begin{align*}
        \int \frac{1-x^2}{1+x^2\sqrt{1+x^4}} \, dx & = \int \frac{1-x^2}{x^2 (x+1/x)\sqrt{x^2 + 1/x^2}} \, dx \\
                                                   & =\int \frac{1}{t\sqrt{t^2-2}} \, dt
    \end{align*}
    となる.ここで,$s=\frac{1}{t}$とすると,$\frac{dt}{ds}=-\frac{1}{s^2}$であり,
    \begin{align*}
        \int \frac{1}{t \sqrt{t^2-2}} \, dt & = \int \frac{1}{\sqrt{1-2s^2}} \, ds                                 \\
                                            & = \frac{1}{\sqrt{2}} \arcsin (\sqrt{2}s)+C                           \\
                                            & = \frac{1}{\sqrt{2}} \arcsin \left(\frac{\sqrt{2}x}{x^2+1} \right)+C
    \end{align*}
    となる.
\end{tleftbar}

\subsection*{p247:1-(xiv)}
\addcontentsline{toc}{subsection}{\texorpdfstring{p247:1-(xiv)}{p247:1-(xiv)}}


\begin{leftbar}
    $1 = (x)'$と見て部分積分する.
    \begin{align*}
        \int \arctan x \,dx
         & = x \arctan x - \int \frac{x}{1+x^2} \,dx \\
         & = x \arctan x - \frac{1}{2} \log (x^2+1).
    \end{align*}
\end{leftbar}

\newpage


\section*{p290:3}
\addcontentsline{toc}{section}{\texorpdfstring{p290:3}{p290:3}}

\begin{leftbar}
    求める面積を $S$とすると,第1象限の面積を$4$倍すればよいので,
    \begin{align*}
        S & =4 \cdot  \frac{1}{2}\int_{0}^{\frac{\pi}{4}} r^2 \, d \theta \\
          & = 2 \int_{0}^{\frac{\pi}{4}} (2a^2 \cos 2\theta) \, d \theta  \\
          & =2\Bigl [ a^2 \sin 2\theta \Bigl ]_{0}^{\frac{\pi}{4}}        \\
          & = 2\left ( a^2 \sin (\pi/2)-a^2 \sin (0) \right)              \\
          & = 2a^2
    \end{align*}
    となる.
\end{leftbar}

\begin{tikzpicture}[scale=1.8]
    % パラメータ
    \def\a{1}

    % 領域を塗りつぶす
    \fill[gray!20, domain=-pi/4:pi/4, variable=\t]
    plot ({\a*sqrt(2*cos(2*\t r)) * cos(\t r)}, {\a*sqrt(2*cos(2*\t r)) * sin(\t r)})
    -- (0,0) -- cycle;


    \fill[gray!20, domain=-pi/4:pi/4, variable=\t]
    plot ({-\a*sqrt(2*cos(2*\t r)) * cos(-\t r)}, {-\a*sqrt(2*cos(2*\t r)) * sin(-\t r)})
    -- (0,0) -- cycle;

    % 軸の描画
    \draw[->] (-1.7,0) -- (1.7,0) node[below] {$x$};
    \draw[->] (0,-1.7) -- (0,1.7) node[left] {$y$};

    % 方程式の描画
    \draw[black, thick, domain=-pi/4:pi/4, variable=\t]
    plot ({\a*sqrt(2*cos(2*\t r)) * cos(\t r)}, {\a*sqrt(2*cos(2*\t r)) * sin(\t r)});


    \draw[black, thick, domain=-pi/4:pi/4, variable=\t]
    plot ({-\a*sqrt(2*cos(2*\t r)) * cos(-\t r)}, {-\a*sqrt(2*cos(2*\t r)) * sin(-\t r)});


    \node at (-1.5,1.5) {$r^2 = 2a^2 \cos(2\theta)$};
\end{tikzpicture}

\newpage


\section*{p290:4}
\addcontentsline{toc}{section}{\texorpdfstring{p290:4}{p290:4}}

\begin{leftbar}
    求める面積を$S$,求める求める体積を$V$とおくと,
    \begin{align*}
        S & = \frac{1}{2} \int_{0}^{2\pi} r^2 \, d \theta                                                           \\
          & =\frac{1}{2} \int_{0}^{2\pi} a^2 (1+ 2\cos \theta + \cos ^2 \theta) \, d \theta                         \\
          & = \frac{1}{2} \int_{0}^{2\pi} a^2 \left (1+ 2\cos \theta + \frac{1+\cos 2\theta}{2} \right) \, d \theta \\
          & = a^2 \Bigl [ \frac{3}{2} x + 2\sin \theta +\frac{\sin 2 \theta}{4} \Bigl ]_{0}^{2\pi}                  \\
          & = \frac{3 \pi a^2}{2}
    \end{align*}
    となる.
\end{leftbar}

\begin{tikzpicture}[scale=1.8]
    % グレーで塗りつぶす領域の描画
    \fill[gray!30, domain=0:2*pi, variable=\t, smooth]
    plot ({\t r}:{1 + cos(\t r)});

    % r=1+cos(theta)の描画
    \draw[domain=0:2*pi, variable=\t, smooth]
    plot ({\t r}:{1 + cos(\t r)});


    % 軸の描画
    \draw[->] (-1.5,0) -- (2.5,0) node[right] {$x$};
    \draw[->] (0,-2.5) -- (0,2.5) node[above] {$y$};

\end{tikzpicture}

\newpage


\section*{p325--326:1}
\addcontentsline{toc}{section}{\texorpdfstring{p325--326:1}{p325--326:1}}


\subsection*{p325--326:1-(i)}
\addcontentsline{toc}{subsection}{\texorpdfstring{p325--326:1-(i)}{p325--326:1-(i)}}

\begin{tleftbar}
    \[
        I = \int_{-\infty}^{\infty} \frac{dx}{(t+x^2)^{n+1}} \quad (t > 0, n \in \mathbb{N})
    \]

    \[
        (1+x^2)^{n+1} = O(x^{2n+2}) \quad (x \to \pm \infty)
    \]

    \[
        \frac{1}{(t+x^2)^{n+1}} = O(x^{-2n-2}) \quad (x \to \pm \infty)
    \]
    ここで,
    \[
        -2n-2 < -1
    \]
    であるから,定理11.3により,広義積分可能である.

    \[
        x = \sqrt{t} \tan \theta
    \]
    とすると.
    \begin{alignat*}{2}
        I & = \int_{-\frac{\pi}{2}}^{\frac{\pi}{2}} \frac{\cos^{2n+2} \theta}{t^{n+1} } \frac{\sqrt{t}}{\cos^2 \theta} \, d \theta &       &                            \\
          & = \frac{1}{t^{n+1/2}} \int_{-\frac{\pi}{2}}^{\frac{\pi}{2}} \cos^{2n} \theta \, d\theta                                &       &                            \\
          & = \frac{1}{t^{n+1/2}} \int_{0}^{\frac{\pi}{2}} 2 \cos^{2n} \theta \, d\theta                                           &       &                            \\
          & = \frac{1}{t^{n+1/2}} \frac{\Gamma (n+1/2) \Gamma (1/2)}{\Gamma(n+1)}                                                  & \quad & \text{($\because$~定理12.4)} \\
          & = \frac{(2n-1)!! / 2^n \sqrt{\pi} \cdot \sqrt{\pi}}{t^{n+1/2} \cdot (n!)}                                              & \quad & \text{($\because$~定理12.4)} \\\
          & = \frac{(2n-1)!!}{(2n)!!} t^{ - \frac{2n+1}{2}} \pi .
    \end{alignat*}
\end{tleftbar}


\newpage


\subsection*{p325--326:1-(ii)}
\addcontentsline{toc}{subsection}{\texorpdfstring{p325--326:1-(ii)}{p325--326:1-(ii)}}

\begin{tleftbar}
    \[
        F(t) = \int_0^\pi \frac{dx}{(t + \cos x)^2} \quad (t > 1)
    \]

    \[
        f(x,t) = - \frac{1}{t+\cos x} \quad ( 0 \leqq x \leqq \pi , ~ t>1)
    \]
    とすると,
    \[
        \frac{\partial}{\partial t}  f(x,t) = \frac{1}{(t+\cos x)^2}.
    \]
    いま,$f(x,t)$,$\partial f(x,t) / \partial t $は$[0,\pi] \times (1,\infty)$で連続であるから,定理14.1により,
    \begin{align*}
        \frac{\partial}{\partial t} \int_0^\pi f(x,t) \, dx & = \int_0^\pi \frac{\partial}{\partial t} f(x,t) \, dx \\
                                                            & = F(t)
    \end{align*}
    となる.

    また,
    \begin{align*}
        \int_0^\pi f(x,t) \, dx & = \int_0^\pi - \frac{dx}{t+\cos x}                                                        \\
                                & = \int_0^\infty \left ( - \frac{1}{t+\dfrac{1-s^2}{1+s^2}} \right) \frac{2ds}{1+s^2}      \\
                                & = -2\int_0^\infty \frac{ds}{(t-1)s^2+t+1}                                                 \\
                                & = -2 \Bigl [ \frac{1}{\sqrt{t+1}} \arctan \sqrt{\frac{t-1}{t+1}}s \Bigr]_{s=0}^{s=\infty} \\
                                & = \frac{-\pi}{\sqrt{t^2-1}}
    \end{align*}
    より,
    \begin{align*}
        F(t) & = \frac{d}{dt} \frac{-\pi}{\sqrt{t-2-1}}                 \\
             & = \pi \left(-\frac{1}{2} \right) (t^2-1)^{-3/2} \cdot 2t \\
             & = -\frac{\pi t}{(t^2-1)^{3/2}}.
    \end{align*}
\end{tleftbar}

\newpage

\part*{第5章:級数}
\addcontentsline{toc}{part}{\texorpdfstring{第5章:級数}{第5章:級数}}


\section*{p366:1}
\addcontentsline{toc}{section}{\texorpdfstring{p366:1}{p366:1}}

\kakko{補題}

$a_{2n} \to \alpha$,$a_{2n+1} \to \beta$$\quad (\alpha,\beta \in [-\infty,\infty])$ならば,集積値全体の集合は$\{\alpha,\beta\}$である.

    \begin{proof}
        $a_n$の収束部分列$a_{n(k)}$を任意にとる.
        $\{n(k):k \in \mathbb{N}\}$は偶数,奇数の少なくとも一方を無限個含む.
        偶数がそうであったとして,$n(k)$から奇数を取り除いて小さい順に並べた列の$l$番目の項を$a_{n(k(l))}$とする.
        $a_{n(k(l))}$は$a_{2n}$の部分列であるから,仮定より$a_{n(k(l))} \to \alpha \quad (l \to \infty)$である.
        $a_{n(k)}$は収束するから,$a_{n(k)} \to \alpha \quad (k \to \infty)$である.
        奇数が無限個含まれる場合も同様に考えて,$a_{n(k)} \to \beta \quad (k \to \infty)$である.
        $a_{n(k)}$の取り方は任意だったから,集積値は$\alpha$と$\beta$のみである.
    \end{proof}

    \subsection*{p366:1-(i)}
    \addcontentsline{toc}{subsection}{\texorpdfstring{p366:1-(i)}{p366:1-(i)}}

    \begin{tleftbar}
        \[
            \lim_{m \to \infty} a_{2m} = 1 , \quad \lim_{m \to \infty} a_{2m-1} = -1
        \]
        であり,$(a_n)_{n \in \mathbb{N}}$の集積値全体の集合は補題により$\{ -1 , 1 \}$である.

        ゆえに,
        \[
            \varlimsup_{n \to \infty} a_n = 1 , \quad \varliminf_{n \to \infty} a_n =-1
        \]
        である.
    \end{tleftbar}


    \subsection*{p366:1-(ii)}
    \addcontentsline{toc}{subsection}{\texorpdfstring{p366:1-(ii)}{p366:1-(ii)}}

    \begin{tleftbar}
        \[
            \lim_{m \to \infty} a_{2m} = \infty , \quad \lim_{m \to \infty} a_{2m-1} = -\infty
        \]
        であり,$(a_n)_{n \in \mathbb{N}}$の集積値全体の集合は補題により$\{ -\infty , \infty \}$である\footnote{補完数直線を考えているため,これでよい.}.

        ゆえに,
        \[
            \varlimsup_{n \to \infty} a_n = \infty , \quad \varliminf_{n \to \infty} a_n =-\infty
        \]
        である.
    \end{tleftbar}

    \subsection*{p366:1-(iii)}
    \addcontentsline{toc}{subsection}{\texorpdfstring{p366:1-(iii)}{p366:1-(iii)}}

    \begin{tleftbar}
        $ a_{n} = a_{n+6}$であることを考慮すると,$(a_n)_{n \in \mathbb{N}}$の集積値全体の集合は$\{ 0 , \pm \frac{\sqrt{3}}{2} \}$である.
        ゆえに,
        \[
            \varlimsup_{n \to \infty} a_n = \frac{\sqrt{3}}{2} , \quad \varliminf_{n \to \infty} a_n =-\frac{\sqrt{3}}{2}
        \]
        である.
    \end{tleftbar}

    \newpage


    証明の前に,$\pm \infty$について演算を定義しておく.
    \begin{align*}
        a + \infty           & = \infty + a = \infty \qquad \text{if $a \in (-\infty,\infty]$},                   \\
        a - \infty           & = a + (-\infty) = -\infty + a = -\infty \qquad \text{if $a \in [-\infty,\infty)$}, \\
        a \cdot (\pm \infty) & = (\pm \infty) \cdot a =
        \begin{cases*}
            \pm \infty \quad & if $a \in (0,\infty]$,  \\
            \mp \infty \quad & if $a \in [-\infty,0)$.
        \end{cases*}
    \end{align*}
$0 \cdot \infty$や$\infty - \infty$などは定義しない.
このとき,補完数直線において次の命題が成り立つ.
\begin{lemm*}
    実数列$\{a_n\}$,$\{b_n\}$が$a_n \xrightarrow{n \to \infty} a$,$b_n \xrightarrow{n \to \infty} b$を満たしており,$a+b$,$ab$が定義されるとき,それぞれ$a_n + b_n \xrightarrow{n \to \infty} a+b$,$a_n b_n \xrightarrow{n \to \infty} ab$が成り立つ.
\end{lemm*}
問題2),3)において,演算が定義されないような場合は除外したものを証明する.


\section*{p366:2}
\addcontentsline{toc}{section}{\texorpdfstring{p366:2}{p366:2}}

\begin{proof}
    \begin{leftbar}
        \[
            \limsup _{n \to \infty} a_n + \liminf _{n \to \infty} b_n \leq \limsup _{n \to \infty} (a_n + b_n) \leq \limsup _{n \to \infty} a_n + \limsup _{n \to \infty} b_n
        \]
        を示す.その他の不等式についても同様である.
        $n \in \mathbb{N}$を任意に取って固定する.上限の定義より,$m \geq n$に対して,
        \begin{align*}
            a_m & \leq \sup _{k \geq n} a_k, \\
            b_m & \leq \sup _{k \geq n} b_k
        \end{align*}
        であるから,
        \[
            a_m + b_m \leq \sup _{k \geq n} a_k + \sup _{k \geq n} b_k
        \]
        が成り立つ.左辺でmについての上限を取ると,
        \[
            \sup _{k \geq n} (a_k + b_k) \leq \sup _{k \geq n} a_k + \sup _{k \geq n} b_k
        \]
        となる.(右辺は$\{a_m + b_m; m \geq n\}$の上界であり,$\sup$は最小上界であるからこの操作は正当.)$n \to \infty$とすれば,
        \[
            \limsup _{n \to \infty} (a_n + b_n) \leq \limsup _{n \to \infty} a_n + \limsup _{n \to \infty} b_n
        \]
        となって右側の不等式が示される.
        また,$m \geq n$に対し,
        \[
            a_m + \inf _{k \geq n} b_k \leq a_m + b_m \leq \sup _{k \geq n} (a_k + b_k)
        \]
        であるから,左辺の上限を取って,
        \[
            \sup _{k \geq n} a_k + \inf _{k \geq n} b_k \leq \sup _{k \geq n} (a_k + b_k)
        \]
        となる.$n \to \infty$とすれば左側の不等式も示される.
    \end{leftbar}
\end{proof}

\section*{p366:3}
\addcontentsline{toc}{section}{\texorpdfstring{p366:3}{p366:3}}

\begin{leftbar}
    \begin{proof}
        $\limsup$についてのみ示す.$\liminf$についても同様である.
        $\limsup_{n \to \infty} a_n = 0$の場合,$\limsup_{n \to \infty} b_n < \infty$であるから$\sup_{k \geq n} b_k < \infty$となり$\left(\sup _{k \geq n} a_k\right) \left(\sup _{k \geq n} b_k\right)$が定義されることに注意する.$a_n, b_n \geq 0$であるから,$m \geq n$に対して,
        \[
            a_m b_m \leq \left(\sup _{k \geq n} a_k\right) \left(\sup _{k \geq n} b_k\right)
        \]
        が成り立つ.左辺で$m$について上限を取ってから$n \to \infty$とすると,
        \[
            \limsup_{n \to \infty} (a_n b_n) \leq \left(\limsup _{n \to \infty} a_n\right) \left(\limsup_{n \to \infty} b_n\right)
        \]
        となる.また,第I章の命題1.4,1.6より,
        \begin{align*}
            \sup_{k \geq n} (c a_k) & = c\sup_{k \geq n} a_k \quad (c > 0),                             \\
            \sup_{k \geq n} (c a_k) & = -\inf_{k \geq n} (-c a_k) = c\inf_{k \geq n} a_k \quad (c < 0),
        \end{align*}
        が成り立つ.$n \to \infty$とすれば残りの等式が得られる.
    \end{proof}
\end{leftbar}

\section*{p366:4}
\addcontentsline{toc}{section}{\texorpdfstring{p366:4}{p366:4}}

\begin{leftbar}
    \begin{proof}
        p16の問題4)の一般化である.右側の不等式のみ示す.
        $\alpha = \limsup _{n \to \infty} \frac{a_{n+1}}{a_{n}}$とする.$\alpha = \infty$ならば明らかであるから$\alpha \neq \infty$の場合を考える.$\varepsilon > 0$を任意に取ると,定理1.4より,
        \[
            \frac{a_{n+1}}{a_{n}} < \alpha + \varepsilon \quad (n \geq N)
        \]
        となるような$N$が取れる.この不等式を繰り返し用いて,
        \begin{align*}
            a_n
             & < (\alpha + \varepsilon) a_{n-1}   \\
             & < (\alpha + \varepsilon)^2 a_{n-2} \\
             & < \ldots                           \\
             & < (\alpha + \varepsilon)^{n-N} a_N
        \end{align*}
        がわかる.両側の式を$1/n$乗してから$\limsup$を取ると,
        \[
            \limsup _{n \to \infty} \sqrt[n]{a_n} < \lim _{n \to \infty} (\alpha + \varepsilon)^{1-N/n} \sqrt[n]{a_N} = \alpha + \varepsilon
        \]
        となる.$\varepsilon \to +0$とすると,
        \[
            \limsup _{n \to \infty} \sqrt[n]{a_n} \leq \alpha
        \]
        である.これが証明すべきことであった.
    \end{proof}
\end{leftbar}

\newpage

\begin{thebibliography}{9}
    \bibitem{sugiura} 杉浦光夫『解析入門I』,東京大学出版会,1980
\end{thebibliography}

\end{document}